\section{Cross Section and Upper Limit}
\label{sec.xsec_ul}

In this section, the measurement of the cross sections for the exclusive $\gamma p \rightarrow \phi \pi^+ \pi^- p$ and $\gamma p \rightarrow \phi f_0 p$ reactions is discussed, as well as the determination of an upper limit on the production cross section of the $Y(2175)$ in $\gamma p \rightarrow Y(2175) p \rightarrow \phi \pi^+ \pi^- p$ and $\gamma p \rightarrow Y(2175) p \rightarrow \phi f_0 p$ reactions.

\subsection{Cross Section for \texorpdfstring{$\bm{\gamma p \rightarrow \phi \pi^{+} \pi^{-} p}$}{} }
\label{sec.xsec_ul.phi2pi}

To study the effect of the photon beam energy ($E_{\gamma}$) in both, the coherent and incoherent region, as well as the momentum transfer (-$t$) dependence on the cross section, the total hadronic cross section for $\gamma p \rightarrow \phi \pi^{+} \pi^{-} p$ reaction in $t$-channel is studied as a function of both $E_{\gamma}$ and -$t$. The cross section is measured in the $E_{\gamma}$ region of 6.5 - 11.6 GeV, distributed equally into 10 bins of 0.51 GeV width, and in the 0 - 4 GeV$^2$ region of -$t$, divided into 10 intervals of 0.4 GeV$^2$ width. The total cross section for the $\gamma p \rightarrow \phi \pi^{+} \pi^{-} p$ reaction is defined as

\begin{equation}
    \label{eq.xsec_ul.phi2pi}
    \sigma_{\gamma p \rightarrow \phi \pi^{+} \pi^{-} p} = \frac{N_{\phi\pi^+\pi^-}^{Data}}{\varepsilon~\mathcal{L}~BR(\phi\rightarrow K^{+}K^{-})},\\
\end{equation}

\noindent The numerator is the number of $\phi\pi^+\pi^-$ signal events observed in real data. The reconstruction efficiency ($\varepsilon$) is ratio of the number of $\phi\pi^+\pi^-$ reconstructed signal events in MC simulation and the total number of generated events. The luminosity ($\mathcal{L}$) is the product of the integrated flux extracted from Fig.~\ref{fig.data_mc.data_tag_flux} and the target thickness of 1.273 b$^{-1}$. The last term is the branching ratio of $\phi\rightarrow K^{+}K^{-}$ taken from~\cite{Tanabashi18}, with $BR(\phi\rightarrow K^{+}K^{-})$ = $0.492 \pm 0.005$.
\par The number of generated events are extracted in bins of $E_{\gamma}$ and -$t$, from the total generated events in the MC samples. The total generated events are distributed over the selected region of $E_{\gamma}$ and -$t$, as shown in Fig.~\ref{fig.xsec_ul.phi2pi.1}.

\begin{figure}[H]
    \centering
    \begin{subfigure}[b]{0.5\textwidth}
        \includegraphics[width=\textwidth]{plots/ctot_beame_tru.eps}
        \caption{}
        \label{fig.xsec_ul.phi2pi.1.a}
    \end{subfigure}\hfill
    \begin{subfigure}[b]{0.5\textwidth}
        \includegraphics[width=\textwidth]{plots/ctot_t_tru.eps}
        \caption{}
        \label{fig.xsec_ul.phi2pi.1.b}
    \end{subfigure}
    \caption{\label{fig.xsec_ul.phi2pi.1}The total generated $\phi \pi^{+} \pi^{-} p$ MC samples distributed in (a) $E_{\gamma}$ and (b) -$t$ bins.}
\end{figure}

The number of $\phi \pi^{+} \pi^{-}$ signal events in both MC and data are extracted from fitting the $K^+K^-$ invariant mass in every $E_{\gamma}$ and -$t$ bin. The correlations between the $K^+K^-$ invariant mass and both $E_{\gamma}$ and -$t$ are shown in Fig.~\ref{fig.xsec_ul.phi2pi.2}. A clear $\phi(1020)$ resonance is seen around the mass of 1.020 GeV/$c^2$, corresponding to the horizontal band in Fig.~\ref{fig.xsec_ul.phi2pi.2}. The signal shape is described by a Voigtian model and the background by a 4$^{th}$ degree Chebyshev polynomial. The $\phi \pi^{+} \pi^{-}$ yields obtained for each $E_{\gamma}$ and -$t$ bin are shown in Fig.~\ref{fig.xsec_ul.phi2pi.7} and Fig.~\ref{fig.xsec_ul.phi2pi.8}, respectively. As expected, the yield is more important in the coherent beam region and at low momentum transfer. The small yield drop in the first -$t$ bin, could be due to the detection loss of the recoiled protons at low momentum.

\begin{center}
\null
\vfill
\begin{figure}[htbp]
    \centering
    \begin{subfigure}[b]{0.5\textwidth}
        \includegraphics[width=\textwidth]{plots/c2_phie_17.eps}
        \caption{}
        \label{fig.xsec_ul.phi2pi.2.a}
    \end{subfigure}\hfill
    \begin{subfigure}[b]{0.5\textwidth}
        \includegraphics[width=\textwidth]{plots/c2_phie_mc_17.eps}
        \caption{}
        \label{fig.xsec_ul.phi2pi.2.b}
    \end{subfigure}
    \begin{subfigure}[b]{0.5\textwidth}
        \includegraphics[width=\textwidth]{plots/c2_phit_17.eps}
        \caption{}
        \label{fig.xsec_ul.phi2pi.2.c}
    \end{subfigure}\hfill
    \begin{subfigure}[b]{0.5\textwidth}
        \includegraphics[width=\textwidth]{plots/c2_phit_mc_17.eps}
        \caption{}
        \label{fig.xsec_ul.phi2pi.2.d}
    \end{subfigure}
    \caption{\label{fig.xsec_ul.phi2pi.2}$K^{+}K^{-}$ invariant mass versus $E_{\gamma}$ in (a) MC and (b) data, as well as versus -$t$ in (c) MC and (d) data, for the 2017 sample. The horizontal narrow band $\sim$ 1.020 GeV/$c^2$ is the $\phi(1020)$ resonance.}
\end{figure}
\null
\vfill
\end{center}

\begin{figure}[H]
    \centering
    \includegraphics[width=1.0\textwidth]{plots/c_phie1_mc_17.eps}
    \caption{\label{fig.xsec_ul.phi2pi.3}$K^{+}K^{-}$ invariant mass in $E_{\gamma}$ bins for 2017 MC sample. The $E_{\gamma}$ bin ranges and the fit parameters for the total (red), signal (blue), and background (dashed) fits are shown.}
\end{figure}

\begin{figure}[H]
    \centering
    \includegraphics[width=1.0\textwidth]{plots/c_phie1_17.eps}
    \caption{\label{fig.xsec_ul.phi2pi.4}$K^{+}K^{-}$ invariant mass in $E_{\gamma}$ bins for 2017 dataset. The $E_{\gamma}$ bin ranges and the fit parameters for the total (red), signal (blue), and background (dashed) fits are shown.}
\end{figure}

\begin{figure}[H]
    \centering
    \includegraphics[width=1.0\textwidth]{plots/c_phit1_mc_17.eps}
    \caption{\label{fig.xsec_ul.phi2pi.5}$K^{+}K^{-}$ invariant mass in -$t$ bins for 2017 MC sample. The -$t$ bin ranges and the fit parameters for the total (red), signal (blue), and background (dashed) fits are shown.}
\end{figure}

\begin{figure}[H]
    \centering
    \includegraphics[width=1.0\textwidth]{plots/c_phit1_17.eps}
    \caption{\label{fig.xsec_ul.phi2pi.6}$K^{+}K^{-}$ invariant mass in -$t$ bins for 2017 data sample. The -$t$ bin ranges and the fit parameters for the total (red), signal (blue), and background (dashed) fits are shown.}
\end{figure}

\begin{figure}[H]
    \centering
    \begin{subfigure}[b]{0.5\textwidth}
        \includegraphics[width=\textwidth]{plots/cmgphie_mc.eps}
        \caption{}
        \label{fig.xsec_ul.phi2pi.7.a}
    \end{subfigure}\hfill
    \begin{subfigure}[b]{0.5\textwidth}
        \includegraphics[width=\textwidth]{plots/cmgphie.eps}
        \caption{}
        \label{fig.xsec_ul.phi2pi.7.b}
    \end{subfigure}
    \caption{\label{fig.xsec_ul.phi2pi.7}$\phi \pi^+ \pi^-$ yields versus $E_{\gamma}$ in (a) MC and (b) data. The yield for 2017 (blue), Spring 2018 (red), and Fall 2018 (magenta) are displayed.}
\end{figure}

\begin{figure}[H]
    \centering
    \begin{subfigure}[b]{0.5\textwidth}
        \includegraphics[width=\textwidth]{plots/cmgphit_mc.eps}
        \caption{}
        \label{fig.xsec_ul.phi2pi.8.a}
    \end{subfigure}\hfill
    \begin{subfigure}[b]{0.5\textwidth}
        \includegraphics[width=\textwidth]{plots/cmgphit.eps}
        \caption{}
        \label{fig.xsec_ul.phi2pi.8.b}
    \end{subfigure}
    \caption{\label{fig.xsec_ul.phi2pi.8}$\phi \pi^+ \pi^-$ yields versus -$t$ in (a) MC and (b) data. The yield for the 2017 (blue), Spring 2018 (red), and Fall 2018 (magenta) are displayed.}
\end{figure}

The efficiency is then calculated as the ratio of the $\phi \pi^+ \pi^-$ reconstructed MC yields and the total number of generated MC events. The results are plotted in Fig.~\ref{fig.xsec_ul.phi2pi.7}, showing the efficiencies versus $E_{\gamma}$ and -$t$ for the different MC samples. The efficiencies are different between the datasets, with almost $\sim$ $50\%$ between each sample. This is mainly due to the different running conditions and random trigger rates included in the different MC sets.

\begin{figure}[H]
    \centering
    \begin{subfigure}[b]{0.5\textwidth}
        \includegraphics[width=\textwidth]{plots/cmgeeff.eps}
        \caption{}
        \label{fig.xsec_ul.phi2pi.9.a}
    \end{subfigure}\hfill
    \begin{subfigure}[b]{0.5\textwidth}
        \includegraphics[width=\textwidth]{plots/cmgteff.eps}
        \caption{}
        \label{fig.xsec_ul.phi2pi.9.b}
    \end{subfigure}
    \caption{\label{fig.xsec_ul.phi2pi.9}The reconstruction efficiency versus (a) $E_{\gamma}$ and (b) -$t$, for $\phi \pi^+ \pi^-$ MC samples of 2017 (blue), Spring 2018 (red), and Fall 2018 (magenta). The relative ratio of Spring 2018 (red), and Fall 2018 (magenta), w.r.t to 2017 datasets are shown in the bottom plot.}
\end{figure}

Finally, having gathered all the ingredients, the cross section is then calculated using Eq.~\ref{eq.xsec_ul.phi2pi}. The yields, efficiencies and cross sections for the datasets are summarized in Tab.~\ref{tab.xsec_ul.phi2pi.2.1} -~\ref{tab.xsec_ul.phi2pi.4.2}. The resultant cross sections versus $E_{\gamma}$ and -$t$ for the different datasets are shown in Fig.~\ref{fig.xsec_ul.phi2pi.10}. The total cross section for the different datasets are consistent within errors, except for the Fall 2018 data that is systematically higher then the other datasets in some bins. This effect is still under investigation. The total errors are a quadratic sum of the statistical and systematic uncertainties, the estimation of systematic errors will be discussed in Sec.~\ref{sec.syserr}.
~\par The cross section measurements for different datasets are used to produce an average total cross section for every $E_{\gamma}$ and -$t$. The method used to average the cross sections is a standard weighted least-squares procedure~\cite{Tanabashi18}. Since the datasets are independent, the cross section measurements are uncorrelated, and the weighted average and error are then calculated by

\begin{equation}
    \label{eq.xsec_ul.phi2pi.2}
    \begin{aligned}
        & \bar{\sigma} \pm \delta\bar{\sigma} = \frac{\sum_{i}w_{i}\sigma_{i}}{\sum_{i}w_{i}} \pm  \left(\sum_{i}w_{i}\right)^{-1/2}~, \\
        \mathrm{with}\\
        & w_{i} = 1/(\delta \sigma_{i})^2
    \end{aligned}
\end{equation}

\noindent here $\sigma_{i}$ and $\delta \sigma_{i}$ are the values and errors of the measured cross sections, with $i=1,2,3$ for the three different datasets, and the sum run over $N=3$ measurements. We then have two main cases depending on the $\chi^{2}/(N-1)$ ratio, with 

\begin{equation}
    \label{eq.xsec_ul.phi2pi.3}
    \chi^{2} = \sum w_{i}(\bar{\sigma}-\sigma_{i})^2
\end{equation}

If this ratio is smaller or equal to 1, then the final average cross section is as defined in Eq.~\ref{eq.xsec_ul.phi2pi.2}. But if the ratio is slightly larger then 1, then we increase our average errors $\delta\bar{\sigma}$ in Eq.~\ref{eq.xsec_ul.phi2pi.2}, by a scale factor $S$ defined as

\begin{equation}
    \label{eq.xsec_ul.phi2pi.4}
    S = [\chi^{2}/(N-1)]^{1/2}
\end{equation}

The idea here is that large value of the $\chi^{2}$ is likely due to underestimation of errors in at least one of the cross section measurements. Since the measurement with the underestimated error is not known, we assume they are all underestimated by the same factor $S$. Scaling up all the cross section errors by this factor, the ratio gets closer to unity, and consequently the average error $\delta\bar{\sigma}$ scales up by the same factor.

\begin{center}
\null
\vfill
\begin{figure}[H]
    \centering
    \begin{subfigure}[b]{0.5\textwidth}
        \includegraphics[width=\textwidth]{plots/cmgexsec.eps}
        \caption{}
        \label{fig.xsec_ul.phi2pi.10.a}
    \end{subfigure}\hfill
    \begin{subfigure}[b]{0.5\textwidth}
        \includegraphics[width=\textwidth]{plots/cmgtxsec.eps}
        \caption{}
        \label{fig.xsec_ul.phi2pi.10.b}
    \end{subfigure}
    \caption{\label{fig.xsec_ul.phi2pi.10}$\gamma p \rightarrow \phi \pi^{+} \pi^{-} p$ total cross section versus (a) $E_{\gamma}$ and (b) -$t$, for 2017 (blue), Spring 2018 (red) and Fall 2018 (magenta). The relative ratio of Spring 2018 (red) and Fall 2018 (magenta), w.r.t to 2017 are shown in the bottom plot.}
\end{figure}
\null
\vfill
\end{center}

\begin{center}
    \null
    \vfill
\begin{figure}[H]
    \centering
    \begin{subfigure}[b]{0.5\textwidth}
        \includegraphics[width=\textwidth]{plots/cgexsec_avg.eps}
        \caption{}
        \label{fig.xsec_ul.phi2pi.11.a}
    \end{subfigure}\hfill
    \begin{subfigure}[b]{0.5\textwidth}
        \includegraphics[width=\textwidth]{plots/cgtxsec_avg.eps}
        \caption{}
        \label{fig.xsec_ul.phi2pi.11.b}
    \end{subfigure}
    \caption{\label{fig.xsec_ul.phi2pi.11}$\gamma p \rightarrow \phi \pi^{+} \pi^{-} p$ average cross section for the 2017, Spring and Fall 2018 datasets, versus (a) $E_{\gamma}$ and (b) -$t$.}
\end{figure}
\null
\vfill
\end{center}

\newpage
\begin{center}
\null
\vfill
\begin{table}[H]
    \caption{$\phi \pi^{+}\pi^{-}$ yields in MC ($N_{MC}$) and data ($N_{Data}$), efficiencies ($\varepsilon$) and cross sections ($\sigma$) in $E_{\gamma}$ for 2017 dataset.}
    \label{tab.xsec_ul.phi2pi.2.1}
    \begin{tabular}{|c|c|c|c|c|c|}
    \hline
    $E_{\gamma}$ (GeV) & $N_{MC}$ & $N_{Data}$ & $\varepsilon$ ($\%$) & $\sigma$ (nb) \\ 
    \hline
    6.50 - 7.01 & 24298 $\pm$ 180 & 4687 $\pm$ 110 & 4.17 $\pm$ 0.03 & 65.47 $\pm$ 1.53 $\pm$ 1.76 \\ 
    7.01 - 7.52 & 31523 $\pm$ 219 & 5299 $\pm$ 119 & 4.60 $\pm$ 0.03 & 60.77 $\pm$ 1.37 $\pm$ 1.85 \\ 
    7.52 - 8.03 & 109098 $\pm$ 428 & 14577 $\pm$ 202 & 4.93 $\pm$ 0.02 & 56.99 $\pm$ 0.79 $\pm$ 1.38 \\ 
    8.03 - 8.54 & 223401 $\pm$ 571 & 23213 $\pm$ 251 & 5.31 $\pm$ 0.01 & 54.84 $\pm$ 0.59 $\pm$ 1.38 \\ 
    8.54 - 9.05 & 205465 $\pm$ 531 & 19140 $\pm$ 227 & 5.47 $\pm$ 0.01 & 52.70 $\pm$ 0.62 $\pm$ 1.30 \\ 
    9.05 - 9.56 & 77856 $\pm$ 348 & 7912 $\pm$ 147 & 5.54 $\pm$ 0.02 & 54.45 $\pm$ 1.01 $\pm$ 1.15 \\ 
    9.56 - 10.07 & 104113 $\pm$ 381 & 9337 $\pm$ 160 & 5.62 $\pm$ 0.02 & 48.81 $\pm$ 0.84 $\pm$ 1.36 \\ 
    10.07 - 10.58 & 100110 $\pm$ 375 & 7914 $\pm$ 149 & 5.72 $\pm$ 0.02 & 48.64 $\pm$ 0.91 $\pm$ 1.13 \\ 
    10.58 - 11.09 & 124527 $\pm$ 406 & 9236 $\pm$ 160 & 5.87 $\pm$ 0.02 & 45.55 $\pm$ 0.79 $\pm$ 1.64 \\ 
    11.09 - 11.60 & 56339 $\pm$ 267 & 2952 $\pm$ 91 & 5.96 $\pm$ 0.03 & 41.70 $\pm$ 1.28 $\pm$ 1.65 \\ 
   \hline
\end{tabular}
\end{table}
\end{center}

\begin{center}
\begin{table}[H]
    \caption{$\phi \pi^{+}\pi^{-}$ yields in MC ($N_{MC}$) and data ($N_{Data}$), efficiencies ($\varepsilon$) and cross sections ($\sigma$) in -$t$ for 2017 dataset.}
    \label{tab.xsec_ul.phi2pi.2.2}
    \begin{tabular}{|c|c|c|c|c|c|}
    \hline
    -t $(GeV/c)^{2}$ & $N_{MC}$ & $N_{Data}$ & $\varepsilon$ ($\%$) & $\sigma$ (nb) \\ 
    \hline
    0.00 - 0.40 & 201927 $\pm$ 555 & 18970 $\pm$ 258 & 4.06 $\pm$ 0.01 & 12.67 $\pm$ 0.17 $\pm$ 0.47 \\ 
    0.40 - 0.80 & 309953 $\pm$ 666 & 33745 $\pm$ 296 & 6.34 $\pm$ 0.01 & 14.42 $\pm$ 0.13 $\pm$ 0.36 \\ 
    0.80 - 1.20 & 213756 $\pm$ 548 & 24016 $\pm$ 236 & 6.33 $\pm$ 0.02 & 10.27 $\pm$ 0.10 $\pm$ 0.23 \\ 
    1.20 - 1.60 & 133609 $\pm$ 431 & 15220 $\pm$ 186 & 5.98 $\pm$ 0.02 & 6.90 $\pm$ 0.08 $\pm$ 0.14 \\ 
    1.60 - 2.00 & 80978 $\pm$ 335 & 8909 $\pm$ 145 & 5.58 $\pm$ 0.02 & 4.33 $\pm$ 0.07 $\pm$ 0.09 \\ 
    2.00 - 2.40 & 48638 $\pm$ 263 & 5479 $\pm$ 116 & 5.18 $\pm$ 0.03 & 2.86 $\pm$ 0.06 $\pm$ 0.09 \\ 
    2.40 - 2.80 & 28255 $\pm$ 198 & 3410 $\pm$ 94 & 4.70 $\pm$ 0.03 & 1.96 $\pm$ 0.05 $\pm$ 0.07 \\ 
    2.80 - 3.20 & 16620 $\pm$ 155 & 2252 $\pm$ 78 & 4.32 $\pm$ 0.04 & 1.41 $\pm$ 0.05 $\pm$ 0.05 \\ 
    3.20 - 3.60 & 9538 $\pm$ 115 & 1440 $\pm$ 65 & 3.92 $\pm$ 0.05 & 1.00 $\pm$ 0.04 $\pm$ 0.05 \\ 
    3.60 - 4.00 & 5578 $\pm$ 89 & 1041 $\pm$ 56 & 3.59 $\pm$ 0.06 & 0.78 $\pm$ 0.04 $\pm$ 0.04 \\    
   \hline
\end{tabular}
\end{table}
\null
\vfill
\end{center}
   
\begin{center}
\null
\vfill
\begin{table}[H]
    \caption{$\phi \pi^{+}\pi^{-}$ yields in MC ($N_{MC}$) and data ($N_{Data}$), efficiencies ($\varepsilon$) and cross sections ($\sigma$) in $E_{\gamma}$ for Spring 2018 dataset.}
    \label{tab.xsec_ul.phi2pi.3.1}
    \begin{tabular}{|c|c|c|c|c|c|}
    \hline
    $E_{\gamma}$ (GeV) & $N_{MC}$ & $N_{Data}$ & $\varepsilon$ ($\%$) & $\sigma$ (nb) \\ 
    \hline
    6.50 - 7.01 & 38158 $\pm$ 230 & 8149 $\pm$ 159 & 2.54 $\pm$ 0.02 & 64.18 $\pm$ 1.25 $\pm$ 2.56 \\ 
    7.01 - 7.52 & 49313 $\pm$ 280 & 9125 $\pm$ 171 & 2.79 $\pm$ 0.02 & 59.59 $\pm$ 1.12 $\pm$ 2.12 \\ 
    7.52 - 8.03 & 173727 $\pm$ 531 & 25904 $\pm$ 290 & 2.97 $\pm$ 0.01 & 59.34 $\pm$ 0.66 $\pm$ 2.07 \\ 
    8.03 - 8.54 & 370754 $\pm$ 726 & 38395 $\pm$ 349 & 3.26 $\pm$ 0.01 & 54.55 $\pm$ 0.50 $\pm$ 2.34 \\ 
    8.54 - 9.05 & 366942 $\pm$ 710 & 35863 $\pm$ 334 & 3.40 $\pm$ 0.01 & 53.26 $\pm$ 0.50 $\pm$ 2.39 \\ 
    9.05 - 9.56 & 121260 $\pm$ 438 & 13848 $\pm$ 215 & 3.46 $\pm$ 0.01 & 52.55 $\pm$ 0.82 $\pm$ 2.35 \\ 
    9.56 - 10.07 & 167596 $\pm$ 496 & 17415 $\pm$ 235 & 3.57 $\pm$ 0.01 & 49.31 $\pm$ 0.66 $\pm$ 2.35 \\ 
    10.07 - 10.58 & 162712 $\pm$ 489 & 14180 $\pm$ 217 & 3.67 $\pm$ 0.01 & 46.60 $\pm$ 0.71 $\pm$ 2.60 \\ 
    10.58 - 11.09 & 204522 $\pm$ 528 & 17442 $\pm$ 235 & 3.78 $\pm$ 0.01 & 44.85 $\pm$ 0.60 $\pm$ 2.77 \\ 
    11.09 - 11.60 & 89142 $\pm$ 341 & 5939 $\pm$ 136 & 3.89 $\pm$ 0.01 & 43.89 $\pm$ 1.00 $\pm$ 2.33 \\
   \hline
\end{tabular}
\end{table}
\end{center}

\begin{center}
\begin{table}[H]
    \caption{$\phi \pi^{+}\pi^{-}$ yields in MC ($N_{MC}$) and data ($N_{Data}$), efficiencies ($\varepsilon$) and cross sections ($\sigma$) in -$t$ for Spring 2018 dataset.}
    \label{tab.xsec_ul.phi2pi.3.2}
    \begin{tabular}{|c|c|c|c|c|c|}
    \hline
    -t $(GeV/c)^{2}$ & $N_{MC}$ & $N_{Data}$ & $\varepsilon$ ($\%$) & $\sigma$ (nb) \\ 
    \hline
    0.00 - 0.40 & 435261 $\pm$ 804 & 44809 $\pm$ 426 & 3.04 $\pm$ 0.01 & 13.88 $\pm$ 0.13 $\pm$ 0.66 \\ 
    0.40 - 0.80 & 540178 $\pm$ 885 & 60810 $\pm$ 419 & 3.98 $\pm$ 0.01 & 14.39 $\pm$ 0.10 $\pm$ 0.66 \\ 
    0.80 - 1.20 & 332059 $\pm$ 688 & 37409 $\pm$ 312 & 3.70 $\pm$ 0.01 & 9.53 $\pm$ 0.08 $\pm$ 0.40 \\ 
    1.20 - 1.60 & 190988 $\pm$ 519 & 22584 $\pm$ 240 & 3.36 $\pm$ 0.01 & 6.33 $\pm$ 0.07 $\pm$ 0.29 \\ 
    1.60 - 2.00 & 107997 $\pm$ 390 & 13394 $\pm$ 189 & 3.06 $\pm$ 0.01 & 4.13 $\pm$ 0.06 $\pm$ 0.20 \\ 
    2.00 - 2.40 & 60580 $\pm$ 292 & 8108 $\pm$ 150 & 2.79 $\pm$ 0.01 & 2.74 $\pm$ 0.05 $\pm$ 0.14 \\ 
    2.40 - 2.80 & 34486 $\pm$ 220 & 4767 $\pm$ 123 & 2.60 $\pm$ 0.02 & 1.73 $\pm$ 0.04 $\pm$ 0.09 \\ 
    2.80 - 3.20 & 19275 $\pm$ 168 & 3203 $\pm$ 104 & 2.38 $\pm$ 0.02 & 1.27 $\pm$ 0.04 $\pm$ 0.06 \\ 
    3.20 - 3.60 & 10510 $\pm$ 125 & 1934 $\pm$ 87 & 2.14 $\pm$ 0.03 & 0.85 $\pm$ 0.04 $\pm$ 0.04 \\ 
    3.60 - 4.00 & 5748 $\pm$ 91 & 1171 $\pm$ 75 & 1.94 $\pm$ 0.03 & 0.57 $\pm$ 0.04 $\pm$ 0.05 \\ 
   \hline
\end{tabular}
\end{table}
\null
\vfill
\end{center}
 
\begin{center}
\null
\vfill
\begin{table}[H]
    \centering
    % \small
    \caption{$\phi \pi^{+}\pi^{-}$ yields in MC ($N_{MC}$) and data ($N_{Data}$), efficiencies ($\varepsilon$) and cross sections ($\sigma$) in $E_{\gamma}$ for Fall 2018 dataset.}
    \label{tab.xsec_ul.phi2pi.4.1}
    \begin{tabular}{|c|c|c|c|c|}
    \hline
    $E_{\gamma}$ (GeV) & $N_{MC}$ & $N_{Data}$ & $\varepsilon$ ($\%$) & $\sigma$ (nb) \\ 
    \hline
    6.50 - 7.01 & 55414 $\pm$ 281 & 12668 $\pm$ 189 & 5.44 $\pm$ 0.03 & 63.89 $\pm$ 0.95 $\pm$ 1.27 \\ 
    7.01 - 7.52 & 68475 $\pm$ 334 & 13239 $\pm$ 195 & 5.83 $\pm$ 0.03 & 64.25 $\pm$ 0.95 $\pm$ 1.05 \\ 
    7.52 - 8.03 & 218913 $\pm$ 592 & 36310 $\pm$ 329 & 6.09 $\pm$ 0.02 & 63.18 $\pm$ 0.57 $\pm$ 1.06 \\ 
    8.03 - 8.54 & 434149 $\pm$ 808 & 57721 $\pm$ 411 & 6.43 $\pm$ 0.01 & 60.87 $\pm$ 0.43 $\pm$ 1.29 \\ 
    8.54 - 9.05 & 430666 $\pm$ 776 & 48422 $\pm$ 368 & 6.62 $\pm$ 0.01 & 60.32 $\pm$ 0.46 $\pm$ 1.24 \\ 
    9.05 - 9.56 & 152565 $\pm$ 489 & 19785 $\pm$ 239 & 6.71 $\pm$ 0.02 & 61.48 $\pm$ 0.74 $\pm$ 1.47 \\ 
    9.56 - 10.07 & 210479 $\pm$ 554 & 25280 $\pm$ 264 & 6.81 $\pm$ 0.02 & 58.54 $\pm$ 0.61 $\pm$ 1.63 \\ 
    10.07 - 10.58 & 193868 $\pm$ 533 & 20278 $\pm$ 240 & 6.93 $\pm$ 0.02 & 57.41 $\pm$ 0.68 $\pm$ 1.54 \\ 
    10.58 - 11.09 & 243906 $\pm$ 578 & 24941 $\pm$ 260 & 7.02 $\pm$ 0.02 & 54.69 $\pm$ 0.57 $\pm$ 1.65 \\ 
    11.09 - 11.60 & 117208 $\pm$ 391 & 9077 $\pm$ 156 & 7.13 $\pm$ 0.02 & 53.00 $\pm$ 0.91 $\pm$ 1.79 \\ 
   \hline
\end{tabular}
\end{table}
\end{center}

\begin{center}
\begin{table}[H]
    \centering
    \caption{$\phi \pi^{+}\pi^{-}$ yields in MC ($N_{MC}$) and data ($N_{Data}$), efficiencies ($\varepsilon$) and cross sections ($\sigma$) in -$t$ for Fall 2018 dataset.}
    \label{tab.xsec_ul.phi2pi.4.2}
    \begin{tabular}{|c|c|c|c|c|c|}
    \hline
    -t $(GeV/c)^{2}$ & $N_{MC}$ & $N_{Data}$ & $\varepsilon$ ($\%$) & $\sigma$ (nb) \\ 
    \hline
    0.00 - 0.40 & 428783 $\pm$ 808 & 52479 $\pm$ 434 & 5.24 $\pm$ 0.01 & 14.59 $\pm$ 0.12 $\pm$ 0.49 \\ 
    0.40 - 0.80 & 599831 $\pm$ 941 & 81009 $\pm$ 464 & 7.44 $\pm$ 0.01 & 15.86 $\pm$ 0.09 $\pm$ 0.32 \\ 
    0.80 - 1.20 & 416252 $\pm$ 776 & 55227 $\pm$ 364 & 7.46 $\pm$ 0.01 & 10.79 $\pm$ 0.07 $\pm$ 0.18 \\ 
    1.20 - 1.60 & 264308 $\pm$ 615 & 34776 $\pm$ 287 & 7.12 $\pm$ 0.02 & 7.12 $\pm$ 0.06 $\pm$ 0.14 \\ 
    1.60 - 2.00 & 164040 $\pm$ 483 & 21106 $\pm$ 228 & 6.76 $\pm$ 0.02 & 4.55 $\pm$ 0.05 $\pm$ 0.09 \\ 
    2.00 - 2.40 & 100037 $\pm$ 377 & 12537 $\pm$ 181 & 6.37 $\pm$ 0.02 & 2.87 $\pm$ 0.04 $\pm$ 0.06 \\ 
    2.40 - 2.80 & 60521 $\pm$ 293 & 7798 $\pm$ 148 & 5.98 $\pm$ 0.03 & 1.90 $\pm$ 0.04 $\pm$ 0.04 \\ 
    2.80 - 3.20 & 37073 $\pm$ 233 & 5177 $\pm$ 124 & 5.73 $\pm$ 0.04 & 1.32 $\pm$ 0.03 $\pm$ 0.04 \\ 
    3.20 - 3.60 & 22113 $\pm$ 181 & 3312 $\pm$ 105 & 5.34 $\pm$ 0.04 & 0.90 $\pm$ 0.03 $\pm$ 0.03 \\ 
    3.60 - 4.00 & 13377 $\pm$ 139 & 2366 $\pm$ 92 & 5.05 $\pm$ 0.05 & 0.68 $\pm$ 0.03 $\pm$ 0.03 \\  
   \hline
\end{tabular}
\end{table}
\null
\vfill
\end{center}

\newpage
\subsection{Upper Limit for \texorpdfstring{$\bm{\gamma p \rightarrow Y(2175) p \rightarrow \phi \pi^{+} \pi^{-} p}$}{}}
\label{sec.xsec_ul.yphi2pi}

In the following, we will study the resonant mode of the previous reaction, with the $Y(2175) \rightarrow \phi \pi^{+} \pi^{-}$ being produced as an intermediate resonance. First, we select $\phi \pi^+ \pi^-$ signal, by subtracting non-$\phi$(1020) background. To achieve this, we study the invariant mass correlation between $K^{+}K^{-}$ and $K^{+}K^{-} \pi^+ \pi^-$, seen in Fig.~\ref{fig.xsec_ul.yphi2pi.1}. Next to the clear horizontal band of the $\phi(1020)$, we see another diagonal band of correlated events. The latter is investigated by performing a one dimensional projection of 50 intervals of $K^{+}K^{-} \pi^+ \pi^-$ on the $K^{+}K^{-}$ invariant mass. The yields of $\phi \pi^+ \pi^-$ are then extracted after fitting the signal and background shapes as shown in Fig.~\ref{fig.xsec_ul.yphi2pi.2}. The resulting $K^{+}K^{-} \pi^+ \pi^-$ invariant mass after background subtraction (Fig.~\ref{fig.xsec_ul.yphi2pi.3}) shows no enhancement around 2175 GeV/c$^2$, leading to the non observation of $Y(2175)$ resonance in the $\phi \pi^+ \pi^-$ channel. 
~\par In the absence of a signal, limits can be set on the the $\gamma p \rightarrow Y(2175) p \rightarrow \phi \pi^{+} \pi^{-} p$ production cross section. We define an upper limit at 90$\%$ Confidence Level (CL) using the maximum likelihood method. For $n$ independent measurements of the cross section $\sigma_{i}$, following a probability density function $f(\sigma_{i};\sigma)$ with the cross section ($\sigma$) as a parameter, the likelihood function is obtained from the probability of the data under assumption of the parameters defined as

\begin{equation}
    \label{eq.xsec_ul.yphi2pi.1}
    \begin{aligned}
        \mathcal{L(\sigma)} = \prod_{n}^{i=1} f(\sigma_{i};\sigma)~.\\
    \end{aligned}
\end{equation}

\noindent The maximum likelihood estimator for $\sigma$ is defined as the values that give the maximum of $\mathcal{L(\sigma)}$. The upper limit ($U\kern-0.14em L_{90}$) is the cross section at 90$\%$ of the profile likelihood distribution, with

\begin{equation}
    \label{eq.xsec_ul.yphi2pi.2}
    \begin{aligned}
        \int_{0}^{UL_{90}} \mathcal{L}(\sigma)~d\sigma = 0.9 \\
    \end{aligned}
\end{equation}

\noindent In this case, the Bayesian approach is used, for which a prior knowledge on the signal cross section is expressed in the sense that a probability for a negative cross section is negligible for real physics processes.
~\par After fixing the $Y(2175)$ signal shape from MC simulation as shown in Fig.~\ref{fig.xsec_ul.yphi2pi.4}, we use the same fit parameters to fit the data, see Fig.~\ref{fig.xsec_ul.yphi2pi.5}. The negative yield extracted from these plots are due to the dips around the Y(2175) mass, which may be due to simple statistical fluctuations or even destructive interference between the resonances participating in this process. We perform multiple fits, varying the signal amplitude parameter around the nominal value by five times the statistical uncertainty on the yield, and extract the profile likelihoods for each variation, as displayed in Fig~\ref{fig.xsec_ul.yphi2pi.6}. In this case, the Likelihood profiles essentially follow Gaussian distributions, so that the mean can be taken as the nominal cross section that maximizes the likelihood, and the standard deviation as the error on the cross section measurement. To take into account the effect of the cross section systematic errors on the upper limit determination, discussed in Sec.~\ref{sec.syserr}, we convolute the obtained likelihood profile distribution with a gaussian function of the same mean and the standard deviation corresponding to the systematic uncertainty. The resulting convolution (Fig~\ref{fig.xsec_ul.yphi2pi.7}) is a gaussian with the nominal cross section given by the mean and the total error (quadratic sum of the statistical and systematic error) given by the standard deviation. The CL corresponds to 90th percentile of the convoluted distribution above zero. The constructed Bayesian Confidence interval should represent a $90\%$ probability to cover the true value of the cross section. This technique is applied for the different datasets, and the results are summarized in the Tab.~\ref{tab.xsec_ul.yphi2pi.1}.
~\par The total upper limit for all the datasets is obtained by a simultaneous fit (Fig~\ref{fig.xsec_ul.yphi2pi.8}), combining all the previous signal and background models from each dataset, with the cross section as a free parameter. After multiple variations of the cross section around its optimal value with 5 standard deviation of its statistical error. The profile likelihood is extracted, and the systematic errors (detailed in Tab.~\ref{tab.syserr.simpdf}) are incorporated using the method discussed above. The resulting total upper limit at $90\%$ CL is 0.60 nb (Fig~\ref{fig.xsec_ul.yphi2pi.9}).

\begin{figure}[H]
    \centering
    \begin{subfigure}[b]{0.49\textwidth}
        \includegraphics[width=\textwidth]{plots/c_phi_y_17.eps}
        \caption{}
        \label{fig.xsec_ul.yphi2pi.1.a}
    \end{subfigure}
    \begin{subfigure}[b]{0.49\textwidth}
        \includegraphics[width=\textwidth]{plots/c_phi_y_18.eps}
        \caption{}
        \label{fig.xsec_ul.yphi2pi.1.b}
    \end{subfigure}
    \begin{subfigure}[b]{0.49\textwidth}
        \includegraphics[width=\textwidth]{plots/c_phi_y_18l.eps}
        \caption{}
        \label{fig.xsec_ul.yphi2pi.1.c}
    \end{subfigure}
    \caption{$K^{+}K^{-}$ versus $K^{+}K^{-} \pi^+ \pi^-$ invariant mass for (a) 2017, (b) Spring 2018 and (c) Fall 2018 datasets.}
    \label{fig.xsec_ul.yphi2pi.1}
\end{figure}

\begin{figure}[H]
    \centering
    \includegraphics[width=0.6\textwidth]{plots/c1_phi_y_16.eps}
    \caption{\label{fig.xsec_ul.yphi2pi.2}Invariant mass of $K^{+}K^{-}$ of one projection of $K^{+}K^{-} \pi^+ \pi^-$ invariant mass. The total fit (red) is composed of signal shape (blue) described by Voigtian model and background (dashed) by polynomial of $4^{th}$ degree.}
\end{figure}

\begin{figure}[H]
    \centering
    \begin{subfigure}[b]{0.49\textwidth}
        \includegraphics[width=\textwidth]{plots/c_gphi_y_17.eps}
        \caption{}
        \label{fig.xsec_ul.yphi2pi.3.a}
    \end{subfigure}
    \begin{subfigure}[b]{0.49\textwidth}
        \includegraphics[width=\textwidth]{plots/c_gphi_y_18.eps}
        \caption{}
        \label{fig.xsec_ul.yphi2pi.3.b}
    \end{subfigure}
    \begin{subfigure}[b]{0.49\textwidth}
        \includegraphics[width=\textwidth]{plots/c_gphi_y_18l.eps}
        \caption{}
        \label{fig.xsec_ul.yphi2pi.3.c}
    \end{subfigure}
    \caption{The yields $\phi \pi^+ \pi^-$ versus $K^{+}K^{-} \pi^+ \pi^-$ invariant mass for (a) 2017, (b) Spring 2018 and (c) Fall 2018 datasets. No observation of the $Y(2175)$ in data.}
    \label{fig.xsec_ul.yphi2pi.3}
\end{figure}

\begin{figure}[H]
    \centering
    \includegraphics[width=0.6\textwidth]{plots/cmc_YMass_postcuts_fitted_17.eps}
    \caption{\label{fig.xsec_ul.yphi2pi.4}Invariant mass of $\phi \pi^+ \pi^-$ in MC simulation. The total fit (red) is composed of signal shape (blue) described by Voigtian and background (dashed) by polynomial of $4^{th}$ degree.}
\end{figure}

\begin{figure}[htbp]
    \centering
    \begin{subfigure}[b]{0.49\textwidth}
        \includegraphics[width=\textwidth]{plots/c_gphiy_17.eps}
        \caption{}
        \label{fig.xsec_ul.yphi2pi.5.a}
    \end{subfigure}
    \begin{subfigure}[b]{0.49\textwidth}
        \includegraphics[width=\textwidth]{plots/c_gphiy_18.eps}
        \caption{}
        \label{fig.xsec_ul.yphi2pi.5.b}
    \end{subfigure}
    \begin{subfigure}[b]{0.49\textwidth}
        \includegraphics[width=\textwidth]{plots/c_gphiy_18l.eps}
        \caption{}
        \label{fig.xsec_ul.yphi2pi.5.c}
    \end{subfigure}
    \caption{The yields of $\phi \pi^+ \pi^-$ versus $K^{+}K^{-} \pi^+ \pi^-$ invariant mass for (a) 2017, (b) Spring 2018 and (c) Fall 2018 datasets. The fit models and parameters are obtained from Fig.~\ref{fig.xsec_ul.yphi2pi.4}.}
    \label{fig.xsec_ul.yphi2pi.5}
\end{figure}

\begin{figure}[htbp]
    \centering
    \begin{subfigure}[b]{0.49\textwidth}
        \includegraphics[width=\textwidth]{plots/c17_profxsec.eps}
        \caption{}
        \label{fig.xsec_ul.yphi2pi.6.a}
    \end{subfigure}
    \begin{subfigure}[b]{0.49\textwidth}
        \includegraphics[width=\textwidth]{plots/c18_profxsec.eps}
        \caption{}
        \label{fig.xsec_ul.yphi2pi.6.b}
    \end{subfigure}
    \begin{subfigure}[b]{0.49\textwidth}
        \includegraphics[width=\textwidth]{plots/c18l_profxsec.eps}
        \caption{}
        \label{fig.xsec_ul.yphi2pi.6.c}
    \end{subfigure}
    \caption{Profile likelihood versus total cross section for (a) 2017, (b) Spring 2018 and (c) Fall 2018 datasets.}
    \label{fig.xsec_ul.yphi2pi.6}
\end{figure}

\begin{figure}[htbp]
    \centering
    \begin{subfigure}[b]{0.49\textwidth}
        \includegraphics[width=\textwidth]{plots/c17_twogaus_conv.eps}
        \caption{}
        \label{fig.xsec_ul.yphi2pi.7.a}
    \end{subfigure}
    \begin{subfigure}[b]{0.49\textwidth}
        \includegraphics[width=\textwidth]{plots/c18_twogaus_conv.eps}
        \caption{}
        \label{fig.xsec_ul.yphi2pi.7.b}
    \end{subfigure}
    \begin{subfigure}[b]{0.49\textwidth}
        \includegraphics[width=\textwidth]{plots/c18l_twogaus_conv.eps}
        \caption{}
        \label{fig.xsec_ul.yphi2pi.7.c}
    \end{subfigure}
    \caption{Convoluted profile likelihood and a gaussian with the nominal cross section as mean and total errors as standard deviation versus total cross section for (a) 2017, (b) Spring 2018 and (c) Fall 2018 datasets. The vertical blue lines are indicating the cross section upper limit at 90$\%$ CL.}
    \label{fig.xsec_ul.yphi2pi.7}
\end{figure}

\begin{figure}[htbp]
    \centering
    \begin{subfigure}[b]{0.49\textwidth}
        \includegraphics[width=\textwidth]{plots/csimpdf_17_yphi2pi.eps}
        \caption{}
        \label{fig.xsec_ul.yphi2pi.8.a}
    \end{subfigure}
    \begin{subfigure}[b]{0.49\textwidth}
        \includegraphics[width=\textwidth]{plots/csimpdf_18_yphi2pi.eps}
        \caption{}
        \label{fig.xsec_ul.yphi2pi.8.b}
    \end{subfigure}
    \begin{subfigure}[b]{0.49\textwidth}
        \includegraphics[width=\textwidth]{plots/csimpdf_18l_yphi2pi.eps}
        \caption{}
        \label{fig.xsec_ul.yphi2pi.8.c}
    \end{subfigure}
    \caption{\label{fig.xsec_ul.yphi2pi.8}The yields of $\phi \pi^+ \pi^-$ versus $K^{+}K^{-} \pi^+ \pi^-$ invariant mass for the different datasets. The simultaneous fit is composed of the models and parameters obtained from Fig.~\ref{fig.xsec_ul.yphi2pi.5}.}
\end{figure}

\begin{figure}[H]
    \centering
    \includegraphics[width=0.6\textwidth]{plots/c_twogaus_conv_simpdf_yphi2pi.eps}
    \caption{\label{fig.xsec_ul.yphi2pi.9}Convoluted profile likelihood and a gaussian with the nominal cross section as mean and total errors as standard deviation versus total cross section for the total datasets. The vertical blue lines are indicating the cross section upper limit at 90$\%$ CL.}
\end{figure}

\begin{table}[!htbp]
    \small
    \centering
    \caption{Total cross sections and upper limits for $\gamma p \rightarrow Y(2175) p \rightarrow \phi \pi^{+} \pi^{-} p$.}
    \label{tab.xsec_ul.yphi2pi.1}
    \begin{tabular}{|c|c|c|c|c|}
        \hline
        Dataset & $N_{measured}$ & $\varepsilon$ ($\%$) & \thead{$\sigma$ (nb) x\\BR[$Y(2175) \rightarrow \phi \pi^{+} \pi^{-}$]} & \thead{Upper Limit\\$90\%$ CL (nb)} \\
        \hline
        2017 & -753 $\pm$ 284 & 7.64 $\pm$ 0.01 & -0.27 $\pm$ 0.10 $\pm$ -0.09 & 0.12 \\
        Spring 2018 & -351 $\pm$ 402 & 4.82 $\pm$ 0.004 & -0.07 $\pm$ 0.08 $\pm$ -0.16 & 0.25 \\
        Fall 2018 & -738 $\pm$ 458 & 9.93 $\pm$ 0.01 & -0.11 $\pm$ 0.07 $\pm$ -0.22 & 0.31 \\
        \hline
    \end{tabular}
\end{table}

\newpage
\subsection{Cross Section for \texorpdfstring{$\bm{\gamma p \rightarrow \phi f_0 p}$}{}}
\label{sec.xsec_ul.phifo}

This section summarizes the study of the non-resonant, without the $Y(2175)$, production of the $\phi f_0 p$ final state. The $\phi(1020)\pi^{+}\pi^{-}$ signal yields are extracted by fitting the $K^{+}K^{-}$ invariant-mass projections in each 0.018 GeV/c$^{2}$ interval of $\pi^{+}\pi^{-}$ invariant mass. The invariant mass correlation between the $K^{+}K^{-}$ and the $\pi^{+}\pi^{-}$ pairs is shown in Fig.~\ref{fig.xsec_ul.phifo.1}. The $f_{0}(980)$ signal shape is well described by the Breit-Wigner model in MC simulation (Fig.~\ref{fig.xsec_ul.phifo.2}), and the signal yield obtained is used for the reconstruction efficiency calculation.
~\par An observation of the $f_{0}(980)$ resonance is clearly seen in the data (Fig.~\ref{fig.xsec_ul.phifo.3}), with the parameters consistent with the PDG data values for this meson. Furthermore, an enhancement near the $\rho(770)$ and $K_{s}^{0}$ mesons are seen near the nominal masses. The $K_{s}^{0}$ is produced in a displaced vertex, leading to yield reductions due to the primary vertex constraint in the kinematic fitting procedure. The cross section of the $\gamma p \rightarrow \phi f_0 p$ is calculated using Eq.~\ref{eq.xsec_ul.phi2pi}, and the results are summarized in Tab.~\ref{tab.xsec_ul.phifo}. The cross sections for all the datasets are consistent within errors.
~\par The average cross section for all the data is 10.80 $\pm$ 2.52 (nb), which is calculated using the method described in sec.~\ref{sec.xsec_ul.phi2pi}.

\begin{table*}[!b]
    \centering
    \caption{A summary of the total cross section and efficiency for $\gamma p \rightarrow \phi f_0 p$. The statistical and systematics errors are displayed for the cross section. The systematic uncertainties will be discussed in Sec.~\ref{sec.syserr}}
    \label{tab.xsec_ul.phifo}
    \begin{tabular}{|c|c|c|c|}
        \hline
        Dataset & $N_{measured}$ & $\varepsilon$ ($\%$) & $\sigma$ $\times$ BR[$f_0(980) \rightarrow \pi^{+} \pi^{-}$] (nb) \\
        \hline
        2017 & 1959 $\pm$ 1034 & 0.50 $\pm$ 0.01 & 10.36 $\pm$ 5.47 $\pm$ 4.26 \\
        Spring 2018 & 4460 $\pm$ 1009 & 0.31 $\pm$ 0.002 & 13.49 $\pm$ 3.05 $\pm$ 5.54 \\
        Fall 2018 & 5036 $\pm$ 1006 & 0.67 $\pm$ 0.003 & 10.28 $\pm$ 2.05 $\pm$ 2.18 \\
        \hline
    \end{tabular}
\end{table*}

\begin{figure}[H]
    \centering
    \begin{subfigure}[b]{0.49\textwidth}
        \includegraphics[width=\textwidth]{plots/c_phifo_17.eps}
        \caption{}
        \label{fig.xsec_ul.phifo.1.a}
    \end{subfigure}
    \begin{subfigure}[b]{0.49\textwidth}
        \includegraphics[width=\textwidth]{plots/c_phifo_18.eps}
        \caption{}
        \label{fig.xsec_ul.phifo.1.b}
    \end{subfigure}
    \begin{subfigure}[b]{0.49\textwidth}
        \includegraphics[width=\textwidth]{plots/c_phifo_18l.eps}
        \caption{}
        \label{fig.xsec_ul.phifo.1.c}
    \end{subfigure}
    \caption{$K^{+}K^{-}$ versus $\pi^+ \pi^-$ invariant mass for (a) 2017, (c) Spring 2018 and (d) Fall 2018 datasets. The horizontal band corresponds to the $\phi(1020)$ and the vertical two bands to the $\rho(770)$ and $K_{s}^{0}$.}
    \label{fig.xsec_ul.phifo.1}
\end{figure}

\begin{figure}[H]
    \centering
    \includegraphics[width=0.6\textwidth]{plots/cmc_foMass_postcuts_fitted_17.eps}
    \caption{\label{fig.xsec_ul.phifo.2}Invariant mass of $\pi^+ \pi^-$ in MC simulation. The total fit (red) is composed of signal shape (blue) described by a Breit-Wigner and a background (dashed) by polynomial of first degree.}
\end{figure}

\begin{center}
\null
\vfill
\begin{figure}[htbp]
    \centering
    \begin{subfigure}[b]{0.49\textwidth}
        \includegraphics[width=\textwidth]{plots/c_gphifo_17.eps}
        \caption{}
        \label{fig.xsec_ul.phifo.3.a}
    \end{subfigure}
    \begin{subfigure}[b]{0.49\textwidth}
        \includegraphics[width=\textwidth]{plots/c_gphifo_18.eps}
        \caption{}
        \label{fig.xsec_ul.phifo.3.b}
    \end{subfigure}
    \begin{subfigure}[b]{0.49\textwidth}
        \includegraphics[width=\textwidth]{plots/c_gphifo_18l.eps}
        \caption{}
        \label{fig.xsec_ul.phifo.3.c}
    \end{subfigure}
    \caption{The yields of $\phi \pi^+ \pi^-$ versus $\pi^+ \pi^-$ invariant mass for (a) 2017, (b) Spring 2018 and (c) Fall 2018 datasets. The total fit (blue) is composed of the signal (red) described by a Breit-Wigner and the background (dashed) described by a second order polynomial.}
    \label{fig.xsec_ul.phifo.3}
\end{figure}
\null
\vfill
\end{center}

\newpage
\subsection{Upper Limit for \texorpdfstring{$\bm{\gamma p \rightarrow Y(2175) p \rightarrow \phi f_0 p}$}{}}
\label{sec.xsec_ul.yphifo}

Following a similar procedure as in Sec.~\ref{sec.xsec_ul.yphi2pi} to reduce the non-$\phi$(1020) background, we obtain the $\phi f_{0}$ yields by fitting the $K^{+}K^{-}$ invariant-mass projections in each 0.02 GeV/c$^{2}$ interval of $K^{+}K^{-}\pi^{+}\pi^{-}$ invariant mass, while requiring the di-pion mass pair within 2.5 times the PDG average mass error of $f_0(980)$. The resulting $K^{+}K^{-}\pi^{+}\pi^{-}$ invariant-mass distribution for $\phi f_{0}$ candidate events is shown in Fig.~\ref{fig.xsec_ul.yphifo.1}. To estimate the $Y(2175)$ contribution within multiple fluctuations in the invariant mass distribution, the likelihood ratio test~\cite{Cow11} is used to determine the significance of the $Y(2175)$ signal. The binned maximum likelihood fits are used for this test.
~\par We define the null hypothesis $H_{0}$ as the condition, in which only the background is observed in the data, and the alternative hypotheses $H_{1}$, in which both signal and background are modeled in data. According to Wilks theorem~\cite{Wil01}, the significance ($Z$) adopted as the test statistics is asymptotically distributed according to the $\chi^{2}$ function, with degrees of freedom equal to the difference between the number of fit parameters. The goal of the profile likelihood ratio in this study is to quantify degree of compatibility (or not) of the data with the hypothesis of the $Y(2175)$ signal being present, which would lead to an observation ($Z \geq 5\sigma$), evidence ($3\sigma < Z < 5\sigma$) or none of both ($Z < 3\sigma$). In the case of one degree of freedom difference between the two hypothesis, the significance is defined as

\begin{equation}
    \label{eq.xsec_ul.yphifo}
    \begin{aligned}
        Z = \sqrt{-2~ln\left(\frac{\mathcal{L}(H_{0})}{\mathcal{L}(H_{1})}\right)}~,
    \end{aligned}
\end{equation}

After describing the $Y(2175)$ signal shape in MC simulation, the same parameters are fixed in the data and the signal amplitude is set as a free parameter, once with the signal and background fit to obtain the likelihood $\mathcal{L}(H_{1})$, and the second with only the background to obtain $\mathcal{L}(H_{0})$. The significance is then calculated using Eq.~\ref{eq.xsec_ul.yphifo}; it is displayed next to the fit parameterization in Fig.~\ref{fig.xsec_ul.yphifo.1}. To estimate the goodness of the fit model to the data, we use the pull distribution, which is defined as the difference between the data and the fit values divided by the data errors. The pull histogram is distributed as a standard Gaussian with a mean of zero and a unit width. If the mean is not centered at zero than there is a bias in the fit model (Fig.~\ref{fig.xsec_ul.yphifo.1}). An enhancement around 2.191 GeV/c$^2$ is observed, with mean and width consistent with the PDG data for the $Y(2175)$. This resonance is seen in both, the largest data samples of Spring and Fall 2018 datasets, with a significance above 3$\sigma$. Despite the enhancement around the mass of interest, we could not claim an observation of the $Y(2175)$ due to systematic and statistical limitations, and to the strong bias in the fits especially at the region of interest. For this reason, we set a $90\%$ CL upper limit on the $\gamma p \rightarrow Y(2175) p \rightarrow \phi f_0 p$ cross section, using the method in Sec.~\ref{sec.xsec_ul.yphi2pi}. The resulting profile likelihoods indicating the upper limits are shown in Fig.~\ref{fig.xsec_ul.yphifo.2}. The summary of the efficiency, cross section and the upper limit values are listed in Tab.\ref{tab.xsec_ul.yphifo}.
~\par Using the similar method in Sec.~\ref{sec.xsec_ul.yphi2pi}, the total upper limit for all the datasets is obtained by a simultaneous fit (Fig.~\ref{fig.xsec_ul.yphifo.3}). After extracting the profile likelihood and including the systematic errors (summarized in Tab.~\ref{tab.syserr.simpdf}), the resulting total upper limit at $90\%CL$ is 0.44 nb (Fig.~\ref{fig.xsec_ul.yphifo.4}).

\begin{center}
    \begin{table}[htbp]
        \centering
        \caption{Summary of efficiency, cross section, and upper limit for different datasets.}
        \label{tab.xsec_ul.yphifo}
        \begin{tabular}{|c|c|c|c|c|}
            \hline
            Data set & $N_{measured}$ & $\varepsilon$ [$\%$] & \thead{$\sigma \times$ \\ $BR_{f_{0}\rightarrow\pi^{+}\pi^{-}}$ $\times BR_{Y\rightarrow \phi f_0}$ [nb]} & \thead{Upper Limit\\$90\%$ CL (nb)}\\
            \hline
            2017 & 135 $\pm$ 107 & 8.40 $\pm$ 0.013 & 0.04 $\pm$ 0.03 $\pm$ 0.13 & 0.24 \\
            Spring 2018 & 729 $\pm$ 150 & 5.32 $\pm$ 0.004 & 0.13 $\pm$ 0.03 $\pm$ 0.22 & 0.46 \\
            Fall 2018 & 691 $\pm$ 178 & 12.17 $\pm$ 0.010 & 0.08 $\pm$ 0.02 $\pm$ 0.14 & 0.29 \\
            \hline
        \end{tabular}
    \end{table}
\end{center}

\begin{figure}[htbp]
    \centering
    \begin{subfigure}[b]{0.49\textwidth}
        \includegraphics[width=\textwidth]{plots/chgphiy_17.eps}
        \caption{}
        \label{fig.xsec_ul.yphifo.1.a}
    \end{subfigure}
    \begin{subfigure}[b]{0.49\textwidth}
        \includegraphics[width=\textwidth]{plots/chgphiy_18.eps}
        \caption{}
        \label{fig.xsec_ul.yphifo.1.b}
    \end{subfigure}
    \begin{subfigure}[b]{0.49\textwidth}
        \includegraphics[width=\textwidth]{plots/chgphiy_18l.eps}
        \caption{}
        \label{fig.xsec_ul.yphifo.1.c}
    \end{subfigure}
    \caption{The invariant mass distribution for $\phi f_0$ candidates for (a) 2017, (b) Spring 2018 and (c) Fall 2018 datasets. The total fit (blue) is composed of the signal (red) described by a Voigtian and the background (dashed) with a third degree Chebyshev polynomial. The total fit (magenta) is performed with only the background. A pull distribution is shown in the bottom of each plot.}
    \label{fig.xsec_ul.yphifo.1}
\end{figure}

\begin{figure}[htbp]
    \centering
    \begin{subfigure}[b]{0.49\textwidth}
        \includegraphics[width=\textwidth]{plots/c17_twogaus_conv_yphifo.eps}
        \caption{}
        \label{fig.xsec_ul.yphifo.2.a}
    \end{subfigure}
    \begin{subfigure}[b]{0.49\textwidth}
        \includegraphics[width=\textwidth]{plots/c18_twogaus_conv_yphifo.eps}
        \caption{}
        \label{fig.xsec_ul.yphifo.2.b}
    \end{subfigure}
    \begin{subfigure}[b]{0.49\textwidth}
        \includegraphics[width=\textwidth]{plots/c18l_twogaus_conv_yphifo.eps}
        \caption{}
        \label{fig.xsec_ul.yphifo.2.c}
    \end{subfigure}
    \caption{Convoluted profile likelihood and a gaussian with the nominal cross section as mean and total errors as standard deviation versus total cross section for (a) 2017, (b) Spring 2018 and (c) Fall 2018 datasets. The vertical blue lines are indicating the cross section upper limit at 90$\%$ CL.}
    \label{fig.xsec_ul.yphifo.2}
\end{figure}

\begin{figure}[htbp]
    \centering
    \begin{subfigure}[b]{0.49\textwidth}
        \includegraphics[width=\textwidth]{plots/csimpdf_17_yphifo.eps}
        \caption{}
        \label{fig.xsec_ul.yphifo.3.a}
    \end{subfigure}
    \begin{subfigure}[b]{0.49\textwidth}
        \includegraphics[width=\textwidth]{plots/csimpdf_18_yphifo.eps}
        \caption{}
        \label{fig.xsec_ul.yphifo.3.b}
    \end{subfigure}
    \begin{subfigure}[b]{0.49\textwidth}
        \includegraphics[width=\textwidth]{plots/csimpdf_18l_yphifo.eps}
        \caption{}
        \label{fig.xsec_ul.yphifo.3.c}
    \end{subfigure}
    \caption{The invariant mass distribution for $\phi f_0$ candidates for (a) 2017, (b) Spring 2018 and (c) Fall 2018 datasets. The simultaneous fit (blue) is composed of the signal (red) described by a Voigtian and the background (dashed) with a third degree Chebyshev polynomial.}
    \label{fig.xsec_ul.yphifo.3}
\end{figure}

\begin{figure}[htbp]
    \centering
    \includegraphics[width=0.6\textwidth]{plots/c_twogaus_conv_simpdf_yphifo.eps}
    \caption{\label{fig.xsec_ul.yphifo.4}Convoluted profile likelihood and a gaussian with the nominal cross section as mean and total errors as standard deviation versus total cross section for the total datasets. The vertical blue lines are indicating the cross section upper limit at 90$\%$ CL.}
\end{figure}