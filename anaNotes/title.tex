\begin{titlepage}
  
  \title{{\bf Search for the hybrid candidate Y(2175)}\\
  GlueX Analysis Note}
  \author{Abdennacer Hamdi}
    % \email{nacer@jlab.org}
    \date{\today}

  \maketitle
  
  \begin{tikzpicture}[remember picture,overlay]
    \node[anchor=north east,inner sep=10pt] at (current page.north east)
    % \node[anchor=north,inner sep=10pt] at (current page.north)
    {\includegraphics[scale=0.1]{plots/gluexlogo.pdf}};
  \end{tikzpicture}
  
  \begin{abstract}
    Understanding the hadron spectrum is one of the primary goals of non-perturbative QCD. Many predictions have experimentally been confirmed, others still remain under experimental investigation. Of particular interest is how gluonic excitations give rise to states with constituent glue. One class of such states are hybrid mesons that are predicted by theoretical models and Lattice QCD calculations. Searching for and understanding the nature of these states is a primary physics goal of the GlueX experiment at the CEBAF accelerator at Jefferson Lab. A search for a $J^{PC}$ = 1$^{--}$ hybrid meson candidate, the $Y(2175)$, in $\phi(1020)\pi^{+}\pi^{+}$ and $\phi(1020)f_{0}(980)$ channels in photoproduction on a proton target has been conducted. A first measurement of non-resonant $\phi(1020)\pi^{+}\pi^{+}$ and $\phi(1020)f_{0}(980)$ total cross sections in photoproduction has been performed. An upper limit on the resonance production cross section for the $Y(2175) \rightarrow \phi(1020)\pi^{+}\pi^{+}$ and $Y(2175) \rightarrow \phi(1020)f_0(980)$ channels are estimated.
  \end{abstract}

\end{titlepage}