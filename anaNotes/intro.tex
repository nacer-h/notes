\section{Introduction}
\label{sec.intro}

Searching for and understanding the nature of the exotic states has been and still is a central goal of hadron spectroscopy. In recent years, many new and unexpected resonance-like signals have been observed in the heavy-quark sector~\cite{Tanabashi18}. Many of these so-called $XYZ$ states are candidates for exotic configurations of mesons. Similar studies are also performed in the light quark sector. Due to the short lifetime of light mesons, the resonances are broad, leading to states overlapping with each other, which hence made their detection challenging. In order to settle the fundamental question of the existence of states beyond the quark model and thus a strong confirmation of QCD or whether they are not realized in nature as expected, large data sets with high statistical precision are needed. The unambiguous identification of exotic states requires experiments with complementary production mechanisms and the analysis of different final states~\cite{Szczepaniak01}.\\
The $Y(2175)$, also denoted as $\phi(2170)$ by the Particle Data Group (PDG)~\cite{Tanabashi18}, was first observed in 2006 by the BaBar collaboration~\cite{Aubert06} in the $e^{+}e^{-}\rightarrow \phi(1020)f_0(980)$ process. Later the analysis was updated~\cite{Aubert12} with twice the integrated luminosity. By fitting the observed cross section for both $e^{+}e^{-}\rightarrow \phi \pi^{+} \pi^{-}$ (Fig.~\ref{fig.1.4.1.a}) and $e^{+}e^{-}\rightarrow \phi f_0(980)$ (Fig.~\ref{fig.1.4.1.b}), they confirmed the presence of the $Y(2175)$ in the data, as well as the presence of the $\phi(1680)$ resonance.\\
It was subsequently confirmed by the Belle collaboration~\cite{Shen09} in both, the reactions $e^{+}e^{-}\rightarrow \phi(1020)\pi^{+}\pi^{-}$ (Fig.~\ref{fig.1.4.2.a}) and $e^{+}e^{-}\rightarrow \phi f_0(980)$ (Fig.~\ref{fig.1.4.2.b}). The analysis is based on a data sample of 673 fb$^{-1}$ collected on and below the $\Upsilon(4S)$ resonance. In order to obtain the parameters of $Y(2175)$ and $\phi(1680)$ resonances, a least squares fit is applied to the measured cross section distribution. An incoherent Breit-Wigner fit for the $Y(2175)$ and $\phi(1680)$ was performed, with an additional function centered near 2.4 GeV/c$^2$, where the statistical significance were 10$\sigma$ for the first two resonances, and only 1.5$\sigma$ for the structure around 2.4 GeV/c$^2$. The cross section were measured from threshold to $\sqrt{s}$ = 3 GeV using initial-state radiation.\\
The $Y(2175)$ was also confirmed by the BESII~\cite{Ablikim08} and BESIII~\cite{Ablikim15}, both in $J/\psi$ hadronic decays (Fig.~\ref{fig.1.4.3}), based on samples of 5.8 x 10$^7$ and 2.25 x 10$^8$ $J/\psi$ events, respectively. The fit yields 471$\pm$54 $Y(2175)$ events with a statistical significance of greater than 10$\sigma$. The fit results show that the significance of the structure around 2.35 GeV/c$^2$ is only 3.8$\sigma$. The resonance was recently also measured for the first time in $e^{+}e^{-}\rightarrow \eta Y(2175)$ process with BESIII~\cite{Ablikim19}. The mass and width of the Y(2175) resonance, in different experiments, are summarized in Tab.~\ref{tab.1.4}.

% \begin{center}
%     \null
%     \vfill
    \begin{figure}[H]
        \centering
        \begin{subfigure}[b]{0.45\textwidth}
            \includegraphics[width=\textwidth]{plots/babar_y2175_2.png}
            \caption{}
            \label{fig.1.4.1.a}
        \end{subfigure}\hfill
        \begin{subfigure}[b]{0.45\textwidth}
            \includegraphics[width=\textwidth]{plots/babar_y2175.png}
            \caption{}
            \label{fig.1.4.1.b}
         \end{subfigure}
         \caption{(a) The fit to the $e^{+}e^{-}\rightarrow \phi \pi^{+} \pi^{-}$ cross section using the model described in Ref.~\cite{Aubert12}, the entire contribution due to the $\phi(1680)$ is shown by the dashed curve. The dotted curve shows the contribution for only the $\phi f_0$ decay. (b) The $e^+e^- \rightarrow \phi f_0$(980) cross section measured in the $K^{+}K^{-}\pi^{+}\pi^{-}$ (solid dots) and $K^{+}K^{-}\pi^{0}\pi^{0}$ (open squares) final states. The solid and dashed curve represents the result of $Y(2175)$ and $\phi(1680)$ resonance fits, respectively. The hatched area and dotted curve show the $Y(2175)$ contribution for two solutions described in Ref.~\cite{Aubert12}. }
        \label{fig.1.4.1}
    \end{figure}
    % \null
    % \vfill
    % \end{center}
    
    \begin{figure}[H]
        \centering
        \begin{subfigure}[b]{0.47\textwidth}
            \includegraphics[width=\textwidth]{plots/belle_y2175.png}
            \caption{}
            \label{fig.1.4.2.a}
        \end{subfigure}\hfill
        \begin{subfigure}[b]{0.44\textwidth}
            \includegraphics[width=\textwidth]{plots/belle_y2175_2.png}
            \caption{}
            \label{fig.1.4.2.b}
         \end{subfigure}
         \caption{(a) The fit to $e^+e^- \rightarrow \phi \pi^+\pi^-$ cross section with two incoherent Breit-Wigner functions, one for the $\phi$(1680) (red dashed line) and the other for the $Y$(2175) (green dashed line). (b) $e^+e^- \rightarrow \phi f_0$(980) cross section with a single Breit-Wigner function that interferes with a nonresonant component. In (b), the dashed and dot-dashed curves are for the destructive and constructive interference solutions described in~\cite{Shen09}, respectively.}
         \label{fig.1.4.2}
    \end{figure}
    
    \begin{figure}[htbp]
        \centering
            \includegraphics[width=0.45\textwidth]{plots/bes3_y2175.png}
            \caption{$\phi f_0$(980) invariant mass spectrum, with an unbinned maximum likelihood fit. The circular and triangular dots show the distribution in the signal and background region, with the backgrounds estimated using sideband regions. The green dashed line represents the direct decay of $J/\psi \rightarrow \eta \phi f_0(980)$. Ref.~\cite{Ablikim15}.}
            \label{fig.1.4.3}
    \end{figure}

\noindent Since it is produced in $e^+e^-$ annihilation, the quantum numbers are $J^{PC} = 1^{--}$. The observation of the $Y(2175)$ stimulated many theoretical interpretations of its nature. There are very few known meson resonances with $I = 0$ near this mass, and therefore it is likely not a standard meson but rather an exotic state. Since the similarity in production mechanism and decay patterns to the $Y(4260)$ in the charm sector and the $\Upsilon$(10860) in the bottom sector, the $Y$(2175) is discussed to be a candidate of a strangeonium hybrid ($s\bar{s}g$)~\cite{Gui07}, a tetraquark ($s\bar{s}s\bar{s}$)~\cite{Chen08}, a $\Lambda \bar{\Lambda}$ bound state~\cite{Klempt07}, an excited $\phi$~\cite{Coito09}, or an ordinary $\phi f_0(980)$ resonance produced by interactions between the final state particles~\cite{Alvarez09}. The quark model predicts two conventional $s\bar{s}$ states near 2175 MeV/$c^2$, the ${3}^{3}\!S_{1}$ and ${2}^{3}\!D_{1}$~\cite{Godfrey85, Barnes97}, however, both of them are predicted to be significantly broader than the $Y(2175)$~\cite{Barnes03, Ding07}.\\
Hadron production induced by photons has been largely studied since it provides an excellent tool to probe the hadron spectrum~\cite{Ballam68, Meyer70, Wang14, Wang17}. The strong affinity of photons for $s\bar{s}$ allows to use photon beams to study the strangeonium-like states, like the observation of the $\phi(1020)$~\cite{Mibe05} and $\phi(1680)$~\cite{Aston81} in $\gamma p \rightarrow K^{+}K^{-}p$ reaction. Since the $Y(2175)$ was observed in the $\phi \pi^{+}\pi^{-}$ and $\phi f_{0}(980)$ channels, indicating a substantial $s\bar{s}$ component in the $Y(2175)$, it would be straightforward to search for the resonance $Y(2175)$ in the reaction of $\gamma p \rightarrow \phi f_{0}(980)p$ and $\gamma p \rightarrow \phi \pi^{+}\pi^{-}p$.\\
In this note, we will search for the $Y(2175)$ resonance in photoproduction, in both $\phi \pi^{+}\pi^{-}$ and $\phi f_{0}(980)$ decay modes, while studying the $\gamma p \rightarrow K^{+}K^{-} \pi^{+}\pi^{-} p$ final state. Thus, we measure cross sections for the $\gamma p \rightarrow Y(2175) p \rightarrow \phi \pi^+ \pi^{-} p$ and $\gamma p \rightarrow Y(2175) p \rightarrow \phi f_{0}(980) p$ resonant and non-resonant (without the $Y(2175)$ state) modes. To achieve that, we start by an event selection to reduce the background, that is followed by a description of the Monte Carlo samples and the real data used in the analysis. We finally report the cross section measurements for the different channels, and discuss the systematic uncertainties associated with theses measurements.

\begin{table}[H]
    \centering
    \small
    \setlength{\tabcolsep}{3pt}
    \caption{Mass and width of the $Y(2175)$ resonance in different experiments, reproduced from~\cite{Aubert06, Aubert12, Shen09, Ablikim08, Ablikim15, Ablikim19}}
    \label{tab.1.4}
    \begin{tabular}{|c|c|c|c|}
        \hline
        Experiments & Reactions & \thead{$Y(2175)$ mass\\(GeV/c$^2$)} & \thead{$Y(2175)$ width \\(GeV/c$^2$)} \\
        \hline
        \multirow{3}{*}{BaBar} 
        & $e^{+}e^{-}\rightarrow \phi(1020)f_0(980)$ & $2.175 \pm 0.010 \pm 0.015$ & $0.058 \pm 0.016 \pm 0.020$ \\ 
        & $e^{+}e^{-}\rightarrow \phi(1020)f_0(980)$ & $2.180 \pm 0.008 \pm 0.008$ & $0.077 \pm 0.015 \pm 0.010$ \\ 
        & $e^{+}e^{-}\rightarrow \phi(1020)\pi^{+}\pi^{-}$ & $2.176 \pm 0.014 \pm 0.004$ & $0.090 \pm 0.022 \pm 0.010$ \\ 
        \hline
        \multirow{2}{*}{Belle}
        & $e^{+}e^{-}\rightarrow \phi(1020)f_0(980)$ & $2.163 \pm 0.032$ & $0.125 \pm 0.040$ \\ 
        & $e^{+}e^{-}\rightarrow \phi(1020)\pi^{+}\pi^{-}$ & $2.079 \pm 0.013$ & $0.192 \pm 0.023$ \\
        \hline
        \multirow{2}{*}{BESII/BESIII}
        & $J/\psi \rightarrow \eta \phi f_0(980)$ & $2.186 \pm 0.010 \pm 0.006$ & $0.065 \pm 0.023 \pm 0.017$ \\ 
        & $J/\psi \rightarrow \eta \phi \pi^{+}\pi^{-}$ & $2.200 \pm 0.006 \pm 0.005$ & $0.104 \pm 0.015 \pm 0.015$ \\ 
        & $e^{+}e^{-}\rightarrow \eta Y(2175)$ & $2.135 \pm 0.008 \pm 0.009$ & $0.104 \pm 0.024 \pm 0.012$ \\ 
        \hline
    \end{tabular}
\end{table}