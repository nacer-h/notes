\section{Summary and Outlook}
\label{sec.summ}
% \addcontentsline{toc}{section}{Summary and Outlook}

The first phase of the GlueX experiment was completed successfully at the end of 2019, with more than 121 pb$^{-1}$ of data collected in the coherent photon beam region. Using the calibrated data sets, a search for the hybrid meson candidate, the $Y(2175)$, in both, the $\phi\pi^{+}\pi^{-}$ and the $\phi(1020) f_0(980)$ exclusive final states has been performed. A first measurement of the photoproduction cross section for both channels has been carried out, and an upper limit on the production cross section of the $Y(2175)$ has been determined for both, $\phi\pi^{+}\pi^{-}$ and $\phi(1020)f_0(980)$ final states.
~\par The observed strong dependence of the $\phi \pi^+\pi^-$ cross section on the momentum transfer could be explained by the presence of intermediate sub-resonances, like the observed $\rho(770)$, or target fragmentation sources in the reaction, like $\Delta^{++} \rightarrow \pi^+ p$. These can lead to a different final state phase-space detector occupation, and given an asymmetric detector acceptance, this could be translated to different efficiencies and thus also cross section measurements.
~\par In the absence of the $Y(2175)$ in the $\phi(1020) \pi^+\pi^-$ and $\phi(1020) f_0(980)$ channels, an upper limit on the measured cross section has been established. We obtain an upper limit at $90\%$ CL of 0.12 nb, 0.25 nb and 0.31 nb for $Y(2175)\rightarrow \phi(1020) \pi^+\pi^-$, and 0.24 nb, 0.46 nb, and 0.29 nb for $Y(2175)\rightarrow \phi(1020) f_0(980$, for the 2017, Spring and Fall 2018 datasets, respectively. The non-observation of the $Y(2175)$ may be an indication for the presence of other sources of background, such as $e.g.$ $\Delta^{++}$ resonance in the reaction. The performed analysis is worth to be revisited with the improved PID capabilities and the higher statistics in GlueX Phase-II.
~\par Detailed studies of the branching ratios of the $Y(2175)$ into different final states may then indicate the nature of this resonance. For instance, if the $\phi(1020) f_0(980)$ decay mode is the dominant one, then the tetraquark picture is favored, with the $Y(2175)$ as an $ss\overline{ss}$, $s\bar{s}s\bar{s}$ or $su\overline{su}$ depending on the structure considered for the $f_0(980)$. 
~\par The next phase of the GlueX program will start soon, with an additional detector system for Detection of Internally Reflected Cherenkov light (DIRC), currently being installed and commissioned. This upgrade will improve the particle identification system (Fig.~\ref{fig.summ}), in order to cleanly select meson and baryon decay channels that include kaons in the final state. Once this detector has been installed and commissioned, the plan is to collect a total of 200 days of physics analysis data at an average intensity of 5x10$^{7}/s$ tagged photons on target. This data sample will provide an order of magnitude statistical improvement over the initial GlueX data set. Together with the developed kaon identification system, the GlueX potential for contributing to the understanding of hybrid mesons, in particular on the nature of the $Y(2175)$, will significantly increase in the near future. It will be worth to repeat the analysis proposed, developed, and carried out in this analysis note.

\begin{figure}[htbp]
    \centering
    \includegraphics[width=0.8\textwidth]{plots/dirc_pvstheta.png}
    \caption{Kaon momentum versus the polar angle in MC, with kaon from the $\gamma p \rightarrow Y(2175) p \rightarrow \phi \mathrm{f}_0  \rightarrow  K^{+} K^{-} \pi^{+} \pi^{-} p$ reaction. The boxes show the TOF (red) and DIRC (purple) coverages.}
    \label{fig.summ}
\end{figure}
