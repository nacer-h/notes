\documentclass[a4paper]{article}
\usepackage[T1]{fontenc}
\usepackage[utf8]{inputenc}
\usepackage[margin=0.5in]{geometry}
\usepackage{amsmath,amssymb,amsfonts}
\usepackage{mathptmx}
\usepackage{graphicx,float}
\usepackage{caption}
\usepackage{subcaption}
\usepackage{xcolor} 
\usepackage{comment}
\usepackage{datetime}
\usepackage{enumitem} % add bullets
\usepackage[pdfusetitle]{hyperref}
\title{Research Notes}
\author{Abdennacer Hamdi}
\date{\today, \currenttime}
\usepackage{bookmark}
\hypersetup{colorlinks, citecolor=black, filecolor=black, linkcolor=blue, linktoc=all, urlcolor=black}
\usepackage{titlesec}
\titleformat{\section}[display]{\normalfont\bfseries}{}{0pt}{\Large}
\usepackage[backend=biber]{biblatex}
\usepackage{filecontents}
\begin{filecontents}{\jobname.bib}
@misc{ref:1, author = "K.A. Olive et al. (Particle Data Group)", year = "2014", title = "PDG"}

\end{filecontents}
\addbibresource{\jobname.bib}
\nocite{*}

% \usepackage{tikz}
% % set up externalization
% \usetikzlibrary{external}
% \tikzset{external/system call={latex \tikzexternalcheckshellescape -halt-on-error-interaction=batchmode -jobname ``\image'' ``\texsource''; dvips -o ``\image''.ps ``\image''.dvi; ps2eps ``\image.ps''}}
% \tikzexternalize


\begin{document}

\maketitle
\tableofcontents
\listoffigures
%%
\clearpage

%%%%%%%%%%%%%%%%%%%%%%%%%%%% Write here %%%%%%%%%%%%%%%%%%%%%%%%%%%%%%%%

\section{Y(2175) simulation studies, attempts to get rid of combinatorics}

\subsection{08 September 2016}

\begin{itemize}
\item Generation and reconstruction of the signal alone:
  $${\gamma}p{\rightarrow}Y(2175)p{\rightarrow}{\phi}(1020)f_0(980)p{\rightarrow}K^+K^-\pi^+\pi^-p$$
  Properties:
  \begin{itemize}[label={--}]
  \item 10K generated signal events,
  \item $P_{beam}$ = 9 GeV/c, t-channel slope = 5.0, the recoiled proton is reconstructed,
  \item $\Gamma_{Y(2175)}$ = 0.061 GeV, $\Gamma_{\phi(1020)}$ = 0.004 GeV and $\Gamma_{f_o(980)}$ = 0.048 GeV.~\cite{ref:1}
  \end{itemize}
\item In the figure~\ref{fig:1.1} representing the number of events as a function of the invariant masses of (a) $\phi(1020)$ ($M_{K^+K^-}$), (b) $f_0(980)$ ($M_{\pi^+\pi^-}$) and (c) Y(2175) ($M_{K^+K^-\pi^+\pi^-}$), we see a lot of combinatoric backgroud, due to miss identification of charged tracks, the extra bumps appearing in the figure (a) is due to miss reconstruction of high momenta Pions as Kaons, and the other appearing in the figure (b), low momenta Kaons miss reconstructed as pions. This effect appears also in figure~\ref{fig:1.2}, representing $M_{K^+K^-}$ as a function of $M_{\pi^+\pi^-}$, where we see a second band formed far from the masses of the two resonances $\phi(1020)$ and $f_0(980)$.

  \begin{figure}[htbp]
    \centering
    \begin{subfigure}[b]{0.3\textwidth}
      \includegraphics[width=\textwidth]{figures20160908/PhiMass_Measured.eps}
      \caption{}
    \end{subfigure}
    \begin{subfigure}[b]{0.3\textwidth}
      \includegraphics[width=\textwidth]{figures20160908/foMass_Measured.eps}
      \caption{}
    \end{subfigure}
    \begin{subfigure}[b]{0.3\textwidth}
      \includegraphics[width=\textwidth]{figures20160908/YMass_Measured.eps}
      \caption{}
    \end{subfigure}
    \caption{Number of events as a function of the measured (reconstructed) invariant masses of:(a)- $\phi(1020)$, (b)- $f_0(980)$ and (c)- Y(2175)}
    \label{fig:1.1}
  \end{figure}

  \begin{figure}[h]
    \centering
    \includegraphics[width=0.4\textwidth]{figures20160908/foPhiMass.eps}
    \caption{ $M_{K^+K^-}$ as a function of $M_{\pi^+\pi^-}$}
    \label{fig:1.2}
  \end{figure}

  \item After applying a Kinematic Fit on the 4-momentum of the final state paticles and on tracks coming from same vertex, we see a an improuvement of the invariant mass resolution of the resonances (red line)~\ref{fig:1.2}.
  \item By applying a selection on the $M_{K^+K^-}<1.05$GeV/$c^2$, the combinatoric backgroud was reduced considerably (doted line)~\ref{fig:1.3}.

  \begin{figure}[htbp]
    \centering
    \begin{subfigure}[b]{0.3\textwidth}
      \includegraphics[width=\textwidth]{figures20160908/PhiMass.eps}
      \caption{}
    \end{subfigure}
    \begin{subfigure}[b]{0.3\textwidth}
      \includegraphics[width=\textwidth]{figures20160908/foMass.eps}
      \caption{}
    \end{subfigure}
    \begin{subfigure}[b]{0.3\textwidth}
      \includegraphics[width=\textwidth]{figures20160908/YMass.eps}
      \caption{}
    \end{subfigure}
    \caption{Number of events as a function of the invariant masses of:(a)- $\phi(1020)$, (b)- $f_0(980)$ and (c)- Y(2175) (Teal), after 4-P and vertex Kinematic fit (red line), and after a $M_{K^+K^-}<1.05$GeV/$c^2$ selection (doted line)}
    \label{fig:1.3}
  \end{figure}

\end{itemize}


\clearpage
  
\subsection{28 September 2016}

 After a 4-momentum and same vertex kinematic fit on the Y(2175) invariant mass (figure~\ref{fig:1.2}, red line), we observe a huge background at high momenta (second peak). \medskip 
 To investigate the source of this background, we studied the energy loss $dE/dx$ in the Central Drift Chamber detector (CDC) (figure~\ref{fig:1.4}(a)) and the velocity $\beta$ in the Time Of Flight detector (TOF)(figure~\ref{fig:1.4}(b)) for different final particle states. the dots represent the measured quantities and the lines are the expected ones. \medskip 
The events at low momenta belong mainly to protons, by a PID selection of these events, the backgroud was decreased effeciently (figure~\ref{fig:1.4}(c)).

\begin{figure}[htbp]
  \centering
  \begin{subfigure}[b]{0.3\textwidth}
    \includegraphics[width=\textwidth]{figures20160928/dEdxvsP.eps}
    \caption{}
  \end{subfigure}
  \begin{subfigure}[b]{0.3\textwidth}
    \includegraphics[width=\textwidth]{figures20160928/betaVsP.eps}
    \caption{}
  \end{subfigure}
  \begin{subfigure}[b]{0.3\textwidth}
    \includegraphics[width=\textwidth]{figures20160928/YMass_cut.eps}
    \caption{}
  \end{subfigure}
  \caption{(a): $dE/dx$ measured (dots) and expexted for $proton$ (black line), $e^{+}$ (green line), $K^{+}$ (red line), $\pi^{+}$ (blue line) as function of momentum. (b): $\beta$ measured (dots) and expexted for $proton$ (black line), $e^{+}$ (green line), $K^{+}$ (red line), $\pi^{+}$ (blue line) as function of momentum. (c): Number of events as a function of the invariant masses of Y(2175) after 4-P and vertex Kinematic fit (Teal), after a $dE/dx$ selection (red), after $\beta$ selection (blue), and after both last selections (black)}
  \label{fig:1.4}
\end{figure}

\printbibliography
\end{document}