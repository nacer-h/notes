\section*{Summary and Outlook}
\addcontentsline{toc}{section}{Summary and Outlook}

~\par L'\'etude du couplage de boson de $Higgs$ au quark $top$ est un outil tr\`es important pour la validation du Mod\`ele Standard, qui pr\'edit ce couplage, mais pas encore observé. Celui-ci sera observé au LHC pendant le run 2 selon les prédictions du MS. Les collisions $e^+e^-$ seront utilisé pour avoir une meilleure pr\'ecision sur la mesurede ce couplage. Nous avons testé une signature avec deux leptons de m\^eme charge dans l'\'etat final, afin de réduire le plus de bruit de fond possible, qui est d\^u essentielement au $t\bar{t}$ provenant de la mauvaise reconstruction de la charge et contamination en leptons secondaires. Ce choit de signature, permet d'obtenir une précision statistique de l'ordre de $20\%$ pouvent être améliorer grâce à la reconstruction et aux mesure sur les jets.\\
La s\'eparation des particules dans le jet est cruciale. Nous avons \'etudiée les performances du SDHCAL lors de passage d'un faisceaux de $\pi^{\pm}$ de diff\'erentes \'energies. La linéarité ainsi que la résolution en énergie on été testées. \\
Un effet de saturation propre aux conditions du faisceau test et qui conduit \`a une d\'egradation de la r\'esolution en \'energie du d\'etecteur a été corrigé. \\
Comme le SDHCAL peut avoir une r\'esolution en temps de l'ordre de 100 ps dans sa version multi-gaps, on a construit une simulation pour d\'emontrer la possibilité de s\'eparation des particules dans le jet par leur temps d'arrivée, et cela serait possible pour les impulsions des constituants des jets typiquement attendues dans les collisions $e^+e^-$ (<p> $<l0$ GeV).
~\par Pour des \'etude post\'erieur, on doit compl\'eter la relation de lin\'earit\'e de l'\'energie reconstruite en fonction du nombre de $hits$ par un quatri\`eme terme li\'e \`a la topologie de la gerbe hadronique.\\
Des simulations complètes doivent êtres utiliser afin de rafiner le potentiel des mesures en temps dans le SDHCAL. De même, de telles simulations permetteraient de rafiner la précision attendue sur la signature $t\bar{t}H$ en $l^{\pm}l^{\pm}$.