\documentclass[a4paper,roman]{article}
\usepackage{graphicx,subfigure,float}
\title{Statement of Research Interests}
\author{Abdennacer HAMDI}
\date{\today}
\pagenumbering{gobble}

%\setlength{\topmargin}{-10mm}
%\setlength{\textwidth}{7in}
%\setlength{\oddsidemargin}{-8mm}
%\setlength{\textheight}{9in}
%\setlength{\footskip}{1in}
\usepackage[scale=0.88]{geometry}

\begin{document}
\fontsize{11}{12}
\selectfont
\maketitle

{\bfseries Strangeonium features and sensitivities at PANDA}

~\par It is with great pleasure that I am writing to you to express my interest in applying for a scholarship as a PhD student to work together with the hadron physics group at GSI to work on physics data analysis and DIRC detector studies for GlueX in view of the upcoming PANDA experiment. This program appeals to me because of my strong interest in particle physics in addition to using the knowledge I have gained through my science classes in the university.

~\par At some point during my early school years, I decided to specialize in nuclear physics and it turned out to be a successful decision. I graduated in 2010, and got a job at the Algiers Nuclear Research Center (Algeria). But it was my pure and deep interest to understand the fundamental building blocks of matter and their interactions that led me towards particle physics. I followed a master degree program at the University of Blaise Pascal (France), and successfully graduated in June 2015.\\
During my master degree, I did two research trainings. The first one was related to the search for a dark matter candidate in events with monotop signature in ATLAS data, being predicted by BSM theories. The second one was with the $e^+e^-$ team on the study of Yukawa coupling of the Higgs boson to the top quark in $e^+e^-$ experiments, and a performance study of a hadronic calorimeter proposed by the CALICE collaboration.\\
All these challenges have strengthened my skills in laboratory work and I have gained experience working with different research teams.

~\par After receiving some courses on hadron physics, I was very interested by the confinement of quarks and gluons inside QCD bound states, and about the predictions of QCD models (Flux Tube Model, Bag Model,...) about the possibility of existence of new exotic matter, due to the gluonic degrees of freedom, like glueballs and hybrids. While looking for a PhD project in that respect, I identified the PANDA experiment planned at FAIR to be a good opportunity to do these kind of investigations. I have been seeking just such an opportunity as this, it is clear to me that the The Helmholtz Graduate School for Hadron and Ion Research and Frankfurt University are genuinely devoted to the advancement of knowledge and education of students, I will have the opportunity to learn from the best and the privilege to work in an international environment. I believe I can contribute to that mission through my passion for physics. My education and background would be a good match for your requirements.

~\par Modern high energy physics experiments are a complicated synthesis of the theory behind an experiment, design and development of the detector to conduct the experiment, monitoring every detail of the extremely complicated detector, and analyzing the obtained data.\\
My research interests revolve around, exploring the experimental nature of elementary particles, especially testing the predictions of the Standard Model, such as investigating both the dynamics governing the interaction of fundamental particles and the existence of new forms of matter, by spectrotroscopy experiments.\\
In that respect, the analysis of data taken by the GlueX experiment, which just started, on the one hand is a great opportunity to detect and analyse these kind of states in the light quark region, like e.g. the exotic strangeonium candidate Y(2175). On the other hand, it offers the possibility to tune trigger algorithms needed for PANDA at FAIR for open and hidden strangeness ($K^{\pm}, K_S, K_L, \phi$) in order to be able to select these signatures with a high efficiency as well as test and further develop the partial wave analysis software necessery to determine their properties from both GlueX and PANDA data.\\
Due to my experience regarding hardware, I see also a large potential to contribute to the DIRC detectors being developed for both experiments, which are key devices for $kaon/pion$ separation and thus are naturally related to the physics aimed for above.

\end{document}