\documentclass[a4paper]{jpconf}
\usepackage{graphicx}
\usepackage{amsmath}
\begin{document}
\title{Search for exotic states in photoproduction at GlueX}

\author{Abdennacer Hamdi (for the GlueX Collaboration)}

\address{GSI Helmholtzzentrum f\"ur Schwerionenforschung GmbH, Planckstr. 1, 64291 Darmstadt, Germany}

\ead{a.hamdi@gsi.de}

\begin{abstract}
    Quantum Chromodynamics (QCD) is the theory that describes how hadrons are built from quarks and gluons via the strong interaction. Many predictions have been experimentally confirmed, but many others remain under experimental investigation.
    Of particular interest is how gluonic excitations give rise to exotic states. One class of such states are hybrid mesons that are predicted by theoretical models and Lattice QCD calculations.
    Search for and understanding the nature of these states is one of the primary physics goal of the GlueX experiment at the CEBAF accelerator at Jefferson Lab in the US.
    We will give an overview on the experiment, and present the status of the search for a hybrid meson candidate, namely the $Y(2175)$.
    This work is supported by HGS-HIRe.
\end{abstract}

\section{Introduction}
Mesons in the constituent quark model are color-singlet bound states of a quark $q$ and antiquark $\overline{q'}$, with allowed quantum numbers $J^{PC} = 0^{-+}, 0^{++}, 1^{--}, 1^{+-}, 1^{++}, 2^{--}, 2^{-+}, 2^{++}, etc$, where $J$, $P$ and $C$ are angular momentum, parity and charge conjugation, respectively. This simple picture has been successful to describe many observed states in the meson spectrum. However a richer spectrum is predicted by Quantum Chromodynamics (QCD), and this by including the gluonic degrees of freedom to the quark and anti-quark system, called hybrid mesons. Since the gluonic field can carry different quantum numbers this introduces many new states to the spectrum, also carrying quantum numbers: $J^{PC} = 0^{--}, 0^{+-}, 1^{-+}, 2^{+-}, etc$, that are not allowed for conventional $q\overline{q'}$ mesons. These latter are the (spin-)exotic hybrid mesons, and their experimental observation will be a smoking gun for the existence of states beyond the constituent quark model. The hybrid mesons are predicted by many phenomenological models~\cite{ref.1} and Lattice QCD is making predictions for their properties like mass and width~\cite{ref.2}, which can be tested experimentally.

\section{Experimental Setup}
The GlueX experiment is dedicated to map the spectrum of hybrid mesons, using a high-energy linearly polarized photon beam, produced by converting 12 GeV electrons by bremsstrahlung radiation on a thin radiator, delivered by the Continuous Photon Beam Facility (CEBAF) at Jefferson Lab. By choosing the crystal axis orientation in the Diamond we produced two data set, parallel (PARA) and perpendicular (PERP), with the electric field vector parallel or perpendicular to the floor of the experimental hall. The energy and intensity of the photon beam are monitored by a pair spectrometer system (dipole magnet and scintillator arrays), and to measure the polarization, a triplet polarimeter ($e^{+}e^{-}$ pair conversion on a thin Be foil) is used. These photons are incident on 30 cm long liquid hydrogen target situated in the center of the detector (see Figure~\ref{fig.1}). Charged particles from the primary interaction first pass through the start counter (scintillator detector), which provides some particle identification through energy loss measurements. Directly surrounding the start counter is the Central Drift Chamber (CDC) (straw tube detector), provides charged tracks position measurements and $dE/dx$ information. Downstream of the CDC are the four packages of the Forward Drift Chamber system (FDC) (planar drift chambers), provides charged track position measurements as well as the $dE/dx$ information. Surrounding the tracking devices is the Barrel Calorimeter (BCAL) (lead scintillator fiber) sensitive to photons between polar angles of $12^{\circ}$ and $160^{\circ}$. Downstream of the BCAL is the Forward Calorimeter (FCAL) (lead-glass crystals), covers polar angles from $1^{\circ}$ to $12^{\circ}$. Upstream of the FCAL is the Time Of Flight wall (TOF) (scintillator bars) providing both timing and energy loss information. For more details about the GlueX detector components and performance see~\cite{ref.3}.

\begin{figure}[h]
    \centering
    \includegraphics[width=16pc]{/Users/nacer/notes/fairness/proceeding/plots/gluex_detector.png}
    \caption{\label{fig.1}The GlueX experimental setup.}
\end{figure}

\section{Beam Asymmetry Measurements}
Understanding the photoproduction mechanism is critical to disentangling $J^{PC}$ quantum numbers of the observed states in the exotic hybrid mesons search. Theoretical models have predicted that the linearly polarized beam asymmetry ($\Sigma$) is sensitive to the contribution from vector $1^{-}$ ($\rho^{0}/\omega$) and axial-vector $1^{+}$ ($b_{1}^{0}/h_{1}$) exchanges for the the $\pi^{0}$ and $\eta$ photoproductions~\cite{ref.4}.\\
The exclusive reaction $\gamma p \rightarrow p \pi^{0}$ and $\gamma p \rightarrow p \eta$ where $\pi^{0}/\eta\rightarrow \gamma\gamma$ final states, are selected. The yields for the PERP and PARA orientations are given by

\begin{align}
Y_{\parallel/\perp} \propto N_{\parallel/\perp}(1 \pm P \Sigma \cos 2\phi_{p})
\label{eq.1}
\end{align}

where $\Sigma$ is the linearly polarized beam asymmetry and $\phi_{p}$ is the azimuthal angle of the production plane defined by the final-state proton.\\
The orthogonality of the PARA and PERP polarization configurations cancels out the $\phi$-dependent instrumental acceptance by

\begin{align}
\frac{Y_{\perp}-F_{R}Y_{\parallel}}{Y_{\perp}+F_{R}Y_{\parallel}} = \frac{(P_{\perp}+P_{\parallel})\Sigma \cos 2\phi_{p}}{2+(P_{\perp}-P_{\parallel})\Sigma \cos 2\phi_{p}}
\label{eq.2}
\end{align}

where $F_{R} = N_{\perp}/N_{\parallel}$ is the ratio of the integrated photon flux between PERP($N_{\perp}$) and PARA ($N_{\parallel}$). Figure~\ref{fig.3} shows the yield asymmetry as a function of $\phi_{p}$, which is fit using the functional form in Eq.(~\ref{eq.2}), where beam asymmetry $\Sigma$ is the only free parameter.

\begin{figure}[h]
    \centering
    \includegraphics[width=16pc]{plots/yield_asymmetry.png}
    \caption{\label{fig.3}The yield asymmetry, fit with Eq.(\ref{eq.2}) to extract $\Sigma$.}
\end{figure}

The beam asymmetry is determined in bins of proton transfer momentum ($-t$) for the $\pi^{0}$ and $\eta$ photoproductions, and the results are shown in Fig.~\ref{fig.4}.

\begin{figure}[h]
    \centering
    \begin{minipage}{16pc}
        \includegraphics[width=16pc]{plots/pi0_sigma.png}
    \end{minipage}\hspace{2pc}%
    \begin{minipage}{16pc}
        \includegraphics[width=16pc]{plots/eta_sigma.png}
    \end{minipage}
    \caption{\label{fig.4} Beam asymmetry $\Sigma$ for (a) $\gamma p \rightarrow p \pi^{0}$ and (b) $\gamma p \rightarrow p \eta$ (black filled circles). Uncorrelated systematic errors are indicated by gray bars and combined statistical and systematic uncertainties are given by the black error bars. The previous SLAC results at $E_{\gamma}$ = 10 $GeV$ (blue open circles) are also shown along with various Regge theory calculations (4, 5, 6, 7, 8 and 9 references found at~\cite{ref.4}).}
\end{figure}

The theoretical models predict a dip near $-t$ = 5 $(GeV/c)^{2}$, due to the contribution from the axial-vector meson exchange that is consistent with previous $\pi^{0}$ measurements at $\bar{E}_{\gamma}$ = 10 $GeV$ from the Stanford Linear Accelerator Center (SLAC). This effect is not observed in the GlueX data, which strongly suggests a vector meson dominance exchange at this energy. The $\eta$ beam asymmetry measurements are the first above 3 $GeV$, this measurements are expected to contribute to the understanding of the photoproduction mechanism.

\section{Charmonium Photoproduction Near Threshold}
Photoproduction of $J/\psi$ near threshold provides a clean signal to study charmonium production due to the lack of contributions from the decays of charmonium states.
The study of $J/\psi$ photoproduction can provide information on the gluonic structure of the nucleus, through measuring the contribution of the leading order (two gluon exchange) or higher twist (three gluon exchange) in the production mechanism. This process can also be used to search for the pentaquark candidates reported by Lhb in the $J/\psi p$ channel of the $\Lambda^{0}_{b} \rightarrow J/\psi p K^{-}$ decay~\cite{ref.5}.\\
The exclusive reaction $\gamma p \rightarrow p e^{+}e^{-}$ was selected, which includes the narrow $\phi$ and $J/\psi$ peaks, and the continuum dominated by the Bethe-Heitler (BH) process. Figure~\ref{fig.5} shows the invariant mass spectrum of $e^{+}e^{-}$ data after the event selection. We normalize the $e^{+}e^{-}$ total cross section to that of BH in the invariant mass range $1.20$ - $2.50$ GeV, thus canceling uncertainties from factors like luminosity and common detector efficiencies.

\begin{figure}[h]
    \centering
    \includegraphics[width=16pc]{plots/jpsi_photoproduction.png}
    \caption{\label{fig.5}Electron-positron invariant mass spectrum from the data. The insert shows the $J/\psi$ region fitted with a linear polynomial plus a Gaussian (fit parameters shown).}
\end{figure}

The measured total cross section in bins of beam energy is shown in Figure~\ref{fig.6}, compared to other measurements from photoproduction experiments, and to the theoretical models, we find that our data do not favor either pure two- or three-hard-gluon exchange separately, and a combination of the two processes is required to fit the data adequately.
The narrow LHCb states, $P^{+}_{c}(4312)$, $P^{+}_{c}(4440)$, and $P^{+}_{c}(4457)$, produced in the s-channel would appear as structures at $E_{\gamma}$ = $9.44$, $10.04$ and $10.12$ GeV in the cross-section, but the results seen in Figure~\ref{fig.6} show no evidence for such structures. Then a model-dependent upper limits on the branching fraction of the LHCb $P^{+}_{c}$ states was estimated.

\begin{figure}[h]
    \centering
    \includegraphics[width=16pc]{plots/jpsi_cross-section.png}
    \caption{\label{fig.6}GlueX results for the $J/\psi$ total cross section vs beam energy, compared to the Cornell and SLAC data, the theoretical predictions, and the JPAC model. All curves are fitted/scaled to the GlueX data only. For our data the quadratic sums of statistical and systematic errors are shown; the overall normalization uncertainty is 27\%.}
\end{figure}

\section{Hybrid Meson Photoproduction Search}
One of the potential candidates for the hybrid mesons are the $Y(2175)$, a possible strangeonium counterpart of the $Y(4260)$ in the charmonium sector, already observed in positron-electron experiments~\cite{ref.6}~\cite{ref.7}. The GlueX experiment offers a new opportunity to search for this state for the first time in photoproduction mechanism.\\
Since the $Y(2175)$ is seen in $\phi(1020)f_{0}(980)$ and $\phi(1020)\pi^{+}\pi^{-}$ invariant masses, we study the exclusive reaction $\gamma p \rightarrow p \pi^{+}\pi^{-}K^{+}K^{-}$, where the $K^{+}K^{-}$ system was constrained to the $\phi(1020)$ mass. In order to remove the background underneath the $\phi(1020)$ in the $K^{+}K^{-}$ invariant mass, we fit a $\phi(1020)$ signal plus background for fine granular $\pi^{+}\pi^{-}$ mass slices individually and this way extract the $\pi^{+}\pi^{-}$ invariant masses dependent $\phi(1020)$ yields in different beam energy (Figure~\ref{fig.7}) and proton momentum transfer bins (Figure~\ref{fig.8}). An important contribution from $\rho(770)$ and a small enhancement around 0.980 $Gev/c^{2}$ are seen. the observation of $f_{0}(980)$ is not confirmed. A similar method will be applied to the $\phi(1020)\pi^{+}\pi^{-}$ for the non $\phi(1020)$ background subtraction, to look for the presence of the $Y(2175)$ sate.

\begin{figure}[h]
    \centering
    \includegraphics[width=30pc]{plots/fo_eg.png}
    \caption{\label{fig.7}$\pi^{+}\pi^{-}$ invariant mass against $\phi(1020)$ yield in different beam energy ($E_{\gamma}$) bins.}
\end{figure}

\begin{figure}[h]
    \centering
    \includegraphics[width=30pc]{plots/fo_t.png}
    \caption{\label{fig.8}$\pi^{+}\pi^{-}$ invariant mass against $\phi(1020)$ yield in different proton momentum transfer ($-t$) bins.}
\end{figure}

\section{Summary \& Outlook}
The first part of the Gluex program is to map the conventional meson spectrum, and this is achieved by beam asymmetry measurments to understand the production mechanism. An opportunistic study of charmonium production at threshold lead to upper limits on the branching fraction of the pentaquark. A DIRC (Detection of Internally Reflected Cherenkov light) detector is currenly installed, which will open a broader program to search for hybrid mesons in the future.

\section*{References}
\begin{thebibliography}{9}
\bibitem{ref.1} C. A. Meyer and E. S. Swanson 2015 {\it Prog. Part. Nucl. Phys.} {\bf 82} 21-58.
\bibitem{ref.2} Jozef J. Dudek 2013 {\it Phys.Rev.} D {\bf 88} 094505.
\bibitem{ref.3} H. Al Ghoul et al. (GlueX Collaboration) 2016 {\it AIP Conf.Proc} {\bf 1735} 020001.
\bibitem{ref.4} H. Al Ghoul et al. (GlueX Collaboration) 2017 {\it Phys.Rev.} C {\bf 95} 042201.
\bibitem{ref.5} A. Ali et al. (GlueX Collaboration) 2019 JLAB-PHY-19-2946 1905.10811.
\bibitem{ref.6} B. Aubert et al. (BABAR Collaboration) 2006 {\it Phys. Rev.} D {\bf 74} 091103(R).
\bibitem{ref.7} C. P. Shen et al. (Belle Collaboration) 2009 {\it Phys. Rev.} D {\bf 80} 031101(R).
\end{thebibliography}

\end{document}


