\documentclass{article}
\RequirePackage[a4paper]{geometry}
\geometry{top=25mm,bottom=25mm,left=25mm,right=25mm,nohead,nofoot,includeheadfoot}
\pagestyle{empty}
\usepackage[utf8]{inputenc} % input encoding can be adjusted as needed
\usepackage{mathptmx,graphicx}
\begin{document}
\begin{center}
{\Large\bfseries Search for exotic states in photo-production at GlueX\par}
\vspace{3ex}
% {\bfseries Abdennacer Hamdi\par}
% {\footnotesize\itshape Institut für Kernphysik, J. W. Goethe-Universität, Max-von-Laue-Str. 1, 60438 Frankfurt am Main, Germany\par}
% \vspace{3ex}
{\bfseries Abdennacer Hamdi$^{1,2}$, Klaus Götzen$^{1}$, Frank Nerling$^{1,2}$, Klaus Peters$^{1,2}$ \par}
{\footnotesize\itshape
1. Institut für Kernphysik, J. W. Goethe-Universität, Max-von-Laue-Str. 1, 60438 Frankfurt am Main, Germany \par
2. GSI Helmholtzzentrum für Schwerionenforschung GmbH, Planckstraße 1, 64291 Darmstadt, Germany \par}
\vspace{1ex}
{\bfseries on behalf of the GlueX Collaboration \par}

\vspace{3ex}
\end{center}
Quantum Chromodynamics (QCD) is the theory that describes how hadrons are built from quarks and gluons via the strong 
interaction. Many predictions have been observed, but many others remain under experimental 
investigation.
Of particular interest is how gluonic excitations give rise to exotic states. One class of such states are hybrid
mesons that are predicted by theoretical models and Lattice QCD calculations.
Search for and understanding the nature of these states is one of the primary physics goal of the GlueX experiment at the CEBAF accelerator at Jefferson Lab in the US.
We will give an overview on the experiment, and present the status of the search for a hybrid meson candidate, namely the $Y(2175)$.
This work is supported by HGS-HIRe.

\setlength\parindent{0pt}\vspace{2ex}
%\textbf{References}

\footnotesize
%\hangindent=1.5em
\phantom{[1] Bernhard Ketzer, Proceedings of the Sixth International Conference on Quarks and Nuclear Physics (QNP2012), \textbf{arXiv:1208.5125v1} (2012).}

%\hangindent=1.5em
%[2] Gui-Jun Ding and Mu-Lin Yan, Phy. Lett.\textbf{B650}, 390 (2007).

\hangindent=1.5em
%[1] The BABAR Collaboration, B. Aubert, et al, Phys.Rev.\textbf{D74}, 091103 (2006).  \par
%\hspace{3ex} BES Collaboration, M. Ablikim, et al, Phys.Rev.Lett.\textbf{100}, 102003 (2008). \par
%\hspace{3ex} The Belle Collaboration, C. P. Shen, et al, Phys.Rev.\textbf{D80}, 031101 (2009).

%\hangindent=1.5em
%[4] BES Collaboration, M. Ablikim, et al, Phys.Rev.Lett.\textbf{100}, 102003 (2008).

%\hangindent=1.5em
%[5] The Belle Collaboration, C. P. Shen, et al, Phys.Rev.\textbf{D80}, 031101 (2009).

\end{document}