\documentclass[a4paper,roman]{article}
\usepackage{anyfontsize}
\usepackage{graphicx,subfigure,float}
\title{Statement of Research Experience and Interests}
\author{Abdennacer Hamdi\\
Goethe University of Frankfurt\\
\texttt{nacer@jlab.org}}
\date{\today}
\pagenumbering{gobble}

%\setlength{\topmargin}{-10mm}
%\setlength{\textwidth}{7in}
%\setlength{\oddsidemargin}{-8mm}
%\setlength{\textheight}{9in}
%\setlength{\footskip}{1in}
\usepackage[scale=0.88]{geometry}

\begin{document}
\fontsize{12}{15}
\selectfont
\maketitle

\section*{Research and Teaching Experience}
~\par At some point during my early school years, I decided to specialize in nuclear physics and it turned out to be a successful decision. I graduated in 2010, and got a job at the Algiers Nuclear Research Center (Algeria), where i had next to my laboratory activities of evaluating and assessing environmental radioactivity, the opportunity to mentor master students in their manipulation of the Gamma spectroscopy detectors during different stages of calibration and manipulations, and offer them basic courses about Radiation protection.

~\par It was my pure and deep interest to understand the fundamental building blocks of matter and their interactions that led me towards particle physics. I followed a master degree programme at the University of Blaise Pascal in France, and successfully graduated in June 2015. During my master degree, i did two research trainings. The first one was related to the search for a dark matter candidate in events with a single top quark and a missing transverse energy in the final state, the Monotop process, with the ATLAS experiment, and i studied the possibilities of extracting this signature from many background sources. The second one was with the $e^+e^-$ team on studying the Yukawa coupling of the Higgs boson to the top quark in $e^+e^-$ experiments, and overcoming the challenges of a multi-jet final state while analyzing the $t\bar{t}H$ signature. I also studied the performance of a hadronic calorimeter proposed by the CALICE collaboration for the future leptonic colliders, where i had the opportunity to participate in taking data at CERN to study the calorimeter prototype. I studied its energy resolution using $\pi^{\pm}$ from test beam and build a Monte Carlo simulation to demonstrate the possibility of separating particles of the jet in the calorimeter by their time of flight in the detector.

~\par After receiving some courses on hadron physics, i become very interested in the confinement of quarks and gluons inside QCD bound states, and about the predictions from theoretical models and lattice QCD calculations about the possibility of the existence of new exotic matter, due to the gluonic degrees of freedom, like glueballs and hybrids. While looking for a PhD project in that respect, i identified the PANDA experiment planned at FAIR to be a good opportunity to do these kind of investigations. As we joined the GlueX collaboration, i had the opportunity to work mainly and represent my analysis team in GlueX. After adapting to the new software environment and analysis methods of the experiment, i studied the $\phi\pi^{+}\pi^{-}p$ channel, both in simulation and data, in the pursuit of a candidate for hybrid mesons, the $Y(2175)$. Due to lack of a good $\pi/K$ separation in the absence of the Cerenkov detector, i had to develop efficient analysis methods to overcome the background. As a good discriminator between protons and pions was the energy loss in the Central Drift Chamber detector (CDC). I delivered the optimal mean truncation, for a better $<dE/dx>$ estimation, and the results are implemented in the current GlueX reconstruction software.\\
All these challenges have strengthened my skills in scientific research and I have gained experience working with different research teams.

\section*{Research Interests}
~\par Modern high energy physics experiments are a complicated synthesis of the theory behind an experiment, design and development of the detector to conduct the experiment, monitoring every detail of the extremely complicated detector, and analyzing the obtained data.\\
My research interests revolve around, exploring the experimental nature of elementary particles, especially testing the predictions of the Standard Model, such as investigating both the dynamics governing the interaction of fundamental particles and the existence of new forms of matter, by spectroscopy experiments.\\
In that respect, the analysis of current and upcoming high luminosity GlueX data, with the Detection of Internally Reflected Cherenkov light detector (DIRC) installation, will open a broader programme to map the spectrum of hadrons. I am confident that i can contribute to the efforts of the team in the physics analysis, particularly studying the $\gamma p \rightarrow \eta' \pi^{0} p$ reaction and perform a moment analysis to compare to the recent theoretical models including resonance in S, P or D waves for the $\eta \pi^{0} p$ final state (Phys. Rev. D 100, 054017).

% In one hand, mapping the baryon spectrum, in particular the cascades will provide more insights into the doubly-strange hyperons. On the other hand it is a great opportunity to detect and further analyse the $\phi\pi^{+}\pi^{-}$ channel, with all its rich potential. Both the $Y(2175)$ and the $\Xi$ are strange rich final states, and having the DIRC installed will improve $K/\pi$ separation power up to $4 GeV/c$. 
Due to my experience regarding particle identification and hardware, I see also a potential to contribute to the efforts of the team addressing the needs of the experiment by developing the new software and hardware, in particular participating in the improvement of the neutron detection in the calorimetry system in the GlueX detector.\\
Being involved in many different experiments have laid a solid ground for me to further pursue an academic career. My ambition is to be a part of the processes that would lead us to a better understanding of matter in the universe.
\end{document}