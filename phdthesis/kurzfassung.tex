\section*{kurzfassung}
\addcontentsline{toc}{chapter}{Kurzfassung}

~\par Das Verst\"andnis des Hadronenspektrums ist eines der Hauptziele der nicht-perturbativen QCD. Viele Vorhersagen sind experimentell best\"atigt worden, andere werden noch experimentell untersucht. Von besonderem Interesse ist, wie gluonische Anregungen zu Zust\"anden mit konstitutivem Kleber f\"uhren.  Eine Klasse solcher Zust\"ande sind hybride Mesonen, die durch theoretische Modelle und Gitter-QCD-Berechnungen vorhergesagt werden. Die Suche nach und das Verst\"andnis der Natur dieser Zust\"ande ist ein prim\"ares physikalisches Ziel des GlueX-Experiments am CEBAF-Beschleuniger am Jefferson Lab. In $\phi(1020)\pi^{+}\pi^{+}$ und $\phi(1020)f_{0}(980)$ wurde eine Suche nach einem $J^{PC}$ = 1$^{--}$ Hybridmeson-Kandidaten, dem $Y(2175)$, in den Kan\"alen $\phi(1020)\pi^{+}\pi^{+}$ und $\phi(1020)f_{0}(980)$ in der Photoproduktion auf einem Protonentarget durchgef\"uhrt. Eine erste Messung der nichtresonanten Gesamtwirkungsquerschnitte $\phi(1020)\pi^{+}\pi^{+}$ und $\phi(1020)f_{0}(980)$ bei der Photoproduktion wurde durchgef\"uhrt.  Es wurde eine Obergrenze f\"ur den Wirkungsquerschnitt f\"ur die Resonanzproduktion f\"ur die Kan\"ale $Y(2175) \rightarrow \phi(1020)\pi^{+}\pi^{+}$ und $Y(2175) \rightarrow \phi(1020)f_0(980)$ gesch\"atzt. Da die Analyse im Wesentlichen von der Qualit\"at der Identifikation des geladenen Kaons abh\"angt, wurde auch eine Optimierung der Teilchenidentifikation durch eine Verbesserung der Energieverlustsch\"atzung in der CDC durch eine Methode des abgeschnittenen Mittelwertes untersucht.
