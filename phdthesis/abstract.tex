\section*{Abstract} % Pas de numérotation
\addcontentsline{toc}{section}{Abstract} % Ajout dans la table des matières

~\par Le Mod\`ele Standard (MS) de la physique des particules \`a \'et\'e couronn\'ee de succ\`es depuis plus de 40 ans, notament avec la d\'ecouverte du boson de $Higgs$ en 2012. Une des pr\'edictions de ce mod\`ele est le couplage yukawa du $Higgs$ aux fermions. Dans cette \'etude, on s'intéresse à la mesure du couplage yukawa \`a la particule la plus massif du MS, le quark $top$, dans les collision $e^+e^-$.\\
Pour cela on va mener une anlyse de la signature $t\bar{t}H$ avec un choit de deux leptons de m\^eme charge comme signature discriminante.\\
Cette mesure comme une grande partie de la physique étudié sur collisionneur $e^+e^-$, fait intervenir des jets. Un calorimètre hadronique est proposé par la colaboration CALICE. On étudie la r\'esolution en \'energie d'un prototype, le SDHCAL, grâce à des données collectées en faisceaux test au CERN.\\
 Enfin, on se basant sur un Toy Monte Carlo, on montre qu'il est possible d'utiliser le temps de vol des particules du jet dans le calorim\`etre affin de les mieux s\'epareés.

\subsection*{Abstract}

~\par The Standard Model (SM) of particle physics has been successful for more than 40 years, especially through the discovery in the summer 2012 of a the Higgs boson. One of the predictions of the SM is the Yukawa coupling of the $Higgs$ boson to fermions, in this study, we measure the Yukawa coupling to the heaviest particle of SM, the $top$ quark, in $e^+e^-$ collisions.\\
To do so, we will make an analysis of $t\bar{t}H$ signature, by chousing the same charge dilepton in the final state as a descriminator. This signature is composed mainly of jets at the final state, to do a good separation of  particles in the jet , a hadronic calorimeter was proposed by the CALICE collaboration, we will study its energy resolution in $\pi^{\pm}$ coming from test beam at CERN.\\
 To complete this study, we build a Toy Monte Carlo simulation to demonstrate the possibility of separating particles of the jet in the calorimeter by thier time of flight in the calorimeter.