\section*{Abstract} % Pas de numérotation
% \addcontentsline{toc}{section}{Abstract} % Ajout dans la table des matières

~\par Understanding the Hadron spectrum is one of the primary goals of non-perturb-ative QCD. Many predictions have been experimentally confirmed, but others remain under experimental investigation. Of particular interest is how gluonic excitations give rise to states with constituent glue. One class of such states are hybrid mesons that are predicted by theoretical models and Lattice QCD calculations. Searching for and understanding the nature of these states is a primary physics goal of the GlueX experiment at the CEBAF accelerator at Jefferson Lab. A search for a 1$^{--}$ hybrid meson candidate, the Y(2175), in $\phi(1020)\pi^{+}\pi^{+}$ and $\phi(1020)f_{0}(980)$ channels in photo-production was conducted. A first measurement of non-resonant $\phi(1020)\pi^{+}\pi^{+}$ and $\phi(1020)f_{0}(980)$ total cross sections in photoprodution were estimated. A significant observation of the $Y(2175)$ in $\phi(1020)f_{0}(980)$ channel, consistent with previous measurements, and an upper limit on the cross section for the $\phi(1020)\pi^{+}\pi^{+}$ channel was established. Since the analysis essentially depends on the quality of the charged kaon identification, an optimization of particle identification through an improvement of the energy loss estimation in the CDC by a truncated mean method was also studied.