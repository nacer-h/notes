\section{Theoretical Context}
\label{p1}

Le Mod\`ele Standard de la physique des particules \'el\'ementaires est la th\'eorie couramment accept\'ee pour d\'ecrire les constituants de la mati\`ere et leurs interactions \'electromagn\'etique, faible et forte.
~\par C'est une th\'eorie quantique et relativiste des champs d\'evelopp\'ee dans le cadre du formalisme lagrangien, regroupement la Th\'eorie Quantique \'electrofaible qui fournit une description unifi\'ee des interactions \'electromagn\'etique et faible, et de la Th\'eorie de la Chromodynamique Quantique $QCD$ (Quantum ChromoDynamics) qui fournit une description de l'interaction forte. 
~\par C'est \'egalement une th\'eorie de jauge qui s'appuie sur des principes d'invariance (sous des transformations de jauges locales et de Lorentz) du lagrangien construit \`a partir des constituants de la mati\`ere, les fermions (leptons et quarks) et bosons.

\subsection{The constituent Quark Model}

Les champs de mati\`ere sont des fermions de spin $\frac{\hbar}{2}$, classifi\'es selon 3 g\'en\'erations et selon des propri\'et\'es communes (leurs nombres quantiques) en doublets de chiralit\'e gauche $"L"$ et singulets de chiralit\'e droite $"R"$ du groupe de sym\'etrie de l'isospin faible $SU(2)_{L}$, et du groupe de l'hypercharge faible $U(1)_{Y}$
~\par $1^{ere}$ g\'en\'eration: $({\nu}_{e},e^{-})_{L}, e^{-}_{R}, (u,d)_{L}, u_{R}, d_{R}$
~\par $2^{eme}$ g\'en\'eration: $({\nu}_{\mu},\mu^{-})_{L}, \mu^{-}_{R}, (c,s)_{L}, c_{R}, s_{R}$
~\par $3^{eme}$ g\'en\'eration: $({\nu}_{\tau},\tau^{-})_{L}, \tau^{-}_{R}, (t,b)_{L}, t_{R}, b_{R}$
~\par Les neutrinos n'existent que dans l'\'etat de chiralit\'e gauche comme composante du doublet, leur contrepartie singulet droit n'existant pas, contrairement au secteur des quarks.
~\par Cette description des propri\'et\'es des constituants correspondant \`a l'interaction \'electrofaible est compl\'et\'ee dans le secteur des quarks par le nombre quantique de couleur et son groupe de sym\'etrie $SU(3)_{c}$ correspondant \`a l'interaction forte d\'ecrite par la Th\'eorie de la Chromodynamique Quantique.
~\par Exp\'erimentalement, 6 saveurs diff\'erentes de quark ont \'et\'e observ\'ees $up (u)$, $down (d)$, $strange (s)$, $charme (c)$, $botom (b)$, $top (t)$ de charges \'electriques $Q=\frac{2e}{3}, \frac{-e}{3}, \frac{-e}{3}, \frac{2e}{3}, \frac{-e}{3}$ et $\frac{2e}{3}$ respectivement, chacune d'elles apparaissant sous trois couleurs $rouge$, $bleu$ et $vert$. Les leptons n'en portent pas et de ce fait ne participent pas \`a l'interaction forte, ils sont six: l'\'electron ($e^-$), le $muon (\mu^-$), le $tau (\tau^-$) de charges \'electriques $Q=-e$ et trois neutrinos associ\'ees: $\nu_e$, $\nu_\mu$ et $\nu_\tau$ de charges \'electriques $Q=0$ (Tableau~\ref{table:1.1}.)
~\par Finalement, les interactions forte, faible et \'electromagn\'etique vont \^etre introduites par les transformations locales du groupe de sym\'etrie r\'eunion des groupes pour les diff\'erentes interactions trait\'ees et sous la condition d'invariance du lagrangien sous ces transformations: 
$${SU(3)_c}\otimes{SU(2)_L}\otimes{U(1)_Y}$$

\begin{table}[h!]
  \centering
  \begin{tabular}{|c|c|c|c|c|c|c|c|}
    \hline
    Fermions & \multicolumn{2}{|c|}{$1^{ere}$ Famille} & \multicolumn{2}{|c|}{$2^{eme}$ Famille} & \multicolumn{2}{|c|}{$3^{eme}$ Famille} & Charge\\
    \cline{2-7}  & Nom & Masse (GeV) & Nom & Masse (GeV) & Nom & Masse(GeV) &  \\
    \hline
    Quarks& \bf{u} & $(1.55-3.3)\times10^{-3}$ & \bf{c} & $1.27$ & \bf{t} & $173.2$ & $\frac{2}{3}$ \\
     & \bf{d} & $(3-6)\times10^{-3}$ & \bf{s} & $(70-130)\times10^{-3}$ & \bf{b} & $4.2-4.7$ & $\frac{-1}{3}$ \\
    \hline
    Leptons & \boldmath$e^-$ & $3.3\times10^{-3}$ & \boldmath$\mu^-$ & $105.7\times10^{-3}$ & \boldmath$\tau^-$ & $1.77$ & ${-1}$ \\
     & \boldmath$\nu_e$ & $<2\times10^{-9}$ & \boldmath$\nu_\mu$ & $<0.19\times10^{-6}$ & \boldmath$\nu_\tau$ & $<18.2\times10^{-6}$ & $0$ \\
    \hline
  \end{tabular}
  \caption{Organisation des fermions en trois familles avec leurs masses et charges respectives. L'anti-particule associ\'ee \`a chaque fermion poss\`ede une charge oppos\'ee.}
  \label{table:1.1}
\end{table}

\subsection{Beyond The Constituent Quark Model}

Le lagrangien de d\'epart des fermions libres \`a masses nulles ne contient que les termes cin\'etiques. Pour satisfaire son invariance sous les transformations locales de jauge des groupes de sym\'etrie, il est n\'ecessaire d'introduire des champs vectoriels et de remplacer les d\'eriv\'ees des termes cin\'etiques par des d\'eriv\'ees covariantes telles que le nouveau lagrangien soit invariant. L'invariance impose \'egalement les lois de transformation de ces champs suppl\'ementaires.
~\par L'apparition de ces nouveaux champs vectoriels entra\^ine l'adjonction d'autres termes cin\'etiques dans le lagrangien, analogues au terme du champ \'electromagn\'etique en $QED$, au changement pr\`es de la d\'eriv\'ee covariante dans la d\'efinition du tenseur des champs.
~\par \`A ce stade, les fermions et les bosons, particules \`a masses nulles, avec des interactions et des couplages impos\'es via les termes cin\'etiques des fermions du lagrangien, et il appara\^it \'egalement des interactions entre bosons vecteurs via leurs tenseurs de champs.
~\par Les groupes de sym\'etrie ${SU(2)_L}$ et ${U(1)_Y}$ contiennent respectivement 3 et 1 g\'en\'erateurs:
~\par - $W_{\mu}^{i}(i={1,2,3})$ avec une constante de couplage $g$ pour ${SU(2)_L}$
~\par - $B_{\mu}$ avec une constante de couplage $g^\prime$ pour ${U(1)_Y}$
~\par \`a chacun de ces g\'en\'erateurs est associ\'e un boson de jauge, champ vectoriel de spin $\hbar$, les bosons $W^+$ et $W^-$ obtenus par m\'elanges des bosons $W^1$ et $W^2$, le boson $Z$ et le photon $\gamma$ obtenus par m\'elange des bosons de jauge $W^3$ et $B$.
~\par Le groupe ${SU(3)_c}$ contient 8 g\'en\'erateurs, chacun d'eux associ\'e \`a un gluon $(g_\alpha, \alpha=1,...,8)$ et de constante de couplage $g_s$ (Tableau~\ref{table:1.2}.)
~\par L'observation du spectre physique des fermions et des bosons nous indique qu'ils sont massifs et que la sym\'etrie \'electrofaible est bris\'ee. Cette brisure spontan\'ee de la sym\'etrie \'electrofaible est appel\'ee m\'ecanisme de Higgs qui permet de g\'en\'erer la masse des bosons et des fermions \`a travers l'int\'eraction avec un champ scalaire fondamental, le champ de Higgs. Ce m\'ecanisme pr\'evoit l'\'existence d'une particule scalaire appel\'ee boson de Higgs d'une masse de $125.9 \pm 0.4$ GeV \ [1] et de spin $0$.

\makeatletter
\setlength{\@fptop}{5pt}
\begin{table}[h!]
  \centering
  \begin{tabular}{|c|c|c|c|}
    \hline   \multicolumn{2}{|c|}{Boson} & Interaction & Charge  \\
    \cline{1-2} Nom & Masse(GeV) &  & \\
    \hline 
    photon \boldmath$\gamma$ & 0 & Electromagn\'etique & 0 \\
    \boldmath$W^{\pm}$, \boldmath$Z^{o}$ & $80.403 , 91.188$ & Faible & $\pm1,0$ \\
    8 gluons \bf{g} & $0$ & Forte & $0$ \\
    Higgs \bf{H} & $125.7$ & M\'ecanisme de Higgs & $0$ \\
    \hline    
  \end{tabular}
  \caption{Les bosons associ\'es aux trois interactions ainsi que le boson de Higgs avec leurs masses et charges respectives.}
  \label{table:1.2}
\end{table}

% \subsection{Le mécanisme de Higgs}

L'interaction faible est transport\'ee par un boson vectoriel massif charg\'e. Une masse est en effet indispensable aux particules de jauge pour d\'ecrire les forces \`a courte port\'ee. Si l'on souhaite donner une masse aux particules de jauge, il est n\'ecessaire de passer par la construction d'une th\'eorie utilisant une brisure (spontan\'ee) de sym\'etrie. Lorsqu'une sym\'etrie continue est spontan\'ement bris\'ee, il appara\^it des particules de masse nulle appel\'ees bosons de Goldstone. Un champs de jauge de masse nulle acquiert une masse en se propageant dans le vide des champs scalaires, appel\'es champs de Higgs. Le ph\'enom\`ene de Higgs permet aux bosons de jauge d'acqu\'erir une masse, tout en pr\'eservant la renormalisabilit\'e: 
$$M_{W^{\pm}}=\frac{g\nu}{2}$$
$$M_{Z}=\frac{1}{2}\nu\sqrt{g^2+{g^\prime}^2}$$
$$\ M_{\gamma}=0$$
~\par o\`u:
\begin{itemize}
  \item[$\bullet$] $g$ et $g^\prime$ sont respectivement les constantes de couplage des interactions faible et \'electromagn\'etique
  \item[$\bullet$] $\nu$ est un param\`etre libre positif
\end{itemize}
~\par Il permet m\^eme d'attribuer une masse pour les autres particules du mod\`ele standard: 
$$M_{fermion}=\frac{1}{\sqrt{2}}g_{fermion}\nu$$ 
$$M_{H}=\sqrt{2{\lambda}}\nu$$
~\par o\`u:
\begin{itemize}
  \item[$\bullet$] $g_{fermion}$ est une constante de couplage de Yukawa
  \item[$\bullet$] $\lambda$ est un param\`etre libre positif
\end{itemize}

Comme la constante de couplage du boson de Higgs au fermion ($g_{fermion}$) est proportionnel \`a la masse de ce dernier, alors le couplage avec le fermion le plus massif du mod\`ele standard qui est le quark $top$ ($M_{t}=173.21 \ GeV \ [1]$) est le plus important.
~\par Une autre motivation \`a la mesure de pr\'ecision du couplage du boson de Higgs au quark top est l'information qu'elle fournit sur l'\'etat de stabilit\'e du vide. En effet, \`a partir du diagramme de phases du vide qui est pr\'esent\'e dans la figure~\ref{figure:1.1}. Les valeurs privil\'egi\'ees par les mesures exp\'erimentales se trouvent dans la r\'egion critique du diagramme de phase, entre la zone de stabilit\'e absolue et la zone d'instabilit\'e,  or nous vivons dans un univers dans lequel le vide est visiblement stable, bien que l'incertitude actuelle ne permette pas de l'affirmer de facon d\'efinitve. C'est pour cela qu'il faut am\'eliorer la pr\'ecision sur la mesure du couplage Yukawa au quark top.

\begin{figure}[h!]
\centering
\includegraphics[width=0.4\columnwidth,keepaspectratio=true]{Plots/metastability.png}
\caption{Diagramme de phases du vide \'electrofaible dans le plan ($m_{top}$-$m_{H}$), masse du quark top et du boson de Higgs. Le point indique la valeur centrale privil\'eegi\'ee par les observations exp\'erimentales.}
\label{figure:1.1}
\end{figure}

~\par Le boson de Higgs se d\'esint\`egre soit en une paire de fermion-antifermion soit dans une paire de bosons. sa largeur de d\'esint\'egration en une paire de fermion est proportionnelle au carr\'e de la masse du fermion final. Pour des masses du boson du higgs sup\'erieurs au seuil de production de paire de bosons de l'interaction faible, il va se d\'esint\'egre pr\'ef\'erentiellement en paire $b\bar{b}$, sinon dans le cas inverse, il va se d\'esint\'egrer en paire de bosons faible (voir la figure~\ref{figure:1.2}.)

\begin{figure}[h!]
\centering
\includegraphics[width=0.4\columnwidth,keepaspectratio=true]{Plots/H_decays.jpg}
\caption{Les diff\'erents rapport de branchement des modes de d\'esint\'egrations du boson de higgs du mod\`ele standard.}
\label{figure:1.2}
\end{figure}