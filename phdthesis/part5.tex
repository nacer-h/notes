\section{Summary and Outlook}
\label{p5}

Il est n\'ecessaire de déterminer les performance du SDHCAL. Pour cela nous avons recours aux faisceaux de particules mis \`a disposition par le CERN aux physiciens de la collaboration CALICE.
\par Ces faisceaux de particules sont contr\^ol\'es par une s\'erie d’aimants et de d\'etecteur pour permettre \`a l’utilisateur de d\'efinir : le profil d’impulsion des particules, l’intensit\'e du flux, le positionnement du point de focalisation par rapport \`a son exp\'erience ainsi que la s\'election des particules ($\pi^{\pm}, \mu^{\pm}, e^{\pm}$). Le faisceau est d\'elivré durant un interval de temps appel\'e $Spill$ de quelque secondes. Il y a environ 2 $spill$ \`a la minute. 
\par En s\'electionnant des pions charg\'es dont l’impulsion est connue, on \'etudie la r\'esolusion en \'energie du d\'etecteur. on regardant le nombre moyen de $hits$ enregistr\'e pour chaque $pion$ d'\'energie : 0, 10, 20, 25, 30, 40, 50, 60 et 70 GeV, la figure~\ref{fig:5.1} illustre le profile de ces $hits$ pour un $\pi^{\pm}$ de 10 GeV dans le SDHCAL.\\
\begin{figure}[H]
  \centering
  \includegraphics[width=0.5\columnwidth]{Plots/DrawShowers_10.png}
  \caption{le profile des $hits$ pour un $\pi^{\pm}$ de 10 GeV dans le SDHCAL}
  \label{fig:5.1}
\end{figure}

En faisant un ajustement Gaussian, on remarque un comportement lin\'eaire (figure~\ref{figure:5.2}.(a)). Pour quantifier la linéarité de la réponse, on ajuste le nombre de $hits$ enregistr\'e dans le SDHCAL en fonction de l'\'energie reconstruite des particules dans le faisceau (figure~\ref{figure:5.2}.(b)). Par une droite, On voit bien une lin\'earit\'e, avec une faible d\'eviation \`a partir d'une impulsion de 50 Gev, qui s'explique par une fuite \`a haute \'energie (Leakage).
~\par Au premier ordre on consid\`ere une d\'ependance lin\'eaire de l'\'energie reconstruite par rapport au nombre total de $hits$, donn\'ee par la relation :~\eqref{eq:5.1}
\begin{equation}
  \label{eq:5.1}
  E_{reconstruite}=\alpha.N_{hits}
\end{equation}
Avec $\alpha$ déterminer par l'ajustement linéaire.
\begin{figure}[H]
  \centering 
  \mbox{\subfigure[\label{}]{\includegraphics[width=0.5\columnwidth]{Plots/nHits.png}} \subfigure[\label{}]{\includegraphics[width=0.5\columnwidth]{Plots/linearity_nhitsvsE.pdf}}}
  \caption{(a) Distribution du nomre de hits produit par des faisceaux de pions de diff\'erentes \'energie et (b) nombre de hits en fonction des \'energies du pion} 
  \label{figure:5.2}
\end{figure}

La r\'esolution sur l'\'energie est \'estim\'ee par un ajustement gaussien (seule la composante gaussienne de la distribution est considérée. Celle ci est déterminée de façon itérative) et est illustr\'ee sur la figure~\ref{figure:5.3}.(a). Le tableau~\ref{tab:5.1} regroupe les résolutions obtenuent à différentes énergies.

\begin{figure}[H]
  \centering 
  \mbox{\subfigure[\label{}]{\includegraphics[width=0.5\columnwidth]{Plots/gaus_avant.pdf}} \subfigure[\label{}]{\includegraphics[width=0.5\columnwidth]{Plots/gaus_apres.pdf}}}
  \caption{Ajustement gaussien du nombre de hits produits avec des faisceaux de pions d'énergies 70 GeV :  (a) données brut, (b) données corrigées} 
  \label{figure:5.3}
\end{figure}
\vfill%
\begin{table}[H]
  \begin{center}
    \begin{tabular}{|c|c|c|c|c|} \hline
      Energy [GeV] & \# Events & Mean & Sigma ($\sigma$) & $\sigma$/mean [\%] \\ \hline
      5 GeV &1.207e+04 & $98.36\pm0.9266$&$20.56\pm1.322$ & $ \bf20.9\pm1.344$ \\ \hline
      10 GeV &3.323e+04 & $190.1\pm0.3735$&$36.29\pm0.455$ & $ \bf19.09\pm0.2394$ \\ \hline
      20 GeV &1.108e+05 & $353.8\pm0.2593$&$52.1\pm0.3386$ & $ \bf14.72\pm0.0957$ \\ \hline
      25 GeV &7.606e+04 & $424.9\pm0.3626$&$58.92\pm0.4575$ & $ \bf13.87\pm0.1077$ \\ \hline
      30 GeV &3.376e+04 & $494.2\pm0.6123$&$63.93\pm0.7856$ & $ \bf12.94\pm0.159$ \\ \hline
      40 GeV &6.22e+04 & $657.2\pm0.672$&$79.79\pm0.8087$ & $ \bf12.14\pm0.1231$ \\ \hline
      50 GeV &1.59e+04 & $795.8\pm1.214$&$91.33\pm1.516$ & $ \bf11.48\pm0.1905$ \\ \hline
      60 GeV &5.373e+04 & $974.1\pm0.895$&$111.5\pm1.106$ & $ \bf11.45\pm0.1135$ \\ \hline
      70 GeV &4.498e+04 & $1124\pm1.194$&$126.2\pm1.529$ & $ \bf11.23\pm0.1361$ \\ \hline
    \end{tabular}
    \caption{Résultats de l'ajustement gaussien de la distribution du nombre de hits}
    \label{tab:5.1}
  \end{center} 
\end{table}

\begin{table}[H]
  \begin{center}
    \begin{tabular}{|c|c|c|c|c|} \hline
      Energy [GeV] & \# Events & Mean & Sigma ($\sigma$) & $\sigma$/mean [\%] \\ \hline
      5 GeV &1.207e+04 & $98.9\pm0.8979$&$21.44\pm1.188$ & $ \bf21.68\pm1.201$ \\ \hline
      10 GeV &3.323e+04 & $189.9\pm0.4668$&$35.08\pm0.5779$ & $ \bf18.47\pm0.3043$ \\ \hline
      20 GeV &1.108e+05 & $361.3\pm0.2396$&$52.7\pm0.2775$ & $ \bf14.59\pm0.07682$ \\ \hline
      25 GeV &7.606e+04 & $439\pm0.33$&$58.34\pm0.4027$ & $ \bf13.29\pm0.09172$ \\ \hline
      30 GeV &3.376e+04 & $520.6\pm0.5927$&$63.23\pm0.7847$ & $ \bf12.15\pm0.1507$ \\ \hline
      40 GeV &6.22e+04 & $671.7\pm0.658$&$79.59\pm0.8167$ & $ \bf11.85\pm0.1216$ \\ \hline
      50 GeV &1.59e+04 & $820.6\pm1.202$&$89.15\pm1.539$ & $ \bf10.86\pm0.1876$ \\ \hline
      60 GeV &5.373e+04 & $1043\pm0.8149$&$105.4\pm0.9863$ & $ \bf10.11\pm0.09458$ \\ \hline
      70 GeV &4.498e+04 & $1173\pm1.155$&$124\pm1.439$ & $ \bf10.57\pm0.1226$ \\ \hline
    \end{tabular}
    \caption{Résultats de l'ajustement gaussien de la distribution du nombre de hits Après correction}
    \label{tab:5.2}
  \end{center} 
\end{table}

~\par L'intensit\'e du faisceau test (100 $\pi^{\pm}$ \`a la seconde sur $2{\times}2 cm^2$) sur le calorim\`etre. Elle représente une condition extrême comparé à ce qui est attendu dans une expérience $e^+e^-$. Cela conduit \`a des pertes en nombre de hits dans le temps ,du à un effet de saturation. Une correction sur le nombre de hits a été ajoutée suivant la relation~\eqref{equation:5.2}:

\begin{equation}
  N_{corrige}=N_{hits}-\alpha.T 
\label{equation:5.2}
\end{equation}

~\par o\`u $N_{corrige}$ et $N_{hits}$ sont le nombre de $hit$ apr\`es et avant correction respectivement. $\alpha$ est ajust\'ee, run par run~\footnote{un run est un ensemble de données collectées avec des conditions expérimentales identiques (énergie, type de particules, intensité).}, et d\'ecrit l'\'evolution du nombre total de $hits$ dans le temps relatif au d\'ebut de $spill$ ($T$) comme illustr\'e sur la figure~\ref{figure:5.4}.  

 \begin{figure}[H]
  \centering 
  \mbox{\subfigure[\label{}Avant correction]{\includegraphics[width=0.5\columnwidth]{Plots/NhitsvsT_avant.pdf}} \subfigure[\label{}Apr\`es correction]{\includegraphics[width=0.5\columnwidth]{Plots/NhitsvsT_apres.pdf}}}
  \caption{Nombre de hits en fonction du temps, (a) avant et (b) apr\`es correction de l'effet de saturation} 
  \label{figure:5.4}
\end{figure}

~\par Cette correction, appliqu\'ee $\pi^{\pm}$ par $\pi^{\pm}$, permet d'am\'eliorer la r\'esolution en \'energie, estimée par ajustement gaussien (voir~\ref{figure:5.3}.(b)) , comme illustr\'e par la figure~\ref{figure:5.5} et le tableau~\ref{tab:5.2}.

 \begin{figure}[H]
  \centering 
  \mbox{\subfigure[\label{}Avant correction]{\includegraphics[width=0.5\columnwidth]{Plots/Avant_ResolutionvsE.pdf}} \subfigure[\label{}Apr\`es correction]{\includegraphics[width=0.5\columnwidth]{Plots/Apres_ResolutionvsE.pdf}}}
  \caption{R\'esolution en \'energie du SDHCAL en fonction de l'\'energie du faisceau, (a) avant et (b) apr\`es correction de l'effet de saturation} 
  \label{figure:5.5}
\end{figure}

~\par Nous avons considéré le SDHCAL dans un mode binaire (nombre total de hits), or on peut exprimer l'\'energie reconstruite des $pions$ en fonction du nombre de $hits$ dans le SDHCAL, en prenant en compte les seuils, par la formule~\eqref{equation:5.3} : 

\begin{equation}
E_{reconstruite}=\alpha(N_{hits}).N_{1}+\gamma(N_{hits}).N_{2}+\beta(N_{hits}).N_{3}
\label{equation:5.3}
\end{equation}

~\par o\`u $N_{hits}$ est le nombre de hits dans la gerbe hadronique, $N_{1}$, $N_{2}$, $N_{3}$ respectivement le nombre de hits de seuil 1, 2 et 3 dans la gerbe et les poids ($\alpha$, $\beta$ et $\gamma$) sont en fonction du nombre total de $hits$.

\begin{equation}
\alpha=p_{1}+p_{2}{\times}N_{hits}+p_{3}{\times}N_{hits}^{2}
\label{equation:5.4}
\end{equation}

\begin{equation}
\beta=p_{4}+p_{5}{\times}N_{hits}+p_{6}{\times}N_{hits}^{2}
\label{equation:5.5}
\end{equation}

\begin{equation}
\gamma=p_{7}+p_{8}{\times}N_{hits}+p_{9}{\times}N_{hits}^{2}
\label{equation:5.6}
\end{equation}

avec $p_{1,2,3,4,5,6,7,8,9}$ des constantes, utilis\'ees pour obtenir une meilleur r\'esolution en \'energie. 

\par Le SDHCAL caract\'eris\'e par ces trois seuils, apporte une information sur la concentration en particules qui touchent une cellule et am\'eliore la r\'esolution en \'energie d'un ordre de $10\%$ en assignant un poids diff\'erent pour chaque seuil lors du calcul de l'\'energie~\cite{8}.