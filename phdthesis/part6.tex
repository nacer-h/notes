\section{Simulation}
\label{p6}
~\par La S\'eparation des particules dans un jet se base sur la reconstruction spatiale des gerbes, or le SDHCAL peut \^etre \'equip\'e de chambres multi-gaps (GRPC) lui conf\'erant une r\'esolution en temps de l'ordre du $100~ps$. On construit un Toy Monte Carlo afin de d\'emontrer le pouvoir de s\'eparation des particules issus des jets, on utilisant leurs temps de vol dans le SDHCAl. Les param\`etres pris en compte sont :\\
\begin{itemize}
\item Temps de vol des : $\pi^{\pm}$ ($59.2\%$), $K^{\pm}$ ($14.1\%$), $p$ ($7\%$), $n$ ($8.5\%$) et $K^{0}$ ($11.3\%$) dans le calorim\`etre. \\
\item Un champs magnetique de 2 Tesla.\\
\item Diff\'erentes distribution de l'impulsion.\\ 
\end{itemize}
 
Comme sur la figure~\ref{figure:6.1}, repr\'esente l'evolution du temps de vol en fonction de l'impulsion des particules. Les particules charg\'ees de faible impulsion n'atteignent pas le calorim\`etre, \`a cause du champ magn\'etique qui fait d\'evier les particules avec un rayon de courbure tr\`es important, alors elles sont pas d\'etect\'e.

\begin{figure}[H]
  \centering
  \includegraphics[width=0.5\columnwidth,keepaspectratio=true]{Plots/TvsP_2.pdf}
  \caption{Evolution du temps de vol en fonction de l'impulsion des particules}
  \label{figure:6.1}
\end{figure}
La figure~\ref{figure:6.2} montre la fraction de paire de hadron chargés et neutres dont le temps d'arrivé dans le calorimètre est supérieur à 100 ps.
\par Pour estimer le pouvoir de s\'eparation des particules dans le SDHCAl, on compare la diff\'erence moyenne du temps de vol des hadron chargés et neutres à la r\'esolusion temporelle du d\'etecteur (figure~\ref{figure:6.2}) exprimée en foction de l'impulsion moyenne des constituants du jet. La confusion entre les dépots d'énergie dans le calorimètre provient essentiellement entre les hadrons chargés et neutres ce qui conduit à une dégradation de la résolution du détecteur. On remarque une bonne s\'eparation des particules d'impulsion inférieur \`a 10 GeV pour un d\'etecteur de r\'esolusion de 100 ps.
\vfill%
\begin{figure}[H]
  \centering
  \includegraphics[width=0.5\columnwidth,keepaspectratio=true]{Plots/DeltaTvarVsPcn_2.pdf}
  \caption{Diff\'erence de temps entres les particule charg\'ees et neutres en fonction de l'impulsion moyenne des particules en unit\'e de $\sigma$ (100 ps).}
  \label{figure:6.2}
\end{figure}
\vfill%
L'efficacité de séparation entre les hadrons chargés et neutres s'améliore dans un milieu où le champ magnétique est de 2 Tesla (figure~\ref{figure:6.3} et~\ref{figure:6.4} ). Pour un champ magnétique intense de 4 Tesla, l'efficacité de séparation se dégrade (figure~\ref{figure:6.5} et~\ref{figure:6.6}), due au rapprochement des temps d'arrivés des hadrons chargés et neutres. Les particules chargés sont retardés par le champs magnétique, les particules neutres sont retardés par leur masse importante.

\begin{figure}[H]
  \centering 
  \mbox{\subfigure[\label{}]{\includegraphics[width=0.5\columnwidth]{Plots/EffvsP_cn_0.png}} \subfigure[\label{}]{\includegraphics[width=0.5\columnwidth]{Plots/EffvsP_cn_2.png}}}
  \caption{Efficacité de séparation des hadrons chargés et neutres pour un détecteur de résolution temporelle de 100 ps (a) sans champ magnétique, (b) avec champs magnétique de 2 Tesla.} 
  \label{figure:6.3}
\end{figure}

\begin{figure}[H]
  \centering 
  \mbox{\subfigure[\label{}]{\includegraphics[width=0.5\columnwidth]{Plots/DeltaTvarVsPcn_0.png}} \subfigure[\label{}]{\includegraphics[width=0.5\columnwidth]{Plots/DeltaTvarVsPcn_2.png}}}
  \caption{Différence de temps entres les particules chargés et neutres en fonction de l'impulsion moyenne des particules en unité de $\sigma$ (100 ps)  : (a) sans champ magnétique, (b) avec champs magnétique de 2 Tesla.} 
  \label{figure:6.4}
\end{figure}

\begin{figure}[H]
  \centering 
  \mbox{\subfigure[\label{}]{\includegraphics[width=0.5\columnwidth]{Plots/EffvsP_cn_2.png}} \subfigure[\label{}]{\includegraphics[width=0.5\columnwidth]{Plots/EffvsP_cn_4.png}}}
  \caption{Efficacité de séparation des hadrons chargés et neutres pour un détecteur de résolution temporelle de 100 ps (a) avec champs magnétique de 2 Tesla, (b) avec champs magnétique de 4 Tesla.} 
  \label{figure:6.5}
\end{figure}

\begin{figure}[H]
  \centering 
  \mbox{\subfigure[\label{}]{\includegraphics[width=0.5\columnwidth]{Plots/DeltaTvarVsPcn_2.png}} \subfigure[\label{}]{\includegraphics[width=0.5\columnwidth]{Plots/DeltaTvarVsPcn_4.png}}}
  \caption{Différence de temps entres les particules chargés et neutres en fonction de l'impulsion moyenne des particules en unité de $\sigma$ (100 ps)  : (a) avec champs magnétique de 2 Tesla, (b) avec champs magnétique de 4 Tesla.} 
  \label{figure:6.6}
\end{figure}

Néanmoins le champs améliore la séparation dans l'espace en séparant les hadrons chargés et neutres. 
~\par Nous avons considéré la séparation dans le temps indépendamment de la séparation dans l'espace. Les deux critères doivent être combinés.