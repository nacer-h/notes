\section{Mean Energy Loss estimation in the Central Drift Chamber}
\label{p3}

Reconstructing the final charged and neutral tracks are essential for particle identification (PID). As an example looking for the Hybrid candidate, $Y(2175)$, in the exclusive reaction $\gamma p \rightarrow Y(2175) p \rightarrow K^{+}K^{-}\pi^{+}\pi^{-}p$ demands a good $K$, $\pi$ and $p$ separation. One of the crucial GlueX spectrometer subsystems for providing the PID information is the CDC detector, by measuring the mean energy loss per length $dE/dx$.
When a charged particle passes through the detector, it can traverse through multiple straws. On the wire in the straw where the particle passed through an electrical signal is induced. From timing characteristics of this signal the distance of closest approach can be reconstructed (hit-radius). When 3 or more hit-radii are reconstructed in an event, a track finding algorithm is applied. Every combination of two hit-radii is taken to provide four possible tracks by tangent lines connecting the two circles defined by the two hit-radii. These trajectories are then used to search other straws with hit-radii within a 4 mm window of these track candidates. The track with the most hits or the lowest $\chi^{2}/dof$ (if the number of hits is the same) is identified as the actual track and passed to the track fitting algorithm.

% Due to their cylindrical geometry, straw tubes can have very low mass, which reduces multiple scattering of charged particles. Individual wire breakage does not affect neighboring wires because each straw is self-contained. Another advantage of straw tubes is that they can support their own wire-tension and in this way the end plates (plates that hold the straws) can be made much thinner compared to a conventional wire chamber, which also reduces multiple scattering. The CDC will consist of 3500, 1.5 m long, straw tubes. The straws are mounted in 28 layers: 12 axial and 16 stereo and the placement of the straws is chosen to minimize tracking ambiguities. The inner and outer diameter of the chamber is 20 and 120 cm respectively. Both end plates, on the side opposite to where the straws end, gas plenums are constructed. These plenums allow gas to be distributed to all the straws at the same time instead of feeding gas to straws individually. High voltage distribution boards and pre-amplifiers are mounted on the upstream gas-plenum, outside the gas volume. The output signals of the pre-amplifiers travel over a 20 m long cable to readout-electronics in VME crates. FIG. 1: Drawing of the final CDC design. The CDC will be operated inside a solenoid that generates a magnetic field of 2.24 T. The goal for position resolution transverse to the wire is 150 µm and 1.5 mm along the axis of the chamber. Full tracking will be achieved using both the CDC and the Forward Drift Chambers, which are located directly downstream the CDC. The target will be located 25 cm upstream from the center of the CDC. The polar angular coverage is between 6 and 165. This detector is designed to enable dE/dx particle identification for pions, kaons, and protons with momenta lower than 450 MeV. A drawing of the final CDC design is shown in Fig. 1. First the straw tube design will be presented with tests concerning sagging and radiation damage. Then the electronics and gas system is introduced followed by studies to characterize the detector. These studies are used to construct a Monte Carlo simulation that describes the detector performance, followed by timing and resolution studies. The possibility of particle identification (PID) by measuring the energy loss of the particles in the straw volume is investigated using the Monte Carlo simulation. Possible improvement of position resolution along the wire by application of charge division is investigated, followed by final remarks and conclusions.
\subsection{Energy Loss Calibration}

\subsection{Mean Energy Loss Estimation}

\subsection{Conclusions}
