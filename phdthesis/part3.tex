\section{Mean Energy Loss estimation in the Central Drift Chamber}
\label{p3}

\subsection{Introduction}

Reconstructing the final charged and neutral tracks are essential for particle identification (PID). Searching for a Hybrid candidate, e.g. $Y(2175)$, in the exclusive reaction $\gamma p \rightarrow Y(2175) p \rightarrow K^{+}K^{-}\pi^{+}\pi^{-}p$ demands a good kaons, pions and protons separation. One of the crucial GlueX spectrometer subsystems for providing the PID information is the CDC detector, through the measurement of energy loss $dE/dx$.

\subsection{Particle Identification in the CDC}

The Central Drift chamber is a gas (mixture of argon and $CO_{2}$) filled detector for the detection of charged particles. A charged particle traversing the gas volume ionises some of the gas atoms, so that electron-ion pairs are produced. Due to the electric field introduced by the Solenoid magnet and applied to the gas volume, the electrons and ions start drifting in opposite directions: the negatively charged electrons to the anode and the positively charged ions to the cathode. For the anode, a thin wire of $20~\mu m$ diameter is used in order to obtain a high electric field nearby. Electrons reaching the vicinity of the anode wire after the drift time are accelerated by this strong field so that they can ionise other gas molecules. The electrons form an avalanche that can be measured as a negative electric pulse. The sum of the integrated charges for each hit in the sense wires is used to calculate the $dE/dx$.
The energy loss $dE/dx$ in the CDC is described by the Bethe-Bloch equation~\ref{eq.3.2}.

\begin{equation}
    \label{eq.3.2}
    -\frac{dE}{dX} = 2 \pi N_A r_e^2 m_e c^2 \rho \frac{Z}{A} \frac{z^2}{\beta^2} \left [ \ln \left( \frac{2 m_e \gamma^2 \beta^2 W_{max}}{I^2} \right) - 2 \beta^2 - \delta - 2\frac{C}{Z} \right].
\end{equation}

\noindent
with $2 \pi N_A r_e^2 m_e c^2 = 0.1535$~MeV$~\!$cm$^2$/gm. The terms in this expression are defined in Table~\ref{tab.3.1}.

\begin{table}[htbp]
    \begin{center}
        \begin{tabular} {||l||l||} \hline
            $r_e$ : classical electron radius   & $\rho$ : density of medium \\
            ~~~~~~ (2.817$\times$10$^{-13}$~cm) & $ze$ : charge of incident particle \\
            $m_e$ : electron mass               & $\beta$ : $v/c$ of incident particle \\
            $N_A$ : Avogadro's number           & $\gamma$ : $1/\sqrt{1 - \beta^2}$ \\
            $I$  : Mean excitation potential    & $\delta$ : density correction \\
            $Z$ : Atomic number of medium       & $C$ : shell correction \\
            $A$ : Atomic weight of medium       & $W_{max}$ : max. energy transfer \\ \hline
        \end{tabular}
    \end{center}
    \caption{\small{Terms in the Bethe-Bloch formula for the energy loss.}}
    \label{tab.3.1}
\end{table}

The maximum energy transfer that can be provided in a head-on collision from the incident particle of mass $M$ to an atomic electron in the medium, $W_{max}$, can be computed as:

\begin{equation}
    \label{eq.3.2}
    W_{max} = \frac{2 m_e c^2 \eta}{1 + 2 s \sqrt{1 + \eta^2} + s^2},
\end{equation}

\noindent
where $s = m_e/M$ and $\eta = \beta \gamma$.

It is a universal function of $\beta \gamma$ for all particle masses. The energy loss as function of momentum shows a characteristic decrease with $1/\beta^{2}$, reaches a minimum around $\beta \gamma$ = 4, and continues with a logarithmic rise "relativistic rise region" until it saturates " Fermi plateau". In practice two corrections are made to this expression to account for density effects ($\delta$) and shell corrections ($C$). The notion of density effects, which are important as the energy of the incident charged particle increases, arises from the fact that the electric field of a charged particle tends to polarize the atoms along its path. Due to this effect, electrons far from the path of the particle are shielded from the full electric field intensity. The notion of shell effects, which are most important at low energies, is needed to account for effects that arise when the velocity of the incident particle is comparable to the orbital velocity of the atomic electrons in the medium. In this case the atomic electrons cannot be assumed as stationary, and capture process is possible. By measuring the momentum of a particle as well as its energy loss, the mass of the particle can be determined. Figure~\ref{fig.3.1} shows the energy loss as a function of reconstructed particle momentum, where we see two distinguished bands for protons and for other lighter particles ($\pi$, $K$ and $e$) for tracks up to $\sim$ 1 GeV/c, beyond that the bands merge resulting in a bad particle identification.

\begin{figure}[h]
    \centering
    \includegraphics[width=30pc]{plots/dedxvsp_cdc.png}
    \caption{\label{fig.3.1}Energy loss dE/dx in the CDC as a function of reconstructed particle momentum.}
\end{figure}

\subsection{Mean Energy Loss Estimation}
The application of an energy loss measurement for the identification of relativistic particles faces several constraints. The energy loss distribution (approximately a Landau distribution) has a large inherent width of 60 to 70$\%$ (FWHM for a path length of 1 cm of gas at atmospheric pressure). This makes it necessary to measure many samples along each track in order to determine the mean energy loss with sufficient accuracy. Reduction of the fluctuations of the energy loss can also be achieved by using thicker samples, for example by increasing the gas pressure. However, increasing the thickness results in a lower Fermi plateau and a different slope of the relativistic rise and therefore smaller differences in the energy loss of different particle species. 

The figure of merit which has to be optimized is not the resolution but the particle separation power expressed as resolution normalized to the energy loss difference.

\subsection{Conclusions}
