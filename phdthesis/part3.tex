\chapter{Estimation of the Mean Energy Loss in the Central Drift Chamber}
\label{p.3}

Reconstructing the final charged and neutral particles is essential for particle identification (PID). Searching for hybrid candidates, $e.g.$ the $Y(2175)$ sate, in the exclusive reaction $\gamma p \rightarrow Y(2175) p \rightarrow K^{+}K^{-}\pi^{+}\pi^{-}p$ demands a good kaon, pion and proton separation. One of the crucial GlueX spectrometer subsystems for providing the PID information is the CDC detector, through the measurement of the energy loss, $dE/dx$.

\section{Particle Identification in the CDC}
\label{p.3.1}

The Central Drift chamber is a gas detector for the detection of charged particles. It is filled with a mixture of $50\%$ argon and $50\%$ CO$_{2}$ at atmospheric pressure. Charged particle traversing the gas volume ionises some of the gas atoms, so that electron-ion pairs are produced. Due to the electric field introduced by the Solenoid magnet and applied to the gas volume, the electrons and ions start drifting in opposite directions: the negatively charged electrons to the anode and the positively charged ions to the cathode. For the anode, a thin wire of $20~\mu$m diameter is used in order to obtain a high electric field nearby. Electrons reaching the vicinity of the anode wire after the drift time are accelerated by this strong field so that they can ionise other gas molecules. The electrons form an avalanche that can be measured as a negative electric pulse. The sum of the integrated charges for each hit in the sense wires can be used to calculate the energy loss~\cite{34}.
~\par In the CDC, the energy loss is obtained from the height of the first peak of each pulse instead of its integral, as this was found to give better resolution. The energy loss and track length in the straws traversed are used finally to compute the $dE/dx$~\cite{35}.
The $dE/dx$ in the material is described by the Bethe-Bloch formula~\cite{36} defined as
\begin{equation}
    \label{eq.3.1}
    -\frac{dE}{dX} = 2 \pi N_A r_e^2 m_e c^2 \rho \frac{Z}{A} \frac{z^2}{\beta^2} \left [ \ln \left( \frac{2 m_e \gamma^2 \beta^2 W_{max}}{I^2} \right) - 2 \beta^2 - \delta - 2\frac{C}{Z} \right]~,
\end{equation}
with $2 \pi N_A r_e^2 m_e c^2 = 0.1535$~MeV$~\!$cm$^2$/gm. The terms in this expression are defined in Tab.~\ref{tab.3.1}.
\begin{table}[b]
    \begin{center}
        \caption{\small{Terms in the Bethe-Bloch formula for the energy loss.}}
        \label{tab.3.1}
        \begin{tabular} {||l||l||} \hline
            $r_e$ : classical electron radius   & $\rho$ : density of medium \\
            ~~~~~~ (2.817$\times$10$^{-13}$~cm) & $ze$ : charge of incident particle \\
            $m_e$ : electron mass               & $\beta$ : $v/c$ of incident particle \\
            $N_A$ : Avogadro's number           & $\gamma$ : $1/\sqrt{1 - \beta^2}$ \\
            $I$  : Mean excitation potential    & $\delta$ : density correction \\
            $Z$ : Atomic number of medium       & $C$ : shell correction \\
            $A$ : Atomic weight of medium       & $W_{max}$ : max. energy transfer \\ \hline
        \end{tabular}
    \end{center}
\end{table}
The maximum energy transfer that can be provided in a head-on collision from the incident particle of mass $M$ to an atomic electron in the medium, $W_{max}$, can be computed as:
\begin{equation}
    \label{eq.3.2}
    W_{max} = \frac{2 m_e c^2 \eta}{1 + 2 s \sqrt{1 + \eta^2} + s^2}~,
\end{equation}
where $s = m_e/M$ and $\eta = \beta \gamma$.
~\par It is a universal function of $\beta \gamma$ for all particle masses. The energy loss as function of momentum shows a characteristic decrease with $1/\beta^{2}$. It reaches a minimum around $\beta \gamma$ = 4, and continues with a logarithmic rise, the so-called "relativistic rise region", until it saturates in form of the "Fermi plateau". In practice two corrections are made to this expression to account for density effects ($\delta$) and shell corrections ($C$). The notion of density effects, which are important as the energy of the incident charged particle increases, arises from the fact that the electric field of a charged particle tends to polarize the atoms along its path. Due to this effect, electrons far from the path of the particle are shielded from the full electric field intensity. The notion of shell effects, which are most important at low energies, is needed to account for effects that arise when the velocity of the incident particle is comparable to the orbital velocity of the atomic electrons in the medium. In this case the atomic electrons cannot be assumed as stationary, and capture process is possible.
~\par The momentum of the charged particles emerging from the interaction point are reconstructed in the CDC, which is defined by

\begin{equation}
    \label{eq.3.2.1}
    p_{total}~=~p_{\perp}~sec~(\lambda)~~~\mathrm{and}~~~\lambda~=~\frac{\pi}{2}-\theta,
\end{equation}

\noindent where the transverse momentum, $p_{\perp}$, and the angle, $\lambda$, are measured from the curvature of the tracks in the magnetic field, with the polar angle, $\theta$, covering a range between $29^{\circ}$ and $123^{\circ}$ in the CDC.
~\par By measuring the momentum of the particle as well as the energy loss, the mass of the particle can be determined. The Figure~\ref{fig.3.1} shows the energy loss as a function of reconstructed particle momentum, where we see two distinguished bands, one for protons at higher $dE/x$ values, and for other lighter particles, pions and kaons, in the horizontal band, at low momentum tracks up to $\sim$ 1 GeV/c, and beyond which the bands merge resulting in a bad PID.

\begin{figure}[H]
    \centering
    \includegraphics[width=0.7\textwidth]{plots/dedxp_all.eps}
    \caption{\label{fig.3.1}$dE/dx$ as a function of reconstructed particle momentum with experimental data in the CDC . At values of $p<0.5$ GeV/c, the protons are in the high $dE/dx$ band, and the kaons and pions are at the lower $dE/dx$ band.}
\end{figure}

\section{Mean Energy Loss Estimation}
\label{p.3.2}

A precise and accurate $dE/dx$ measurement is crucial for the quality of particle identification. A study of the $\gamma p \rightarrow \pi^{+} \pi^{-} p$ phase space shows that the recoiled proton mostly hits the CDC, see Fig.~\ref{fig.3.2}.
Measuring the $dE/dx$ will be the main information to identify these recoiled proton.

\begin{figure}[H]
    \centering
    \includegraphics[width=0.8\textwidth]{plots/top_view_detector_dedx.png}
    \caption{\label{fig.3.2}Top view of the detector, with reconstructed tracks of final state particles in the $\gamma p \rightarrow \pi^{+} \pi^{-} p$ reaction. The tracks at higher polar angles are protons that hit mostly the CDC, while the pions are moving forward, ending at the TOF wall.}
\end{figure}

Before the mean energy loss of a track can be calculated, many corrections were required due to the variations in temperature, atmospheric pressure and other environmental conditions which will affect the $dE/dx$ measurements. Furthermore, the geometric structure of the CDC, the charge-particle trajectory and non-uniformity of the electric and magnetic fields, also cause biases in the $dE/dx$ measurement~\cite{37,38,39}.
~\par The $dE/dx$ distribution has a long tail that follows a Landau-like distribution. The high energy losses are mainly caused by particles having direct collisions with electrons~\cite{36}. Due to the high-energy tail, the mean moves towards higher energy loss, and the mean energy loss value is higher than the most probable value see (Fig.~\ref{fig.3.3}, left). This makes it necessary to measure many samples along each track in order to determine the mean energy loss with sufficient accuracy. Reduction of the fluctuations of the energy loss can also be achieved by using thicker samples, for example by increasing the gas pressure. However, increasing the thickness results in a lower Fermi plateau and a different slope of the relativistic rise, and therefore resulting in a smaller differences in the energy loss of different particle species.
~\par For a better estimate of the mean Energy loss, we use the truncated mean method, due to its simplicity and very good accuracy~\cite{37,38,39}. This method rejects a certain percentage of the lowest and highest $dE/dx$ measurements from the calculation of the mean energy loss. Fig.~\ref{fig.3.3} shows the energy loss for one track in the CDC. We observe that the $dE/dx$ distribution before the truncation (Fig.~\ref{fig.3.3}, left) has a mean of about 6 keV/cm, while the most probable $dE/dx$ value is around 4 keV/cm. After a 40$\%$ truncation at high $dE/dx$ (Fig.~\ref{fig.3.3}, right), a better mean energy loss accuracy is achieved.

\begin{figure}[H]
    \centering
    \begin{subfigure}[b]{0.5\textwidth}
        \includegraphics[width=\textwidth]{plots/dedxav_onetrack_pretrunc.eps}
    \end{subfigure}\hfill
    \begin{subfigure}[b]{0.5\textwidth}
        \includegraphics[width=\textwidth]{plots/dedxav_onetrack_postrunc.eps}
    \end{subfigure}
    \caption{Energy loss for one track in the CDC before truncation (left) and after 40$\%$ truncation (right). An average of 25 hits per track is seen before the truncation. The $dE/dx$ distributions are fitted with Landau and Gauss functions, before and after truncation, respectively.}
    \label{fig.3.3}
\end{figure}
\vfill

~\par The fraction of hits to be rejected is determined by optimizing three figure of merits: the average energy loss resolution of both the proton and pions, the separation power between particle species, defined in Eq.~\ref{eq.3.3.1}, and the mis-identification fraction (mis-PID) between particles, given by Eq.~\ref{eq.3.3.2}. These figure of merits are illustrated in Fig.~\ref{fig.3.2.3}. The mis-PID is the fraction of events in the $dE/dx$ distribution, that is mixed between particle species. The latter was introduced due to the fact that in some cases, where even with a good resolution and separation power, there is still an important mix between the tails of the energy loss distributions of the different particles.

\begin{figure}[H]
    \centering
    \includegraphics[width=0.7\textwidth]{plots/fom_trunc.png}
    \caption{\label{fig.3.2.3} Illustration of the three figure of merits: $dE/dx$ resolution of particle 1 ($\sigma_{1}$) and particle 2 ($\sigma_{2}$), separation power ($Z$), and finally the mis-PID, as shown by the blue hatched area.}
\end{figure}

\begin{equation}
    \label{eq.3.3.1}
    Z = \frac{separation}{resolution} = \frac{dE/dx_{1} - dE/dx_{2}}{\sqrt{\sigma_{1}^2 + \sigma_{2}^2}}~,
\end{equation}

\noindent where $Z$ is separation power, $dE/dx_{1/2}$ and $\sigma_{1/2}$ are the average energy loss and its resolution for particles 1/2, respectively.
% \vspace{-5mm}%Put here to reduce too much white space after your text

Considering the two $dE/dx$ distributions, $G(x;\sigma_{1})$ and $G(x;\sigma_{2})$, of the particle 1 and 2, respectively. The reconstruction efficiency, $\epsilon$, of particle 1 is defined as

\begin{equation}
    \label{eq.3.3.2.1}
    \epsilon = \int_{-\infty}^{x_{0}} G(x;\sigma_{1})~dx~,
\end{equation}

\noindent where $x_{0}$ is $dE/dx$ value at which we can reach a certain level of reconstruction efficiency of particle 1, as shown in Fig.~\ref{fig.3.2.3}, represented by the dashed vertical line around $\sim$ 2. The mis-PID is then the fraction of particle 2 that is incorrectly identified as the particle 1, due to the overlap of the two particle species $dE/dx$ distributions in this region. 

\begin{equation}
    \label{eq.3.3.2}
    \mathrm{mis-PID} = \int_{-\infty}^{x_{0}} G(x;\sigma_{2})~dx~.
\end{equation}

\section{Simulation}
\label{p.3.3}

In the following section, we will focus on optimizing the truncation based on the mis-PID, and finally compare to the other two figure of merits mentioned in Sec.~\ref{p.3.2}.
~\par We generate protons and pions over a symmetric phase-space in momentum and polar angle of the tracks, and we measure their energy loss in the CDC. The $dE/dx$ distributions for the proton (left) and pions (right) versus the track momentum are sown in Fig.~\ref{fig.3.4}.

\begin{figure}[H]
    \centering
    \begin{subfigure}[b]{0.45\textwidth}
        \includegraphics[width=\textwidth]{plots/cdedxp_proton_0_0.eps}
    \end{subfigure}\hfill
    \begin{subfigure}[b]{0.45\textwidth}
        \includegraphics[width=\textwidth]{plots/cdedxp_pip_0_0.eps}
    \end{subfigure}
    \caption{Energy loss versus track momentum for protons (left) and pions (right).}
    \label{fig.3.4}
\end{figure}

The average energy loss for protons and pions are shown Fig.~\ref{fig.3.5}, where the mis-PID fractions is shown in yellow, the latter is defined as the fraction of pion events that mix with protons, considering a reconstruction efficiency of proton of 95$\%$. In the following we will search for the optimal truncation combinations between low and high tail of average $dE/dx$ distribution, that delivers the smallest mis-PID and the strongest separation power.

\begin{figure}[H]
    \centering
    \includegraphics[width=0.45\textwidth]{plots/hdedxmisid_proton_pip_0_0.eps}
    \caption{\label{fig.3.5}Average energy loss in the CDC for protons (Red) and Pions (blue), mis-PID (yellow).}
\end{figure}
Looking at the mis-PID fraction for different truncation combinations in every momentum and polar angle bin on Fig.~\ref{fig.3.6}, from no truncation (0$\%$,0$\%$) on the top left up to (80$\%$,0$\%$) on the bottom right, that smallest mis-PID is a result of a (0$\%$,20$\%$) truncation.
% \begin{figure}[H]
%     \centering
%     \includegraphics[width=1.0\textwidth]{plots/dedxav_proton_pip_all.eps}
%     \caption{\label{fig.3.6}Average energy loss in the CDC for protons (Red) and Pions (blue).}
% \end{figure}
% Similar results are seen when looking in details for the optimal truncation in every momentum and polar angle bin, see Fig.~\ref{fig.3.7}.
\begin{figure}[H]
    \centering
    \includegraphics[width=1.0\textwidth]{plots/cpthetamisid_all_proton_pip.png}
    \caption{\label{fig.3.6}Mis-PID fraction in bins of momentum and polar angle between protons and pions.}
\end{figure}
Studying the correlation between the optimal truncations for every momentum and polar angle of protons and pions (Fig.~\ref{fig.3.7}), shows a concentration of the optimal combination of truncations between 20$\%$ and 40$\%$ on the high tail of dEdx, for both the mis-PID (left) and separation power (right).
\begin{figure}[H]
    \centering
    \begin{subfigure}[b]{0.45\textwidth}
        \includegraphics[width=\textwidth]{plots/ctruncmisidcorr_proton_pip.eps}
    \end{subfigure}\hfill
    \begin{subfigure}[b]{0.45\textwidth}
        \includegraphics[width=\textwidth]{plots/ctruncsepcorr_proton_pip.eps}
    \end{subfigure}
    \caption{Energy loss versus track momentum for protons (left) and pions (right).}
    \label{fig.3.7}
\end{figure}

Averaging over all the polar angles oh the tracks, reflected on the error bars in Fig.~\ref{fig.3.8}, we show the mis-PID (top left), separation power (top right), $dE/dx$ resolution of protons (bottom left) and of pions (bottom right), in track momentum dependence for four different truncation combinations. In the four cases the optimal truncation is 20$\%$ - 40$\%$ on the high tail of energy loss distribution.

\begin{figure}[H]
    \centering
    \begin{subfigure}[b]{0.45\textwidth}
        \includegraphics[width=\textwidth]{plots/cmgergrmisidp_proton_pip.eps}
    \end{subfigure}\hfill
    \begin{subfigure}[b]{0.45\textwidth}
        \includegraphics[width=\textwidth]{plots/cmgergrsepp_proton_pip.eps}
    \end{subfigure}\hfill
    \begin{subfigure}[b]{0.45\textwidth}
        \includegraphics[width=\textwidth]{plots/cmgergrresp_proton.eps}
    \end{subfigure}\hfill
    \begin{subfigure}[b]{0.45\textwidth}
        \includegraphics[width=\textwidth]{plots/cmgergrresp_pip.eps}
    \end{subfigure}
    \caption{mis-PID (top left), separation power (top right), $dE/dx$ resolution of protons (bottom left) and of pions (bottom right) versus the momentum, the error bars are the averaging over polar angles.}
    \label{fig.3.8}
\end{figure}

\section{Conclusions}
\label{p.3.4}

We studied the optimal truncation of the average energy loss for pions and protons in the CDC, by optimizing the mis-PID, separation power and the $dE/dx$ resolution. The optimal truncation is 20$\%$ - 40$\%$ on the high tail of energy loss distribution. A more conservative truncation of 20$\%$ is now included in the GlueX reconstruction software.

