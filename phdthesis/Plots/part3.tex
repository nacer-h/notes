\section{Potentiel de mesure du processus $\mathbf{t\bar{t}H}$}
\label{p3}

\subsection{Le choix de la signature déscriminante}

Un des processus les plus importants qui appara\^it \`a partir de l'\'energie de collision dans le centre de masse de $500GeV$ est le processus $e^{+}e^{-}{\rightarrow}t\bar{t}H$. Le quark $top$, \'etant le fermion le plus massif du Mod\`ele Standard, a le plus important couplage au boson du Higgs, sa mesure est cruciale pour la compr\'ehension du m\'ecanisme de g\'en\'eration de la masse des fermions.
~\par La signature qui retient particuli\`erement notre attention dans cette \'etude est celle d'une production de paire de quark top ($t$),anti-top ($\bar{t}$) accompagnée d'une radiation de boson de higgs ($H$) \`a partir d'une collision d'electron ($e^{-}$) et de positron ($e^{+}$) représenté dans le diagramme de feynman (figure :~\ref{figure:3.1}).

\begin{figure}[H]
\centering
\includegraphics[width=0.4\columnwidth,keepaspectratio=true]{Plots/ee-ttH.jpg}
\caption{diagramme de feynman de premier ordre pour le proccessus $e^{+}e^{-}{\rightarrow}t\bar{t}H$.}
\label{figure:3.1}
\end{figure}

Le quark $t(\bar{t})$ se d\'esint\`egre principalement en quark $b(\bar{b})$ et boson $W^+(W^-)$, le boson du Higgs lui se d\'esint\`egre en paire de boson $W^+W^-$ dans $21\%$ des cas\cite{1}. Le boson $W^+(W^-)$ ce d\'esint\`egre \`a son tour soit de mani\`ere hadronique en une paire de quark $q\bar{q}$, o\`u $q(\bar{q})=u(\bar{d}),c(\bar{s})$ dans $45.7\%$ des cas, soit de mani\`ere leptonique en lepton et neutrino ($l^+(l^-)\nu(\bar{\nu})$ o\`u $l^+(l^-)=e^{\pm},\mu^{\pm},\tau^{\pm}$ dans $10.5\%$ des cas, soit de mani\`ere semi-leptonique en $q\bar{q}l^+\nu$ ($q\bar{q}l^-\bar{\nu}$), dans $43.8\%$ des cas. Les configurations principales obtenues dans l'\'etat finale sont list\'ee dans le tableau~\ref{table:3.1}.

\begin{table}[H]
  \centering
  \begin{tabular}{|c|c|c|}
    \hline Etat finale & Configuration & Taux de branchement(\%) \\
    \hline 
    \multirow{3}{*}{$2\gamma$} & $b\bar{b}l^{+}{\nu}l^{-}\bar{\nu}\gamma\gamma$ & 0.025  \rule[-7pt]{0pt}{20pt}\\
      & $b\bar{b}l^{+}{\nu}q\bar{q}\gamma\gamma$ & 0.112  \rule[-7pt]{0pt}{20pt}\\
      & $b\bar{b}q\bar{q}q\bar{q}\gamma\gamma$ & 0.125  \rule[-7pt]{0pt}{20pt}\\
    \hline
    \multirow{3}{*}{$4b$} &  $b\bar{b}l^{+}{\nu}l^{-}\bar{\nu}b\bar{b}$ & 5.193  \rule[-7pt]{0pt}{20pt}\\
      & $b\bar{b}l^{+}{\nu}q\bar{q}b\bar{b}$ & 23.195  \rule[-7pt]{0pt}{20pt}\\
      & $b\bar{b}q\bar{q}q\bar{q}b\bar{b}$ & 25.901  \rule[-7pt]{0pt}{20pt}\\
    \hline
    \multirow{5}{*}{$multileptons$} &  $b\bar{b}l^{+}{\nu}q\bar{q}l^{+}{\nu}q\bar{q}$ & 5.27  \rule[-7pt]{0pt}{20pt}\\
      & $b\bar{b}l^{+}{\nu}l^{-}\bar{\nu}l^{+}{\nu}q\bar{q}$ & 0.777  \rule[-7pt]{0pt}{20pt}\\
      & $b\bar{b}l^{+}{\nu}l^{-}\bar{\nu}l^{+}{\nu}\bar{\nu}q\bar{q}$ & 0.058  \rule[-7pt]{0pt}{20pt}\\
      & $b\bar{b}l^{+}{\nu}l^{-}\bar{\nu}l^{+}{\nu}\bar{\nu}l^{-}\bar{\nu}{\nu}$ & 0.147  \rule[-7pt]{0pt}{20pt}\\
      & $b\bar{b}q\bar{q}q\bar{q}l^{+}l^{-}l^{+}l^{-}$ & 11.945  \rule[-7pt]{0pt}{20pt}\\
    \hline    
  \end{tabular}
  \caption{Les configurations principales dans l'\'etat finale du proccessus $e^{+}e^{-}{\rightarrow}t\bar{t}H$.}
  \label{table:3.1}
\end{table}

La s\'election permet de distinguer le signal recherch\'e des bruits de fond, on va se servir des propri\'et\'es de notre signal. Comme les états finaux avec 2$\gamma$ sont rarent et ceux avec 4b dont un grand nombre d'évènements mais la difficulté de modèlisé le bruit fond le plus important $t\bar{t}b\bar{b}$. Donc le choix est porté sur un \'etat final contenant deux leptons de m\^eme signe ($l^{\pm}l^{\pm}$).

\subsection{Bruits de fond potentiel}

~\par Les bruits de fond associés \`a cette signature sont:
\begin{itemize}
\item[$\bullet$] $\ e^+e^-{\rightarrow}t\bar{t}$
\item[$\bullet$] $\ e^+e^-{\rightarrow}t\bar{t}Z$
\item[$\bullet$] $\ e^+e^-{\rightarrow}HZ$   
\end{itemize}
En supposant que l'efficacit\'e de reconstruction des leptons dans le d\'etecteur est de 95 $\%$, les 5 $\%$ restantes peuvent conduirent, soit \`a une perte de leptons dans le processus $e^+e^-{\rightarrow}t\bar{t}Z{\rightarrow}l^+\nu \bar{b}q\bar{q}bl^{+} \mathbf{l^-}$, soit \`a une une mal identification de la charge du lepton dans les processus $e^+e^-{\rightarrow}t\bar{t}{\rightarrow}b\nu l^+ \bar{b}\bar{\nu} \mathbf{l^+}$ et $e^+e^-{\rightarrow}HZ{\rightarrow}b\bar{b}l^+ \mathbf{l^-}$, soit à un lepton isolé reconstruit comme prompt dans le processus $e^+e^-{\rightarrow}t\bar{t}{\rightarrow}b\nu l^+\bar{b}q\bar{q} \mathbf{l^+}$, dans les trois cas, on obtient des leptons de m\^eme signe dans l'\'etat final, ce qui conduit \`a plus de bruit de fond apr\`es la reconstruction des leptons.

\subsection{Optimisation du signal}

~\par Le nombre d'\'ev\`enements dans les \'etats finaux $t\bar{t}Z$ ($N_{t\bar{t}Z}$) et $t\bar{t}$ ($N_{t\bar{t}}$) sont exprimer en fonction de l'efficacit\'e de reconstruction ($\epsilon_{rec}$) et d'identification de charge ($\epsilon_{charge}$) des leptons dans le d\'etecteur, repr\'esent\'es dans les figure~\ref{figure:3.2}.(a) et (b) respectivements, sont donn\'es par les relations~\eqref{eq:3.1} et~\eqref{eq:3.2} respectivement: 

\begin{equation}
  \label{eq:3.1}
  N_{t\bar{t}Z}=(1-\epsilon_{rec}){\times}\epsilon_{rec}^2{\times}BR_{(t\bar{t}Z{\rightarrow}l^+{\nu}\bar{b}q\bar{q}bl^+l^-)}{\times}\sigma_{t\bar{t}Z}{\times}\mathcal{L}  
\end{equation}

\begin{equation}
  \label{eq:3.2}
  N_{t\bar{t}}=2{\times}(1-\epsilon_{charge}){\times}\epsilon_{charge}{\times}BR_{(t\bar{t}{\rightarrow}l^+{\nu}\bar{b}l^-\bar{\nu}\bar{b})}{\times}\sigma_{t\bar{t}}{\times}\mathcal{L}  
\end{equation}

~\par o\`u la luminosit\'e int\'egr\'e $\mathcal{L}=1000\ \rm{fb^{-1}}$, les sections efficaces $\sigma_{t\bar{t}Z}=4.04$ fb et $\sigma_{t\bar{t}}=1633$ fb et les taux de branchements $BR_{(t\bar{t}Z{\rightarrow}l^+{\nu}\bar{b}q\bar{q}bl^+l^-)}=4\%$ et $BR_{(t\bar{t}{\rightarrow}l^+{\nu}\bar{b}l^-\bar{\nu}\bar{b})}=9\%$.

\begin{figure}[H]
  \centering
  \subfigure[]{
    \includegraphics[width=0.45\columnwidth,keepaspectratio=true]{Plots/eff_ttbarZ_reco.pdf}
  }
  \quad
  \subfigure[]{
    \includegraphics[width=0.45\columnwidth,keepaspectratio=true]{Plots/eff_ttbar_charge.pdf}
  }
  \caption{(a)- Le nombre d'\'ev\`enements $t\bar{t}Z$ ($N_{t\bar{t}Z}$) en fonction de l'efficacit\'e de reconstruction ($\epsilon_{rec}$) et (b)- Le nombre d'\'ev\`enements $t\bar{t}$ ($N_{t\bar{t}}$) en fonction de l'efficacit\'e d'identification de charge ($\epsilon_{charge}$) des leptons dans le d\'etecteur.}
  \label{figure:3.2}
\end{figure}

~\par Les cofigurations finales du signal et des bruits de fond contiennent : 6 jets dans $t\bar{t}H$, 4 jets dans $t\bar{t}Z$ et 2 ou 4 jets dans $t\bar{t}$. Avec un bruit de fond plus important qui provient de la mal identification de charge et les leptons secondaires dans $t\bar{t}$.\\
La section efficace mesuré du signal ($\sigma$) et de son incertitde sont , donn\'ee par les relations~\eqref{eq:3.3} et~\eqref{eq:3.4}respectivements : 
\begin{equation}
  \label{eq:3.3}
  \sigma=\frac{N_{Data}-N_{Bkg}}{\epsilon_{t\bar{t}H}.\mathcal{L}}
\end{equation}

\begin{equation}
  \label{eq:3.4}
 \frac{ \Delta \sigma}{\sigma}=\frac{1}{\delta}
\end{equation}
o\`u $N_{Data}$ nombre total d'\'ev\`enements,  $N_{Bkg}$ nombre d'\'ev\`enement du bruit de fond et $\epsilon_{t\bar{t}H}$ l'\'efficacit\'e du signal calcul\'ee par Monte Carlo. La significance ($\delta$) donn\'ee par la formule~\eqref{eq:3.5} : 

\begin{equation}
  \label{eq:3.5}
  \delta=\frac{s}{\sqrt{s+b}}
\end{equation}

o\`u $s$ est le nombre d'\'ev\`enement du signal, et $b$ le nombre d'\'ev\`enement du bruit de fond $t\bar{t}$.\\

 Le tableau~\ref{table:3.2}  r\'esume les sections efficaces ($\sigma$), le nombre d'\'ev\`enements du signal et des bruits de fonds, ainsi que le type des bruits de fond pour une \'energie de collision dans le centre de masse de 500 GeV.
\begin{table}[H]
\centering
\begin{tabular}{|c|c|c|c|} \hline
  signal et bruits de fond & $\sigma$(fb) & Nombre d'\'ev\`enements ($\times 1000 $ fb$^{-1}$)  &  Type de Bruit de fond\\ \hline
  $e^+e^-{\rightarrow}t\bar{t}H$ & 1.07 & 149.8 &  \\ \hline
  \multirow{2}{*}{$e^+e^-{\rightarrow}t\bar{t}Z$} & \multirow{2}{*}{4.04} & \multirow{-1}{*}{2} & \multirow{-1}{*}{irréductible} \\ \cline{3-4}
  &  & 4.90 & Acceptance de lepton \\ \hline
  \multirow{3}{*}{$e^+e^-{\rightarrow}t\bar{t}$} & \multirow{3}{*}{1633} & \multirow{-1}{*}{$\approx$ 0} &  \multirow{-1}{*}{irréductible} \\ \cline{3-4}
  &  & 631.17 & mal identification de charge  \\ \cline{3-4}
  &  & 163 & Lepton secondaire  \\ \hline
  $e^+e^-{\rightarrow}HZ$ & 24.8 & 4.14 & mal identification de charge \\\hline
\end{tabular}
\caption{les sections efficaces et le nombre d'\'ev\`enements du signal et des bruits de fonds, ainsi que le type des bruits de fond pour une \'energie de collision dans le centre de masse de 500 GeV.}
\label{table:3.2}
\end{table}