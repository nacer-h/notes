% \phantomsection\addcontentsline{toc}{section}{Bibliography}
% \begin{thebibliography}{ABC}	
%     \bibitem[1]{Fritzsch94} H. Fritzsch, M. Gell-Mann and H. Leutwyler. Advantages of the Color Octet Gluon Picture. \emph{Phys. Lett.} \textbf{47}B, 365 (1973).
%     \bibitem[2]{2} M. Gell-Mann. A Schematic Model of Baryons and Mesons. \emph{Phys.Lett.} \textbf{8}, 214-215 (1964).
%     \bibitem[3]{3} M. Gell-Mann. Symmetries of baryons and mesons. \emph{Phys.Rev.} \textbf{125}, 1067 (1962).
%     \bibitem[4]{4} Y.Ne'eman. Derivation of strong interactions from a gauge invariance. \emph{Nucl.Phys.} \textbf{26}, 1067 (1961).
%     \bibitem[5]{5} G. Zweig. An $SU(3)$ model for strong interaction symmetry and its breaking. \emph{CERN Preprint 8182/Th} (1964).
%     \bibitem[6]{6} D. J. Gross, The discovery of asymptotic freedom and the emergence of QCD. \emph{Rev. Mod. Phys.} \textbf{77}, 837 (2005).
%     \bibitem[7]{7} F. Wilczek. Asymptotic freedom: From paradox to paradigm. \emph{Rev. Mod. Phys.} \textbf{77}, 857 (2005).
%     \bibitem[8]{8} M. Tanabashi et al. (Particle Data Group). Review of Particle Physics. \emph{Phys. Rev.} D \textbf{98}, 030001 (2018).
%     \bibitem[9]{9} C. A. Meyer and E. S. Swanson. Hybrid Mesons. \emph{Progress in Particle and Nuclear Physics.} B\textbf{82}, 21 (2015).
%     \bibitem[10]{10} Gui-Jun Ding, Mu-Lin Yan. A candidate for $1^{--}$ strangeonium hybrid. \emph{Phys. Lett.} B\textbf{650}, 390 (2007).
%     \bibitem[11]{11} H. Al Ghoul et al. (GlueX Collaboration). \emph{Phys.Rev.} C \textbf{95} 042201(R), (2017).
%     \bibitem[12]{12} J. J. Dudek, R. G. Edwards, P. Guo, and C. E. Thomas (for Hadron Spectrum Collaboration), Toward the excited isoscalar meson spectrum from lattice QCD. \emph{Phys. Rev.} D \textbf{88}, 094505 (2013).
%     \bibitem[13]{13} N. Isgur, R. Kokoski, and J. Paton. Gluonic excitations of mesons: Why they are missing and where to find them. \emph{Phys. Rev. Lett.} \textbf{54}, 869 (1985).
%     \bibitem[14]{14} A. P. Szczepaniak, M. Swat. Role of Photoproduction in Exotic Meson Searches. \emph{Phys.Lett.} B \textbf{516}, 72 (2001).
%     \bibitem[15]{15} S. Godfrey and J. Napolitano. Light-meson spectroscopy. \emph{Rev. Mod. Phys.} \textbf{71}, 1411 (1999).
%     \bibitem[16]{16} B. Aubert et al. (BaBar Collaboration). Structure at 2175 MeV in $e^+e^- \rightarrow \phi f_0(980)$ observed via initial-state radiation. \emph{Phys. Rev.} D \textbf{74}, 091103(R) (2006).
%     \bibitem[17]{17} B. Aubert et al. (BaBar Collaboration). Cross sections for the reactions $e^{+}e^{-} \rightarrow K^+K^{-}\pi^{+}\pi^{-}$, $K^{+}K^{-}\pi^{0}\pi^{0}$ and $K^{+}K^{-}K^{+}K^{-}$ measured using initial-state radiation events. \emph{Phys. Rev.} D \textbf{86}, 012008 (2012).
%     \bibitem[18]{18} C. P. Shen et al. (Belle). Observation of the $\phi(1680)$ and the $Y(2175)$ in $e^{+}e^{-} \rightarrow \phi \pi^{+} \pi^{-}$. \emph{Phys. Rev.} D \textbf{80}, 031101 (2009).
%     \bibitem[19]{19} M. Ablikim et al. (BES). Observation of $Y(2175)$ in $e^{+}e^{-} \rightarrow \eta \phi f_{0}(980)$. \emph{Phys. Rev. Lett.} \textbf{100}, 102003 (2008).
%     \bibitem[20]{20} M. Ablikim et al. (BESIII). Study of $J/\psi \rightarrow \eta \phi \pi^{+} \pi^{-}$. \emph{Phys. Rev.} D \textbf{91}, 052017 (2015).
%     \bibitem[21]{21} M. Ablikim et al. (BESIII). Observation of $e^{+}e^{-} \rightarrow \eta Y(2175)$ at center-of-mass energies above 3.7 GeV. \emph{Phys. Rev.} D \textbf{99}, 012014 (2019).
%     \bibitem[22]{22} H. X. Chen et al. $Y(2175)$ state in the QCD sum rule. \emph{Phys. Rev.} D \textbf{78}, 034012 (2008).
%     \bibitem[23]{23} E. Klempt and A. Zaitsev. Glueballs, hybrids, multiquarks: Experimental facts versus QCD inspired concepts. \emph{Phys. Rept.} \textbf{454}, 1 (2007).
%     \bibitem[24]{24} S. Coito, G. Rupp and E. van Beveren. Multichannel calculation of excited vector $\phi$ resonances and the $\phi(2170)$. \emph{Phys.Rev.} D \textbf{80}, 094011 (2009).
%     \bibitem[25]{25} L. Alvarez-Ruso, J.A. Oller, J.M. Alarcon. $\phi(1020)f_{0}(980)$ $S$-wave scattering and the $Y(2175)$ resonance. \emph{Phys.Rev.} D \textbf{80}, 054011 (2009).
%     \bibitem[26]{26} S. Godfrey and N. Isgur. Mesons in a relativized quark model with chromodynamics. \emph{Phys.Rev.} D \textbf{32}, 189 (1985).
%     \bibitem[27]{27} T. Barnes, F. E. Close, P. R. Page, and E. S. Swanson. Higher quarkonia. \emph{Phys.Rev.} D \textbf{55}, 4157 (1997).
%     \bibitem[28]{28} T. Barnes, N. Black, and P. R. Page. Strong decays of strange quarkonia. \emph{Phys.Rev.} D \textbf{68}, 054014 (2003).
%     \bibitem[29]{29} G.-J. Ding, M.-L. Yan. $Y(2175)$: Distinguish Hybrid State from Higher Quarkonium. \emph{Phys.Lett.} B \textbf{657}, 49 (2007).
%     \bibitem[30]{30} Charles E. Reece. Continuous wave superconducting radio frequency electron linac for nuclear physics research. \emph{Phys. Rev. Accel.} Beams \textbf{19}, 124801 (2016).
%     \bibitem[31]{31} The GlueX Collaboration. Hall D / GlueX Technical Design Report (1997).
%     \bibitem[32]{32} M. Dugger et al. Design and construction of a high-energy photon polarimeter. \emph{Nucl.Instrum.Meth.} A \textbf{867} 115 (2017).
%     \bibitem[33]{33} H. Al Ghoul et al. (The GlueX Collaboration). First Results from The GlueX Experiment. \emph{AIP Conf. Proc.} \textbf{1735} 020001 (2016).
%     \bibitem[34]{34} Y. Van Haarlem et al. (The GlueX Collaboration). The Central Drift Chamber for GlueX. \emph{Nucl. Instrum. Meth.} A \textbf{662} 142 (2010).
%     \bibitem[35]{35} N.S.Jarvis et al.. The Central Drift Chamber for GlueX. \emph{Nucl. Instrum. Meth.} \textbf{962} 163727 (2020).
%     \bibitem[36]{36} W. R. Leo. Techniques for Nuclear and Particle Physics Experiments: A How-to Approach. Second Revised Edition, \emph{Springer-Verlag} 24 (1994).
%     \bibitem[37]{37} Wa. Blum, W. Riegler and L. Rolandi. Particle Detection with Drift Chambers. \emph{Springer-Verlag} 331- (2008).
%     \bibitem[38]{38} Xue-Xiang Cao et al. Studies of dE/dx measurements with the BESIII. \emph{Chin.Phys.} C \textbf{34} 1852 (2010).
%     \bibitem[39]{39} M. Hauschild. Progress in $dE/dx$ techniques used for particle identification. \emph{Nucl. Instrum. Meth.} A \textbf{379} 436 (1996).
%     \bibitem[40]{40} T. H. Bauer, R. D. Spital, D. R. Yennie and F. M. Pipkin. The hadronic properties of the photon in high-energy interactions. \emph{Rev. Mod.Phys.} \textbf{50} 261 (1978).

% \end{thebibliography}

