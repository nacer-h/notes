\chapter{Search For the Y(2175) in Photoproduction at GlueX}
\label{chap.y2175}

\section{Introduction}
\label{chap.y2175.intro}

The discovered structure at 2175 MeV in e$^+$e$^{-}$ collider experiments is claimed as an isospin singlet, and its spin-parity is determined to be $J^{PC}$ = 1$^{--}$. While the PDG meanwhile introduced the notation $\phi(2170)$, within this work the state will still be denoted $Y(2175)$. The observation of this resonance, with its peculiar width and mass, besides a seemingly preferential decay modes, has triggered many theoretical interpretations, most of which propose exotic solutions. 
~\par Despite all previous experimental efforts, our knowledge of the $Y(2175)$ is not sufficient to confirm or suppress one of the theoretical interpretations. So far, all the experimental information about the $Y(2175)$ are limited to the e$^+$e$^{-}$ annihilation and $J/\psi$ hadronic decay. The $Y(2175)$ production in other processes will help to understand its nature.
~\par Hadron production induced by photons has been largely studied since it provides an excellent tool to probe the hadron spectrum~\cite{Ballam68, Meyer70, Wang14, Wang17}. The strong affinity of photons for $s\bar{s}$ allows to use photon beams to study the strangeonium-like states, like the observation of the $\phi(1020)$~\cite{Mibe05} and $\phi(1680)$~\cite{Aston81} in $\gamma p \rightarrow K^{+}K^{-}p$ reaction. Since the $Y(2175)$ was observed in the $\phi \pi^{+}\pi^{-}$ and $\phi f_{0}(980)$ channels, indicating a substantial $s\bar{s}$ component in the $Y(2175)$, it would be straightforward to search for the resonance $Y(2175)$ in the reaction of $\gamma p \rightarrow \phi f_{0}(980)p$ and $\gamma p \rightarrow \phi \pi^{+}\pi^{-}p$.
~\par In this chapter, we will search for the $Y(2175)$ resonance in photoproduction, in both $\phi \pi^{+}\pi^{-}$ and $\phi f_{0}(980)$ decay modes, while studying the $\gamma p \rightarrow K^{+}K^{-} \pi^{+}\pi^{-} p$ final state. Thus, we measure cross sections for the $\gamma p \rightarrow Y(2175) p \rightarrow \phi \pi^+ \pi^{-} p$ and $\gamma p \rightarrow Y(2175) p \rightarrow \phi f_{0}(980) p$ resonant and non-resonant (without the $Y(2175)$ state) modes. To achieve that, we start by an event selection to reduce the background, that is followed by a description of the Monte Carlo samples and the real data used in the analysis. We finally report the cross section measurements for the different channels, and discuss the systematic uncertainties associated with theses measurements.

\section{Data and Simulation}
\label{chap.y2175.data_mc}

\subsection{Data}

The first phase running of the GlueX experiment, GlueX Phase-I, was completed at the end of 2018. It has started collecting data since 2016, with four run periods. Theses data sets are used in this study.

\subsubsection{Data Samples}
\label{chap.y2175.data_mc.data_samples}

The data are organized into a number of "runs" that correspond to $\sim$ 2 hours of data collection. In 2016, most of the runs were spent on studying hardware performance, thus only a subset of runs (from 11366 to 11555) are selected, which was the optimal running conditions of the detector.
~\par During the selected set of runs, the polarized photon beam was produced on a thin diamond radiator and passed through the collimator. The diamond was rotated between two perpendicular orientations, parallel and perpendicular polarizations, with respect to the floor. A small set of the selected runs are produced with unpolarized photon beam, using an aluminum radiator instead of the diamond radiator. Since the importance of having a large data sample to increase the probability of finding the $Y(2175)$ in photoproduction, all the different polarizations are combined in a dataset.
~\par The CEBAF accelerator delivers a 250 MHz electron beam, corresponding to a beam bunch spacing of 4.008 ns, with different average intensities in the datasets. Due to large quantity of data collection, and to efficiently store and process these data, a set of conditions were implemented to save only events of potential physics interest. These events that pass the trigger conditions are referred as triggers. For instance, a minimum energy deposition in the FCAL and/or BCAL is used to determine a good event, which is used in the 2017 dataset triggers. A summary of the luminosity in the coherent peak region, the number of triggers, and the running conditions for the different datasets is presented in Tab.~\ref{tab.y2175.data_mc.data_samples}.

\begin{table}[H]
    \centering
    % \small
    \caption{Summary of GlueX Phase-I selected dataset}
    \label{tab.y2175.data_mc.data_samples}
    \begin{tabular}{|c|c|c|c|c|c|}%|c|c|c|c|c|c|
        \hline
        \multirow{2}{*}{\thead{Run\\Period}} & \multirow{2}{*}{\thead{Coherent\\Peak\\Luminosity\\(pb$^-1$)}} & \multirow{2}{*}{\thead{Number\\of\\Triggers\\(x $10^9$)}} & \multicolumn{3}{c|}{Running Conditions} \\ [1ex]
        \cline{4-6}
         & & & \thead{Beam\\Intensity\\(nA)} & \thead{Radiator\\Thickness\\($\mu$m)} & \thead{Collimator\\Diameter\\(mm)}\\
        \cline{4-6}
        \hline
        2016 & 2.0 & 6.1 & 160, 200 & 50 & 3.4 \\
        % \hline
        2017 & 21.8 & 49.6 & 100, 150  & 58 & 5.0 \\
        % \hline
        Spring 2018 & 58.4 & 146.0 & 50 - 250 & 17, 58 & 3.4, 5.0 \\
        % \hline
        Fall 2018 & 39.2 & 80.14 & 450, 200 & 17, 47 & 5.0 \\
        \hline
    \end{tabular}
\end{table}

\subsubsection{Data Processing}
\label{chap.y2175.data_mc.data_process_comb}

The triggered events are stored in a raw format, which is then processed and used to reconstruct the four-momentum vectors, positions of the tracks and showers, and many other important quantities, like particle identification information. After every improvement in reconstruction and calibration the data is processed again to produce the above quantities with a better precision. The latest reconstruction versions are used in this analysis.

\subsubsection{Tagged Photon Flux}
\label{chap.y2175.data_mc.data_tag_flux}

The tagged photon flux for a data run period is determined using the hit coincidence between the PS and the TAGM or TAGH, including the PS acceptance correction. The tagged flux integrated over the run periods of the different datasets used in this study is shown in Fig.~\ref{fig.y2175.data_mc.data_tag_flux}. The data collected during the running experiment  is not always recorded, due to detector and data acquisition limitations. Thus the measured photon flux has to account for the live time, the time that the data acquisition was ready to record events in the experiment, and correct the measured photon flux.
~\par The increase in the number of triggers from 2016 to 2018 spring datasets, is reflected in the flux yields seen in Fig.~\ref{fig.y2175.data_mc.data_tag_flux}. As expected, the coherent photons region is produced between 8 - 9.2 GeV, with a shift in the 2016 dataset energy peak towards a higher energy, due to the decrease in the electron beam energy delivered to the GlueX experiment from $\sim$ 12 GeV in 2016 data to 11.6 GeV in the rest datasets.

\begin{figure}[H]
    \centering
    \includegraphics[width=0.8\textwidth]{plots/ctot_tagged_flux.eps}
    \caption{\label{fig.y2175.data_mc.data_tag_flux}The tagged photon flux versus the photon beam energy distributions for 2016 (black), 2017 (blue), Spring 2018 (red), and Fall 2018 (magenta) datasets.}
\end{figure}

\subsection{Monte Carlo Simulation}
\label{chap.y2175.data_mc.mc}

To understand our experimental data and obtain the event reconstruction efficiency, a Monte Carlo simulation (MC) is used. The MC samples are generated based on an isobar model, where a meson decays into two particles. The widths and masses of the generated particles are extracted from PDG data~\cite{Tanabashi18}, with the $Y(2175)$ parameters taken from a weighted data average over multiple experimental measurements. The generator produces four-vectors for a given topology, where the generated final state particles did not include any spin information. The generated beam energy distribution and the momentum transfer are based on the beam properties and t-slope from each datasets, respectively. Four samples for each reaction matching the corresponding year of data taking are generated, and a set of random triggers are included to simulate the detector noise during data collection for every run. The generated events were then passed through the modeled GlueX detectors based on Geant4, to simulate their response. In addition, the results were then smeared to model the detector resolution and efficiency. Finally, the simulated events were then reconstructed and analyzed in the same way as real data. A summary of theses samples with the number of generated events are shown in Tab.~\ref{tab.y2175.data_mc.mc}   

\begin{table}[H]
    \centering
    \caption{Monte Carlo samples}
    \label{tab.y2175.data_mc.mc}
    \begin{tabular}{|c|c|c|c|c|}
        \hline
        MC samples & 2016 & 2017 & Spring 2018 & Fall 2018 \\
        \hline
        $\gamma p \rightarrow \phi \pi^+ \pi^- p$ & 2 M & 10 M & 9.7 M & 10 M \\
        \hline
        $\gamma p \rightarrow Y(2175) p \rightarrow \phi \pi^+ \pi^- p$ & 2 M & 10 M & 9.7 M & 10 M \\
        \hline
        $\gamma p \rightarrow \phi f_0 p$ & 2 M & 10 M & 9.7 M & 10 M \\
        \hline
        $\gamma p \rightarrow Y(2175) p \rightarrow \phi f_0 p$ & 2 M & 10 M & 9.7 M & 10 M \\
        \hline
    \end{tabular}
\end{table}

The phase space kinematics of the final state particles in the $\gamma p \rightarrow \phi \pi^+ \pi^- p$, $\phi \rightarrow K^+ K^-$ process is represented in the Fig.~\ref{fig.y2175.data_mc.mc}, where momentum and polar angle for these particles are provided. The pions and kaons receiving a higher momentum will preferentially travel towards the FDC, FCAL and TOF, while the recoiled protons with lower momentum will move with a higher open angle $\sim$ [40$^{\circ}$ - 60$^{\circ}$] relative to the beam direction, to hit mostly the CDC and BCAL.

\begin{center}
\null
\vfill
\begin{figure}[htbp]   
    \centering
        \includegraphics[width=1.0\textwidth]{plots/cphi2pi_17_chi100_tot_ptheta.eps}
        \caption{Momentum versus polar angle for the $\gamma p \rightarrow \phi \pi^+ \pi^- p$, $\phi \rightarrow K^+ K^-$ topology, with the $K^{+},~K^{-},~\pi^{+},~\pi^{-}$, and proton final state particles after reconstruction in the GlueX detector.}
        \label{fig.y2175.data_mc.mc}
\end{figure}
\null
\vfill
\end{center}

\section{Event Selection}
\label{chap.y2175.evt_sel}

In order to search for $Y(2175)$ in the decay modes $\phi \pi^+ \pi^-$ and $\phi f_0(980)$, with $\phi \rightarrow K^+ K^- $ and $f_0 \rightarrow \pi^+ \pi^-$, we study the reactions of the form $\gamma p \rightarrow K^+ K^- \pi^+ \pi^- p$. The purpose of the event selection procedure is to subtract as much as possible the background events that mimics our signal, as well as keeping as much as possible the signal events. This is realized by cutting on different variables, then followed by selecting the exclusive $\phi \pi^+ \pi^-$ events, since the $\phi f_0(980)$ is a subsample of the $\phi \pi^+ \pi^-$.

\subsection{Particle Combinations}
\label{chap.y2175.evt_sel.par_com}

We start selecting the candidates for the reaction $\gamma p \rightarrow K^+ K^- \pi^+ \pi^- p$ by requiring one tagged photon beam, three reconstructed positively charged tracks, and two reconstructed negatively charged track, which altogether create a single combination matching to the desired decay. Multiple combinations of the reconstructed particles lead to the possibility of multiple hypotheses for a single event. To prevent double counting of events, we keep track of the particles used in a combination.
% In addition, three extra "good" charged tracks are allowed in the event, that have a matching hit in one of the detectors besides the drift chambers. These extra tracks could be a potential background, but could not be removed from the events at this early stage of the analysis.

\subsection{Beam Photon Accidentals Subtraction}
\label{chap.y2175.evt_sel.bea_pho_acc_sub}

In an event, one or more tagged photons could arrive in the same time window of $4.008$ ns to the target. Using one of these wrong photons 'accidentally' arriving in the current time window to constrain momentum and energy for exclusive event reconstruction leads to indistinguishable peaking background in quantities of interest. Since the primary photon and the accidentals arrive in the same time, a selection of the beam bunch time alone will not be sufficient. For this reason, a statistical method is used to remove the contribution from the accidental photons. We estimate the number of events generated from photons outside the beam bunch time, since the behavior of these photons is similar to the accidental photons but that are for certain not part of the current reaction. This is achieved by estimating the accidental contribution as an average over 8 adjacent time windows (4 before and 4 after the current signal time window) and assigning a weight of $1.0$ and $-1/8$ to all the combinations inside and outside the main beam bunch time, respectively. Finally, these weights are used in the analysis to subtract the contribution from the accidental photons. The time difference between the time of the reconstructed tagged photons, and the Radio Frequency (RF) time, which is coming from the accelerator clock corresponding to the incoming beam photon time at the center of the target, is shown in Fig.~\ref{fig.chap.y2175.evt_sel.bea_pho_acc_sub}. The primary photon beam bunch appear centered near $\Delta t_{Beam-RF} = 0$. In addition to this main peak, four beam bunches in each side equally spaced in time period of 4.008 ns, since the accelerator delivers micro-pulses at 249.5 MHz. These eight peaks are mainly caused by real electron hits in other tagger channels near to the primary photon energy.

\begin{figure}[H]
    \centering
        \includegraphics[width=0.8\textwidth]{plots/cpippimkpkm_17_chi100_TaggerAccidentals.eps}
        \caption{Time difference between the tagged photons and the RF clock time. The primary photon beam is shown in the middle peak after accidental subtraction (red), and the near beam bunches are shown in each side of the main peak, separated by 4.008 ns.}
        \label{fig.chap.y2175.evt_sel.bea_pho_acc_sub}
\end{figure}

\subsection{Track Energy Loss Selection}
\label{chap.y2175.evt_sel.pid_dedx_sel}

As discussed in chapter~\ref{p.3}, we isolate the recoiled protons from the pions and kaons detected in the CDC, by applying a cut on the dE/dx. To select between the different mass hypotheses for charged tracks, we use an exponential function to select proton candidates as described by Eq.~\ref{eq.y2175.evt_sel.pid_dedx_sel.1} and lighter particle candidates (pions and kaons) as given by Eq.~\ref{eq.y2175.evt_sel.pid_dedx_sel.2}:

\begin{equation}
    \begin{aligned}
        \label{eq.y2175.evt_sel.pid_dedx_sel.1}
        \frac{dE}{dx}~>~e^{(-4.0~p + 2.25)} + 1.0~,~\mathrm{and}
    \end{aligned}
\end{equation}
\begin{equation}
    \begin{aligned}
        \label{eq.y2175.evt_sel.pid_dedx_sel.2}
        \frac{dE}{dx}~<~e^{(-7.0~p + 3.0)} + 6.2~,
    \end{aligned}
\end{equation}

where $p$ is the momentum of the particles.
Fig.~\ref{fig.y2175.evt_sel.pid_dedx_sel} shows the energy loss of positively charged particles as a function of their momentum. According to the Bethe-Bloch formula, lower momentum protons deposit more energy (curved band in Fig.~\ref{fig.y2175.evt_sel.pid_dedx_sel}) than lighter particles (horizontal band in Fig.~\ref{fig.y2175.evt_sel.pid_dedx_sel}) for the same momentum. A good separation between the particles is seen up to $\sim$ 1 GeV/$c^{2}$ momentum. A more conservative cut is applied on the $dE/dx$ to avoid throwing potential good events that are closer to the region where the two bands merge.

\begin{figure}[H]
    \centering
        \includegraphics[width=0.65\textwidth]{plots/c_dedx_cdc.eps}
        \caption{dE/dx of positively charged particle as a function of their momentum in the CDC. The curved and horizontal bands represent protons and lighter particles (kaons and pions) candidates, respectively. The proton candidates above the red curve (Eq.~\ref{eq.y2175.evt_sel.pid_dedx_sel.1}) are kept.}
        \label{fig.y2175.evt_sel.pid_dedx_sel}
\end{figure}

\subsection{Timing Selection}
\label{chap.y2175.evt_sel.pid_tim_sel}

Comparing the RF beam bunch time and the track vertex time for every final state particle candidate $K^{+}$, $K^{-}$, $\pi^{+}$, $\pi^{-}$ and proton candidates, provides a good PID, and a timing cut was made in each subdetector. The vertex time is the time of the matched hit, propagated to the point of closest approach to the beamline. Since the reference plane for timing is chosen to be at the center of the liquid hydrogen target, a correction is made to the vertex time to account for the distance between the vertex location and the reference plane. Fig.~\ref{fig.y2175.evt_sel.pid_tim_sel} shows this timing difference for the TOF detector for proton candidates both in data (fig.~\ref{fig.y2175.evt_sel.pid_tim_sel.a}) and MC simulation (fig.~\ref{fig.y2175.evt_sel.pid_tim_sel.b}), as a function of particle momentum. The protons appear in the range [-0.3,+0.3] (ns), corresponding to a 3$\sigma$ cut around the mean, where $\sigma \sim 100$ ps is the TOF detector time resolution. The other entries outside this time window are pions and/or kaons, which is the reason for their strong suppression in MC simulation. A loose selection is also applied on the rest af the particle candidates, due to the non trivial particle bands distinction. A summary of the timing cuts are listed in tab.~\ref{tab.y2175.evt_sel.pid_tim_sel}.

\begin{figure}[H]
    \centering
    \begin{subfigure}[b]{0.5\textwidth}
        \includegraphics[width=\textwidth]{plots/c2_pippimkpkm_17_chi100_p_dttof.eps}
        \caption{}
        \label{fig.y2175.evt_sel.pid_tim_sel.a}
    \end{subfigure}\hfill
    \begin{subfigure}[b]{0.5\textwidth}
        \includegraphics[width=\textwidth]{plots/c2_phi2pi_17_chi100_p_dttof.eps}
        \caption{}
        \label{fig.y2175.evt_sel.pid_tim_sel.b}
    \end{subfigure}
    \caption{The difference between the time measured by TOF after propagation to interaction vertex and the time delivered by the RF clock for protons as a function of particle momentum in (a) data and (b) MC simulation. The time window selected is shown between the two red lines, corresponding to the proton candidates. The curved time band is due to mis-identified protons with lighter particle (pions and kaons) arriving earlier in time to the TOF detector.}
    \label{fig.y2175.evt_sel.pid_tim_sel}
\end{figure}

\begin{table}[H]
    \centering
    \small
    \caption{Events selection using the difference between the RF and vertex time at each detector system.}
    \label{tab.y2175.evt_sel.pid_tim_sel}
    \begin{tabular}{|c|c|c|}
        \hline
        Candidate & Detector System & $\Delta T_{detector-RF}$ Cut (ns) \\
        \hline
        $\pi^{\pm}$ & TOF & $\pm$ 0.5 \\
        \hline
        $\pi^{\pm}$ & BCAL & $\pm$ 1.0 \\
        \hline
        $\pi^{\pm}$ & FCAL & $\pm$ 2.0 \\
        \hline
        $\pi^{\pm}$ & SC & $\pm$ 2.5 \\
        \hline
        $K^{\pm}$ & TOF &  $\pm$ 0.3 \\
        \hline
        $K^{\pm}$ & BCAL & $\pm$ 0.75 \\
        \hline
        $K^{\pm}$ & FCAL & $\pm$ 2.5 \\
        \hline
        $K^{\pm}$ & SC & $\pm$ 2.5 \\
        \hline
        proton & TOF & $\pm$ 0.3 \\
        \hline
        proton & BCAL & $\pm$ 1.0 \\
        \hline
        proton & FCAL & $\pm$ 2.0 \\
        \hline
        proton & SC & $\pm$ 2.5 \\
        \hline
    \end{tabular}
\end{table}

\subsection{Kinematic Fitting}
\label{chap.y2175.evt_sel.kin_fit}

Kinematic fitting is a mathematical procedure in which we rely on physics principles governing the particles in the reaction or decay process to improve the measured quantities, $e.g.$: energy, momentum, position,..., $etc$. For instance, considering the reaction, $\gamma p \rightarrow K^+ K^- \pi^+ \pi^- p$. The fact that the five final state particles are coming from a common vertex position can be used to improve the measured position and momentum vectors. The total four-momentum of the final states must equal to the initial beam four-momentum, thus improving the energy and momentum resolution measured of these particles. The fit takes a fully specified reaction 4-momenta and covariance matrices for all initial and final state particles, and the results of this fit can be used to  provide criteria for rejecting background events that does not satisfy the fit constrains and to improve measured quantities.
~\par The kinematic fitting is performed on the measured parameters $y$ (4-momenta, position), together with the errors and correlations among each other, represented by the covariance matrix $V_{y}^{-1}$. The estimated fit quantities are obtained after minimizing the $\chi^{2}$ of the overall kinematic fit, satisfying each of the different constrains. The $\chi^{2}$ is defined as

\begin{equation}
    \label{eq.y2175.evt_sel.kin_fit.1}
    \begin{aligned}
        \chi^{2} = (y-\eta)^{T}V_{y}^{-1}(y-\eta)~,
    \end{aligned}
\end{equation}

~\par If the formulated hypothesis matches the true reaction then the kinematic fit $\chi^{2}/ndf \sim 1$ corresponding to $\chi^{2} \sim 11$ in our case, where the number of degrees of freedom, $ndf$ = number of observables - number of constrains = 11. In data, mainly the non-matching hypotheses ($i.e.$ no $K^{+}K^{-}\pi^{+}\pi^{-}$ event) make the distribution differ from MC, leading to higher tales in the kinematic fit $\chi^{2}$ distribution. The normalized distributions of kinematic fit $\chi^{2}$ for the different datasets in data and MC simulation are shown in Fig.~\ref{fig.y2175.evt_sel.kin_fit.1.a} and Fig.~\ref{fig.y2175.evt_sel.kin_fit.1.b}, respectively. The $\chi^{2}$ distributions are consistent between the different datasets, except for the 2018 Spring dataset, which shows a less converging $\chi^{2}$ in data, and is still under investigation. To insure the minimization of the $\chi^{2}$, the kinematic fit is required to converge. The $\chi^{2}$ cut is selected based on the optimal significance ($Z$) defined by

\begin{equation}
    \label{eq.y2175.evt_sel.kin_fit.2}
    \begin{aligned}
        Z = \frac{S}{\sqrt{S+B}}~,
    \end{aligned}
\end{equation}

Where $S$ and $B$ are the number of $\phi \pi^{+}\pi^{-}$ data signal and background events, respectively.
~\par The signal and background events are extracted by fitting the $K^+K^-$ invariant mass and integrating in the [1, 1.05 GeV/c$^2$] mass region. The $S$ and $B$ are obtained after different kinematic fit $\chi^2$ cuts, from a $\chi^2<100$ to $\chi^2<5$ in 20 steps, a subset is seen in Fig.~\ref{fig.y2175.evt_sel.kin_fit.2}. The signal shape is described by a Voigtian model ($V$), which is a convolution of a Breit-Wigner ($BW$) and a Gaussian ($G$) functions, defined as

\begin{equation}
\label{eq.y2175.evt_sel.kin_fit.3}
    \begin{aligned}
        & V(x;\sigma,\Gamma) = \int_{-\infty}^{+\infty} G(x';\sigma) BW(x-x';\Gamma)dx'~,\\
        & BW(x;\Gamma) =  \frac{A}{2\pi}\frac{\Gamma}{(x-\mu)^2+(\frac{\Gamma}{2})^2}~,\\
        & G(x;\sigma) = \frac{1}{\sigma \sqrt{2\pi}}e^{-\frac{1}{2}(\frac{x-\mu}{\sigma})^2}
    \end{aligned}
\end{equation}

\noindent Here, $\Gamma$ is the Full-width at half-maximum (FWHM) of the Breit-Wigner profile and $\sigma$ is the standard deviation of the Gaussian profile. The amplitude ($A$), mean ($\mu$),  $\sigma$, and $\Gamma$ are the fit parameters. The background shape is described by the Chebyshev polynomial $T_n(x)$ of second degree. For any degree $n$ the Chebyshev polynomial is defined as
 
\begin{equation}
    \label{eq.y2175.evt_sel.kin_fit.4}
    \begin{aligned}
        T_n(x) = \frac{(-2)^{n}n!}{(2n)!}\sqrt{1-x^2}\frac{d^n}{dx^n}(1-x^2)^{n-1/2}~,
    \end{aligned}
\end{equation}

After extracting the signal and background events, the significance is calculated for each $\chi^{2}$ cut using Eq.~\ref{eq.y2175.evt_sel.kin_fit.2}. The resulting significance as a function of the selection variable is displayed in Fig.~\ref{fig.y2175.evt_sel.kin_fit.3}. The optimal significance is realized by a cut of $\sim$ $\chi^2<55$, and this selection is used through all the following analysis. The optimal $\chi^{2}$ cut is indicated by the vertical red line in Fig.~\ref{fig.y2175.evt_sel.kin_fit.1}.

%~ add desired spacing between images, e. g. ~, \quad, \qquad, \hfill etc. (or a blank line to force the subfigure onto a new line)
\begin{figure}[H]
    \centering
    \begin{subfigure}[b]{0.5\textwidth}
        \includegraphics[width=\textwidth]{plots/cpippimkpkm_kin_chisq_all.eps}
        \caption{}
        \label{fig.y2175.evt_sel.kin_fit.1.a}
    \end{subfigure}\hfill
    \begin{subfigure}[b]{0.5\textwidth}
        \includegraphics[width=\textwidth]{plots/cphi2pi_kin_chisq_all.eps}
        \caption{}
        \label{fig.y2175.evt_sel.kin_fit.1.b}
    \end{subfigure}
    \caption{Kinematic fit $\chi^2$ normalized distributions in (a) data and (b) MC simulation, for the different datasets.}
    \label{fig.y2175.evt_sel.kin_fit.1}
\end{figure}

\begin{figure}[H]
    \centering
        \includegraphics[width=1.0\textwidth]{plots/cdata_PhiMass_17_chi2cut.eps}
        \caption{$K^+K^-$ invariant mass after each kinematic fit $\chi^2$ cut, as shown on the top of the histograms. The signal (red) and background (dashed line) fits are described by Eq.~\ref{eq.y2175.evt_sel.kin_fit.3} and Eq.~\ref{eq.y2175.evt_sel.kin_fit.4}, respectively. The total fit is shown in blue. The number of signal ($N_{Sig}$) and background ($N_{Bkg}$) events are displayed for every cut.}
        \label{fig.y2175.evt_sel.kin_fit.2}
\end{figure}

\begin{figure}[H]
    \centering
        \includegraphics[width=0.7\textwidth]{plots/cgrdata_PhiMass_17_chi2cut.eps}
        \caption{Significance as a function of the kinematic fit $\chi^{2}$ cuts. The red vertical line shows the optimal significance and the corresponding best cut.}
        \label{fig.y2175.evt_sel.kin_fit.3}
\end{figure}

\subsection{Missing Mass Squared}
\label{chap.y2175.evt_sel.mis_mass_sqrt}

The conservation of the four-momentum in the exclusive reaction is required, and since all the final state particles were reconstructed, the missing mass, defined in Eq.~\ref{eq.y2175.evt_sel.mis_mass_sqrt}, should be negligible. However, the missing mass is not vanishing due to the detection uncertainty in identification of the particle masses, which represents a source of background. The normalized missing mass squared distributions for the different datasets, both in data and MC simulation are shown in Fig.~\ref{fig.y2175.evt_sel.mis_mass_sqrt.1.a} and Fig.~\ref{fig.y2175.evt_sel.mis_mass_sqrt.1.b}, respectively. The distributions are very consistent between the datasets in data, with a small variations in the missing mass resolution in MC. To reduce this background, we select events with a missing mass squared ($MM$) close to 0, and the $MM^{2}$ cut will be determined again based on the optimal significance defined previously in Eq.~\ref{eq.y2175.evt_sel.kin_fit.2}. The significance is calculated after every $MM^{2}$ symmetric cut, from $\pm 0.1$ (GeV/$c^2$)$^2$ down to 0 in 20 steps of 0.005 (GeV/$c^2$)$^2$, a subset is shown in Fig.~\ref{fig.y2175.evt_sel.mis_mass_sqrt.2}. The maximum significance is reached for a $MM^{2}$ cut in the range [-0.035,+0.035] (GeV/$c^2$)$^2$, indicated by the vertical red dashed line both, in Fig.\ref{fig.y2175.evt_sel.mis_mass_sqrt.3} and Fig.\ref{fig.y2175.evt_sel.mis_mass_sqrt.1}.
% \setlength{\belowdisplayskip}{15pt}
% \setlength{\abovedisplayskip}{15pt}
\begin{equation}
    \label{eq.y2175.evt_sel.mis_mass_sqrt}
    \begin{aligned}
        MM^2 &= \left(\sum P_{i} - \sum P_{f}\right)^2 \\
             &= [(P_{\gamma} + P_{proton}) - (P_{k^+} + P_{k^-} + P_{\pi^+} + P_{\pi^-} + P_{p^{\prime}})]^2~,
    \end{aligned}    
\end{equation}

\noindent the $P_i$ and $P_f$ are the four-momenta of the initial and final particles, respectively. The $P_{p^{\prime}}$ is the four-momentum of the recoiling proton.

\begin{figure}[H]
    \centering
    \begin{subfigure}[b]{0.5\textwidth}
        \includegraphics[width=\textwidth]{plots/cpippimkpkm_mm2_all.eps}
        \caption{}
        \label{fig.y2175.evt_sel.mis_mass_sqrt.1.a}
    \end{subfigure}\hfill
    \begin{subfigure}[b]{0.5\textwidth}
        \includegraphics[width=\textwidth]{plots/cphi2pi_mm2_all.eps}
        \caption{}
        \label{fig.y2175.evt_sel.mis_mass_sqrt.1.b}
    \end{subfigure}
    \caption{The missing mass squared normalized distributions in (a) data and (b) MC simulation, for the different datasets.}
    \label{fig.y2175.evt_sel.mis_mass_sqrt.1}
\end{figure}

\begin{figure}[H]
    \centering
        \includegraphics[width=1.0\textwidth]{plots/cdata_PhiMass_17_mm2cut.eps}
        \caption{$K^+K^-$ invariant mass after each $MM^2$ cut, as shown on the top of the histograms. The signal (red) and background (dashed line) fits are described by Eq.~\ref{eq.y2175.evt_sel.kin_fit.3} and Eq.~\ref{eq.y2175.evt_sel.kin_fit.4}, respectively. The total fit is shown in blue. The number of signal ($N_{Sig}$) and background ($N_{Bkg}$) events are displayed for every cut.}
        \label{fig.y2175.evt_sel.mis_mass_sqrt.2}
\end{figure}

\begin{figure}[htbp]
    \centering
        \includegraphics[width=0.7\textwidth]{plots/cgrdata_PhiMass_17_mm2cut.eps}
        \caption{Significance as a function of cuts on the missing mass squared. The red vertical line shows the optimal significance and the corresponding best cut.}
        \label{fig.y2175.evt_sel.mis_mass_sqrt.3}
\end{figure}

\section{Cross Section and Upper Limit}
\label{chap.y2175.xsec_ul}

In this section, the measurement of the cross sections for the exclusive $\gamma p \rightarrow \phi \pi^+ \pi^- p$ and $\gamma p \rightarrow \phi f_0 p$ reactions is discussed, as well as the determination of an upper limit on the production cross section of the $Y(2175)$ in $\gamma p \rightarrow Y(2175) p \rightarrow \phi \pi^+ \pi^- p$ and $\gamma p \rightarrow Y(2175) p \rightarrow \phi f_0 p$ reactions.

\subsection{Cross Section for \texorpdfstring{$\bm{\gamma p \rightarrow \phi \pi^{+} \pi^{-} p}$}{} }
\label{chap.y2175.xsec_ul.phi2pi}

To study the effect of the photon beam energy ($E_{\gamma}$) in both, the coherent and incoherent region, as well as the momentum transfer (-$t$) dependence on the cross section, the total hadronic cross section for $\gamma p \rightarrow \phi \pi^{+} \pi^{-} p$ reaction in $t$-channel is studied as a function of both $E_{\gamma}$ and -$t$. The cross section is measured in the $E_{\gamma}$ region of 6.5 - 11.6 GeV, distributed equally into 10 bins of 0.51 GeV width, and in the 0 - 4 GeV$^2$ region of -$t$, divided into 10 intervals of 0.4 GeV$^2$ width. The total cross section for the $\gamma p \rightarrow \phi \pi^{+} \pi^{-} p$ reaction is defined as

\begin{equation}
    \label{eq.y2175.xsec_ul.phi2pi}
    \sigma_{\gamma p \rightarrow \phi \pi^{+} \pi^{-} p} = \frac{N_{\phi\pi^+\pi^-}^{Data}}{\varepsilon~\mathcal{L}~BR(\phi\rightarrow K^{+}K^{-})},\\
\end{equation}

\noindent The numerator is the number of $\phi\pi^+\pi^-$ signal events observed in real data. The reconstruction efficiency ($\varepsilon$) is ratio of the number of $\phi\pi^+\pi^-$ reconstructed signal events in MC simulation and the total number of generated events. The luminosity ($\mathcal{L}$) is the product of the integrated flux extracted from Fig.~\ref{fig.y2175.data_mc.data_tag_flux} and the target thickness of 1.273 b$^{-1}$. The last term is the branching ratio of $\phi\rightarrow K^{+}K^{-}$ taken from~\cite{Tanabashi18}, with $BR(\phi\rightarrow K^{+}K^{-})$ = $0.492 \pm 0.005$.
\par The number of generated events are extracted in bins of $E_{\gamma}$ and -$t$, from the total generated events in the MC samples. The total generated events are distributed over the selected region of $E_{\gamma}$ and -$t$, as shown in Fig.~\ref{fig.y2175.xsec_ul.phi2pi.1}.

\begin{figure}[H]
    \centering
    \begin{subfigure}[b]{0.5\textwidth}
        \includegraphics[width=\textwidth]{plots/ctot_beame_tru.eps}
        \caption{}
        \label{fig.y2175.xsec_ul.phi2pi.1.a}
    \end{subfigure}\hfill
    \begin{subfigure}[b]{0.5\textwidth}
        \includegraphics[width=\textwidth]{plots/ctot_t_tru.eps}
        \caption{}
        \label{fig.y2175.xsec_ul.phi2pi.1.b}
    \end{subfigure}
    \caption{\label{fig.y2175.xsec_ul.phi2pi.1}The total generated $\phi \pi^{+} \pi^{-} p$ MC samples distributed in (a) $E_{\gamma}$ and (b) -$t$ bins. The low number of the 2016 MC sample (black squares) reflects the number of events generated for this sample of only 2 M events compared to the other samples of 10 M events each.}
\end{figure}

The number of $\phi \pi^{+} \pi^{-}$ signal events in both MC and data are extracted from fitting the $K^+K^-$ invariant mass in every $E_{\gamma}$ and -$t$ bin. The correlations between the $K^+K^-$ invariant mass and both $E_{\gamma}$ and -$t$ are shown in Fig.~\ref{fig.y2175.xsec_ul.phi2pi.2}. A clear $\phi(1020)$ resonance is seen around the mass of 1.020 GeV/$c^2$, corresponding to the horizontal band in Fig.~\ref{fig.y2175.xsec_ul.phi2pi.2}. The signal shape is described by a Voigtian model (Eq.~\ref{eq.y2175.evt_sel.kin_fit.3}) and the background by a 4$^{th}$ degree Chebyshev polynomial (Eq.~\ref{eq.y2175.evt_sel.kin_fit.4}). The $\phi \pi^{+} \pi^{-}$ yields obtained for each $E_{\gamma}$ and -$t$ bin are shown in Fig.~\ref{fig.y2175.xsec_ul.phi2pi.7} and Fig.~\ref{fig.y2175.xsec_ul.phi2pi.8}, respectively. As expected, the yield is more important in the coherent beam region and at low momentum transfer. The small yield drop in the first -$t$ bin, could be due to the detection loss of the recoiled protons at low momentum. 

\begin{center}
\null
\vfill
\begin{figure}[htbp]
    \centering
    \begin{subfigure}[b]{0.5\textwidth}
        \includegraphics[width=\textwidth]{plots/c2_phie_17.eps}
        \caption{}
        \label{fig.y2175.xsec_ul.phi2pi.2.a}
    \end{subfigure}\hfill
    \begin{subfigure}[b]{0.5\textwidth}
        \includegraphics[width=\textwidth]{plots/c2_phie_mc_17.eps}
        \caption{}
        \label{fig.y2175.xsec_ul.phi2pi.2.b}
    \end{subfigure}
    \begin{subfigure}[b]{0.5\textwidth}
        \includegraphics[width=\textwidth]{plots/c2_phit_17.eps}
        \caption{}
        \label{fig.y2175.xsec_ul.phi2pi.2.c}
    \end{subfigure}\hfill
    \begin{subfigure}[b]{0.5\textwidth}
        \includegraphics[width=\textwidth]{plots/c2_phit_mc_17.eps}
        \caption{}
        \label{fig.y2175.xsec_ul.phi2pi.2.d}
    \end{subfigure}
    \caption{\label{fig.y2175.xsec_ul.phi2pi.2}$K^{+}K^{-}$ invariant mass versus $E_{\gamma}$ in (a) MC and (b) data, as well as versus -$t$ in (c) MC and (d) data, for the 2017 sample. The horizontal narrow band $\sim$ 1.020 GeV/$c^2$ is the $\phi(1020)$ resonance.}
\end{figure}
\null
\vfill
\end{center}

\begin{figure}[H]
    \centering
    \includegraphics[width=1.0\textwidth]{plots/c_phie1_mc_17.eps}
    \caption{\label{fig.y2175.xsec_ul.phi2pi.3}$K^{+}K^{-}$ invariant mass in $E_{\gamma}$ bins for 2017 MC sample. The $E_{\gamma}$ bin ranges and the fit parameters for the total (red), signal (blue), and background (dashed) fits are shown.}
\end{figure}

\begin{figure}[H]
    \centering
    \includegraphics[width=1.0\textwidth]{plots/c_phie1_17.eps}
    \caption{\label{fig.y2175.xsec_ul.phi2pi.4}$K^{+}K^{-}$ invariant mass in $E_{\gamma}$ bins for 2017 dataset. The $E_{\gamma}$ bin ranges and the fit parameters for the total (red), signal (blue), and background (dashed) fits are shown.}
\end{figure}

\begin{figure}[H]
    \centering
    \includegraphics[width=1.0\textwidth]{plots/c_phit1_mc_17.eps}
    \caption{\label{fig.y2175.xsec_ul.phi2pi.5}$K^{+}K^{-}$ invariant mass in -$t$ bins for 2017 MC sample. The -$t$ bin ranges and the fit parameters for the total (red), signal (blue), and background (dashed) fits are shown.}
\end{figure}

\begin{figure}[H]
    \centering
    \includegraphics[width=1.0\textwidth]{plots/c_phit1_17.eps}
    \caption{\label{fig.y2175.xsec_ul.phi2pi.6}$K^{+}K^{-}$ invariant mass in -$t$ bins for 2017 data sample. The -$t$ bin ranges and the fit parameters for the total (red), signal (blue), and background (dashed) fits are shown.}
\end{figure}

\begin{figure}[H]
    \centering
    \begin{subfigure}[b]{0.5\textwidth}
        \includegraphics[width=\textwidth]{plots/cmgphie_mc.eps}
        \caption{}
        \label{fig.y2175.xsec_ul.phi2pi.7.a}
    \end{subfigure}\hfill
    \begin{subfigure}[b]{0.5\textwidth}
        \includegraphics[width=\textwidth]{plots/cmgphie.eps}
        \caption{}
        \label{fig.y2175.xsec_ul.phi2pi.7.b}
    \end{subfigure}
    \caption{\label{fig.y2175.xsec_ul.phi2pi.7}$\phi \pi^+ \pi^-$ yields versus $E_{\gamma}$ in (a) MC and (b) data. The yield for the 2016 (black), 2017 (blue), Spring 2018 (red), and Fall 2018 (magenta) are displayed. The low yields in 2016 reflects the low number of events generated and the low number of triggers in MC and data, respectively.}
\end{figure}

\begin{figure}[H]
    \centering
    \begin{subfigure}[b]{0.5\textwidth}
        \includegraphics[width=\textwidth]{plots/cmgphit_mc.eps}
        \caption{}
        \label{fig.y2175.xsec_ul.phi2pi.8.a}
    \end{subfigure}\hfill
    \begin{subfigure}[b]{0.5\textwidth}
        \includegraphics[width=\textwidth]{plots/cmgphit.eps}
        \caption{}
        \label{fig.y2175.xsec_ul.phi2pi.8.b}
    \end{subfigure}
    \caption{\label{fig.y2175.xsec_ul.phi2pi.8}$\phi \pi^+ \pi^-$ yields versus -$t$ in (a) MC and (b) data. The yield for the 2016 (black), 2017 (blue), Spring 2018 (red), and Fall 2018 (magenta) are displayed. The low yields in 2016 reflects the low number of events generated and the low number of triggers in MC and data, respectively.}
\end{figure}

The efficiency is then calculated as the ratio of the $\phi \pi^+ \pi^-$ reconstructed MC yields and the total number of generated MC events. The results are plotted in Fig.~\ref{fig.y2175.xsec_ul.phi2pi.7}, showing the efficiencies versus $E_{\gamma}$ and -$t$ for the different MC samples. The 2017 and Spring 2018 efficiencies are very comparable to each other, and almost $\sim$ $50\%$ lower then for the 2016 and Fall 2018 datasets. This is mainly due to the different running conditions and random trigger rates included in the different MC sets.

\begin{figure}[t]
    \centering
    \begin{subfigure}[b]{0.5\textwidth}
        \includegraphics[width=\textwidth]{plots/cmgeeff.eps}
        \caption{}
        \label{fig.y2175.xsec_ul.phi2pi.9.a}
    \end{subfigure}\hfill
    \begin{subfigure}[b]{0.5\textwidth}
        \includegraphics[width=\textwidth]{plots/cmgteff.eps}
        \caption{}
        \label{fig.y2175.xsec_ul.phi2pi.9.b}
    \end{subfigure}
    \caption{\label{fig.y2175.xsec_ul.phi2pi.9}The reconstruction efficiency versus (a) $E_{\gamma}$ and (b) -$t$, for $\phi \pi^+ \pi^-$ MC samples of 2016 (black), 2017 (blue), Spring 2018 (red), and Fall 2018 (magenta). The relative ratio of 2017 (blue), Spring 2018 (red), and Fall 2018 (magenta), w.r.t to 2016 datasets are shown in the bottom plot.}
\end{figure}

Finally, having gathered all the ingredients, the cross section is then calculated using Eq.~\ref{eq.y2175.xsec_ul.phi2pi}. The yields, efficiencies and cross sections for the datasets are summarized in Tab.~\ref{tab.y2175.xsec_ul.phi2pi.1.1} -~\ref{tab.y2175.xsec_ul.phi2pi.4.2}. The resultant cross sections versus $E_{\gamma}$ and -$t$ for the different datasets are shown in Fig.~\ref{fig.y2175.xsec_ul.phi2pi.10}. The total cross section for the different datasets are very close and in some points are consistent within errors, except for the 2107 data that is systematically higher then the other datasets. This effect is still under investigation. The total errors are a quadratic sum of the statistical and systematic uncertainties, the estimation of systematic errors will be discussed in Sec.~\ref{chap.y2175.syserr}.
~\par In order to check the consistency of the cross section shape also for 2017 data, the cross section measurements for the 2016, Spring and Fall 2018 are used to produce an average total cross section for every $E_{\gamma}$ and -$t$. The method used to average the cross sections is a standard weighted least-squares procedure~\cite{Tanabashi18}. Since the datasets are independent, the cross section measurements are uncorrelated, and the weighted average and error are then calculated by

\begin{equation}
    \label{eq.y2175.xsec_ul.phi2pi.2}
    \begin{aligned}
        & \bar{\sigma} \pm \delta\bar{\sigma} = \frac{\sum_{i}w_{i}\sigma_{i}}{\sum_{i}w_{i}} \pm  \left(\sum_{i}w_{i}\right)^{-1/2}~, \\
        \mathrm{with}\\
        & w_{i} = 1/(\delta \sigma_{i})^2
    \end{aligned}
\end{equation}

\noindent Here $\sigma_{i}$ and $\delta \sigma_{i}$ are the values and errors of the measured cross sections, with $i=1,2,3$ for the three different datasets, and the sum run over $N=3$ measurements. We then have two main cases depending on the $\chi^{2}/(N-1)$ ratio, with 

\begin{equation}
    \label{eq.y2175.xsec_ul.phi2pi.3}
    \chi^{2} = \sum w_{i}(\bar{\sigma}-\sigma_{i})^2
\end{equation}

If this ratio is smaller or equal to 1, then the final average cross section is as defined in Eq.~\ref{eq.y2175.xsec_ul.phi2pi.2}. But if the ratio is slightly larger then 1, then we increase our average errors $\delta\bar{\sigma}$ in Eq.~\ref{eq.y2175.xsec_ul.phi2pi.2}, by a scale factor $S$ defined as

\begin{equation}
    \label{eq.y2175.xsec_ul.phi2pi.4}
    S = [\chi^{2}/(N-1)]^{1/2}
\end{equation}

The idea here is that large value of the $\chi^{2}$ is likely due to underestimation of errors in at least one of the cross section measurements. Since the measurement with the underestimated error is not known, we assume they are all underestimated by the same factor $S$. Scaling up all the cross section errors by this factor, the ratio gets closer to unity, and consequently the average error $\delta\bar{\sigma}$ scales up by the same factor.
~\par After the calculations of the average cross section for the 2016, Spring and Fall 2018 data, as seen in Fig.~\ref{fig.y2175.xsec_ul.phi2pi.11}, the 2017 cross section is scaled by an empirical constant factor of $0.76$ and $0.63$ in $E_{\gamma}$ and -$t$, respectively. The resultant 2017 corrected cross sections are now consistent within errors with the average cross sections, except at low -$t$, where the difference between the measurements have increased. These scaling factors will help to quantify the sources of the cross section measurement discrepancies between the different datasets.

\begin{center}
\null
\vfill
\begin{figure}[H]
    \centering
    \begin{subfigure}[b]{0.5\textwidth}
        \includegraphics[width=\textwidth]{plots/cmgexsec.eps}
        \caption{}
        \label{fig.y2175.xsec_ul.phi2pi.10.a}
    \end{subfigure}\hfill
    \begin{subfigure}[b]{0.5\textwidth}
        \includegraphics[width=\textwidth]{plots/cmgtxsec.eps}
        \caption{}
        \label{fig.y2175.xsec_ul.phi2pi.10.b}
    \end{subfigure}
    \caption{\label{fig.y2175.xsec_ul.phi2pi.10}$\gamma p \rightarrow \phi \pi^{+} \pi^{-} p$ total cross section versus (a) $E_{\gamma}$ and (b) -$t$, for 2016 (black), 2017 (blue), Spring 2018 (red), and Fall 2018 (magenta). The relative ratio of 2017 (blue), Spring 2018 (red), and Fall 2018 (magenta), w.r.t to 2016 are shown in the bottom plot.}
\end{figure}
\null
\vfill
\end{center}

\begin{center}
    \null
    \vfill
\begin{figure}[H]
    \centering
    \begin{subfigure}[b]{0.5\textwidth}
        \includegraphics[width=\textwidth]{plots/cgexsec_avg.eps}
        \caption{}
        \label{fig.y2175.xsec_ul.phi2pi.11.a}
    \end{subfigure}\hfill
    \begin{subfigure}[b]{0.5\textwidth}
        \includegraphics[width=\textwidth]{plots/cgtxsec_avg.eps}
        \caption{}
        \label{fig.y2175.xsec_ul.phi2pi.11.b}
    \end{subfigure}
    \caption{\label{fig.y2175.xsec_ul.phi2pi.11}$\gamma p \rightarrow \phi \pi^{+} \pi^{-} p$ average cross section (Brown) for the 2016, Spring and Fall 2018 datasets, versus (a) $E_{\gamma}$ and (b) -$t$. The 2017 results before (blue full circles) and after correction (open circles) cross sections are shown.}
\end{figure}
\end{center}

\begin{center}
\begin{table}[h]
    \caption{$\phi \pi^{+}\pi^{-}$ yields in MC ($N_{MC}$) and data ($N_{Data}$), efficiencies ($\varepsilon$) and cross sections ($\sigma$) in $E_{\gamma}$ for 2016 dataset.}
    \label{tab.y2175.xsec_ul.phi2pi.1.1}
    \begin{tabular}{|c|c|c|c|c|c|}
    \hline
    $E_{\gamma}$ (GeV) & $N_{MC}$ & $N_{Data}$ & $\varepsilon$ ($\%$) & $\sigma$ (nb) \\ 
    \hline
   6.50 - 7.01 & 1839 $\pm$ 51 & 588 $\pm$ 40 & 6.90 $\pm$ 0.19 & 65.63 $\pm$ 4.48 $\pm$ 2.14 \\ 
   7.01 - 7.52 & 3406 $\pm$ 69 & 590 $\pm$ 40 & 6.88 $\pm$ 0.14 & 60.48 $\pm$ 4.12 $\pm$ 4.34 \\ 
   7.52 - 8.03 & 5952 $\pm$ 98 & 1049 $\pm$ 54 & 7.31 $\pm$ 0.12 & 67.24 $\pm$ 3.45 $\pm$ 2.97 \\ 
   8.03 - 8.54 & 23186 $\pm$ 183 & 2229 $\pm$ 80 & 7.98 $\pm$ 0.06 & 55.12 $\pm$ 1.98 $\pm$ 1.72 \\ 
   8.54 - 9.05 & 47841 $\pm$ 247 & 3830 $\pm$ 100 & 8.17 $\pm$ 0.04 & 51.02 $\pm$ 1.34 $\pm$ 2.15 \\ 
   9.05 - 9.56 & 9415 $\pm$ 125 & 828 $\pm$ 51 & 8.03 $\pm$ 0.11 & 52.50 $\pm$ 3.23 $\pm$ 1.51 \\ 
   9.56 - 10.07 & 11060 $\pm$ 127 & 1067 $\pm$ 54 & 8.18 $\pm$ 0.09 & 51.33 $\pm$ 2.59 $\pm$ 1.35 \\ 
   10.07 - 10.58 & 12474 $\pm$ 132 & 1116 $\pm$ 55 & 7.97 $\pm$ 0.08 & 51.77 $\pm$ 2.54 $\pm$ 1.61 \\ 
   10.58 - 11.09 & 10958 $\pm$ 124 & 935 $\pm$ 50 & 7.96 $\pm$ 0.09 & 53.20 $\pm$ 2.84 $\pm$ 2.32 \\ 
   11.09 - 11.60 & 15368 $\pm$ 141 & 1007 $\pm$ 54 & 8.02 $\pm$ 0.07 & 41.92 $\pm$ 2.23 $\pm$ 1.65 \\ 
   \hline
\end{tabular}
\end{table}
\null
\vfill
\end{center}

\newpage
\begin{center}
\null
\vfill    
\begin{table}[h]
    \caption{$\phi \pi^{+}\pi^{-}$ yields in MC ($N_{MC}$) and data ($N_{Data}$), efficiencies ($\varepsilon$) and cross sections ($\sigma$) in -$t$ for 2016 dataset.}
    \label{tab.y2175.xsec_ul.phi2pi.1.2}
    \begin{tabular}{|c|c|c|c|c|c|}
    \hline
    -t $(GeV/c)^{2}$ & $N_{MC}$ & $N_{Data}$ & $\varepsilon$ ($\%$) & $\sigma$ (nb) \\ 
    \hline
   0.00 - 0.40 & 26291 $\pm$ 192 & 2593 $\pm$ 96 & 5.57 $\pm$ 0.04 & 14.77 $\pm$ 0.55 $\pm$ 0.53 \\ 
   0.40 - 0.80 & 40108 $\pm$ 239 & 4758 $\pm$ 110 & 8.88 $\pm$ 0.05 & 17.00 $\pm$ 0.39 $\pm$ 0.41 \\ 
   0.80 - 1.20 & 28242 $\pm$ 197 & 3255 $\pm$ 87 & 9.23 $\pm$ 0.06 & 11.19 $\pm$ 0.30 $\pm$ 0.44 \\ 
   1.20 - 1.60 & 17977 $\pm$ 159 & 2186 $\pm$ 70 & 9.04 $\pm$ 0.08 & 7.67 $\pm$ 0.25 $\pm$ 0.19 \\ 
   1.60 - 2.00 & 11562 $\pm$ 126 & 1320 $\pm$ 55 & 9.06 $\pm$ 0.10 & 4.62 $\pm$ 0.19 $\pm$ 0.17 \\ 
   2.00 - 2.40 & 6855 $\pm$ 97 & 751 $\pm$ 43 & 8.51 $\pm$ 0.12 & 2.80 $\pm$ 0.16 $\pm$ 0.14 \\ 
   2.40 - 2.80 & 4248 $\pm$ 74 & 468 $\pm$ 35 & 8.34 $\pm$ 0.15 & 1.78 $\pm$ 0.13 $\pm$ 0.11 \\ 
   2.80 - 3.20 & 2510 $\pm$ 59 & 312 $\pm$ 30 & 7.93 $\pm$ 0.19 & 1.25 $\pm$ 0.12 $\pm$ 0.06 \\ 
   3.20 - 3.60 & 1623 $\pm$ 46 & 159 $\pm$ 22 & 8.09 $\pm$ 0.23 & 0.63 $\pm$ 0.09 $\pm$ 0.06 \\ 
   3.60 - 4.00 & 891 $\pm$ 34 & 128 $\pm$ 21 & 7.29 $\pm$ 0.28 & 0.56 $\pm$ 0.09 $\pm$ 0.09 \\ 
   \hline
\end{tabular}
\end{table}
\end{center}
   
\begin{center}
\begin{table}[H]
    \caption{$\phi \pi^{+}\pi^{-}$ yields in MC ($N_{MC}$) and data ($N_{Data}$), efficiencies ($\varepsilon$) and cross sections ($\sigma$) in $E_{\gamma}$ for 2017 dataset.}
    \label{tab.y2175.xsec_ul.phi2pi.2.1}
    \begin{tabular}{|c|c|c|c|c|c|}
    \hline
    $E_{\gamma}$ (GeV) & $N_{MC}$ & $N_{Data}$ & $\varepsilon$ ($\%$) & $\sigma$ (nb) \\ 
    \hline
    6.50 - 7.01 & 7469 $\pm$ 101 & 4791 $\pm$ 112 & 2.90 $\pm$ 0.04 & 96.14 $\pm$ 2.25 $\pm$ 2.59 \\ 
    7.01 - 7.52 & 9877 $\pm$ 121 & 5505 $\pm$ 124 & 3.24 $\pm$ 0.04 & 89.69 $\pm$ 2.02 $\pm$ 2.14 \\ 
    7.52 - 8.03 & 34681 $\pm$ 239 & 15112 $\pm$ 210 & 3.49 $\pm$ 0.02 & 83.52 $\pm$ 1.16 $\pm$ 2.68 \\ 
    8.03 - 8.54 & 71072 $\pm$ 322 & 24142 $\pm$ 261 & 3.80 $\pm$ 0.02 & 79.73 $\pm$ 0.86 $\pm$ 2.67 \\ 
    8.54 - 9.05 & 65893 $\pm$ 301 & 19915 $\pm$ 237 & 3.93 $\pm$ 0.02 & 76.39 $\pm$ 0.91 $\pm$ 1.98 \\ 
    9.05 - 9.56 & 24954 $\pm$ 194 & 8237 $\pm$ 154 & 4.02 $\pm$ 0.03 & 78.13 $\pm$ 1.46 $\pm$ 2.56 \\ 
    9.56 - 10.07 & 33567 $\pm$ 217 & 9743 $\pm$ 167 & 4.08 $\pm$ 0.03 & 70.16 $\pm$ 1.20 $\pm$ 2.11 \\ 
    10.07 - 10.58 & 32601 $\pm$ 215 & 8299 $\pm$ 157 & 4.19 $\pm$ 0.03 & 69.73 $\pm$ 1.32 $\pm$ 2.28 \\ 
    10.58 - 11.09 & 40464 $\pm$ 236 & 9631 $\pm$ 167 & 4.28 $\pm$ 0.02 & 65.15 $\pm$ 1.13 $\pm$ 1.88 \\ 
    11.09 - 11.60 & 18492 $\pm$ 154 & 3101 $\pm$ 95 & 4.40 $\pm$ 0.04 & 59.28 $\pm$ 1.82 $\pm$ 1.66 \\ 
   \hline
\end{tabular}
\end{table}
\null
\vfill
\end{center}

\begin{center}
\null
\vfill
\begin{table}[H]
    \caption{$\phi \pi^{+}\pi^{-}$ yields in MC ($N_{MC}$) and data ($N_{Data}$), efficiencies ($\varepsilon$) and cross sections ($\sigma$) in -$t$ for 2017 dataset.}
    \label{tab.y2175.xsec_ul.phi2pi.2.2}
    \begin{tabular}{|c|c|c|c|c|c|}
    \hline
    -t $(GeV/c)^{2}$ & $N_{MC}$ & $N_{Data}$ & $\varepsilon$ ($\%$) & $\sigma$ (nb) \\ 
    \hline
    0.00 - 0.40 & 81182 $\pm$ 351 & 19995 $\pm$ 273 & 3.63 $\pm$ 0.02 & 14.94 $\pm$ 0.20 $\pm$ 0.38 \\ 
    0.40 - 0.80 & 109014 $\pm$ 393 & 35122 $\pm$ 309 & 4.95 $\pm$ 0.02 & 19.22 $\pm$ 0.17 $\pm$ 0.46 \\ 
    0.80 - 1.20 & 65924 $\pm$ 304 & 24860 $\pm$ 245 & 4.33 $\pm$ 0.02 & 15.56 $\pm$ 0.15 $\pm$ 0.52 \\ 
    1.20 - 1.60 & 36291 $\pm$ 229 & 15695 $\pm$ 192 & 3.60 $\pm$ 0.02 & 11.79 $\pm$ 0.14 $\pm$ 0.50 \\ 
    1.60 - 2.00 & 21377 $\pm$ 174 & 9238 $\pm$ 151 & 3.26 $\pm$ 0.03 & 7.67 $\pm$ 0.13 $\pm$ 0.36 \\ 
    2.00 - 2.40 & 12172 $\pm$ 132 & 5702 $\pm$ 122 & 2.87 $\pm$ 0.03 & 5.38 $\pm$ 0.11 $\pm$ 0.30 \\ 
    2.40 - 2.80 & 6790 $\pm$ 99 & 3543 $\pm$ 98 & 2.51 $\pm$ 0.04 & 3.82 $\pm$ 0.11 $\pm$ 0.16 \\ 
    2.80 - 3.20 & 3953 $\pm$ 76 & 2330 $\pm$ 82 & 2.28 $\pm$ 0.04 & 2.76 $\pm$ 0.10 $\pm$ 0.18 \\ 
    3.20 - 3.60 & 2304 $\pm$ 56 & 1476 $\pm$ 67 & 2.10 $\pm$ 0.05 & 1.91 $\pm$ 0.09 $\pm$ 0.15 \\ 
    3.60 - 4.00 & 1347 $\pm$ 44 & 1063 $\pm$ 58 & 1.91 $\pm$ 0.06 & 1.51 $\pm$ 0.08 $\pm$ 0.10 \\    
   \hline
\end{tabular}
\end{table}
\end{center}
   
\begin{center}
\begin{table}[H]
    \caption{$\phi \pi^{+}\pi^{-}$ yields in MC ($N_{MC}$) and data ($N_{Data}$), efficiencies ($\varepsilon$) and cross sections ($\sigma$) in $E_{\gamma}$ for Spring 2018 dataset.}
    \label{tab.y2175.xsec_ul.phi2pi.3.1}
    \begin{tabular}{|c|c|c|c|c|c|}
    \hline
    $E_{\gamma}$ (GeV) & $N_{MC}$ & $N_{Data}$ & $\varepsilon$ ($\%$) & $\sigma$ (nb) \\ 
    \hline
    6.50 - 7.01 & 5512 $\pm$ 89 & 8203 $\pm$ 160 & 2.28 $\pm$ 0.04 & 72.20 $\pm$ 1.41 $\pm$ 1.69 \\ 
    7.01 - 7.52 & 7441 $\pm$ 105 & 9221 $\pm$ 173 & 2.58 $\pm$ 0.04 & 65.15 $\pm$ 1.22 $\pm$ 1.25 \\ 
    7.52 - 8.03 & 26267 $\pm$ 208 & 26104 $\pm$ 292 & 2.74 $\pm$ 0.02 & 64.71 $\pm$ 0.72 $\pm$ 1.26 \\ 
    8.03 - 8.54 & 56031 $\pm$ 281 & 38176 $\pm$ 347 & 3.05 $\pm$ 0.02 & 57.99 $\pm$ 0.53 $\pm$ 1.48 \\ 
    8.54 - 9.05 & 56863 $\pm$ 283 & 35720 $\pm$ 333 & 3.25 $\pm$ 0.02 & 55.50 $\pm$ 0.52 $\pm$ 1.47 \\ 
    9.05 - 9.56 & 18752 $\pm$ 173 & 13843 $\pm$ 215 & 3.33 $\pm$ 0.03 & 54.50 $\pm$ 0.85 $\pm$ 1.65 \\ 
    9.56 - 10.07 & 26379 $\pm$ 199 & 17460 $\pm$ 235 & 3.47 $\pm$ 0.03 & 50.93 $\pm$ 0.69 $\pm$ 1.58 \\ 
    10.07 - 10.58 & 25829 $\pm$ 198 & 14090 $\pm$ 216 & 3.59 $\pm$ 0.03 & 47.40 $\pm$ 0.73 $\pm$ 1.91 \\ 
    10.58 - 11.09 & 33036 $\pm$ 213 & 17359 $\pm$ 234 & 3.78 $\pm$ 0.02 & 44.68 $\pm$ 0.60 $\pm$ 2.07 \\ 
    11.09 - 11.60 & 14387 $\pm$ 137 & 5906 $\pm$ 135 & 3.87 $\pm$ 0.04 & 43.80 $\pm$ 1.00 $\pm$ 1.78 \\
   \hline
\end{tabular}
\end{table}
\null
\vfill
\end{center}

\begin{center}
    \null
    \vfill
\begin{table}[H]
    \caption{$\phi \pi^{+}\pi^{-}$ yields in MC ($N_{MC}$) and data ($N_{Data}$), efficiencies ($\varepsilon$) and cross sections ($\sigma$) in -$t$ for Spring 2018 dataset.}
    \label{tab.y2175.xsec_ul.phi2pi.3.2}
    \begin{tabular}{|c|c|c|c|c|c|}
    \hline
    -t $(GeV/c)^{2}$ & $N_{MC}$ & $N_{Data}$ & $\varepsilon$ ($\%$) & $\sigma$ (nb) \\ 
    \hline
    0.00 - 0.40 & 75845 $\pm$ 334 & 44569 $\pm$ 424 & 3.24 $\pm$ 0.01 & 12.97 $\pm$ 0.12 $\pm$ 0.50 \\ 
    0.40 - 0.80 & 86033 $\pm$ 350 & 60630 $\pm$ 418 & 3.87 $\pm$ 0.02 & 14.78 $\pm$ 0.10 $\pm$ 0.43 \\ 
    0.80 - 1.20 & 49379 $\pm$ 263 & 37272 $\pm$ 311 & 3.35 $\pm$ 0.02 & 10.49 $\pm$ 0.09 $\pm$ 0.25 \\ 
    1.20 - 1.60 & 27246 $\pm$ 197 & 22620 $\pm$ 241 & 2.92 $\pm$ 0.02 & 7.29 $\pm$ 0.08 $\pm$ 0.16 \\ 
    1.60 - 2.00 & 15196 $\pm$ 147 & 13324 $\pm$ 188 & 2.62 $\pm$ 0.03 & 4.78 $\pm$ 0.07 $\pm$ 0.12 \\ 
    2.00 - 2.40 & 8339 $\pm$ 108 & 8164 $\pm$ 151 & 2.35 $\pm$ 0.03 & 3.27 $\pm$ 0.06 $\pm$ 0.08 \\ 
    2.40 - 2.80 & 4656 $\pm$ 81 & 4771 $\pm$ 123 & 2.14 $\pm$ 0.04 & 2.11 $\pm$ 0.05 $\pm$ 0.06 \\ 
    2.80 - 3.20 & 2453 $\pm$ 58 & 3156 $\pm$ 103 & 1.85 $\pm$ 0.04 & 1.61 $\pm$ 0.05 $\pm$ 0.06 \\ 
    3.20 - 3.60 & 1322 $\pm$ 42 & 1918 $\pm$ 86 & 1.65 $\pm$ 0.05 & 1.10 $\pm$ 0.05 $\pm$ 0.04 \\ 
    3.60 - 4.00 & 719 $\pm$ 31 & 1179 $\pm$ 76 & 1.49 $\pm$ 0.06 & 0.75 $\pm$ 0.05 $\pm$ 0.05 \\ 
   \hline
\end{tabular}
\end{table}
\end{center}
 
\begin{center}
\begin{table}[H]
    \centering
    % \small
    \caption{$\phi \pi^{+}\pi^{-}$ yields in MC ($N_{MC}$) and data ($N_{Data}$), efficiencies ($\varepsilon$) and cross sections ($\sigma$) in $E_{\gamma}$ for Fall 2018 dataset.}
    \label{tab.y2175.xsec_ul.phi2pi.4.1}
    \begin{tabular}{|c|c|c|c|c|}
    \hline
    $E_{\gamma}$ (GeV) & $N_{MC}$ & $N_{Data}$ & $\varepsilon$ ($\%$) & $\sigma$ (nb) \\ 
    \hline
    6.50 - 7.01 & 14531 $\pm$ 144 & 12668 $\pm$ 189 & 4.91 $\pm$ 0.05 & 70.88 $\pm$ 1.06 $\pm$ 1.99 \\ 
    7.01 - 7.52 & 16340 $\pm$ 157 & 13202 $\pm$ 194 & 5.33 $\pm$ 0.05 & 70.10 $\pm$ 1.03 $\pm$ 1.57 \\ 
    7.52 - 8.03 & 52390 $\pm$ 282 & 36354 $\pm$ 329 & 5.70 $\pm$ 0.03 & 67.51 $\pm$ 0.61 $\pm$ 1.49 \\ 
    8.03 - 8.54 & 102113 $\pm$ 384 & 57786 $\pm$ 412 & 6.10 $\pm$ 0.02 & 64.29 $\pm$ 0.46 $\pm$ 1.22 \\ 
    8.54 - 9.05 & 101185 $\pm$ 369 & 48509 $\pm$ 369 & 6.28 $\pm$ 0.02 & 63.72 $\pm$ 0.48 $\pm$ 1.06 \\ 
    9.05 - 9.56 & 35097 $\pm$ 232 & 19732 $\pm$ 239 & 6.57 $\pm$ 0.04 & 62.69 $\pm$ 0.76 $\pm$ 1.19 \\ 
    9.56 - 10.07 & 48747 $\pm$ 261 & 25116 $\pm$ 263 & 6.66 $\pm$ 0.04 & 59.46 $\pm$ 0.62 $\pm$ 1.18 \\ 
    10.07 - 10.58 & 45465 $\pm$ 252 & 20185 $\pm$ 239 & 6.84 $\pm$ 0.04 & 57.91 $\pm$ 0.69 $\pm$ 1.24 \\ 
    10.58 - 11.09 & 55880 $\pm$ 272 & 24692 $\pm$ 258 & 6.97 $\pm$ 0.03 & 54.52 $\pm$ 0.57 $\pm$ 1.15 \\ 
    11.09 - 11.60 & 27397 $\pm$ 186 & 8996 $\pm$ 155 & 7.24 $\pm$ 0.05 & 51.79 $\pm$ 0.89 $\pm$ 1.41 \\ 
   \hline
\end{tabular}
\end{table}
\null
\vfill
\end{center}

\begin{center}
\null
\vfill
\begin{table}[H]
    \centering
    \caption{$\phi \pi^{+}\pi^{-}$ yields in MC ($N_{MC}$) and data ($N_{Data}$), efficiencies ($\varepsilon$) and cross sections ($\sigma$) in -$t$ for Fall 2018 dataset.}
    \label{tab.y2175.xsec_ul.phi2pi.4.2}
    \begin{tabular}{|c|c|c|c|c|c|}
    \hline
    -t $(GeV/c)^{2}$ & $N_{MC}$ & $N_{Data}$ & $\varepsilon$ ($\%$) & $\sigma$ (nb) \\ 
    \hline
    0.00 - 0.40 & 115135 $\pm$ 411 & 51942 $\pm$ 430 & 5.42 $\pm$ 0.02 & 13.96 $\pm$ 0.12 $\pm$ 0.36 \\ 
    0.40 - 0.80 & 149523 $\pm$ 460 & 80765 $\pm$ 463 & 7.12 $\pm$ 0.02 & 16.52 $\pm$ 0.09 $\pm$ 0.23 \\ 
    0.80 - 1.20 & 94271 $\pm$ 368 & 55187 $\pm$ 363 & 6.97 $\pm$ 0.03 & 11.54 $\pm$ 0.08 $\pm$ 0.21 \\ 
    1.20 - 1.60 & 56537 $\pm$ 279 & 34861 $\pm$ 288 & 6.49 $\pm$ 0.03 & 7.83 $\pm$ 0.06 $\pm$ 0.16 \\ 
    1.60 - 2.00 & 34580 $\pm$ 220 & 21177 $\pm$ 228 & 6.18 $\pm$ 0.04 & 5.00 $\pm$ 0.05 $\pm$ 0.11 \\ 
    2.00 - 2.40 & 21262 $\pm$ 171 & 12579 $\pm$ 182 & 5.90 $\pm$ 0.05 & 3.11 $\pm$ 0.04 $\pm$ 0.09 \\ 
    2.40 - 2.80 & 13154 $\pm$ 135 & 7804 $\pm$ 148 & 5.64 $\pm$ 0.06 & 2.02 $\pm$ 0.04 $\pm$ 0.06 \\ 
    2.80 - 3.20 & 7799 $\pm$ 104 & 5224 $\pm$ 125 & 5.19 $\pm$ 0.07 & 1.47 $\pm$ 0.04 $\pm$ 0.06 \\ 
    3.20 - 3.60 & 4662 $\pm$ 81 & 3324 $\pm$ 106 & 4.78 $\pm$ 0.08 & 1.01 $\pm$ 0.03 $\pm$ 0.05 \\ 
    3.60 - 4.00 & 2800 $\pm$ 62 & 2383 $\pm$ 92 & 4.50 $\pm$ 0.10 & 0.77 $\pm$ 0.03 $\pm$ 0.04 \\  
   \hline
\end{tabular}
\end{table}
\null
\vfill
\end{center}

\begin{comment}
~\par Similar method is used to subtract the non-$\phi$(1020) background both in MC sample Fig.~\ref{fig.4.5.1.2} and real data Fig.~\ref{fig.4.5.1.3}. The $\phi$(1020) signal shape is extracted from MC sample and used in the real data, for every beam energy bin. This method is also applied on all different data sets.

\begin{figure}[H]
    \centering
    \includegraphics[width=1.0\textwidth]{plots/c17_phie1_mc.eps}
    \caption{\label{fig.4.5.1.2}Beam energy-mass dependent invariant mass in simulation.}
\end{figure}

\begin{figure}[H]
    \centering
    \includegraphics[width=1.0\textwidth]{plots/c17_phie1.eps}
    \caption{\label{fig.4.5.1.3}$K^+ K^-$ beam energy-mass dependent invariant mass in data.}
\end{figure}

After extracting the number of $\phi \pi^+ \pi^-$ signal events ($N_{\phi}^{MC}$) in simulation for every beam energy, we compute the reconstruction efficiency ($\epsilon$) by Eq.~\ref{eq.4.5.1.1}

\begin{equation}
    \label{eq.4.5.1.1}
    \begin{aligned}
        \epsilon=\frac{N_{\phi}^{MC}}{N_{\phi}^{Generated}}~,
    \end{aligned}
\end{equation}

where $N_{\phi}^{Generated}$ is the number of $\phi \pi^+ \pi^-$ signal events generated in beam energy.
\par The reconstruction efficiency for the different data sets are shown in Fig.~\ref{fig.4.5.1.4.a}. An average of 1.5$\%$ is seen in the 2017 and 2018 data, and a higher efficiency is observed in the 2016 data.
The next step is to measure the total cross section in beam energy bins using Eq.~\ref{eq.4.5.1.2}.

\begin{equation}
    \label{eq.4.5.1.2}
    \begin{aligned}
        \sigma = \frac{N_{\phi}^{Data}}{\epsilon~\mathcal{L}~BR(\phi\rightarrow K^{+}K^{-})}~,
    \end{aligned}
\end{equation}

where $N_{\phi(Data)}$ is the number of $\phi \pi^+ \pi^-$ signal events in real data extracted from Fig.~\ref{fig.4.5.1.3}. $BR(\phi\rightarrow K^{+}K^{-})$ = 0.492 is the branching ratio of the $\phi\rightarrow K^{+}K^{-}$ channel, the value used is from PDG data. $\mathcal{L}$ is the integrated luminosity, defined in Eq.~\ref{eq.4.5.1.3}

\begin{equation}
    \label{eq.4.5.1.3}
    \begin{aligned}
        \mathcal{L}=I \times T,~~~ I=\int dN_{\gamma}^{Tagged}/dE,~~~ T = 1.273~b^{-1}~,
    \end{aligned}
\end{equation}

were $I$ is the integrated flux, extracted from the yield of the tagged photons, and $T$ is the target thickness.
\par The total cross section for different data sets is shown in Fig.\ref{fig.4.5.1.4.b}

\begin{figure}[H]
    \centering
    \begin{subfigure}[b]{0.45\textwidth}
        \includegraphics[width=\textwidth]{plots/cmgeeff.eps}
        \caption{}
        \label{fig.4.5.1.4.a}
    \end{subfigure}
    \begin{subfigure}[b]{0.45\textwidth}
        \includegraphics[width=\textwidth]{plots/cmgexsec.eps}
        \caption{}
        \label{fig.4.5.1.4.b}
    \end{subfigure}
    \caption{Beam energy-dependent reconstruction efficiency (a) and total cross section (b).}
    \label{fig:4.5.1.4}
\end{figure}

We observe a similar cross sections in 2016 and 2017 data within errors, and a discrepancy with the 2018 data, due to a gas supply issue in the CDC during 2018 data taking.  
\par Another variable of interest is the momentum transfer ($t$) defined in Eq.~\ref{eq.4.5.1.4}.

\begin{equation}
    \label{eq.4.5.1.4}
    \begin{aligned}
        t = (P_{p} - P_{p'})^2~, 
    \end{aligned}
\end{equation}

where $P_{p}$ and $P_{p'}$ are the four-momenta of the target and recoiled proton respectively. The distribution of the momentum transfer is shown in Fig.~\ref{fig.4.5.1.5}.

\begin{figure}[H]
    \centering
    \includegraphics[width=0.5\textwidth]{plots/cpippimkpkm_17_t_kin.eps}
    \caption{\label{fig.4.5.1.5}Momentum transfer distribution in data.}
\end{figure}

Dividing the momentum transfer distribution into 10 equidistant bins, and using a similar method to extract $\phi \pi^+ \pi^-$ signal both in MC and data. We then compute the reconstruction efficiency and the total cross section for different data sets, shown in Fig.~\ref{fig.4.5.1.6}. A stronger dependence of both the efficiency and cross section of the momentum transfer is seen. The cross section difference between the different data sets is similar to the case of the beam energy dependent ones. 

\begin{figure}[H]
    \centering
    \begin{subfigure}[b]{0.45\textwidth}
        \includegraphics[width=\textwidth]{plots/cmgteff.eps}
        \caption{}
        \label{fig.4.5.1.6.a}
    \end{subfigure}
    \begin{subfigure}[b]{0.45\textwidth}
        \includegraphics[width=\textwidth]{plots/cmgtxsec.eps}
        \caption{}
        \label{fig.4.5.1.6.b}
    \end{subfigure}
    \caption{Proton momentum transfer-dependent reconstruction efficiency (a) and total cross section (b).}
    \label{fig.4.5.1.6}
\end{figure}
\end{comment}

\newpage
\subsection{Upper Limit for \texorpdfstring{$\bm{\gamma p \rightarrow Y(2175) p \rightarrow \phi \pi^{+} \pi^{-} p}$}{}}
\label{chap.y2175.xsec_ul.yphi2pi}

In the following, we will study the resonant mode of the previous reaction, with the $Y(2175) \rightarrow \phi \pi^{+} \pi^{-}$ being produced as an intermediate resonance. First, we select $\phi \pi^+ \pi^-$ signal, by subtracting non-$\phi$(1020) background. To achieve this, we study the invariant mass correlation between $K^{+}K^{-}$ and $K^{+}K^{-} \pi^+ \pi^-$, seen in Fig.~\ref{fig.y2175.xsec_ul.yphi2pi.1}. Next to the clear horizontal band of the $\phi(1020)$, we see another diagonal band of correlated events. The latter is investigated by performing a one dimensional projection of 50 intervals of $K^{+}K^{-} \pi^+ \pi^-$ on the $K^{+}K^{-}$ invariant mass. The yields of $\phi \pi^+ \pi^-$ are then extracted after fitting the signal and background shapes as shown in Fig.~\ref{fig.y2175.xsec_ul.yphi2pi.2}. The resulting $K^{+}K^{-} \pi^+ \pi^-$ invariant mass after background subtraction (Fig.~\ref{fig.y2175.xsec_ul.yphi2pi.3}) shows no enhancement around 2175 GeV/c$^2$, leading to the non observation of $Y(2175)$ resonance in the $\phi \pi^+ \pi^-$ channel. 
~\par In the absence of a signal, limits can be set on the the $\gamma p \rightarrow Y(2175) p \rightarrow \phi \pi^{+} \pi^{-} p$ production cross section. We define an upper limit at 90$\%$ Confidence Level (CL) using the maximum likelihood method. For $n$ independent measurements of the cross section $\sigma_{i}$, following a probability density function $f(\sigma_{i};\sigma)$ with the cross section ($\sigma$) as a parameter, the likelihood function is obtained from the probability of the data under assumption of the parameters defined as

\begin{equation}
    \label{eq.y2175.xsec_ul.yphi2pi.1}
    \begin{aligned}
        \mathcal{L(\sigma)} = \prod_{n}^{i=1} f(\sigma_{i};\sigma)~.\\
    \end{aligned}
\end{equation}

\noindent The maximum likelihood estimator for $\sigma$ is defined as the values that give the maximum of $\mathcal{L(\sigma)}$. The upper limit ($U\kern-0.14em L_{90}$) is the cross section at 90$\%$ of the profile likelihood distribution, with

\begin{equation}
    \label{eq.y2175.xsec_ul.yphi2pi.2}
    \begin{aligned}
        \int_{0}^{UL_{90}} \mathcal{L}(\sigma)~d\sigma = 0.9 \\
    \end{aligned}
\end{equation}

\noindent In this case, the Bayesian approach is used, for which a prior knowledge on the signal cross section is expressed in the sense that a probability for a negative cross section is negligible for real physics processes.
~\par After fixing the $Y(2175)$ signal shape from MC simulation as shown in Fig.~\ref{fig.y2175.xsec_ul.yphi2pi.4}, we use the same fit parameters to fit the data, see Fig.~\ref{fig.y2175.xsec_ul.yphi2pi.5}. The negative yield extracted from these plots are due to the dips around the Y(2175) mass, which may be due to simple statistical fluctuations or even destructive interference between the resonances participating in this process. We perform multiple fits, varying the signal amplitude parameter around the nominal value by five times the statistical uncertainty on the yield, and extract the profile likelihoods for each variation, as displayed in Fig~\ref{fig.y2175.xsec_ul.yphi2pi.6}. In this case, the Likelihood profiles essentially follow Gaussian distributions, so that the mean can be taken as the nominal cross section that maximizes the likelihood, and the standard deviation as the error on the cross section measurement. To take into account the effect of the cross section systematic errors on the upper limit determination, discussed in Sec.~\ref{chap.y2175.syserr}, we convolute the obtained likelihood profile distribution with a gaussian function of the same mean and the standard deviation corresponding to the systematic uncertainty. The resulting convolution (Fig~\ref{fig.y2175.xsec_ul.yphi2pi.7}) is a gaussian with the nominal cross section given by the mean and the total error (quadratic sum of the statistical and systematic error) given by the standard deviation. The CL corresponds to 90th percentile of the convoluted distribution above zero. The constructed Bayesian Confidence interval should represent a $90\%$ probability to cover the true value of the cross section. This technique is applied for the different datasets, and the results are summarized in the Tab.~\ref{tab.y2175.xsec_ul.yphi2pi.1}.
% this means that the $Y(2175)$ has only $10\%$ probability of being observed, and therefore we are 90$\%$ confident that the true measured cross section is lower than this upper limit and and any measurement outside this 90$\%$ interval is excluded.

\begin{figure}[H]
    \centering
    \begin{subfigure}[b]{0.49\textwidth}
        \includegraphics[width=\textwidth]{plots/c_phi_y_16.eps}
        \caption{}
        \label{fig.y2175.xsec_ul.yphi2pi.1.a}
    \end{subfigure}
    \begin{subfigure}[b]{0.49\textwidth}
        \includegraphics[width=\textwidth]{plots/c_phi_y_17.eps}
        \caption{}
        \label{fig.y2175.xsec_ul.yphi2pi.1.b}
    \end{subfigure}
    \begin{subfigure}[b]{0.49\textwidth}
        \includegraphics[width=\textwidth]{plots/c_phi_y_18.eps}
        \caption{}
        \label{fig.y2175.xsec_ul.yphi2pi.1.c}
    \end{subfigure}
    \begin{subfigure}[b]{0.49\textwidth}
        \includegraphics[width=\textwidth]{plots/c_phi_y_18l.eps}
        \caption{}
        \label{fig.y2175.xsec_ul.yphi2pi.1.d}
    \end{subfigure}
    \caption{$K^{+}K^{-}$ versus $K^{+}K^{-} \pi^+ \pi^-$ invariant mass for (a) 2016, (b) 2017, (c) Spring 2018 and (d) Fall 2018 datasets.}
    \label{fig.y2175.xsec_ul.yphi2pi.1}
\end{figure}

\begin{figure}[H]
    \centering
    \includegraphics[width=0.6\textwidth]{plots/c1_phi_y_16.eps}
    \caption{\label{fig.y2175.xsec_ul.yphi2pi.2}Invariant mass of $K^{+}K^{-}$ of one projection of $K^{+}K^{-} \pi^+ \pi^-$ invariant mass. The total fit (red) is composed of signal shape (blue) described by Eq.~\ref{eq.y2175.evt_sel.kin_fit.3} and background (dashed) by polynomial of $4^{th}$ degree.}
\end{figure}

\begin{figure}[H]
    \centering
    \begin{subfigure}[b]{0.49\textwidth}
        \includegraphics[width=\textwidth]{plots/c_gphi_y_16.eps}
        \caption{}
        \label{fig.y2175.xsec_ul.yphi2pi.3.a}
    \end{subfigure}
    \begin{subfigure}[b]{0.49\textwidth}
        \includegraphics[width=\textwidth]{plots/c_gphi_y_17.eps}
        \caption{}
        \label{fig.y2175.xsec_ul.yphi2pi.3.b}
    \end{subfigure}
    \begin{subfigure}[b]{0.49\textwidth}
        \includegraphics[width=\textwidth]{plots/c_gphi_y_18.eps}
        \caption{}
        \label{fig.y2175.xsec_ul.yphi2pi.3.c}
    \end{subfigure}
    \begin{subfigure}[b]{0.49\textwidth}
        \includegraphics[width=\textwidth]{plots/c_gphi_y_18l.eps}
        \caption{}
        \label{fig.y2175.xsec_ul.yphi2pi.3.d}
    \end{subfigure}
    \caption{The yields $\phi \pi^+ \pi^-$ versus $K^{+}K^{-} \pi^+ \pi^-$ invariant mass for (a) 2016, (b) 2017, (c) Spring 2018 and (d) Fall 2018 datasets. No observation of the $Y(2175)$ in all the four datasets.}
    \label{fig.y2175.xsec_ul.yphi2pi.3}
\end{figure}

\begin{figure}[H]
    \centering
    \includegraphics[width=0.6\textwidth]{plots/cmc_YMass_postcuts_fitted.eps}
    \caption{\label{fig.y2175.xsec_ul.yphi2pi.4}Invariant mass of $\phi \pi^+ \pi^-$ in MC simulation. The total fit (red) is composed of signal shape (blue) described by Eq.~\ref{eq.y2175.evt_sel.kin_fit.3} and background (dashed) by polynomial of $4^{th}$ degree.}
\end{figure}

\begin{center}
\null
\vfill
\begin{figure}[htbp]
    \centering
    \begin{subfigure}[b]{0.49\textwidth}
        \includegraphics[width=\textwidth]{plots/c_gphiy_16.eps}
        \caption{}
        \label{fig.y2175.xsec_ul.yphi2pi.5.a}
    \end{subfigure}
    \begin{subfigure}[b]{0.49\textwidth}
        \includegraphics[width=\textwidth]{plots/c_gphiy_17.eps}
        \caption{}
        \label{fig.y2175.xsec_ul.yphi2pi.5.b}
    \end{subfigure}
    \begin{subfigure}[b]{0.49\textwidth}
        \includegraphics[width=\textwidth]{plots/c_gphiy_18.eps}
        \caption{}
        \label{fig.y2175.xsec_ul.yphi2pi.5.c}
    \end{subfigure}
    \begin{subfigure}[b]{0.49\textwidth}
        \includegraphics[width=\textwidth]{plots/c_gphiy_18l.eps}
        \caption{}
        \label{fig.y2175.xsec_ul.yphi2pi.5.d}
    \end{subfigure}
    \caption{The yields of $\phi \pi^+ \pi^-$ versus $K^{+}K^{-} \pi^+ \pi^-$ invariant mass for (a) 2016, (b) 2017, (c) Spring 2018 and (d) Fall 2018 datasets. The fit models and parameters are obtained from Fig.~\ref{fig.y2175.xsec_ul.yphi2pi.4}.}
    \label{fig.y2175.xsec_ul.yphi2pi.5}
\end{figure}
\null
\vfill
\end{center}

\begin{center}
\null
\vfill
\begin{figure}[htbp]
    \centering
    \begin{subfigure}[b]{0.49\textwidth}
        \includegraphics[width=\textwidth]{plots/c16_profxsec.eps}
        \caption{}
        \label{fig.y2175.xsec_ul.yphi2pi.6.a}
    \end{subfigure}
    \begin{subfigure}[b]{0.49\textwidth}
        \includegraphics[width=\textwidth]{plots/c17_profxsec.eps}
        \caption{}
        \label{fig.y2175.xsec_ul.yphi2pi.6.b}
    \end{subfigure}
    \begin{subfigure}[b]{0.49\textwidth}
        \includegraphics[width=\textwidth]{plots/c18_profxsec.eps}
        \caption{}
        \label{fig.y2175.xsec_ul.yphi2pi.6.c}
    \end{subfigure}
    \begin{subfigure}[b]{0.49\textwidth}
        \includegraphics[width=\textwidth]{plots/c18l_profxsec.eps}
        \caption{}
        \label{fig.y2175.xsec_ul.yphi2pi.6.d}
    \end{subfigure}
    \caption{Profile likelihood versus total cross section for (a) 2016, (b) 2017, (c) Spring 2018 and (d) Fall 2018 datasets.}
    \label{fig.y2175.xsec_ul.yphi2pi.6}
\end{figure}
\null
\vfill
\end{center}

\begin{figure}[H]
    \centering
    \begin{subfigure}[b]{0.49\textwidth}
        \includegraphics[width=\textwidth]{plots/c16_twogaus_conv.eps}
        \caption{}
        \label{fig.y2175.xsec_ul.yphi2pi.7.a}
    \end{subfigure}
    \begin{subfigure}[b]{0.49\textwidth}
        \includegraphics[width=\textwidth]{plots/c17_twogaus_conv.eps}
        \caption{}
        \label{fig.y2175.xsec_ul.yphi2pi.7.b}
    \end{subfigure}
    \begin{subfigure}[b]{0.49\textwidth}
        \includegraphics[width=\textwidth]{plots/c18_twogaus_conv.eps}
        \caption{}
        \label{fig.y2175.xsec_ul.yphi2pi.7.c}
    \end{subfigure}
    \begin{subfigure}[b]{0.49\textwidth}
        \includegraphics[width=\textwidth]{plots/c18l_twogaus_conv.eps}
        \caption{}
        \label{fig.y2175.xsec_ul.yphi2pi.7.d}
    \end{subfigure}
    \caption{Convoluted profile likelihood and a gaussian with the nominal cross section as mean and total errors as standard deviation versus total cross section for (a) 2016, (b) 2017, (c) Spring 2018 and (d) Fall 2018 datasets. The vertical blue lines are indicating the cross section upper limit at 90$\%$ CL.}
    \label{fig.y2175.xsec_ul.yphi2pi.7}
\end{figure}

\begin{table}[!htbp]
    \small
    \centering
    \caption{Total cross sections and upper limits for $\gamma p \rightarrow Y(2175) p \rightarrow \phi \pi^{+} \pi^{-} p$.}
    \label{tab.y2175.xsec_ul.yphi2pi.1}
    \begin{tabular}{|c|c|c|c|c|}
        \hline
        Dataset & $N_{measured}$ & $\varepsilon$ ($\%$) & \thead{$\sigma$ (nb) x\\BR[$Y(2175) \rightarrow \phi \pi^{+} \pi^{-}$]} & \thead{Upper Limit\\$90\%$ CL (nb)} \\
        \hline
        2016 & 77 $\pm$ 100 & 11.89 $\pm$ 0.034 & 0.21 $\pm$ 0.27 $\pm$ 0.18 & 0.68 \\
        2017 & -999 $\pm$ 333 & 5.33 $\pm$ 0.013 & -0.51 $\pm$ 0.17 $\pm$ 0.21 & 0.24 \\
        Spring 2018 & -606 $\pm$ 436 & 4.72 $\pm$ 0.012 & -0.12 $\pm$ 0.09 $\pm$ 0.14 & 0.21 \\
        Fall 2018 & -1056 $\pm$ 459 & 8.24 $\pm$ 0.014 & -0.19 $\pm$ 0.08 $\pm$ 0.26 & 0.35 \\
        \hline
    \end{tabular}
\end{table}

\newpage
\subsection{Cross Section for \texorpdfstring{$\bm{\gamma p \rightarrow \phi f_0 p}$}{}}
\label{chap.y2175.xsec_ul.phifo}

This section summarizes the study of the non-resonant ($i.e.$ without the $Y(2175)$) production of the $\phi f_0 p$ final state. The $\phi(1020)\pi^{+}\pi^{-}$ signal yields are extracted by fitting the $K^{+}K^{-}$ invariant-mass projections in each 0.018 GeV/c$^{2}$ interval of $\pi^{+}\pi^{-}$ invariant mass. The invariant mass correlation between the $K^{+}K^{-}$ and the $\pi^{+}\pi^{-}$ pairs is shown in Fig.~\ref{fig.y2175.xsec_ul.phifo.1}. The $f_{0}(980)$ signal shape is well described by the Breit-Wigner model in MC simulation (Fig.~\ref{fig.y2175.xsec_ul.phifo.2}), and the signal yield obtained is used for the reconstruction efficiency calculation.
~\par An observation of the $f_{0}(980)$ resonance is clearly seen in the data (Fig.~\ref{fig.y2175.xsec_ul.phifo.3}), with the parameters consistent with the PDG data values for this meson. Furthermore, an enhancement near the $\rho(770)$ and $K_{s}^{0}$ mesons are seen near the nominal masses. The $K_{s}^{0}$ is produced in a displaced vertex, leading to yield reductions due to the primary vertex constraint in the kinematic fitting procedure. The cross section of the $\gamma p \rightarrow \phi f_0 p$ is calculated using Eq.~\ref{eq.y2175.xsec_ul.phi2pi}, and the results are summarized in Tab.~\ref{tab.y2175.xsec_ul.phifo}. The cross sections for all the datasets are consistent within errors.

\begin{table*}[!b]
    \centering
    \caption{A summary of the total cross section and efficiency for $\gamma p \rightarrow \phi f_0 p$. The statistical and systematics errors are displayed for the cross section. The systematic uncertainties will be discussed in Sec.~\ref{chap.y2175.syserr}}
    \label{tab.y2175.xsec_ul.phifo}
    \begin{tabular}{|c|c|c|c|}
        \hline
        Dataset & $N_{measured}$ & $\varepsilon$ ($\%$) & $\sigma$ $\times$ BR[$f_0(980) \rightarrow \pi^{+} \pi^{-}$] (nb) \\
        \hline
        2016 & 596 $\pm$ 125 & 0.72 $\pm$ 0.01 & 26.18 $\pm$ 5.51 $\pm$ 3.22 \\
        2017 & 2188 $\pm$ 695 & 0.26 $\pm$ 0.01 & 22.67 $\pm$ 7.20 $\pm$ 3.18 \\
        Spring 2018 & 5023 $\pm$ 1384 & 0.23 $\pm$ 0.02 & 20.61 $\pm$ 5.68 $\pm$ 1.05 \\
        Fall 2018 & 4400 $\pm$ 859 & 0.48 $\pm$ 0.01 & 12.51 $\pm$ 2.44 $\pm$ 1.44 \\
        \hline
    \end{tabular}
\end{table*}

\begin{figure}[H]
    \centering
    \begin{subfigure}[b]{0.49\textwidth}
        \includegraphics[width=\textwidth]{plots/c_phifo_16.eps}
        \caption{}
        \label{fig.y2175.xsec_ul.phifo.1.a}
    \end{subfigure}
    \begin{subfigure}[b]{0.49\textwidth}
        \includegraphics[width=\textwidth]{plots/c_phifo_17.eps}
        \caption{}
        \label{fig.y2175.xsec_ul.phifo.1.b}
    \end{subfigure}
    \begin{subfigure}[b]{0.49\textwidth}
        \includegraphics[width=\textwidth]{plots/c_phifo_18.eps}
        \caption{}
        \label{fig.y2175.xsec_ul.phifo.1.c}
    \end{subfigure}
    \begin{subfigure}[b]{0.49\textwidth}
        \includegraphics[width=\textwidth]{plots/c_phifo_18l.eps}
        \caption{}
        \label{fig.y2175.xsec_ul.phifo.1.d}
    \end{subfigure}
    \caption{$K^{+}K^{-}$ versus $\pi^+ \pi^-$ invariant mass for (a) 2016, (b) 2017, (c) Spring 2018 and (d) Fall 2018 datasets. The horizontal band corresponds to the $\phi(1020)$ and the vertical two bands to the $\rho(770)$ and $K_{s}^{0}$.}
    \label{fig.y2175.xsec_ul.phifo.1}
\end{figure}

\begin{figure}[H]
    \centering
    \includegraphics[width=0.6\textwidth]{plots/cmc_foMass_postcuts_fitted_16.eps}
    \caption{\label{fig.y2175.xsec_ul.phifo.2}Invariant mass of $\pi^+ \pi^-$ in MC simulation. The total fit (red) is composed of signal shape (blue) described by a Breit-Wigner and a background (dashed) by polynomial of first degree.}
\end{figure}

\begin{center}
\null
\vfill
\begin{figure}[htbp]
    \centering
    \begin{subfigure}[b]{0.49\textwidth}
        \includegraphics[width=\textwidth]{plots/c_gphifo_16.eps}
        \caption{}
        \label{fig.y2175.xsec_ul.phifo.3.a}
    \end{subfigure}
    \begin{subfigure}[b]{0.49\textwidth}
        \includegraphics[width=\textwidth]{plots/c_gphifo_17.eps}
        \caption{}
        \label{fig.y2175.xsec_ul.phifo.3.b}
    \end{subfigure}
    \begin{subfigure}[b]{0.49\textwidth}
        \includegraphics[width=\textwidth]{plots/c_gphifo_18.eps}
        \caption{}
        \label{fig.y2175.xsec_ul.phifo.3.c}
    \end{subfigure}
    \begin{subfigure}[b]{0.49\textwidth}
        \includegraphics[width=\textwidth]{plots/c_gphifo_18l.eps}
        \caption{}
        \label{fig.y2175.xsec_ul.phifo.3.d}
    \end{subfigure}
    \caption{The yields of $\phi \pi^+ \pi^-$ versus $\pi^+ \pi^-$ invariant mass for (a) 2016, (b) 2017, (c) Spring 2018 and (d) Fall 2018 datasets. The total fit (blue) is composed of the signal (red) described by a Breit-Wigner and the background (dashed) described by a second order polynomial.}
    \label{fig.y2175.xsec_ul.phifo.3}
\end{figure}
\null
\vfill
\end{center}

\newpage
\subsection{Upper Limit for \texorpdfstring{$\bm{\gamma p \rightarrow Y(2175) p \rightarrow \phi f_0 p}$}{}}
\label{chap.y2175.xsec_ul.yphifo}

Following a similar procedure as in Sec.~\ref{chap.y2175.xsec_ul.yphi2pi} to reduce the non-$\phi$(1020) background, we obtain the $\phi f_{0}$ yields by fitting the $K^{+}K^{-}$ invariant-mass projections in each 0.02 GeV/c$^{2}$ interval of $K^{+}K^{-}\pi^{+}\pi^{-}$ invariant mass, while requiring the di-pion mass pair within 2.5 times the PDG average mass error of $f_0(980)$. The resulting $K^{+}K^{-}\pi^{+}\pi^{-}$ invariant-mass distribution for $\phi f_{0}$ candidate events is shown in Fig.~\ref{fig.y2175.xsec_ul.yphifo.1}. To estimate the $Y(2175)$ contribution within multiple fluctuations in the invariant mass distribution, the likelihood ratio test~\cite{Cow11} is used to determine the significance of the $Y(2175)$ signal. The binned maximum likelihood fits are used for this test.
~\par We define the null hypothesis $H_{0}$ as the condition, in which only the background is observed in the data, and the alternative hypotheses $H_{1}$, in which both signal and background are modeled in data. According to Wilks theorem~\cite{Wil01}, the significance ($Z$) adopted as the test statistics is asymptotically distributed according to the $\chi^{2}$ function, with degrees of freedom equal to the difference between the number of fit parameters. The goal of the profile likelihood ratio in this study is to quantify degree of compatibility (or not) of the data with the hypothesis of the $Y(2175)$ signal being present, which would lead to an observation ($Z \geq 5\sigma$), evidence ($3\sigma < Z < 5\sigma$) or none of both ($Z < 3\sigma$). In the case of one degree of freedom difference between the two hypothesis, the significance is defined as

\begin{equation}
    \label{eq.y2175.xsec_ul.yphifo}
    \begin{aligned}
        Z = \sqrt{-2~ln\left(\frac{\mathcal{L}(H_{0})}{\mathcal{L}(H_{1})}\right)}~,
    \end{aligned}
\end{equation}

After describing the $Y(2175)$ signal shape in MC simulation, the same parameters are fixed in the data and the signal amplitude is set as a free parameter, once with the signal and background fit to obtain the likelihood $\mathcal{L}(H_{1})$, and the second with only the background to obtain $\mathcal{L}(H_{0})$. The significance is then calculated using Eq.~\ref{eq.y2175.xsec_ul.yphifo}; it is displayed next to the fit parameterization in Fig.~\ref{fig.y2175.xsec_ul.yphifo.1}. To estimate the goodness of the fit model to the data, we use the pull distribution, which is defined as the difference between the data and the fit values divided by the data errors. The pull histogram is distributed as a standard Gaussian with a mean of zero and a unit width. If the mean is not centered at zero than there is a bias in the fit model (Fig.~\ref{fig.y2175.xsec_ul.yphifo.1}). An enhancement around 2.191 GeV/c$^2$ is observed, with mean and width consistent with the PDG data for the $Y(2175)$. This resonance is seen in both, the largest data samples of Spring and Fall 2018 datasets, with a significance above 3$\sigma$. Despite the enhancement around the mass of interest, we could not claim an observation of the $Y(2175)$ due to systematic and statistical limitations, and to the strong bias in the fits especially at the region of interest. For this reason, we set a $90\%$ CL upper limit on the $\gamma p \rightarrow Y(2175) p \rightarrow \phi f_0 p$ cross section, using the method in Sec.~\ref{chap.y2175.xsec_ul.yphi2pi}. The resulting profile likelihoods indicating the upper limits are shown in Fig.~\ref{fig.y2175.xsec_ul.yphifo.2}. The summary of the efficiency, cross section and the upper limit values are listed in Tab.\ref{tab.y2175.xsec_ul.yphifo}.

\begin{center}
    \null
    \vfill
    \begin{table}[htbp]
        \centering
        \caption{Summary of efficiency, cross section, and upper limit for different datasets.}
        \label{tab.y2175.xsec_ul.yphifo}
        \begin{tabular}{|c|c|c|c|c|}
            \hline
            Data set & $N_{measured}$ & $\varepsilon$ [$\%$] & $\sigma \times BR_{f_{0}\rightarrow\pi^{+}\pi^{-}}$ $\times BR_{Y\rightarrow \phi f_0}$ [nb] & \thead{Upper Limit\\$90\%$ CL (nb)}\\
            \hline
            2016 & 8 $\pm$ 40 & 12.70 & 0.02 $\pm$ 0.10 $\pm$ 0.16 & 0.33 \\
            2017 & 112 $\pm$ 116 & 5.25 & 0.06 $\pm$ 0.06 $\pm$ 0.26 & 0.48 \\
            Spring 2018 & 773 $\pm$ 156 & 4.90 & 0.15 $\pm$ 0.03 $\pm$ 0.19 & 0.43 \\
            Fall 2018 & 573 $\pm$ 175 & 9.23 & 0.09 $\pm$ 0.03 $\pm$ 0.20 & 0.39 \\
            \hline
        \end{tabular}
    \end{table}
    \null
    \vfill
\end{center}

\begin{center}
\null
\vfill
\begin{figure}[H]
    \centering
    \begin{subfigure}[b]{0.49\textwidth}
        \includegraphics[width=\textwidth]{plots/chgphiy_16.eps}
        \caption{}
        \label{fig.y2175.xsec_ul.yphifo.1.a}
    \end{subfigure}
    \begin{subfigure}[b]{0.49\textwidth}
        \includegraphics[width=\textwidth]{plots/chgphiy_17.eps}
        \caption{}
        \label{fig.y2175.xsec_ul.yphifo.1.b}
    \end{subfigure}
    \begin{subfigure}[b]{0.49\textwidth}
        \includegraphics[width=\textwidth]{plots/chgphiy_18.eps}
        \caption{}
        \label{fig.y2175.xsec_ul.yphifo.1.c}
    \end{subfigure}
    \begin{subfigure}[b]{0.49\textwidth}
        \includegraphics[width=\textwidth]{plots/chgphiy_18l.eps}
        \caption{}
        \label{fig.y2175.xsec_ul.yphifo.1.d}
    \end{subfigure}
    \caption{The invariant mass distribution for $\phi f_0$ candidates for (a) 2016, (b) 2017, (c) Spring 2018 and (d) Fall 2018 datasets. The total fit (blue) is composed of the signal (red) described by a Voigtian and the background (dashed) with a third degree Chebyshev polynomial. The total fit (magenta) is performed with only the background. A pull distribution is shown in the bottom of each plot.}
    \label{fig.y2175.xsec_ul.yphifo.1}
\end{figure}
\null
\vfill
\end{center}

\begin{center}
\null
\vfill
\begin{figure}[H]
    \centering
    \begin{subfigure}[b]{0.49\textwidth}
        \includegraphics[width=\textwidth]{plots/c16_twogaus_conv_yphifo.eps}
        \caption{}
        \label{fig.y2175.xsec_ul.yphifo.2.a}
    \end{subfigure}
    \begin{subfigure}[b]{0.49\textwidth}
        \includegraphics[width=\textwidth]{plots/c17_twogaus_conv_yphifo.eps}
        \caption{}
        \label{fig.y2175.xsec_ul.yphifo.2.b}
    \end{subfigure}
    \begin{subfigure}[b]{0.49\textwidth}
        \includegraphics[width=\textwidth]{plots/c18_twogaus_conv_yphifo.eps}
        \caption{}
        \label{fig.y2175.xsec_ul.yphifo.2.c}
    \end{subfigure}
    \begin{subfigure}[b]{0.49\textwidth}
        \includegraphics[width=\textwidth]{plots/c18l_twogaus_conv_yphifo.eps}
        \caption{}
        \label{fig.y2175.xsec_ul.yphifo.2.d}
    \end{subfigure}
    \caption{Convoluted profile likelihood and a gaussian with the nominal cross section as mean and total errors as standard deviation versus total cross section for (a) 2016, (b) 2017, (c) Spring 2018 and (d) Fall 2018 datasets. The vertical blue lines are indicating the cross section upper limit at 90$\%$ CL.}
    \label{fig.y2175.xsec_ul.yphifo.2}
\end{figure}
\null
\vfill
\end{center}

\section{Systematic Uncertainties}
\label{chap.y2175.syserr}

In order to determine the systematic errors, multiple variations in the analysis chain are tested, resulting in a different cross section measurements around the nominal value. The relative amount of deviation from the nominal cross section measurement are identified using the sample standard deviation defined as

\begin{equation}
    \label{eq.y2175.syserr.4.5}
    \begin{aligned}
        \delta_{i} = \frac{1}{\sigma_{nom}} \sqrt{\frac{\sum\limits_{i=1}^{N} (\sigma_{i} - \sigma_{mean})^2}{N-1}}~,\\
    \end{aligned}
\end{equation}

\noindent where $N$ is the total number of variations of the measured cross section ($\sigma_{i}$), with respect to the mean value ($\sigma_{mean}$), and $\sigma_{nom}$ is the nominal cross section. The main sources of systematic errors associated with the cross section measurements in Sec.~\ref{chap.y2175.xsec_ul}, and their estimation are discussed in this section.

\subsection{Signal width and Mean}
\label{chap.y2175.syserr.sig}

Alterations in the signal resonance shape could be introduced by the detector resolution and calibration effects. To account for this effect, variations in the signal shape are allowed around the nominal parameters. The $Y(2175)$ signal mean and width are allowed to vary around their nominal value by the PDG average errors, $\pm 0.01$ GeV/c$^2$ and $\pm 0.012$ GeV/c$^2$, respectively. The relative errors for the mean and width uncertainties are estimated and summarized in Tab.~\ref{tab.y2175.syserr.y}.

\subsection{Background Polynomial Order}
\label{chap.y2175.syserr.bkg}

The background model described by the Chebyshev polynomial of degree $n$ as defined in Eq.~\ref{eq.y2175.evt_sel.kin_fit.4}, is varied around the nominal degree by $n$-1 and $n$+1 in order to estimate the uncertainty due to background parameterization. Similar variations are also allowed for the other background polynomials. The cross section measured for every background order is then used as input to Eq.~\ref{eq.y2175.syserr.4.5}, and the resulting relative errors are listed in Tab.~\ref{tab.y2175.syserr.phi2pi.1.1} - \ref{tab.y2175.syserr.yphifo.1.1}.

\subsection{Fitting region}
\label{chap.y2175.syserr.range}

To study the impact of the fit window on the cross section measurement, the $\phi(1020)$, $f_{0}(980)$, and the $Y(2175)$  resonances fit regions are varied around their nominal range of [0.99, 1.2 GeV/c$^2$], [0.83, 1.14 GeV/c$^2$], and [2, 3 GeV/c$^2$], respectively. The $\phi(1020)$ fit range was varied to [0.99, 1.15 GeV/c$^2$] and [0.99, 1.25 GeV/c$^2$], the $f_{0}(980)$ was varied to [0.83, 1.12 GeV/c$^2$] and [0.83, 1.16 GeV/c$^2$], and finally the $Y(2175)$ fit range to [2.0, 2.9 GeV/c$^2$] and [2.0, 3.1 GeV/c$^2$]. The cross section is measured for each range, and the estimated relative errors are summarized in Tab.~\ref{tab.y2175.syserr.phi2pi.1.1} - \ref{tab.y2175.syserr.yphifo.1.1}.

\subsection{Finite binning}
\label{chap.y2175.syserr.bin}

To study the impact of number of data point on the quality of the $\phi(1020)$ model fit, the number of bins in the $K^+K^-$ invariant mass are varied from the nominal value of 100, to 90 and 110 bins. The effect of these modifications on the nominal cross section is then estimated by Eq.~\ref{eq.y2175.syserr.4.5}, and summarized in Tab.~\ref{tab.y2175.syserr.phi2pi.1.1} - \ref{tab.y2175.syserr.yphifo.1.1}.

\subsection{Event Selection Variation}
\label{chap.y2175.syserr.evt}

The variables with stronger effect on the event selection are varied around their nominal cut, to estimate the errors on the final measured cross section.
\par The accidental subtraction using four out-of-time beam bunches on each sides of the prompt beam bunch for the nominal cut is varied to two and one beam bunches. The symmetric timing cut used to select protons in the TOF detector is also varied from $\pm 0.3$ ns to $\pm 0.2$ and $\pm 0.4$ ns. The $\chi^2$ of the kinematic fit as well was varied from $\chi^{2}<55$ to $\chi^{2}<45$ and $\chi^{2}<65$. Finally, the missing mass squared symmetric cut was varied from $\pm 0.035$ to $\pm 0.025$ and $\pm 0.045$. The cross section is measured after every variation and the relative errors for each source is estimated and summarized in Tab.~\ref{tab.y2175.syserr.phi2pi.1.1} - \ref{tab.y2175.syserr.yphifo.1.1}.
~\par Finally, the above potential systematic errors, treated independently from each other, are added in quadrature to calculate the total systematic errors. The latter is quoted in the cross section measurements in Sec.~\ref{chap.y2175.xsec_ul}. The individual systematic uncertainties for the non-resonant $\phi \pi^{+} \pi^{-}$ and $\phi f_0$ final states are reasonably small, and the total systematic uncertainties are comparable and in some cases smaller than the statistical ones. We conclude that the cross section measurements for the non-resonant channels are statistically limited and a collection of a larger data sample in the future will improve the measurements precision. However, the statistic and systematic uncertainties in the resonant modes $Y(2175) \rightarrow \phi \pi^{+} \pi^{-}$ and $Y(2175) \rightarrow \phi f_0$ are large, due to the multiple fluctuations around the $Y(2175)$ and a bigger sensitivity to the small parameter variations. Therefore, these measurements are considered statistically and systematically limited, and a carefully detailed analysis of a larger data sample is required to improve theses measurements.

\newpage
\begin{center}
\null
\vfill
\begin{table}[!htbp]
    \small
    \centering
    \caption{Summary of systematic uncertainties for the $\gamma p \rightarrow \phi \pi^{+} \pi^{-} p$ cross section measurements for the 2016 dataset in $E_{\gamma}$.}
    \label{tab.y2175.syserr.phi2pi.1.1}
    \begin{tabular}{|c|c|c|c|c|c|c|c|}
        \hline
        $E_{\gamma}$ (GeV) & \thead{Bkg deg\\(\%)} & \thead{Fit range\\(\%)} & \thead{binning\\(\%)} & \thead{Accidental\\Subtraction\\(\%)} & \thead{Timing\\Cut\\(\%)} & \thead{Kinematic\\Fit $\chi^{2}$(\%)} & \thead{$MM^{2}$\\(\%)} \\
        \hline
        6.50 - 7.01 & 0.81 & 0.58 & 0.58 & 1.17 & 1.46 & 2.03 & 0.80 \\ 
        7.01 - 7.52 & 2.52 & 2.09 & 0.92 & 1.13 & 1.47 & 5.91 & 0.73 \\ 
        7.52 - 8.03 & 1.87 & 0.80 & 0.03 & 1.58 & 1.88 & 2.83 & 0.50 \\ 
        8.03 - 8.54 & 1.92 & 1.18 & 0.33 & 1.09 & 0.56 & 1.12 & 0.89 \\ 
        8.54 - 9.05 & 0.18 & 1.28 & 0.20 & 0.45 & 1.16 & 3.64 & 0.43 \\ 
        9.05 - 9.56 & 1.93 & 0.57 & 0.88 & 0.60 & 0.21 & 0.37 & 1.38 \\ 
        9.56 - 10.07 & 0.23 & 1.72 & 0.23 & 0.67 & 1.28 & 0.79 & 0.29 \\ 
        10.07 - 10.58 & 2.40 & 1.42 & 0.43 & 0.19 & 0.30 & 0.24 & 0.70 \\ 
        10.58 - 11.09 & 2.41 & 1.61 & 0.66 & 1.12 & 0.24 & 2.65 & 0.88 \\ 
        11.09 - 11.60 & 0.30 & 3.08 & 0.74 & 0.43 & 0.24 & 2.04 & 0.09 \\
        \hline
    \end{tabular}
\end{table}
\end{center}

\begin{center}
\begin{table}[!htbp]
    \small
    \centering
    \caption{Summary of systematic uncertainties for the $\gamma p \rightarrow \phi \pi^{+} \pi^{-} p$ cross section measurements for the 2016 dataset in -$t$.}
    \label{tab.y2175.syserr.phi2pi.1.2}
    \begin{tabular}{|c|c|c|c|c|c|c|c|}
        \hline
        -$t$ $(GeV/c)^{2}$&\thead{Bkg deg\\(\%)}&\thead{Fit range\\(\%)}&\thead{binning\\(\%)}&\thead{Accidental\\Subtraction\\(\%)}&\thead{Timing\\Cut\\(\%)}&\thead{Kinematic\\Fit $\chi^{2}$(\%)}&\thead{$MM^{2}$\\(\%)}\\
        \hline
        0.00 - 0.40 & 0.18 & 2.05 & 0.26 & 1.52 & 0.16 & 2.24 & 0.62 \\ 
        0.40 - 0.80 & 0.14 & 1.10 & 0.08 & 0.51 & 0.61 & 1.71 & 0.23 \\ 
        0.80 - 1.20 & 0.23 & 1.48 & 0.10 & 0.16 & 1.42 & 3.20 & 0.43 \\ 
        1.20 - 1.60 & 0.18 & 1.66 & 0.30 & 0.52 & 1.02 & 0.89 & 0.49 \\ 
        1.60 - 2.00 & 1.16 & 1.06 & 0.10 & 0.36 & 0.83 & 2.96 & 0.56 \\ 
        2.00 - 2.40 & 2.21 & 1.42 & 0.52 & 0.60 & 2.18 & 3.58 & 0.24 \\ 
        2.40 - 2.80 & 2.74 & 1.10 & 0.75 & 0.52 & 1.74 & 4.73 & 1.43 \\ 
        2.80 - 3.20 & 1.85 & 1.06 & 0.54 & 1.92 & 2.09 & 3.15 & 0.97 \\ 
        3.20 - 3.60 & 5.88 & 3.33 & 3.56 & 2.37 & 3.33 & 3.46 & 1.91 \\ 
        3.60 - 4.00 & 6.12 & 5.96 & 2.68 & 1.82 & 5.03 & 12.39 & 2.07 \\
        \hline
    \end{tabular}
\end{table}
\null
\vfill
\end{center}

\begin{center}
\null
\vfill
\begin{table}[!htbp]
    \small
    \centering
    \caption{Summary of systematic uncertainties for the $\gamma p \rightarrow \phi \pi^{+} \pi^{-} p$ cross section measurements for the 2017 dataset in $E_{\gamma}$.}
    \label{tab.y2175.syserr.phi2pi.2.1}
    \begin{tabular}{|c|c|c|c|c|c|c|c|}
        \hline
        $E_{\gamma}$ (GeV)&\thead{Bkg deg\\(\%)}&\thead{Fit range\\(\%)}&\thead{binning\\(\%)}&\thead{Accidental\\Subtraction\\(\%)}&\thead{Timing\\Cut\\(\%)}&\thead{Kinematic\\Fit $\chi^{2}$(\%)}&\thead{$MM^{2}$\\(\%)}\\
        \hline
        6.50 - 7.01 & 0.17 & 0.83 & 0.38 & 0.80 & 1.33 & 1.69 & 0.30 \\ 
        7.01 - 7.52 & 0.13 & 0.96 & 0.02 & 0.28 & 1.26 & 1.41 & 0.22 \\ 
        7.52 - 8.03 & 0.25 & 1.31 & 0.17 & 0.15 & 1.20 & 2.43 & 0.33 \\ 
        8.03 - 8.54 & 0.11 & 1.13 & 0.06 & 0.52 & 1.25 & 2.62 & 0.44 \\ 
        8.54 - 9.05 & 0.09 & 0.98 & 0.25 & 0.43 & 1.13 & 1.79 & 0.17 \\ 
        9.05 - 9.56 & 0.32 & 1.23 & 0.08 & 0.55 & 0.85 & 2.63 & 0.40 \\ 
        9.56 - 10.07 & 0.14 & 1.78 & 0.10 & 0.52 & 0.83 & 1.84 & 0.70 \\ 
        10.07 - 10.58 & 0.09 & 1.65 & 0.03 & 0.18 & 0.92 & 2.40 & 0.59 \\ 
        10.58 - 11.09 & 0.35 & 2.25 & 0.32 & 0.55 & 0.92 & 0.79 & 0.45 \\ 
        11.09 - 11.60 & 0.12 & 1.99 & 0.39 & 0.57 & 1.38 & 0.62 & 0.21 \\
        \hline
    \end{tabular}
\end{table}
\end{center}

\begin{center}
\begin{table}[!htbp]
    \small
    \centering
    \caption{Summary of systematic uncertainties for the $\gamma p \rightarrow \phi \pi^{+} \pi^{-} p$ cross section measurements for the 2017 dataset in -$t$.}
    \label{tab.y2175.syserr.phi2pi.2.2}
    \begin{tabular}{|c|c|c|c|c|c|c|c|}
        \hline
        -$t$ $(GeV/c)^{2}$&\thead{Bkg deg\\(\%)}&\thead{Fit range\\(\%)}&\thead{binning\\(\%)}&\thead{Accidental\\Subtraction\\(\%)}&\thead{Timing\\Cut\\(\%)}&\thead{Kinematic\\Fit $\chi^{2}$(\%)}&\thead{$MM^{2}$\\(\%)}\\
        \hline
        0.00 - 0.40 & 0.26 & 2.06 & 0.18 & 0.10 & 0.23 & 0.76 & 0.71 \\ 
        0.40 - 0.80 & 0.11 & 1.03 & 0.02 & 0.30 & 0.96 & 1.59 & 0.04 \\ 
        0.80 - 1.20 & 0.09 & 1.12 & 0.10 & 0.12 & 1.25 & 2.70 & 0.09 \\ 
        1.20 - 1.60 & 0.23 & 1.12 & 0.27 & 0.26 & 1.05 & 3.76 & 0.28 \\ 
        1.60 - 2.00 & 0.31 & 1.66 & 0.10 & 0.22 & 1.65 & 3.96 & 0.46 \\ 
        2.00 - 2.40 & 0.33 & 0.82 & 0.25 & 0.45 & 2.12 & 4.84 & 0.32 \\ 
        2.40 - 2.80 & 0.81 & 1.02 & 0.32 & 0.08 & 2.75 & 2.28 & 1.22 \\ 
        2.80 - 3.20 & 1.77 & 2.12 & 0.49 & 0.87 & 2.05 & 5.01 & 1.54 \\ 
        3.20 - 3.60 & 1.11 & 0.89 & 0.37 & 0.79 & 4.85 & 5.73 & 1.20 \\ 
        3.60 - 4.00 & 1.90 & 1.20 & 0.71 & 1.30 & 2.62 & 4.71 & 2.32 \\
        \hline
    \end{tabular}
\end{table}
\null
\vfill
\end{center}

\begin{center}
\null
\vfill
\begin{table}[!htbp]
    \small
    \centering
    \caption{Summary of systematic uncertainties for the $\gamma p \rightarrow \phi \pi^{+} \pi^{-} p$ cross section measurements for the Spring 2018 dataset in $E_{\gamma}$.}
    \label{tab.y2175.syserr.phi2pi.3.1}
    \begin{tabular}{|c|c|c|c|c|c|c|c|}
        \hline
        $E_{\gamma}$ (GeV)&\thead{Bkg deg\\(\%)}&\thead{Fit range\\(\%)}&\thead{binning\\(\%)}&\thead{Accidental\\Subtraction\\(\%)}&\thead{Timing\\Cut\\(\%)}&\thead{Kinematic\\Fit $\chi^{2}$(\%)}&\thead{$MM^{2}$\\(\%)}\\
        \hline
        6.50 - 7.01 & 0.19 & 0.61 & 0.28 & 0.14 & 0.92 & 1.76 & 0.14 \\ 
        7.01 - 7.52 & 0.17 & 0.81 & 0.34 & 0.57 & 0.76 & 0.95 & 0.22 \\ 
        7.52 - 8.03 & 0.18 & 0.75 & 0.08 & 0.24 & 0.88 & 1.10 & 0.32 \\ 
        8.03 - 8.54 & 0.31 & 0.97 & 0.20 & 0.52 & 0.81 & 1.85 & 0.16 \\ 
        8.54 - 9.05 & 0.22 & 0.53 & 0.18 & 0.60 & 0.76 & 2.16 & 0.17 \\ 
        9.05 - 9.56 & 0.19 & 0.73 & 0.09 & 0.10 & 0.78 & 2.62 & 0.10 \\ 
        9.56 - 10.07 & 0.37 & 1.36 & 0.09 & 0.10 & 0.66 & 2.43 & 0.47 \\ 
        10.07 - 10.58 & 0.37 & 1.48 & 0.11 & 0.54 & 0.70 & 3.48 & 0.21 \\ 
        10.58 - 11.09 & 0.30 & 1.49 & 0.12 & 0.06 & 0.58 & 4.21 & 0.04 \\ 
        11.09 - 11.60 & 0.47 & 2.30 & 0.16 & 0.28 & 0.59 & 3.09 & 0.21 \\
        \hline
    \end{tabular}
\end{table}
\end{center}

\begin{center}
\begin{table}[!htbp]
    \small
    \centering
    \caption{Summary of systematic uncertainties for the $\gamma p \rightarrow \phi \pi^{+} \pi^{-} p$ cross section measurements for the Spring 2018 dataset in -$t$.}
    \label{tab.y2175.syserr.phi2pi.3.2}
    \begin{tabular}{|c|c|c|c|c|c|c|c|}
        \hline
        -$t$ $(GeV/c)^{2}$&\thead{Bkg deg\\(\%)}&\thead{Fit range\\(\%)}&\thead{binning\\(\%)}&\thead{Accidental\\Subtraction\\(\%)}&\thead{Timing\\Cut\\(\%)}&\thead{Kinematic\\Fit $\chi^{2}$(\%)}&\thead{$MM^{2}$\\(\%)}\\
        \hline
        0.00 - 0.40 & 0.39 & 1.44 & 0.06 & 0.48 & 0.07 & 3.39 & 0.17 \\ 
        0.40 - 0.80 & 0.18 & 0.81 & 0.10 & 0.17 & 0.51 & 2.51 & 0.20 \\ 
        0.80 - 1.20 & 0.26 & 0.84 & 0.10 & 0.45 & 0.73 & 1.71 & 0.42 \\ 
        1.20 - 1.60 & 0.16 & 1.16 & 0.07 & 0.10 & 0.87 & 1.33 & 0.32 \\ 
        1.60 - 2.00 & 0.14 & 1.21 & 0.09 & 0.23 & 1.08 & 1.25 & 0.74 \\ 
        2.00 - 2.40 & 0.14 & 1.20 & 0.19 & 0.39 & 1.33 & 1.12 & 0.47 \\ 
        2.40 - 2.80 & 0.19 & 1.27 & 0.36 & 1.01 & 1.61 & 1.05 & 0.62 \\ 
        2.80 - 3.20 & 0.29 & 2.24 & 0.46 & 1.06 & 2.06 & 1.53 & 0.19 \\ 
        3.20 - 3.60 & 0.93 & 1.76 & 0.32 & 1.47 & 2.00 & 2.25 & 0.51 \\ 
        3.60 - 4.00 & 0.28 & 4.91 & 1.70 & 2.43 & 3.26 & 0.66 & 0.43 \\
        \hline
    \end{tabular}
\end{table}
\null
\vfill
\end{center}

\begin{center}
\null
\vfill
\begin{table}[!htbp]
    \small
    \centering
    \caption{Summary of systematic uncertainties for the $\gamma p \rightarrow \phi \pi^{+} \pi^{-} p$ cross section measurements for the Fall 2018 dataset in $E_{\gamma}$.}
    \label{tab.y2175.syserr.phi2pi.4.1}
    \begin{tabular}{|c|c|c|c|c|c|c|c|}
        \hline
        $E_{\gamma}$ (GeV)&\thead{Bkg deg\\(\%)}&\thead{Fit range\\(\%)}&\thead{binning\\(\%)}&\thead{Accidental\\Subtraction\\(\%)}&\thead{Timing\\Cut\\(\%)}&\thead{Kinematic\\Fit $\chi^{2}$(\%)}&\thead{$MM^{2}$\\(\%)}\\
        \hline
        6.50 - 7.01 & 0.14 & 0.64 & 0.01 & 0.31 & 1.54 & 1.97 & 0.26 \\ 
        7.01 - 7.52 & 0.11 & 0.50 & 0.26 & 0.37 & 1.15 & 1.48 & 0.08 \\ 
        7.52 - 8.03 & 0.18 & 0.42 & 0.16 & 0.31 & 1.07 & 1.52 & 0.16 \\ 
        8.03 - 8.54 & 0.19 & 0.50 & 0.13 & 0.65 & 1.09 & 0.78 & 0.18 \\ 
        8.54 - 9.05 & 0.17 & 0.59 & 0.04 & 0.51 & 0.84 & 0.63 & 0.06 \\ 
        9.05 - 9.56 & 0.13 & 1.33 & 0.16 & 0.18 & 0.80 & 0.26 & 0.16 \\ 
        9.56 - 10.07 & 0.16 & 1.30 & 0.11 & 0.40 & 0.93 & 0.29 & 0.20 \\ 
        10.07 - 10.58 & 0.20 & 1.49 & 0.14 & 0.18 & 0.98 & 0.23 & 0.46 \\ 
        10.58 - 11.09 & 0.27 & 1.52 & 0.06 & 0.07 & 0.61 & 0.74 & 0.24 \\ 
        11.09 - 11.60 & 0.23 & 2.25 & 0.25 & 0.33 & 0.71 & 0.37 & 0.66 \\
        \hline
    \end{tabular}
\end{table}
\end{center}

\begin{center}
\begin{table}[!htbp]
    \small
    \centering
    \caption{Summary of systematic uncertainties for the $\gamma p \rightarrow \phi \pi^{+} \pi^{-} p$ cross section measurements for the Fall 2018 dataset in -$t$.}
    \label{tab.y2175.syserr.phi2pi.4.2}
    \begin{tabular}{|c|c|c|c|c|c|c|c|}
        \hline
        -$t$ $(GeV/c)^{2}$&\thead{Bkg deg\\(\%)}&\thead{Fit range\\(\%)}&\thead{binning\\(\%)}&\thead{Accidental\\Subtraction\\(\%)}&\thead{Timing\\Cut\\(\%)}&\thead{Kinematic\\Fit $\chi^{2}$(\%)}&\thead{$MM^{2}$\\(\%)}\\
        \hline
        0.00 - 0.40 & 0.25 & 1.83 & 0.13 & 0.25 & 0.14 & 1.45 & 0.35 \\ 
        0.40 - 0.80 & 0.17 & 0.55 & 0.08 & 0.40 & 0.66 & 0.09 & 0.13 \\ 
        0.80 - 1.20 & 0.15 & 0.58 & 0.05 & 0.33 & 0.77 & 1.05 & 0.08 \\ 
        1.20 - 1.60 & 0.13 & 0.59 & 0.04 & 0.13 & 0.98 & 1.28 & 0.08 \\ 
        1.60 - 2.00 & 0.16 & 0.56 & 0.04 & 0.21 & 1.05 & 1.39 & 0.02 \\ 
        2.00 - 2.40 & 0.08 & 1.17 & 0.04 & 0.34 & 1.45 & 2.08 & 0.34 \\ 
        2.40 - 2.80 & 0.12 & 0.50 & 0.09 & 0.13 & 1.56 & 1.89 & 0.46 \\ 
        2.80 - 3.20 & 0.14 & 1.52 & 0.14 & 0.04 & 2.25 & 2.93 & 0.25 \\ 
        3.20 - 3.60 & 0.21 & 1.86 & 0.34 & 0.44 & 2.58 & 3.70 & 0.51 \\ 
        3.60 - 4.00 & 0.15 & 1.73 & 0.64 & 1.40 & 3.31 & 3.55 & 0.47 \\
        \hline
    \end{tabular}
\end{table}
\null
\vfill
\end{center}

\begin{table}[!htbp]
    \small
    \centering
    \caption{Summary of systematic uncertainties for the $\gamma p \rightarrow Y(2175) p \rightarrow \phi \pi^{+} \pi^{-} p$ cross section measurements.}
    \label{tab.y2175.syserr.yphi2pi.1.1}
    \begin{tabular}{|c|c|c|c|c|c|c|c|}
        \hline
        Dataset&\thead{Bkg deg\\(\%)}&\thead{Fit range\\(\%)}&\thead{binning\\(\%)}&\thead{Accidental\\Subtraction\\(\%)}&\thead{Timing\\Cut\\(\%)}&\thead{Kinematic\\Fit $\chi^{2}$(\%)}&\thead{$MM^{2}$\\(\%)}\\
        \hline
        2016 & 27.99 & 53.16 & 25.24 & 16.84 & 22.23 & 34.81 & 37.48 \\ 
        2017 & 25.38 & 21.20 & 9.89 & 17.07 & 3.28 & 4.55 & 5.21 \\ 
        2018 Spring & 47.24 & 31.86 & 49.63 & 7.03 & 18.17 & 30.22 & 5.86 \\ 
        2018 Fall & 53.79 & 71.75 & 5.67 & 7.51 & 26.80 & 55.23 & 82.87 \\
        \hline
    \end{tabular}
\end{table}

\begin{table}[!htbp]
    \small
    \centering
    \caption{Summary of systematic uncertainties for the $\gamma p \rightarrow \phi f_{0} p$ cross section measurements.}
    \label{tab.y2175.syserr.phifo.1.1}
    \begin{tabular}{|c|c|c|c|c|c|c|c|}
        \hline
        Dataset&\thead{Bkg deg\\(\%)}&\thead{Fit range\\(\%)}&\thead{binning\\(\%)}&\thead{Accidental\\Subtraction\\(\%)}&\thead{Timing\\Cut\\(\%)}&\thead{Kinematic\\Fit $\chi^{2}$(\%)}&\thead{$MM^{2}$\\(\%)}\\
        \hline
        2016 & 7.86 & 5.59 & 0.96 & 2.16 & 4.17 & 3.46 & 4.70 \\ 
        2017 & 10.26 & 4.76 & 1.17 & 1.54 & 2.07 & 7.35 & 2.43 \\ 
        2018 Spring & 2.27 & 3.68 & 0.87 & 1.82 & 0.88 & 0.64 & 0.86 \\ 
        2018 Fall & 9.47 & 5.62 & 0.16 & 1.39 & 2.15 & 1.74 & 0.38 \\
        \hline
    \end{tabular}
\end{table}

\begin{table}[!htbp]
    \small
    \centering
    \caption{Summary of systematic uncertainties for the $\gamma p \rightarrow Y(2175) p \rightarrow \phi f_{0} p$ cross section measurements.}
    \label{tab.y2175.syserr.yphifo.1.1}
    \begin{tabular}{|c|c|c|c|c|c|c|c|}
        \hline
        Dataset&\thead{Bkg deg\\(\%)}&\thead{Fit range\\(\%)}&\thead{binning\\(\%)}&\thead{Accidental\\Subtraction\\(\%)}&\thead{Timing\\Cut\\(\%)}&\thead{Kinematic\\Fit $\chi^{2}$(\%)}&\thead{$MM^{2}$\\(\%)}\\
        \hline
        2016 & 534.68 & 169.65 & 560.01 & 162.14 & 219.87 & 227.61 & 80.38 \\ 
        2017 & 441.04 & 73.17 & 27.75 & 24.77 & 6.85 & 44.97 & 8.95 \\ 
        2018 Spring & 127.22 & 13.37 & 12.03 & 12.18 & 1.64 & 12.83 & 5.35 \\ 
        2018 Fall & 215.43 & 14.29 & 8.32 & 20.35 & 8.79 & 0.57 & 17.55 \\
        \hline
    \end{tabular}
\end{table}

\begin{table}[!htbp]
    \small
    \centering
    \caption{Summary of systematic uncertainties on the cross section measurements due to resonance parameter variations.}
    \label{tab.y2175.syserr.y}
    \begin{tabular}{|c|c|c|c|c|}
        \hline
        \multirow{2}{*}{Dataset} & \multicolumn{2}{c|}{$Y(2175)\rightarrow \phi \pi^{+} \pi^{-}$} & \multicolumn{2}{c|}{$Y(2175)\rightarrow \phi f_0$} \\
        \cline{2-5}
        & \thead{$Y(2175)$ Mean\\($\%$)} & \thead{$Y(2175)$ Width\\($\%$)} & \thead{$Y(2175)$ Mean\\($\%$)} & \thead{$Y(2175)$ Width\\($\%$)} \\
        \hline
        2016 & 4.68 & 12.86 & 32.16 & 31.58 \\ 
        2017 & 10.52 & 11.15 & 59.48 & 18.11 \\ 
        2018 Spring & 75.21 & 21.91 & 8.73 & 16.11 \\ 
        2018 Fall & 21.57 & 10.70 & 39.57 & 20.88 \\
        \hline
    \end{tabular}
\end{table}

\section{Conclusion}
\label{chap.y2175.conc}

First preliminary measurements of the photoproduction cross section for exclusive $\phi(1020) \pi^+\pi^-$ and $\phi(1020) f_0(980)$ final states with the GlueX experiment for a photon beam energy range [6.5 - 11.6 GeV] are presented. The observed strong dependence of the $\phi \pi^+\pi^-$ cross section on the momentum transfer could be explained by the presence of intermediate sub-resonances, like the observed $\rho(770)$, or target fragmentation sources in the reaction, like $\Delta^{++} \rightarrow \pi^+ p$. These can lead to a different final state phase-space detector occupation, and given an asymmetric detector acceptance, this could be translated to different efficiencies and thus also cross section measurements.
~\par In the absence of the $Y(2175)$ in the $\phi(1020) \pi^+\pi^-$ and $\phi(1020) f_0(980)$ channels, an upper limit on the measured cross section has been established. We obtain an upper limit at $90\%$ CL of 0.67 nb, 0.24 nb, 0.20 nb and 0.35 nb for $Y(2175)\rightarrow \phi(1020) \pi^+\pi^-$, and 0.33 nb, 0.48 nb, 0.43 nb, and 0.39 nb for $Y(2175)\rightarrow \phi(1020) f_0(980$, for the 2016, 2017, and Spring and Fall 2018 datasets, respectively. The non-observation of the $Y(2175)$ may be an indication for presence of other sources of background, such as $e.g.$ $\Delta^{++}$ resonance in the reaction. The performed analysis is worth to be revisited with the improved PID capabilities and the higher statistics in GlueX Phase-II.
~\par Detailed studies of the branching ratios of the $Y(2175)$ into different final states may then indicate the nature of this resonance. For instance, if the $\phi(1020) f_0(980)$ decay mode is the dominant one, then the tetraquark picture is favored, with the $Y(2175)$ as an $ss\overline{ss}$, $s\bar{s}s\bar{s}$ or $su\overline{su}$ depending on the structure considered for the $f_0(980)$.