\section{Search For the Y(2175) Meson in Photoproduction at GlueX}
\label{p4}
\vfill
\subsection{Introduction}
\vfill
Les Collisionneurs $e^{+}e^{-}$ sont en deux cat\'egorie : Circulaire et Lineaire.\\
Dans les Collisionneurs circulaires en retrouvent le {\bfseries LEP} (Large Electron Positron collider) qui utilisait un faisceau d'\'electrons et de positrons jusqu'\`a $200\ GeV$ dans les derniers mois de son fonctionnement en 2000, et le {\bfseries FCC} (Futur Circular Collider) de l’ordre de 100 km circonf\'erence fournissant des faisceaux \`a des \'energies de 45 GeV \`a 175 GeV (pour comparaison, le LEP2 \'etait, au maximum, \`a 104 GeV par faisceau). Dans ces acc\'el\'erateurs circulaires, les cavités accélératrices sont utilis\'es de nombreuses fois pour chaque particule ce qui en fait des machines à priori plus \'economiques. Malheureusement les \'electrons et les positrons (du fait de leur faible masse) sont tr\`es sensibles \`a l'effet de rayonnement de freinage (Bremsstrahlung) limitant l'énergied'un acc\'el\'erateur $e^{+}e^{-}$ circulaire.
~\par Des acc\'el\'erateurs lin\'eaires ont été considérés. Il utilisent des cavit\'es supraconductrices \`a haut gradient, tel que {\bfseries CLIC} (Collisionneur lin\'eaire compact), qui cible une \'energie de collision de 0,5 \`a 3 TeV et une luminosit\'e de $2{\times}10^{34} cm^{-2}s^{-1}$ sur une distance de 15 \`a 50 km, il s’appuie sur un concept novateur d’acc\'el\'eration \`a deux faisceaux \`a la fr\'equence de 12 GHz avec un gradient acc\'el\'erateur \'elev\'e de 100 MV/m. L'{\bfseries ILC} (International Linear Collider), est un projet de collisionneur lin\'eaire $e^{+}e^{-}$ d\'elivrant une \'energie de $90$ jusqu'au TeV avec  une luminosit\'e attendu de $2\times10^{34} cm^{-2}s^{-2}$.
~\par Les Collisionneurs $e^{+}e^{-}$ ne permettront pas d'atteindre les \'energies du LHC ($7\ TeV$ - $14\ TeV$ ) mais seront un outil de pr\'ecision. Etant donn\'ee la nature leptonique des faisceaux, l'\'energie de la collision sera connue avec une bonne pr\'ecision ($\leq1\%$) et les \'ev\'enements seront nettement plus facile \`a analyser. Dans un tel environnement, il est possible de suivre les particules charg\'ees avec une grande pr\'ecision et de reconstruire les \'energies dans les calorim\`etres avec une grande r\'esolution. Autrement dit les r\'esolutions qui y seront atteintes sont sans pr\'ec\'edent.
~\par Leurs but principal sera de produire et d'\'etudier le boson de Higgs potentiellement d\'ecouvert au LHC en mesurant sa masse et ses propri\'et\'es avec plus de pr\'ecision. Les autres \'etudes seront d'une part des mesures de pr\'ecision pour la $QCD$ ( constante de couplage fort $\alpha_{s}$ ) et la masse du quark top et d'autre part la recherche de nouvelle physique ( mati\`ere noire et particules supersym\'etrique l\'eg\`eres).
~\par Dans cette \'etude, on s'int\'eresse \`a l'un des collisionneurs $e^+e^-$, qui est l'ILC. Au point d'interaction, deux diff\'erents d\'etecteurs, l'ILD (International Large Detector) et SiD (Silicon Detector) seraient install\'es sur plate-forme mobile afin de pouvoir les inter-changer.
~\par Au niveau de la calorim\'etrie, les deux d\'etecteurs serait \'equip\'es de calorim\`etres \'electromagn\'etiques et hadroniques diff\'erents. Le calorim\`etre hadronique SDHCAL d\'evelopp\'e \`a L'IPNL, est un des deux concepts de calorimètre proposé pour le détecteur ILD. Sa portabilité vers d'autre collisionneurs $e^+e^-$ est en cours d'étude.

\subsection{Event Selection}

L'ILD est un d\'etecteur g\'en\'erique $4\pi$ avec une structure standard en physique des particules dite en couche d'oignons. La figure~\ref{figure:4.2}.montre une vue en coupe du d\'etecteur ILD. Au plus proche de la ligne de faisceau se trouve un d\'etecteur \`a pixel pour tracker les particules charg\'ees finement. Il permet de reconstruire les vertex de d\'esintégrations des particules \`a courte dur\'ee de vie comme les m\'esons $B^{0}$. La couche suivante du d\'etecteur, le trajectom\`etre, reconstruit lui aussi les trajectoire des particules charg\'ees mais sur une grande distance. Un aimant supra-conducteur situ\'e apr\`es les calorim\`etres permet de courber les trajectoire des particules charg\'ees. La courbure des particules charg\'ees est directement reli\'ee \`a l'impulsion via la relation :
$$p = 300BR$$
~\par o\`u
\begin{itemize}
  \item[$\bullet$] $p$ est l'impulsion de la particule ($MeV/c$).
  \item[$\bullet$] $B$ est le champ magn\'etique ($Tesla$).
  \item[$\bullet$] $R$ est le rayon de courbure de la trajectoire de la particule ($m$).
\end{itemize}
~\par Ainsi, le trajectom\`etre apporte l'information de l'impulsion. Apr\`es le trajectom\`etre se trouve un calorim\`etre \'electromagn\'etique (ECAL) suivi d'un calorim\`etre hadronique (HCAL). Le premier permet d'arr\^eter et de mesurer l'\'energie des particule de type \'electromagn\'etique, c'est \`a dire les photons, les \'electrons et les positrons ; le deuxi\`eme, les Jets produit par hadronisation des quarks et gluons. Les deux sont des calorim\`etres \`a \'echantillonnage c'est \`a dire une alternance de couches d'absorbeur, qui favorisent l'interaction des particules et de couches de milieu actif qui d\'etectent une partie des particules produites. Les calorim\`etres permettent de mesurer l'\'energie des particules. Apr\`es les calorim\`etres se trouve l'aimant supra-conducteur qui permet de courber la trajectoire des particules charg\'ees. La derni\`ere couche est constitu\'e de chambres \`a muons afin de reconstruire les traces de ces derniers. En combinant la mesure de l'impulsion dans le trajectom\`etre et la mesure de l'\'energie dans les calorim\`etres pour une particule, on peut alors reconstruire son quadri-vecteur impulsion pour remonter \`a la cin\'ematique de l'\'ev\`enement.
     
\begin{figure}[H]
\centering
\includegraphics[width=0.6\columnwidth,keepaspectratio=true]{Plots/ILD.png}
\caption{Vue en coupe du d\'etecteur ILD et de ses sous-d\'etecteurs.}
\label{figure:4.2}
\end{figure}

\subsection{Monte Carlo Simulation}

Afin de pouvoir exploiter pleinement le potentiel de l'acc\'el\'erateur, le d\'etecteur devra \^etre performant, c'est-\`a-dire permettre la reconstruction des \'ev\`enements de la maniè\`ere la plus exacte possible. Pour remplir cette condition une fa\c{c}on originale de reconstruire les \'ev\`enements sera utilis\'ee pour l'ILD : le "Particle Flow Algorithm". Les collisions g\'en\'er\'ees dans l'ILC produiront des jets hadroniques en grande quantit\'e. Par exemple les bosons vecteurs, tr\`es importants pour le programme de physique, se d\'esint\`egrent principalement en quarks. Dans un d\'etecteur de particules les jets sont les objets les moins bien mesur\'es apr\`es \'evidemment les neutrinos qui \'echappent \`a toute d\'etection directe. Les jets sont constitu\'es par un ensemble de particules issues d'un quark ou d'un gluon qui s'hadronise.
~\par L'\'energie des jets provient en moyenne \`a $65\%$ des particules charg\'ees, \`a $26\%$ des photons et \`a $9\%$ des neutrons et des hadrons neutres. Il parait naturel d'utiliser le trajectographe pour estimer l'\'energie des traces charg\'ees. En effet celui-ci permet d'atteindre une grande pr\'ecision sur la mesure de l'impulsion. Les calorim\`etres sont alors d\'edi\'es aux particules neutres et doivent \^etre capables de s\'eparer les contributions dues aux particules neutres de celles provenant des particules charg\'ees. Dans cette m\'ethode chacune des particules est reconstruite individuellement comme dans une chambre \`a bulles qui est la meilleure reconstruction que l'on puisse imaginer. Cette m\'ethode est appel\'ee "Particle Fow".
~\par L'erreur sur la mesure de l'\'energie d'un jet peut \^etre d\'ecompos\'ee en plusieurs termes :
$${\sigma}_{jet}^{2}={\sigma}_{particules\ charg\acute{e}es}^{2}+{\sigma}_{\gamma}^{2}+{\sigma}_{hadrons\ neutres}^{2}+{\sigma}_{confusion}^{2}+{\sigma}_{seuil}^{2}+{\sigma}_{pertes}^{2}$$
Les termes ${\sigma}_{particules\ charg\acute{e}es}^{2}$, ${\sigma}_{\gamma}^{2}$ et ${\sigma}_{hadrons\ neutres}^{2}$ représentent respectivement la r\'eesolution sur les particules charg\'ees, les photons et les particules neutres. Les autres termes prennent en compte la d\'egradation de la r\'esolution due : - aux zones de confusion (chevauchement des d\'ep\^ots d'\'energie dans les calorim\`etres), - aux effets de seuil (rejet des particules de faible \'energie), - aux pertes (inefficacit\'e de d\'etection). En prenant en compte des performances d\'ej\`a obtenues sur des exp\'eriences ant\'erieures, on obtient pour chacun des sous-d\'etecteurs une r\'esolution intrins\`eque:
\begin{itemize}
\item $\Delta P/P\simeq{10^{-5}}$ pour les particules charg\'ees mesur\'ees dans le trajectographe,
\item $\Delta E/E\simeq{12\%}$ pour les photons mesur\'es par le calorim\`etre \'electromagn\'etique,
\item $\Delta E/E\simeq{45\%}$ pour les hadrons neutres mesur\'es dans le calorim\`etre hadronique.
\end{itemize}
Cela conduit pour un d\'etecteur id\'eal (${\sigma}_{confusion} = {\sigma}_{seuil} = {\sigma}_{pertes} = 0$) \`a une r\'esolution des jets de ${\sigma}_{jet}=\frac{14,2\%}{\sqrt{E}(GeV)}$. Cette r\'esolution est \`a comparer avec celle du LEP qui \'etait d'environ $60\%$. Ce r\'esultat montre le potentiel d'un d\'etecteur qui minimisera les termes de confusion, de seuil et de pertes. Le passage de la r\'esolution des jets de $60\%$ (comme c'\'etait le cas pour le LEP) \`a $30\%$ a \'et\'e \'etudi\'e pour diff\'erents canaux. Il a \'et\'e par exemple montr\'e qu'une r\'esolution de $30\%$ permet de s\'eparer les bosons Z des W provenant de la production bien connue d'\'ev\'enements donnant ensuite 4 jets : $W^+W^- \nu \bar{\nu}$ et $ZZ \nu \bar{\nu}$. La fgure~\ref{figure:4.3}. met en \'evidence cette s\'eparation apr\`es reconstruction de leur masse invariante pour les deux r\'esolutions consid\'er\'ees. La s\'eparation qui am\'eliore la qualit\'e des r\'esultats est \'equivalente \`a une augmentation de la luminosit\'e de $60\%$.
~\par En pratique cette m\'ethode impose des contraintes fortes sur la conception des diff\'erents sous-d\'etecteurs. les calorim\`etres m\'electromagnm\'etiques et hadroniques devront \^etre finement segmentm\'es afin de pouvoir distinguer les diffm\'erentes contributions des particules venant s'y d\'eposer et le trajectom\`etre devra avoir le plus long rayon externe possible afin de mieux reconstruire les traces des particules charg\'ees et de mieux identifer les points d'entr\'ees de ces traces dans le calorim\`etre \'electromagn\'etique.

\begin{figure}[H]
\centering
\includegraphics[width=0.5\columnwidth,keepaspectratio=true]{Plots/PFA.png}
\caption{Reconstruction des masses invariantes des boson $Z$ et $W$ provenant des \'ev\`enements $W^+W^- \nu \bar{\nu}$ et $ZZ \nu \bar{\nu}$ donnant 4 jets pour une r\'esolution des jets de $\frac{30\%}{\sqrt{E}(GeV)}$ \`a gauche et $\frac{60\%}{\sqrt{E}(GeV)}$ \`a droite.}
\label{figure:4.3}
\end{figure} 

\subsection{Cross Section Upper Limits}

Un prototype de calorim\`etre hadronique \`a haute granularit\'e \`a \'et\'e d\'evelopp\'e dans le cadre de la collaboration CALICE (figure~\ref{figure:4.4}.(a)). C'est un calorim\`etre \`a \'echantillonnage constitu\'e de 48 chambres \`a plaques de verres r\'esistif (GRPC) espac\'ees d'absorbeurs d'acier (longueur d'interaction nucl\'eaire
 ${\lambda}_{I} = 16.76 cm$ et une longueur de radiation $X_{0} = 1.77 cm$) de $15 mm$ d'\'epaisseur. Les chambres GRPC sont ins\'er\'ees dans des cassettes en acier de 5 mm d'\'epaisseur formant ainsi une barri\`ere totale d'acier de 20 mm entre chaque zone active. Les 48 couches d'acier repr\'esentent au total environ 6 longueurs d'interaction soit une \'epaisseur th\'eoriquement suffisante pour arr\^eter la plupart des gerbes hadroniques qui atteindront le HCAL dans l'ILD.
~\par Les chambres \`a plaque de verre r\'esistif (GRPC) sont des d\'etecteurs gazeux (voir figure~\ref{figure:4.4}.(b)). Lorsqu'une particule charg\'ee traverse la couche de gaz de la chambre, celle ci va ioniser le gaz pour produire des \'electrons et des ions. En appliquant une tension de part et d'autre de la couche de gaz, les \'electrons vont d\'eriver vers l'anode et la charge va \^etre r\'ecolter sur les carreaux de cuivre. Si le champs \'electrique fourni est suffisamment fort dans la GRPC, comme dans notre cas, les \'electrons qui d\'erivent vers l'anode produisent \`a leur tour des d'autres \'electrons, cr\'eant ainsi un ph\'enom\`ene d'avalanche \'electronique. C'est le principe de base d'un d\'etecteur gazeux.
~\par La segmentation du calorim\`etre est directement reli\'e \`a la dimension des cellules en cuivre utilis\'ees pour r\'ecolter la charge d\'eriv\'ee dans la couche de gaz. Celle ci est fix\'ee \`a $1 cm^2$ et fait partie int\'egrante de l'\'electronique d'acquisition.

\begin{figure}[H]
  \centering
  \subfigure[]{
    \includegraphics[height=4cm]{Plots/SDHCAL.png}
  }
  \quad
  \subfigure[]{
    \includegraphics[height=4cm]{Plots/GRPC.png}
  }
  \caption{le sous d\'etecteur SDHCAL (a) et la chambre GRPC (b).}
  \label{figure:4.4}
\end{figure}

~\par La quantit\'e de charge \'electrique g\'en\'er\'ee apr\`es le passage d'une particule dans le calorim\`etre est enregistr\'ee, si elle dépasse un seuil prédifinit et modifiable : nous obtenons alors un $hit$. Le SDHCAL est caract\'eris\'e par trois seuils de d\'eclenchement, correspondant a des intervalles de charges d\'efinis pour chaque seuil (tableau~\ref{table:4.1}). L'objectif de ces seuils, est d'obtenir une information sur la densité de particules. 

\begin{table}[H]
  \centering
  \begin{tabular}{|c|c|c|c|}
    \hline
    Seuil  & 1  & 2  & 3 \\ \hline
    Intervalle de charge (pC) & 0.114 < c < 5 & 5 < c < 15 & c > 15 \\ \hline
  \end{tabular}
  \caption{Intervalle de charge des diff\'erents seuils}
  \label{table:4.1}
\end{table}

\subsection{Conclusions}