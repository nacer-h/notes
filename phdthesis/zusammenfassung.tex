\chapter{Ausf\"uhrliche Zusammenfassung}
\label{zusammen}

Die erste Phase des GlueX-Experiments wurde Ende 2019 erfolgreich abgeschlossen, wobei mehr als vier Petabyte Daten f\"ur die Analyse gesammelt wurden. Anhand der kalibrierten Datens\"atze wurde eine Suche nach dem hybriden Mesonkandidaten, dem $Y(2175)$, in beiden $\phi\pi^{+}\pi^{-}$ und $\phi$(1020)$f_0$(980) Kan\"ale durchgef\"uhrt wurden. Eine erste Beobachtung der $Y(2175)$ in $\phi$(1020)$f_0$(980) Fotoproduktion mit einer Bedeutung von 6$\sigma$ und 12$\sigma$ auf 2017 und 2018 Daten jeweils. Eine Obergrenze f\"ur die $\phi\pi^{+}\pi^{-}$ Querschnitt bei 90$\%$ Konfidenzniveau von 0.5nb, 0.3nb and -0.5nb f\"ur das Jahr 2016, 2017 and 2018 Datens\"atze jeweils gemessen wurde. Im nicht-resonanten Modus, ohne die Anwesenheit von $Y(2175)$, n den beiden vorherigen Kan\"alen wurden Querschnittsmessungen durchgef\"uhrt. Zur besseren Absch\"atzung des Energieverlustes in der CDC wurde eine Optimierung der Truncated Mean-Methode durchgef\"uhrt und eine 20$\%$ Das Abschneiden der Treffer bei der Hochgeschichte der $dE/dx$-Verteilung wurde abgeschlossen, die nun in der Rekonstruktionssoftware GlueX enthalten ist.
~\par Die n\"achste Phase des GlueX-Programms wird in K\"urze beginnen, wobei ein zus\"atzliches Detektorsystem, der DIRC, zur Detektion des intern reflektierten Cherenkov-Lichtdetektors, derzeit installiert und in Betrieb genommen wird. Dieses Upgrade wird das Partikel-Identifikationssystem verbessern. Um Meson- und Baryon-Zerfallskan\"ale, die Kaonen enthalten, sauber auszuw\"ahlen. Sobald dieser Detektor installiert und in Betrieb genommen ist, planen wir, insgesamt 200 Tage physikalische Analysedaten mit einer durchschnittlichen Intensit\"at von 5x10$^7$ markierte Photonen auf dem Ziel pro Sekunde. Diese Datenprobe wird eine statistische Verbesserung um eine Gr\"o{\ss}enordnung gegen\"uber dem urspr\"unglichen GlueX-Datensatz und mit dem entwickelten Kaon-Identifikationssystem eine signifikante Steigerung des Potenzials von GlueX bieten, um in naher Zukunft wichtige experimentelle Fortschritte in unserem Wissen \"uber hybride Mesonen zu erzielen.