\chapter{Ausf\"uhrliche Zusammenfassung}
\label{chap.zusammen}

Die erste Phase des GlueX-Experiments wurde Ende 2019 erfolgreich abgeschlossen, wobei mehr als 121 pb$^{-1}$ Daten im koh\"arenten Photonenstrahlbereich gesammelt wurden. Mit den kalibrierten Datens\"atzen wird eine Suche nach dem hybriden Mesonenkandidaten, dem $Y(2175)$, in den beiden Kan\"alen $\phi\pi^{+}\pi^{-}$ und $\phi(1020) f_0(980)$ durchgef\"uhrt. Eine erste Messung des Wirkungsquerschnitts in der Photoproduktion für beide Kan\"ale wird durchgef\"uhrt. Da weder in der Reaktion $\gamma p \rightarrow \phi \pi^+\pi^- p$ noch in der Reaktion $\gamma p \rightarrow \phi f_0 p$ eine signifikante Produktion des $Y(2175)$ beobachtet werden konnte wurde in beiden F\"allen ein einseitges, oberes $90\%$-Konfidenzlimit auf den Wirkungsquerschnitt ermittelt.
~\par Um den Energieverlust in der CDC besser absch\"atzen zu k\"onnen, wurde eine Optimierung der Methode des sog. 'truncated mean' durchgef\"uhrt und eine 20$\%$-Trefferabschneidung beim High Tale der dE/dx-Verteilung abgeschlossen, die jetzt in der Rekonstruktionssoftware GlueX enthalten ist.
~\par Die n\"achste Phase des GlueX-Programms beginnt in K\"urze mit einem zus\"atzlichen Detektorsystem, dem DIRC (detection of internally reflected Cherenkov radiation), das derzeit installiert und in Betrieb genommen wird. Dieses Upgrade wird das Partikel-Identifikationssystem (Abb.~\ref{fig.summ}) verbessern, um Mesonen- und Baryon-Zerfallskan\"ale, die im Endzustand Kaonen enthalten, sauber zu identifizieren. Sobald dieser Detektor installiert und in Betrieb genommen worden ist, folgt eine geplante Datennahme von 200 Tagen mit einer durchschnittlichen Intensit\"at von 5x10$^7$ Photonen pro Sekunde statistische Verbesserung um eine Gr\"oßenordnung gegen\"uber dem urspr\"unglichen GlueX-Datensatz, und mit dem entwickelten Kaon-Identifikationssystem eine signifikante Steigerung des Potentials von GlueX, in naher Zukunft entscheidende experimentelle Fortschritte in unserem Wissen \"uber hybride Mesonen zu erzielen.