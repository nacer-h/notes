\chapter{Zusammenfassung}
\label{chap.zusammen}

Unser Verständnis der grundlegenden Bausteine der Materie hat in den letzten Jahrzehnten große Fortschritte gemacht. Vor fast einem halben Jahrhundert wurde die Quantenchromodynamik (QCD) entwickelt, eine revolutionäre Idee, dass Protonen, Neutronen und alle anderen stark wechselwirkenden Teilchen, die so genannten Hadronen, aus Quarks bestehen, die über den Austausch von Gluonen miteinander wechselwirken. Im Laufe der Jahre hat sich dieser Vorschlag fest etabliert, auch wenn wir freie Quarks aufgrund des Phänomens des Confinement nicht direkt beobachten konnten. Trotz jahrzehntelanger Forschung fehlt uns immer noch ein detailliertes quantitatives Verständnis der Art und Weise, wie die QCD das Spektrum der Hadronen erzeugt. Eine breit angelegte experimentelle Forschungskampagne wird durchgeführt, um ein neues Licht auf das Hadronen-Anregungsspektrum und die Dynamik von Hadronen zu werfen und dabei zu helfen, die theoretischen Modelle zu verbessern und zu testen. Ein wichtiger Akteur bei der Untersuchung dieser Eigenschaften ist das GlueX-Experiment, das darauf abzielt, die Eigenschaften des Beitrags des Gluonischen Feldes zu den Quantenzahlen des Quark-Antiquark gebundenen Systems, den hybriden Mesonen, zu entdecken und zu untersuchen.
~\par Mesonen im Konstituentenquarkmodell sind farbsingletgebundene Zustände eines Quarks $q$ und eines Antiquarks $\overline{q}$, mit den Quantenzahlen $J^{PC} = 0^{-+}, 0^{++}, 1^{--}, 1^{+-}, 1^{++}, 2^{--}, 2^{-+}, 2^{++}, usw$, wobei $J$, $P$ und $C$ die Gesamtdrehimpuls-, Paritäts- bzw. Ladungskonjugation des Fermionensystems sind. Dieses einfache Bild hat erfolgreich viele beobachtete Zustände im Mesonenspektrum beschrieben. Die QCD lässt jedoch ein reichhaltigeres Spektrum zu, das die gluonischen Freiheitsgrade im Quark und Anti-Quark-System einschließt. Da das gluonische Feld verschiedene Quantenzahlen tragen kann, führt dies viele neue Zustände in das Spektrum ein, einschließlich solcher, die Quantenzahlen tragen: $J^{PC} = 0^{--}, 0^{+-}, 1^{-+}, 2^{+-}, usw$, die für konventionelle $q\overline{q}$-Mesonen nicht erlaubt sind. Letztere sind die (Spin-)exotischen Hybridmesonen, und ihre experimentelle Beobachtung wird ein Beweis für die Existenz solcher Zustände jenseits des Konstituentenquarkmodells sein. Die hybriden Mesonen werden von vielen phänomenologischen Modellen vorhergesagt, wobei die Gitter-QCD Vorhersagen für ihre Eigenschaften wie die Masse macht, die experimentell getestet werden kann.
~\par Das GlueX-Experiment ist der Abbildung des Spektrums hybrider Mesonen gewidmet, wobei ein hochenergetischer, linear polarisierter Photonenstrahl verwendet wird, der von einem 12 GeV-Elektronenstrahl durch kohärente Bremsstrahlung auf einem Diamantstrahler erzeugt wird. Durch die Wahl der Kristallachsenorientierung des Diamanten erzeugten wir vier Datensätze mit 2 Sätzen paralleler ($0^{\circ}/90^{\circ}$) bzw. senkrechter ($45^{\circ}/135^{\circ}$) Polarisationsorientierungen. Die Energie und Intensität des Photonenstrahls werden mit einem Paar-Spektrometersystem (Dipolmagnet- und Szintillator-Arrays) überwacht, und zur Messung der Polarisation wird ein Triplett-Polarimeter ($\gamma e^{-} \rightarrow e^{+}e^{-}e^{+}$ Streuung an einer dünnen Be-Folie) verwendet. Der Photonenstrahl trifft auf ein 30 cm langes Flüssigwasserstoff-Target, das entlang der Mittelachse des Detektors positioniert ist (siehe Abb.~\ref{fig.2.2.1}). Der zentrale Bereich des Detektors befindet sich in einem Solenoidmagneten mit einer Magnetfeldstärke von $\sim$ 2T auf seiner Mittelachse. Teilchen aus der primären Wechselwirkung durchlaufen zunächst den Startzähler (Szintillator-Detektor), mit dessen Hilfe der Strahlbecher, der das Ereignis erzeugt hat, identifiziert werden kann. Unmittelbar um den Startzähler herum befindet sich die zentrale Driftkammer (CDC) (Strohhalm Detektor), die Informationen zur Verfolgung und zum Energieverlust ($dE/dx$) liefert. Der CDC nachgeschaltet sind die vier Pakete der Forward Drift Chamber (FDC) (planare Driftkammern), die sowohl die Verfolgung als auch die $dE/dx$-Informationen liefern. Um die Spurdetektor herum befindet sich das Barrel Calorimeter (BCAL) (Bleiszintillatorfaser), das für Photonen zwischen Polarwinkeln von $11^{\circ}$ und $126^{\circ}$ empfindlich ist. Hinter der Magnetspule befindet sich das Forward Calorimeter (FCAL) (Blei-Glasblöcke), das Polarwinkel von $1^{\circ}$ bis $11^{\circ}$ abdeckt. Vor dem FCAL befindet sich die Time Of Flight-Wand (TOF) (Szintillatorbalken), die Zeitinformationen liefert.
~\par Die erste Phase des GlueX-Experiments wurde Ende 2019 erfolgreich abgeschlossen, wobei im Bereich des kohärenten Photonenstrahls mehr als 121 pb$^{-1}$ an Daten gesammelt wurden. Unter Verwendung der kalibrierten Datensätze wurde eine Suche nach dem $1^{--}$ Hybrid-Mesonenkandidaten, dem $Y(2175)$, sowohl in den $\phi\pi^{+}\pi^{-}$ als auch in den $\phi(1020) f_0(980)$ exklusiven Endzuständen durchgeführt. Ein mögliches Strangeonium-Pendant des $Y(4260)$ im Charmonium-Sektor, das bereits bei Positron-Elektronen-Experimenten beobachtet wurde. Trotz aller bisherigen experimentellen Bemühungen reicht unser Wissen über den $Y(2175)$ nicht aus, um eine der theoretischen Interpretationen zu bestätigen oder zu unterdrücken. Bis jetzt beschränken sich alle experimentellen Informationen über den $Y(2175)$ auf die e$^+$e$^{-}$ Vernichtung und den $J/\psi$ hadronischen Zerfall. Die $Y(2175)$-Produktion in anderen Prozessen wird zum Verständnis seiner Natur beitragen. Das GlueX-Experiment bietet eine neue Möglichkeit, diesen Zustand zum ersten Mal in der Fotoproduktion zu suchen. Da der $Y(2175)$ in den Zuständen $\phi(1020)f_{0}(980)$ und $\phi(1020)\pi^{+}\pi^{-}$ gesehen wird, haben wir die exklusive Reaktion $\gamma p \rightarrow p \pi^{+}\pi^{-}K^{+}K^{-}$ untersucht. Die Rekonstruktion der Teilchen im Endzustand ist für die physikalische Analyse mit einer guten Kaon-, Pion- und Protonentrennung unerlässlich. Eines der entscheidenden GlueX-Spektrometer-Subsysteme für die Bereitstellung der PID-Informationen ist der CDC-Detektor durch die Messung des Energieverlustes, $dE/dx$. Für eine bessere Abschätzung des mittleren Energieverlustes im CDC wurde eine optimale Trunkierung des durchschnittlichen $dE/dx$ für Pionen und Protonen untersucht. Der Anteil der zurückzuweisenden Treffer wird durch die Optimierung von drei Gütezahlen bestimmt: die mittlere Energieverlustauflösung sowohl für Protonen als auch für Pionen, die Trennschärfe zwischen den Teilchenarten und die Fehlidentifizierungsfraktion (mis-PID) zwischen den Teilchen. Der Fehlidentifikationsanteil (mis-PID) ist der Anteil der Treffer aller rekonstruierten Spuren in der $dE/dx$-Verteilung, der zwischen den Partikelspezies gemischt wird. Durch die Optimierung des mis-PID, der Trennleistung und der $dE/dx$-Auflösung wird eine optimale Trunkierung erreicht, die auf $\sim$ 20$\%$ auf die hohen $dE/dx$-Werte geschätzt wird und inzwischen offiziell in der Rekonstruktionssoftware GlueX enthalten ist.
~\par Um nach $Y(2175)$ in den Zerfallsmodi $\phi \pi^+ \pi^-$ und $\phi f_0(980)$, mit $\phi \rightarrow K^+ K^- $ und $f_0 \rightarrow \pi^+ \pi^-$ zu suchen, haben wir die Reaktionen der Form $\gamma p \rightarrow K^+ K^- \pi^+ \pi^- p$ untersucht. Ein Ereignisauswahlverfahren wird angewandt, um die Hintergrundereignisse, die unser Signal nachahmen, so weit wie möglich zu subtrahieren und die Signalereignisse so weit wie möglich beizubehalten. Dies wird durch Schnitte auf verschiedene Variablen gefolgt von der Auswahl der exklusiven $\phi \pi^+ \pi^-$-Ereignisse realisiert, da das $\phi f_0(980)$ eine Unterstichprobe des $\phi \pi^+ \pi^-$ ist. Um den Untergrund unter dem $\phi(1020)$ in der $K^{+}K^{-}$-invarianten Masse zu entfernen, fügen wir ein $\phi(1020)$-Signal plus Untergrund in Abhängigkeit von den $\pi^{+}\pi^{-}$ und $K^+K^-\pi^{+}\pi^{+}\pi^{-}$-invarianten Massen ein und extrahieren auf diese Weise die massenabhängigen $\pi^{+}\pi^{-}$ und $K^+K^-\pi^{+}\pi^{-}$ invarianten $\phi(1020)$ Einrträge. Eine erste Messung des Fotoproduktionsquerschnitts für die Kanäle $\phi\pi^{+}\pi^{-}\pi^{-}$ und $\phi(1020) f_0(980)$ wurde durchgeführt. Da die $Y(2175)$ in beiden Kanälen nicht gefunden wurden, wurde eine Obergrenze für den gemessenen Querschnitt festgelegt. Wir erhalten eine Obergrenze bei $90\%$ CL von 0,67 nb, 0,24 nb, 0,20 nb und 0,35 nb für $Y(2175)\rightarrow \phi(1020) \pi^+\pi^-$, und 0,33 nb, 0. 48 nb, 0,43 nb und 0,39 nb für $Y(2175)\rightarrow \phi(1020) f_0(980$, für die Datensätze 2016, 2017 sowie Frühjahr und Herbst 2018. Darüber hinaus werden die potenziellen Quellen systematischer Fehler durch mehrere Variationen in der Analysekette geschätzt und in quadratisch zu den Gesamtfehlern bei den Querschnittsmessungen addiert.
~\par Die n\"achste Phase des GlueX-Programms beginnt in K\"urze mit einem zus\"atzlichen Detektorsystem, dem DIRC (detection of internally reflected Cherenkov radiation), das derzeit installiert und in Betrieb genommen wird. Dieses Upgrade wird die Teilchen-Identifikation (Abb.~\ref{fig.summ}) verbessern, um Mesonen- und Baryon-Zerfallskan\"ale, die im Endzustand Kaonen enthalten, sauber zu identifizieren.
~\par Sobald dieser Detektor installiert und in Betrieb genommen worden ist, folgt eine geplante Datennahme von 200 Tagen mit einer durchschnittlichen Intensit\"at von 5x10$^{7}/s$ Photonen, welches einer statistischen Verbesserung um eine Gr\"oßenordnung gegen\"uber dem urspr\"unglichen GlueX-Datensatz entspricht. Damit sowie mit dem neu entwickelten Kaon-Identifikationssystem wird eine signifikante Steigerung des Potentials von GlueX erzielt, um in naher Zukunft entscheidend zu unserem Wissen \"uber hybride Mesonen beizutragen. Die in dieser Arbeit beschriebene, entwickelte und durchgef\"uhrte Suche sowie Messungen sollten basierend auf den Daten der GlueX Phase-II wiederholt werden.

\begin{comment}
Die erste Phase des GlueX-Experiments wurde Ende 2019 erfolgreich abgeschlossen, wobei mehr als 121 pb$^{-1}$ Daten im koh\"arenten Photonenstrahlbereich gesammelt wurden. Mit den kalibrierten Datens\"atzen wurde eine Suche nach dem hybriden Mesonenkandidaten, dem $Y(2175)$, in den beiden Kan\"alen $\phi\pi^{+}\pi^{-}$ und $\phi(1020) f_0(980)$ durchgef\"uhrt. Eine erste Messung des Wirkungsquerschnitts in der Photoproduktion für beide Kan\"ale wurde durchgef\"uhrt. Da weder in der Reaktion $\gamma p \rightarrow \phi \pi^+\pi^- p$ noch in der Reaktion $\gamma p \rightarrow \phi f_0 p$ eine signifikante Produktion des $Y(2175)$ beobachtet werden konnte, wurde in beiden F\"allen ein einseitiges oberes $90\%$-Konfidenzlimit auf den Wirkungsquerschnitt ermittelt.
~\par Um den Energieverlust in der CDC besser absch\"atzen zu k\"onnen, wurde eine Optimierung der Methode des sog. 'truncated mean' durchgef\"uhrt und eine 20$\%$-Treffer-Reduzierung der dE/dx-Verteilung abgeschlossen, die inzwischen offiziell in der GlueX Rekonstruktions-Software enthalten ist.
~\par Die n\"achste Phase des GlueX-Programms beginnt in K\"urze mit einem zus\"atzlichen Detektorsystem, dem DIRC (detection of internally reflected Cherenkov radiation), das derzeit installiert und in Betrieb genommen wird. Dieses Upgrade wird das Teilchen-Identifikation (Abb.~\ref{fig.summ}) verbessern, um Mesonen- und Baryon-Zerfallskan\"ale, die im Endzustand Kaonen enthalten, sauber zu identifizieren.
~\par Sobald dieser Detektor installiert und in Betrieb genommen worden ist, folgt eine geplante Datennahme von 200 Tagen mit einer durchschnittlichen Intensit\"at von 5x10$^{7}/s$ Photonen, welches einer statistischen Verbesserung um eine Gr\"oßenordnung gegen\"uber dem urspr\"unglichen GlueX-Datensatz entspricht. Damit sowie mit dem neu entwickelten Kaon-Identifikationssystem wird eine signifikante Steigerung des Potentials von GlueX erzielt, in naher Zukunft entscheidend zu unserem Wissen \"uber hybride Mesonen beizutragen. Die in dieser Arbeit beschriebene, entwickelte und durchgef\"uhrte Suche sowie Messungen solten basierend auf den Daten der GlueX Phase-II wiederholt werden.    
\end{comment}

 
