% \phantomsection\addcontentsline{toc}{section}{Bibliography}
% \begin{thebibliography}{ABC}	
%     \bibitem[1]{1} H Fritzsch, M Gell-Mann and H Leutwyler \emph{Advantages of the Color Octet Gluon Picture}. Phys. Lett. 47B 365  (1973).
%     % \bibitem[2]{2} Fedor Bezrukov and Mikhail Shaposhnikov. \emph{Why should we care about the top quark Yukawa coupling?}. CERN, CH-1211 Genève 23, Switzerland, November 10, 2014. \url{http://arxiv.org/abs/1411.1923v1}
%     % \bibitem[3]{3} Ryo Yonamine, Katsumasa Ikematsu, Tomohiko Tanabe, Keisuke Fujii, Yuichiro Kiyo, Yukinari Sumino and Hiroshi Yokoya. \emph{Measuring the top Yukawa coupling at the ILC at $\sqrt{s}$ = 500 GeV}. 2011. \url{http://arxiv.org/abs/1104.5132v1}
%     %   \bibitem[4]{4} M. Battaglia, T. Barklow, M. E. Peskin, Y. Okada, S. Yamashita and, P. Zerwas.  \emph{Physics Benchmarks for the ILC Detectors}. 2006. \url{http://arxiv.org/abs/hep-ex/0603010v1}.
%     % \bibitem[5]{5} The International Linear Collider. \emph{Technical Design Report Volume 2: Physics} , 2013.
%     % \bibitem[6]{6} The International Linear Collider. \emph{Technical Design Report Volume 4: Detectors} , 2013.
%     % \bibitem[7]{7}  Alexey PETRUKHIN, \emph{Energy measurement with the SDHCAL prototype}. CALICE Collaboration, IPNL/CNRS, France, 2014. \url{http://arxiv.org/abs/1406.7111v1}.
%     % \bibitem[8]{8} Rémi Été. \emph{Séparation de gerbes hadroniques proches dans le Calorimètre Hadronique Semi-Digital SDHCAL pour ILC }. Rapport de Stage de Master II Physique Fondamentale et Astrophysique, Institut de Physique Nucléaire de Lyon, Université Claude Bernard Lyon 1. \url{http://rete.github.io/reports/M2.pdf}.
% \end{thebibliography}
