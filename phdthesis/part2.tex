\section{The GlueX Experiment}
\label{p2}

\subsection{Photon Beamline}

L'unification des interactions \'electromagn\'etique et faible voit le jour en 1967 gr\^ace au travail de Glashow, Weinberg et Salam, et l'existence de deux nouveaux bosons vecteurs de l’interaction \'electrofaible, le $Z$ et le $W$, sont postul\'es. Il a fallu attendre 1983 pour qu'ils soient d\'ecouverts par les exp\'eriences UA1 et UA2 aupr\`es du collisionneur proton-antiproton du CERN.
Cependant, des mesures de pr\'ecision voisine de $0.1\%$ sur l'ensemble des param\`etres du $Z$ et $W$ mais aussi des pr\'ecisions sur les masses du $top$, $Higgs$ et de leurs couplages manquent cruellement pour affirmer que le Mod\`ele Standard est une th\'eorie coh\'erente pour les particules et leurs interactions. Cette pr\'ecision de $0.1\%$ est n\'ecessaire pour v\'erifier la coh\'erence interne du Mod\`ele Standard, les pr\'edictions de la m\'ecanique quantique sur laquelle il est fond\'e, et celles du m\'ecanisme de Higgs. Pour cela, des collisions $e^{+}e^{-}$ sont \'etudi\'ees avec une grande attention en physique des particules, à la fois pour l'\'etude des aspects \'electrofaibles et ainsi que pour la recherche d'\'ev\`enements de nouvelles physique.
~\par Les collisions $e^{+}e^{-}$ poss\`edent de nombreux avantages. En particulier: 
~\par- Comme les particules de l'\'etat initiales ($e^{+}, e^{-}$) sont \'el\'ementaires, cela permet une connaissance pr\'ecise de l'\'energie et l'impulsion des particules initiales et, par conservation du quadri-impulsion, de l'\'etat final. Dans les collisions $pp$ o\`u le $proton$ est une particule composite ($partons$), on ne peut pas acc\'eder \`a l'\'enegie-impulsion des particules \'el\'ementaires, seulement la distributions de l'\'energie-impulsion entre les $partons$ (PDF: Parton Distribution Function).
~\par- La forte dépendance de la section efficace en fonction de l'\'energie de collision, permet de choisir la physique \`a \'etudier, par exemple : seuil de production $t\bar{t}$ pour étudier le quark $top$, mass du $Z$ pour des mesures électrofaibles...etc . Alors que dans collisions $pp$, la seul information est que la section efficace d'interaction augmente avec l'\'energie de collision.
~\par- Des calculs limit\'es à l'electrodynamique quantique (QED:Quantum ElectroDynamics) dans le premier ordre, ce qui permet des calculs simples de sections efficaces avec des incertitudes \`a quelque pourcents alors que dans les caculs QCD il faut faire des calculs tr\`es compliqu\'es jusqu'au second-second ordre pour obtenir des incertitudes inf\'erieur \`a $10\%$.
~\par- Enfin les conditions exp\'erimentales où le phénomène d'empilement est limité (pileup), ce qui permet une bonne identification des particules. En revanche dans les collisions $pp$ \`a cause du taux \'elev\'e d'interaction, plusieurs particules vont \^etre produites simultanément (dans la fen\^etre des mesures), ce qui va engendrer des erreurs sur le d\'enombrement et l'identification de l'origine des particules secondaires. \\
Ces diff\'erences entre les collisions $e^{+}e^{-}$ et $p-p$ sont list\'e dans le tableau~\ref{tab:2.1}.
~\par N\'eanmoins, la mont\'ee en \'energie dans ces collisions est techniquement limit\'ee , de m\^eme le taux d'\'ev\`enements est faible d\^u aux faibles sections efficaces de collisions $e^{+}e^{-}$.

\begin{table}[H]
  \centering
  \begin{tabular}[H]{|c|c|c|}
    \hline Collision & $e^{+}e^{-}$ & $pp$ \\
    \hline Quadri-impulsion & connaissance totale & \pbox{20cm}{acc\`es sur la composante\\transversale} \\
    \hline \pbox{20cm}{ Dépendance de la section efficace\\de l'\'energie de collision} & Effet de seuil & faible \\
    \hline Section efficace totale ($mb$) & $5\times10^{-9}$ & $110$ \\
    \hline Section efficace de production ${t\bar{t}}(mb)$ pour $\sqrt{s}=1\rm{TeV}$ & $0.5\times10^{-12}$ & $172\times10^{-9}$ \\
    \hline calculs de sections efficaces dans le $1^{er}$ ordre & $QED$ & $QCD,QED$ \\
    \hline  
  \end{tabular}
  \caption{Comparaison entre les collisions $e^{+}e^{-}$ et $proton(p)-proton(p)$ en terme de quadri-impulsion, de la distribution de section efficace en fonction de l'\'energie de collision dans le centre de masse, de la section efficace totale, de la la section efficace de production d'une paire top et des calculs de section efficaces dans le premier ordre.}
  \label{tab:2.1}
\end{table}

\subsection{The GlueX Spectrometer}

~\par La d\'ecouverte par les exp\'eriences ATLAS et CMS d'une nouvelle particule de masse 125 GeV dont les caract\'eristiques s'apparentent \`a celle du boson de Higgs standard, ouvre une opportunit\'e pour identifier l'origine de la brisure spontann\'ee de la sym\`etrie \'electrofaible. Des collisions $e^{+}e^{-}$  dans le centre de masse entre $91$ GeV et $1$ TeV est un instrument id\'eal pour une \'etude de pr\'ecision in\'egalée sur les param\`etres \'electrofaibles et le boson de Higgs.
~\par Ce programme d'\'etude est class\'e selon les \'energies de collisions $e^{+}e^{-}$ dans le centre de masse ($\sqrt{s}$) en:

\begin{itemize}
\item[$\bullet$] $91$ GeV${\leq}\sqrt{s}{\leq}160$ GeV, production r\'esonante de $Z$ et de paire de $W^+W^-$, ce qui permettra des mesures de pr\'ecisions sur le couplage et l'asym\'etrie du $Z$ et sur la masse du $W$, avec un ordre de grandeur du MeV.\\
\item[$\bullet$] $\sqrt{s}=250$ GeV, pr\`es du pick de section efficace de production du processus $e^{+}e^{-}{\rightarrow}ZH$. Le Tagging du boson $Z$ permet une mesure direct des taux de branchements du Higgs, et ainsi identifier ses diff\'erents modes de d\'esint\'egrations: Jets ($b\bar{b},c\bar{c}$ et $gg$), invisible ($\nu{\bar{\nu}}$) ou rare ($\gamma\gamma,\mu^-\mu^+$,particule exotiques) dans l'\'etat finale, aussi permet une d\'etermination pr\'ecise de la masse du Higgs.\\
\item[$\bullet$] $350$ GeV${\leq}\sqrt{s}{\leq}400$ GeV, en plus d'augmenter la pr\'ecision de mesure des param\`etres pr\'ec\'edants, il y a la possibilit\'e de produire une paire de quark top-antitop, qui permettra une mesure de pr\'ecision sur la masse du top de l'ordre du 100 MeV et de mettre des contraintes sur les param\`etres de brisure de la sym\'etrie \'electrofaible du faite que le couplage du top au Higgs est le plus important des couplage de Yukawa au fermions ($g_{top-Higgs}\approx1$).\\
\item[$\bullet$] $\sqrt{s}=500$ GeV, en plus d'augmenter la pr\'ecision de mesure des param\`etres pr\'ec\'edants, permet une mesure de pr\'ecision de la largeur total de d\'esint\'egration du Higgs ind\'epandement du mod\`ele, des mesures de couplges du boson de Higgs au $top$ ($t\bar{t}$) et du l'auto-couplage du Higgs ($HHZ$), dont ce dernier permettra de d\'eterminer le potentiel du Higgs.\\
\item[$\bullet$] $\sqrt{s}=1$ TeV, acc\`es à tous les modes de production du boson de Higgs avec une meilleur pr\'ecision de mesure sur les couplage de Yukawa au fermions ou bosons et auto-couplage du Higgs, ce qui permettra de mettre des contraintes sur les mod\`ele du Higgs et sur les particules exotiques.
\end{itemize}
~\par Les processus majeurs \'etudiés aupr\`es des collisionneurs $e^+e^-$ avec les \'energies de collisions correspandantes dans le centre de masse et les objectifs physique sont repr\'esent\'ees dans le tableau~\ref{table:2.2}. 

\begin{table}[H]
  \small
  \centering
  \begin{tabular}{c c c}
    \hline 
    Energie(GeV) & Processus & Objectifs physique\\
    \hline 
    $91$ & $e^{+}e^{-}{\rightarrow}Z$ &  ultra-pr\'ecision sur les param\`etres \'electrofaibles \\
    \hline 
    $160$ & $e^{+}e^{-}{\rightarrow}W^{+}W^{-}$ & ultra-pr\'ecision sur la masse du $W$ \\
    \hline 
    $250$ & \pbox{20cm}{$e^{+}e^{-}{\rightarrow}ZH$\\$e^{+}e^{-}{\rightarrow}t\bar{t}$}  & \pbox{20cm}{pr\'ecision sur la largeur totale et les couplages du Higgs\\masse du quark top et son couplage}\rule[-7pt]{0pt}{20pt}\\
    \hline 
    $350-400$ & \pbox{20cm}{$e^{+}e^{-}{\rightarrow}W^+W^-$\\$e^{+}e^{-}{\rightarrow}\nu\bar{\nu}H$\\$e^{+}e^{-}{\rightarrow}t\bar{t}H$} & \pbox{20cm}{pr\'ecision sur le couplage du $W$\\pr\'ecision sur le couplage du Higgs\\pr\'ecision sur le couplage du Higgs au top}\rule[-7pt]{0pt}{20pt}\\
    \hline 
    $500$ & \pbox{20cm}{$e^{+}e^{-}{\rightarrow}ZHH$\\$e^{+}e^{-}{\rightarrow}{\nu}\bar{\nu}HH$} & mesure de Self-coupling du Higgs\rule[-10pt]{0pt}{20pt}\\
    \hline 
    $700-1000$ & $e^{+}e^{-}{\rightarrow}\tilde{t}\tilde{t^*}$ & recherche de particules supersym\'etriques\rule[-7pt]{0pt}{20pt}\\
    \hline    
  \end{tabular}
  \caption{les processus physiques \'etudier dans les collisions $e^{+}e^{-}$, avec les plage d'\'energie de collision dans le centre de masse et le but physique r\'ealis\'e.}
  \label{table:2.2}
\end{table}

% \subsection{Les signatures principales dans les collisions $e^{+}e^{-}$ (benchmarks)}
~\par Des signatures importantes (benchmarks) ont \'et\'e d\'efinies dans les collisions $e^{+}e^{-}$ pour optimiser les performances des d\'etecteurs. Elles constituent des mesures phares pou lesquelles l'instrument (accélérateur, détecteurs et  algorithmes) doit être optimisé. Ces signatures sont caract\`eris\'ees par le fait qu'elles recouvrent la majorit\'e des sc\'enarios en physique des particules, contiennent les processus plus importants qui justifie le développement de nouveaux acc\'elerateurs et d\'etecteurs. 
~\par Les performances des sous-d\'etecteurs (d\'etecteur de Vertex, Tracker, Calorim\`etre) \'etudiés dans les collisions $e^{+}e^{-}$ sont:
\begin{itemize}
  \item[$\bullet$] haute r\'esolution sur l'\'energie des Jets reconstruits et sur la masse invariante des di-Jet, pour atteindre cet objectif des calorim\`etres et algorithmes de reconstruction (PFA: Particle Flow Algorithm) sp\'ecifiques ont \'et\'es developp\'es .
  \item[$\bullet$] Une haute granularit\'e des calorim\`etres electromagn\'etique et hadronique est motiv\'ee par la nécessit\'e de s\'eparer les traces des particules charg\'ees et neutres (photon et hadron neutres).
  \item[$\bullet$] la r\'esolutoin sur l'impulsion des traces charg\'ees dans le Tracker est guid\'ee par la signature $e^{+}e^{-}{\rightarrow}ZH{\rightarrow}l^+l^-X$, o\`u la reconstruction du Higgs est associée au canal leptonique du $Z$.
  \item[$\bullet$] L'efficacit\'e du Tagging des vertex d\'eplac\'es associés aux quark $b$ et $c$ et au lepton $\tau$ dans le Tracker, est testée dans les signatures $e^{+}e^{-}{\rightarrow}ZH,H{\rightarrow}b\bar{b}/c\bar{c}/\tau^+\tau^-$.
\end{itemize}
~\par Les Benchmarks principaux dans les collision $e^{+}e^{-}$ avec les param\`etres des d\'etecteurs testés pour chaque signature sont résumés dans le tableau~\ref{table:2.1}.

\makeatletter
\setlength{\@fptop}{5pt}
\begin{table}[H]
  \small
  \centering
  \begin{tabular}{|c|c|c|c|c|c|}
    \hline 
    Sous-détecteur & Vertex & \multicolumn{2}{|c|}{Tracker} & \multicolumn{2}{|c|}{Calorim\`etre}\\
    \hline 
          Paramètre pertinent    & \pbox{20cm}{param\`etre\\d'impacte} & \pbox{20cm}{r\'esolution\\sur l'impulsion\\($\frac{\delta p}{p^2}$)}  & \pbox{20cm}{Efficacit\'e\\ ($\epsilon$)} & \pbox{20cm}{r\'esolution\\sur l'\'energie\\($\delta{E}$)}  & \pbox{20cm}{r\'solution\\sur la position\\($\delta{\theta},\ \delta{\phi}$)} \\
    \hline 
$e^{+}e^{-}{\rightarrow}ZH{\rightarrow}l^{+}l^{-}X$ &   & x &   &  & \\
$e^{+}e^{-}{\rightarrow}ZH{\rightarrow}jjb\bar{b}$ & x & x & x &  & \\
$e^{+}e^{-}{\rightarrow}ZH$, $H{\rightarrow}b\bar{b}/c\bar{c}/\tau^+\tau^-$ & x &  & x &  & \\
$e^{+}e^{-}{\rightarrow}ZH$, $H{\rightarrow}W^+W^-$ & x &  & x &  & x \\
$e^{+}e^{-}{\rightarrow}ZH$, $H{\rightarrow}\mu^+\mu^-$ & x & x & & &  \\
$e^{+}e^{-}{\rightarrow}ZH$, $H{\rightarrow}\gamma\gamma$ &   &   &  & x & x \\
$e^{+}e^{-}{\rightarrow}ZH$, $H{\rightarrow}invisible$ &   &   & x &  &  \\
$e^{+}e^{-}{\rightarrow}\nu\bar{\nu}{H}$ & x & x & x & x &  \\
$e^{+}e^{-}{\rightarrow}t\bar{t}H$ & x & x & x & x & x \\
$e^{+}e^{-}{\rightarrow}ZHH,\ \nu\bar{\nu}{HH}$ & x & x & x & x & x \\
    \hline    
  \end{tabular}
  \caption{Principaux Benchmarks dans les collision $e^{+}e^{-}$ avec les param\`etres des d\'etecteurs pertinents pour chaque signature.}
  \label{table:2.1}
\end{table}