\chapter{The GlueX Experiment}
\label{p.2}

The GlueX experiment is installed at Jefferson Lab in Newport News, Virginia. The experiment is located at the end of the new {beamline of the Continuous Electron Beam Accelerator Facility (CEBAF) (Fig.~\ref{fig.2}). At CEBAF, electron beam bunches are produced and accelerated by cryomodules containing superconducting radio frequency cavities. One pass corresponds to about a 2.2 GeV energy increase to the electron beam. The electron beam is extracted for three experimental halls (A,B and C), after 1-5 passes through the accelerator. The GlueX experimental hall, hall D, receives a 249.5 MHz electron beam at 5.5 passes in order to reach high enough energy to produce mesons of masses close to 3.5 GeV~\cite{Charles16}. The 12 GeV electron beam is used to produce linearly polarized photons for the experimental hall D. These high energy photons will allow GlueX for accessing different kinematics to produce the lightest hybrid spectrum. An overview of the GlueX photon beamline and the detector is described in the following sections.

\begin{center}
\null
\vfill
\begin{figure}[H]
    \centering
        \includegraphics[width=0.8\textwidth]{plots/cebaf.png}
        \caption{The Jefferson Lab CEBAF accelerator. The electron beam begins its first orbit at the injector. The linear accelerator, using the cryomodules, drive electrons to higher energies. The arcing magnets in both sides steer the electron beam from one straight section of the tunnel to the next for up to five orbits. In the middle the Helium liquifier provides liquid helium for ultra low temperature, during the superconducting operation. The beam is delivered to the 4 experimental halls A, B, C and D~\cite{Charles16}}
        \label{fig.2}
\end{figure}
\null
\vfill
\end{center}

\section{Photon Beamline}
\label{p.2.1}

In the photon beamline, the linearly polarized photon beam is produced, that will be then impinged on the hydrogen target. Measuring the energy, polarization and the flux of the incident photon beam is performed by the beamline components. A schematic illustration of the beamline is shown in Fig.~\ref{fig.2.1}.

\begin{figure}[H]
    \centering
        \includegraphics[width=1.0\textwidth]{plots/beamline_1.png}
        \caption{Schematic layout of the Hall D complex, showing the Tagger Hall, Hall D, and several of the key beamline devices. Also indicated are the locations of the 5C11B and AD00C beam position monitors~\cite{Charles16}.}
        \label{fig.2.1}
\end{figure}

\subsection{Diamond Radiator and Collimator}
\label{p.2.1.1}

The electron beam provided by the CEBAF accelerator is converted on a thin (50, 58, 47, and 17 $\mu$m) diamond radiator to a 9 GeV photon beam via the Bremsstrahlung process. Conservation of four-momenta in the reaction allows the determination of the outgoing photon energy. The lattice structure of the diamond radiator was aligned with the beam to produce coherent bremsstrahlung, with the coherent photon intensity peaking in specific energy ranges where the photons are linearly polarized relative to the crystal axis in the diamond as shown in Fig.~\ref{fig.2.1.1}. Two different diamond orientations are used, with the electric field vector parallel (PARA) or perpendicular (PERP) to the floor of the experimental hall. This process produces also secondary enhancements in the radiated photon energy spectrum due to integer multiples of the reciprocal lattice vector.
~\par For a 12 GeV electron beam, the GlueX experiment reaches more than 40$\%$ intensity of the emitted coherent photons, with a maximum at about 9 GeV photon energy. The emitted bremsstrahlung radiation is found within the characteristic angle $\theta_{CB}$ defined as

\begin{equation}
    \label{eq.2.1.1}
    \begin{aligned}
        \theta_{CB} = \frac{m_{e}}{E}~,
    \end{aligned}
\end{equation}

\noindent where $m_{e}$ is the electron rest mass, and $E$ are the energy of the incident electron.
~\par The photons traverse a 75 m long lead-block collimator, with two diameters configurations: 3.4 mm or 5.0 mm diameter, to suppress the wider angular spread of the incoherent photons, and keep the coherent component of the beam. The effect of collimation on the beam energy spectrum is shown in Fig.~\ref{fig.2.1.1}.

\begin{figure}[H]
    \centering
        \includegraphics[width=0.6\textwidth]{plots/coherent_spectrum.png}
        \caption{The beam profile in different collimation conditions. Three configurations are shown, with the absence of the collimator (black), with a 5 mm diameter collimator (blue), and a 3.4 mm collimator (red). The peak between 8.4 - 9 GeV is representing the coherent photon beam component~\cite{GlueX97}.}
        \label{fig.2.1.1}
\end{figure}

\subsection{Photon Tagging System}
\label{p.2.1.2}

After hitting the radiator, the scattered electron is bent by a dipole magnet of 1.8 T into the tagging spectrometer. The radius of the curvature and thus the deflection angle determines the recoiled electron momentum, and by knowing the initial energy of the incident electron beam, this allows to measure the photon energies. The tagging spectrometer is composed of two scintillation detectors, The Tagger Hodoscope (TAGH) and Tagger Microscope (TAGM) that measures the energy of photons in the incoherent and coherent energy region, respectively. The electrons not interacting with the diamond are stopped in the beam dump. More information on the photon tagger can be found in Ref.~\cite{GlueX97}.

\subsection{Photon Beam Flux and Polarization}
\label{p.2.1.3}

A thin beryllium (Be) foil is placed downstream of the collimator, where the beam photons interact with the electrons of the Be atom through the reaction $\gamma e^- \rightarrow e^-e^+e^-$.
~\par The beam flux is measured by detection of the produced $e^+e^-$ pairs in the Pair Spectrometer (PS). It consists of a dipole magnet to separate the electron pair paths into two scintillating detector arms, and a hodoscope to measure the energy and time using Silicon Photo Multipliers (SiPM).
~\par The beam polarization is determined by the Triplet Polarimeter (TPOL), using the recoiled electron ($e^-$) from the Be foil. It is a silicon strip detector consisting of 32 azimuthal components and 24 concentric circles. The angular distribution of the recoil electron ($\phi_{e^-}$) provides information about the beam polarization. For linearly polarized photons, the cross-section ($\sigma$) is defined as

\begin{equation}
    \label{eq.2.1.3}
    \begin{aligned}
        \sigma = \sigma_0[1+P\Sigma \cos (2\phi_{e^-})]~,
    \end{aligned}
\end{equation}

\noindent where $P$ is the linear polarization, $\sigma_0$ is the unpolarized cross section, and $\Sigma$ is the analyzing power~\cite{Dugger17}.
~\par The photon spectrum and its polarization with a 12 GeV electron beam are shown in Fig.~\ref{fig.2.1.3}, where $\sim 40\%$ of linear polarization after collimation is expected in the energy range of 8.4 – 9 GeV.

\begin{figure}[H]
    \centering
        \includegraphics[width=0.6\textwidth]{plots/tpol_beam_energy.png}
        \caption{(a) Photon beam intensity versus energy as measured by the pair spectrometer (not corrected for instrumental acceptance). (b) Photon beam polarization as a function of beam energy, as measured by the triplet polarimeter, with data points offset horizontally by $\pm$0.015 GeV for clarity~\cite{GlueX97}.}
        \label{fig.2.1.3}
\end{figure}


\section{The GlueX Spectrometer}
\label{p.2.2}

The linearly polarized photon beam is then delivered to the main spectrometer. The GlueX spectrometer is a multilayer detector, composed of a barrel shaped central spectrometer and further sub-detectors in forward direction. Surrounding a 30 cm long liquid hydrogen target in the center, with a decreasing diameter from 2.42 upstream to 1.56 cm downstream, a set of devices for particle identification with the Start Counter (SC) and Time of Flight (TOF) detectors are installed. Another set of devices for charged particle tracking is performed by the Central (CDC) and Forward (FDC) Drift Chambers. And finally, the detection of neutral particles is performed by the barrel (BCAL) and the forward (FCAL) calorimeters. For charged particle momenta measurements, a solenoid magnet is installed, delivering a magnetic field of 2.08 T. A schematic overview of the detector elements is shown in Fig.~\ref{fig.2.2.1}

\begin{figure}[H]
    \centering
        \includegraphics[width=0.8\textwidth]{plots/gluex_detector.png}
        \caption{The GlueX beamline and spectrometer. The photon beam extracted from the tagger hall, left side of the figure, is imping on the iquid hydrogen (LH$_2$) target in the center of the main detector. A 2.08 T solenoidal magnet surrounds the tracking system (green)~\cite{GlueX97}.}
        \label{fig.2.2.1}
\end{figure}

\subsection{Particle Identification Detectors}
\label{p.2.2.1}

\subsubsection{Start Counter (SC)}
The SC covers about 90$\%$ of 4$\pi$ solid angle coverage for particles originating from the center of the target. It provides timing information about the outgoing particles, to select the beam bunch of the photon that initiated the event. The detector with a timing resolution of roughly 300 ps~\cite{GlueX97} is made of segmented plastic scintillator that is bent to taper around the target cell (see Fig.~\ref{fig.2.2.1.1}).

\begin{figure}[H]
    \centering
        \includegraphics[width=0.7\textwidth]{plots/sc.png}
        \caption{The GlueX Start Counter mounted to the liquid H2 target assembly. The beam goes from left to right down the central axis.~\cite{GlueX97}.}
        \label{fig.2.2.1.1}
\end{figure}

\subsubsection{Time of flight Detector (TOF)}
The TOF detector is a wall of scintillators located about 5.5 m downstream from the target and covers an angular region from 0.6$^{\circ}$ to 13$^{\circ}$ in polar angle~\cite{GlueX97}, shown in Fig.~\ref{fig.2.2.1.2.a}. By combining the path length of the particle from tracking with the timing information from the TOF, we can compare the measured with the expected flight time for a given particle species. The measured velocity $\beta$ as function of momentum for different particles based on their mass hypothesis is shown in Fig.~\ref{fig.2.2.1.2.b}.

\begin{figure}[H]
    \centering
    \begin{subfigure}[H]{0.5\textwidth}
        \includegraphics[width=\textwidth]{plots/tof.png}
        \caption{}
        \label{fig.2.2.1.2.a}
    \end{subfigure}\vfill
    \begin{subfigure}[H]{0.6\textwidth}
        \includegraphics[width=\textwidth]{plots/betavsp.png}
        \caption{}
        \label{fig.2.2.1.2.b}
        \vspace{1pt}
    \end{subfigure}
    \caption{TOF detector mounted in Hall D (a), and velocity ($\beta$) versus particle momenta for positively charged particles to demonstrate the PID capability in the TOF detector (b)~\cite{Ghoul16}.}
    \label{fig.4.2.2}
\end{figure}

\subsection{Charged Particle Tracking}
\label{p.2.2.2}

Charged particle tracking is achieved by two detectors contained inside the solenoid magnet, the Central Drift Chamber (CDC) and Forward Drift Chamber (FDC) detectors, as shown in Fig.~\ref{fig.2.2.2}. These measure the position and time information of charged particles to reconstruct their trajectories and momenta. Both are ionization gas chambers filled with a mixture of CO$_2$ and Ar gas. They are comprised of high voltage electrodes that create an electric field between a cathode and anode, the CDC uses straw tubes and the FDC uses planes of wire packages to serve as their anodes. The charged particles will ionize the gas and the free electrons drift towards the anode at a well-defined velocity. The high field gradient near the anode causes an amplification of the initial free electron, this avalanche of electrons is converted to an electrical signal and used to detect the position of the charged particle track. Layers with skewed straw tubes (stereo layers) in the CDC allow the reconstruction of z-coordinates. Using the radius of the track curvature in the magnetic field, the particles momentum is determined. The CDC provides polar angle coverage from 6$^{\circ}$ to 128$^{\circ}$ and the FDC covers angles up to 20$^{\circ}$~\cite{GlueX97}. The CDC also plays an important role in PID by measuring the energy lost per unit length using flash ADCs, (cf. Sec.~\ref{p.3}).

\begin{figure}[H]
    \centering
    \begin{subfigure}[b]{0.7\textwidth}
        \includegraphics[width=\textwidth]{plots/cdc.png}
        \caption{}
        \label{fig.2.2.2.a}
    \end{subfigure}
    \vfill
    \begin{subfigure}[b]{0.7\textwidth}
        \includegraphics[width=\textwidth]{plots/fdc.png}
        \caption{}
        \label{fig.2.2.2.b}
    \end{subfigure}
    \caption{(a) The Central Drift Chamber layer of stereo straw tubes is shown, surrounding a layer of straw tubes at the opposite stereo angle. Part of the carbon fiber endplate, two temporary tension rods and some of the 12 permanent support rods linking the two endplates can also be seen. (b) The Forward Drift Chamber, consisting of 24 disk-shaped planar drift chambers of 1m diameter. They are grouped into four packages.}
    \label{fig.2.2.2}
\end{figure}

\subsection{Calorimeters}
\label{p.2.2.3}

Neutral particles energy and direction are measured by the Barrel Calorimeter (BCAL) and Forward Calorimeter (FCAL), shown in Fig.~\ref{fig.2.2.3}. The particles interacting with these detectors create electromagnetic showers which are used to reconstruct the reaction decay products. The BCAL is a lead-scintillating fiber calorimeter with readout on both the upstream and downstream ends. The FCAL is located 6 m downstream of the target and consists of 2800 lead-glass blocks oriented such that the FCAL acceptance is azimuthal symmetric for polar angles less than 11.5$^{\circ}$. The BCAL covers polar angles spanning from 11$^{\circ}$ to 126$^{\circ}$~\cite{GlueX97}. The GlueX detector readout and data acquisition is triggered by a significant energy deposit in the BCAL or FCAL.

\begin{figure}[H]
    \centering
    \begin{subfigure}[b]{0.4\textwidth}
        \includegraphics[width=\textwidth]{plots/fcal.png}
        \caption{}
        \label{fig.2.2.3.a}
    \end{subfigure}\hfill
    \begin{subfigure}[b]{0.54\textwidth}
        \includegraphics[width=\textwidth]{plots/bcal.png}
        \caption{}
        \label{fig.2.2.3.b}
    \end{subfigure}
    \caption{(a) Picture of the FCAL detector showing the individual lead-glass blocks. (b) View of the upstream face of the BCAL before being inserted into the solenoid bore.}
    \label{fig.2.2.3}
\end{figure}


