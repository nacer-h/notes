\section*{Introduction} % Pas de numérotation
\addcontentsline{toc}{section}{Introduction} % Ajout dans la table des matières

Le Mod\`ele Standard de la physique des particules est l’ensemble des th\'eories int\'egrant toutes les connaissance actuelles sur les particules et les forces fondamentales. Ce mod\`ele a \'et\'e couronn\'e de succ\`es depuis plus de 30 ans sur le plans exp\'erimental notamment avec la d\'ecouverte d'une nouvelle particule de masse 125 GeV dont les caract\'eristiques s'apparentent \`a celle du boson de Higgs standard par les exp\'eriences ATLAS et CMS.
~\par Dans le Mod\`ele Standard les couplages des fermions au boson de Higgs (couplages de Yukawa) sont proportionnels \`a la masse des fermions. Le quark top \'etant le fermion le plus lourd, son couplage de Yukawa doit \^etre \'etudi\'e pr\'ecis\'ement afin de comprendre le m\'ecanisme de brisure \'electro-faible, responsable de la g\'en\'eration des masses des particules, ou une nouvelle physique pouvant se manifester par des couplages diff\'erents de ceux pr\'edits. 
~\par Pour cela des collisionneurs \'electrons-positons avec leurs d\'etecteurs sont en cours d’\'elaboration. La nature ponctuelle des particules initiales (\'electrons et positrons) ainsi que le contr\^ole des bruits de fond en font l’instrument id\'eal pour mesurer avec pr\'ecision les propri\'et\'es du boson de Higgs. La collaboration CALICE propose des calorim\`etres optimis\'es pour les collisionneurs leptoniques. Ces calorim\`etres se caract\'erisent par une forte granularit\'e.
~\par C'est dans ce cadre que s'inscrit ce stage dont la th\'ematique est le potentiel de mesure des couplages du boson de Higgs avec une exp\'erience $e^{+}e^{-}$. Etude de calorim\`etre \`a haute granularit\'e. 
~\par Dans un premier temps, on va commenc\'e par une bri\`eve d\'escription du Mod\`ele Standard et la motivation d\'eri\`er l'\'etude du couplage du boson de $Higgs$ au quark $top$, puis une d\'escription des collisions $e^{+}e^{-}$, la physique \'etudier et les signatures principales (benchmarks) dans ces collisions, apr\`es cela on abord l'analyse de la signature $t\bar{t}H$.\\
Dans un second temps, on d\'ecrit brievement les outils \`experimentaux dont : les collisionneurs $e^+e^-$, le d\'etecteur ILD (International Large Detector), le Particle Flow Algorithme (PFA) et le calorim\`etre hadronique semi-digital (SDHCAL). Puis on va analyser les donn\'ees recolter aupr\`es des faisceaux tests au CERN (Centre Europ\'een de Recherche Nucl\'eaire) dont le but d\'etudier la r\'eponse du d\'etecteur SDHCAL aux gerbes hadroniques, et ensuite, on abordera la simulation Toy Monte Carlo pour \'etudier la possibiliter de s\'parer les particule \`a l'aide de leurs temps de vol dans le d\'etecteur. Finalement, on terminera avec des conclusions et perspectives.