\chapter{Summary and Outlook}
\label{chap.summ}
% \addcontentsline{toc}{section}{Summary and Outlook}

The first phase of the GlueX experiment was completed successfully at the end of 2019, with more than 121 pb$^{-1}$ of data collected in the coherent photon beam region. Using the calibrated data sets, a search for the hybrid meson candidate, the $Y(2175)$, in both, the $\phi\pi^{+}\pi^{-}$ and the $\phi(1020) f_0(980)$ exclusive final states has been performed. A first measurement of the photoproduction cross section for both channels has been carried out, and an upper limit on the production cross section of the $Y(2175)$ has been determined for both, $\phi\pi^{+}\pi^{-}$ and $\phi(1020)f_0(980)$ final states.
~\par For a better estimation of the energy loss in the CDC, an optimization of the truncated mean method was conducted, and a 20$\%$ hit-truncation at the high tale of the dE/dx distribution was concluded, which is meanwhile officially included in the GlueX reconstruction software.
~\par The next phase of the GlueX program will start soon, with an additional detector system for Detection of Internally Reflected Cherenkov light (DIRC), currently being installed and commissioned. This upgrade will improve the particle identification system (Fig.~\ref{fig.summ}), in order to cleanly select meson and baryon decay channels that include kaons in the final state. Once this detector has been installed and commissioned, the plan is to collect a total of 200 days of physics analysis data at an average intensity of 5x10$^{7}/s$ tagged photons on target. This data sample will provide an order of magnitude statistical improvement over the initial GlueX data set. Together with the developed kaon identification system, the GlueX potential for contributing to the understanding of hybrid mesons, in particular on the nature of the $Y(2175)$, will significantly increase in the near future. It will be worth to repeat the analysis proposed, developed, and carried out as described in this thesis.

\begin{figure}[htbp]
    \centering
    \includegraphics[width=0.8\textwidth]{plots/dirc_pvstheta.png}
    \caption{Kaon momentum versus the polar angle in MC, with kaon from the $\gamma p \rightarrow Y(2175) p \rightarrow \phi \mathrm{f}_0  \rightarrow  K^{+} K^{-} \pi^{+} \pi^{-} p$ reaction. The boxes show the TOF (red) and DIRC (purple) coverages.}
    \label{fig.summ}
\end{figure}