\section{Summary and Outlook}
\label{p.5}
% \addcontentsline{toc}{section}{Summary and Outlook}

~\par The first phase of the GlueX experiment was completed successfully at the end of 2019, collecting more than four petabytes of data for analysis. Using the calibrated data sets, a search for the hybrid meson candidate, the Y(2175), in both $\phi\pi^{+}\pi^{-}$ and $\phi$(1020)f$_0$(980) channels were conducted. A first observation of the Y(2175) in $\phi$(1020)f$_0$(980) photo-production with a significance of 6$\sigma$ and 12$\sigma$ for 2017 and 2018 data respectively. An upper limit on the $\phi\pi^{+}\pi^{-}$ cross section at 90$\%$ confidence level of 0.5nb, 0.3nb and -0.5nb for the 2016, 2017 and 2018 data sets respectively was measured. In the non-resonant mode ,without the presence of Y(2175), in both the previous channels, cross section measurements were established. For a better estimation of the energy loss in the CDC, an optimization of the truncated mean method was conducted, and a 20$\%$ hit truncation at the high tale of the dE/dx distribution was concluded, which is now included in the GlueX reconstruction software.

~\par The next phase of the GlueX program will start soon, with an additional detector system the DIRC, for Detection of Internally Reflected Cherenkov light detector, currently being installed and commissioned. This upgrade will improve the particle identification system (Fig.~\ref{fig.5.1}), in order to cleanly select meson and baryon decay channels that include kaons. Once this detector has been installed and commissioned, we plan to collect a total of 200 days of physics analysis data at an average intensity of 5x10$^7$ tagged photons on target per second. This data sample will provide an order of magnitude statistical improvement over the initial GlueX data set and, with the developed kaon identification system, a significant increase in the potential for GlueX to make key experimental advances in our knowledge of hybrid mesons in the near future.

\begin{figure}[H]
    \centering
        \includegraphics[width=0.5\textwidth]{plots/dirc_pvstheta.png}
        \caption{Kaon momentum versus the polar angle in MC, with kaon from the $\gamma p \rightarrow Y(2175) p \rightarrow \phi \mathrm{f}_0  \rightarrow  K^{+} K^{-} \pi^{+} \pi^{-} p$ reaction. The boxes show the TOF (red) and DIRC (purple) coverages.}
        \label{fig.5.1}
\end{figure}