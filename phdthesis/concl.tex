\chapter{Summary and Outlook}
\label{chap.summ}
% \addcontentsline{toc}{section}{Summary and Outlook}

The first phase of the GlueX experiment was completed successfully at the end of 2019, with more than 121 pb$^{-1}$ of data collected in the coherent photon beam region. Using the calibrated data sets, a search for the hybrid meson candidate, the $Y(2175)$, in both, the $\phi\pi^{+}\pi^{-}$ and the $\phi(1020) f_0(980)$ exclusive final states has been performed. A first measurement of the photoproduction cross section for both channels has been carried out, and an upper limit on the production cross section of the $Y(2175)$ has been determined for both, $\phi\pi^{+}\pi^{-}$ and $\phi(1020)f_0(980)$ final states.
~\par For a better estimation of the energy loss in the CDC, an optimization of the truncated mean method was conducted, and a 20$\%$ hit-truncation at the high tale of the dE/dx distribution was concluded, which is meanwhile officially included in the GlueX reconstruction software.
~\par The next phase of the GlueX program will start soon, with an additional detector system for Detection of Internally Reflected Cherenkov light (DIRC), currently being installed and commissioned. This upgrade will improve the particle identification system (Fig.~\ref{fig.summ}), in order to cleanly select meson and baryon decay channels that include kaons in the final state. Once this detector has been installed and commissioned, the plan is to collect a total of 200 days of physics analysis data at an average intensity of 5x10$^{7}/s$ tagged photons on target. This data sample will provide an order of magnitude statistical improvement over the initial GlueX data set. Together with the developed kaon identification system, the GlueX potential for contributing to the understanding of hybrid mesons, in particular on the nature of the $Y(2175)$, will significantly increase in the near future. It will be worth to repeat the analysis proposed, developed, and carried out as described in this thesis.    


\begin{figure}[htbp]
    \centering
    \includegraphics[width=0.8\textwidth]{plots/dirc_pvstheta.png}
    \caption{Kaon momentum versus the polar angle in MC, with kaon from the $\gamma p \rightarrow Y(2175) p \rightarrow \phi \mathrm{f}_0  \rightarrow  K^{+} K^{-} \pi^{+} \pi^{-} p$ reaction. The boxes show the TOF (red) and DIRC (purple) coverages.}
    \label{fig.summ}
\end{figure}

\begin{comment}
    Our understanding of the fundamental building blocks of matter has advanced greatly in the last few decades. It is nearly half a century ago that Quantum Chromodynamics (QCD) was developed, a revolutionary idea that protons, neutrons and all other strongly interacting particles, the so-called hadrons, are made of quarks interacting with each other via the exchange of gluons. Over the years, this proposal has become firmly established even though we have not observed free quarks directly, due to the phenomenon of confinement. Despite decades of research, we still lack a detailed quantitative understanding of the way QCD generates the spectrum of hadrons. A wide experimental research campaign is conducted to shed new light on the hadron excitation spectrum and the dynamics of hadrons, helping to improve and test the theoretical models. A key player to study these properties is the GlueX experiment, which aims to discover and study the properties of the gluonic field contribution to the quantum numbers of the quark-antiquark bound system, the hybrid mesons.
    ~\par Mesons in the constituent quark model are color-singlet bound states of a quark $q$ and antiquark $\overline{q}$, with quantum numbers $J^{PC} = 0^{-+}, 0^{++}, 1^{--}, 1^{+-}, 1^{++}, 2^{--}, 2^{-+}, 2^{++}, etc$, where $J$, $P$ and $C$ are total angular momentum, parity and charge conjugation of the fermion system, respectively. This simple picture has successfully described many observed states in the meson spectrum. However a richer spectrum is allowed by QCD that includes the gluonic degrees of freedom in the quark and anti-quark system. Since the gluonic field can carry different quantum numbers this introduces many new states to the spectrum, including those carrying quantum numbers: $J^{PC} = 0^{--}, 0^{+-}, 1^{-+}, 2^{+-}, etc$, that are not allowed for conventional $q\overline{q}$ mesons. These latter are the (spin-)exotic hybrid mesons, and their experimental observation will be a proof of the existence of such states beyond the constituent quark model. The hybrid mesons are predicted by many phenomenological models, and Lattice QCD is making predictions for their properties like the mass, which can be tested experimentally.
    ~\par The GlueX experiment is dedicated to the mapping of the spectrum of hybrid mesons, using a high-energy linearly polarized photon beam produced by a 12 GeV electron-beam through coherent bremsstrahlung on a diamond radiator. By choosing the crystal axis orientation of the diamond we produced four data sets, with 2 sets of parallel ($0^{\circ}/90^{\circ}$) and perpendicular ($45^{\circ}/135^{\circ}$) polarization orientations, respectively. The energy and intensity of the photon beam are monitored by a pair spectrometer system (dipole magnet and scintillator arrays), and to measure the polarization, a triplet polarimeter ($\gamma e^{-} \rightarrow e^{+}e^{-}e^{+}$ scattering on a thin Be foil) is used. The photon beam impinges on a 30 cm long liquid hydrogen target positioned along the central axis of the detector (see Figure~\ref{fig.2.2.1}). The central region of the detector is contained in a solenoid magnet with $\sim$ 2T on its central axis. Particles from the primary interaction first pass through the Start Counter (scintillator detector), which helps identify the beam bucket which generated the event. Directly surrounding the Start Counter is the Central Drift Chamber (CDC) (straw tube detector), providing tracking and energy loss ($dE/dx$) information. Downstream of the CDC are the four packages of the Forward Drift Chamber system (FDC) (planar drift chambers), providing tracking as well as the $dE/dx$ information. Surrounding the tracking devices is the Barrel Calorimeter (BCAL) (lead scintillator fiber) sensitive to photons between polar angles of $11^{\circ}$ and $126^{\circ}$. Downstream of the solenoid is the Forward Calorimeter (FCAL) (lead-glass blocks), which covers polar angles from $1^{\circ}$ to $11^{\circ}$. In front of the FCAL is the Time Of Flight wall (TOF) (scintillator bars) providing timing information.
    ~\par The first phase of the GlueX experiment was completed successfully at the end of 2019, with more than 121 pb$^{-1}$ of data collected in the coherent photon beam region. Using the calibrated data sets, a search for the $1^{--}$ hybrid meson candidate, the $Y(2175)$, in both the $\phi\pi^{+}\pi^{-}$ and the $\phi(1020) f_0(980)$ exclusive final states has been performed. A possible strangeonium counterpart of the $Y(4260)$ in the charmonium sector, which has been already observed in positron-electron experiments. Despite all previous experimental efforts, our knowledge of the $Y(2175)$ is not sufficient to confirm or suppress one of the theoretical interpretations. So far, all the experimental information about the $Y(2175)$ are limited to the e$^+$e$^{-}$ annihilation and $J/\psi$ hadronic decay. The $Y(2175)$ production in other processes will help to understand its nature. The GlueX experiment offers a new opportunity to search for this state for the first time in photoproduction. Since the $Y(2175)$ is seen in $\phi(1020)f_{0}(980)$ and $\phi(1020)\pi^{+}\pi^{-}$ states, we have studied the exclusive reaction $\gamma p \rightarrow p \pi^{+}\pi^{-}K^{+}K^{-}$. Reconstructing the final state particles is essential for the physics analysis, with a good kaon, pion and proton separation. One of the crucial GlueX spectrometer subsystems for providing the PID information is the CDC detector, through the measurement of the energy loss, $dE/dx$. For a better mean energy loss estimation in the CDC, an optimal truncation of the average $dE/dx$ for pions and protons has been studied. The fraction of hits to be rejected is determined by optimizing three figures of merit: the mean energy loss resolution of both protons and pions, the separation power between particle species, and the mis-identification fraction (mis-PID) between particles. The mis-PID is the fraction of hits, from all the reconstructed tracks, in the $dE/dx$ distribution, that is mixed between particle species. By optimizing the mis-PID, separation power and the $dE/dx$ resolution, an optimal truncation is achieved, and estimated to be $\sim$ 20$\%$ on the high $dE/dx$ values, which is meanwhile officially included in the GlueX reconstruction software.
    ~\par In order to search for $Y(2175)$ in the decay modes $\phi \pi^+ \pi^-$ and $\phi f_0(980)$, with $\phi \rightarrow K^+ K^- $ and $f_0 \rightarrow \pi^+ \pi^-$, we have studied the reactions of the form $\gamma p \rightarrow K^+ K^- \pi^+ \pi^- p$. An event selection procedure is applied to subtract as much as possible the background events that mimics our signal, as well as keeping as much as possible the signal events. This is realized by cutting on different variables, then followed by selecting the exclusive $\phi \pi^+ \pi^-$ events, since the $\phi f_0(980)$ is a subsample of the $\phi \pi^+ \pi^-$. In order to remove the background underneath the $\phi(1020)$ in the $K^{+}K^{-}$ invariant mass, we fit a $\phi(1020)$ signal plus background as a function of the $\pi^{+}\pi^{-}$ and $K^+K^-\pi^{+}\pi^{-}$ invariant masses, and this way extract the $\pi^{+}\pi^{-}$ and $K^+K^-\pi^{+}\pi^{-}$ invariant mass-dependent $\phi(1020)$ yields. A first measurement of the photoproduction cross section for both, $\phi\pi^{+}\pi^{-}$ and the $\phi(1020) f_0(980)$ channels has been carried out. In the absence of the $Y(2175)$ in both channels, an upper limit on the measured cross section has been established. We obtain an upper limit at $90\%$ CL of 0.67 nb, 0.24 nb, 0.20 nb and 0.35 nb for $Y(2175)\rightarrow \phi(1020) \pi^+\pi^-$, and 0.33 nb, 0.48 nb, 0.43 nb, and 0.39 nb for $Y(2175)\rightarrow \phi(1020) f_0(980$, for the 2016, 2017, and Spring and Fall 2018 datasets, respectively. In addition, the potential sources of systematic errors are estimated, through multiple variations in the analysis chain, and added in quadrature to the total errors on the cross section measurements.
    ~\par The next phase of the GlueX program will start soon, with an additional detector system for Detection of Internally Reflected Cherenkov light (DIRC), currently being installed and commissioned. This upgrade will improve the particle identification system (Fig.~\ref{fig.summ}), in order to cleanly select meson and baryon decay channels that include kaons in the final state. Once this detector has been installed and commissioned, the plan is to collect a total of 200 days of physics analysis data at an average intensity of 5x10$^{7}/s$ tagged photons on target. This data sample will provide an order of magnitude statistical improvement over the initial GlueX data set. Together with the developed kaon identification system, the GlueX potential for contributing to the understanding of hybrid mesons, in particular on the nature of the $Y(2175)$, will significantly increase in the near future. It will be worth to repeat the analysis proposed, developed, and carried out as described in this thesis.
\end{comment}