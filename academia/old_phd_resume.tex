\documentclass[11pt,a4paper,roman]{moderncv}
\moderncvtheme[blue]{classic} % optional argument are 'blue' (default), 'orange', 'red', 'green', 'grey' 
%\moderncvtheme[green]{classic}                % idem
\usepackage[T1]{fontenc}
% character encoding
\usepackage[utf8x]{inputenc}                   % replace by the encoding you are using
\usepackage[english]{babel}
\usepackage{bm}
% adjust the page margins
\usepackage[scale=0.9]{geometry} %0.9
\recomputelengths                             % required when changes are made to page layout lengths
\fancyfoot{} % clear all footer fields
% \fancyfoot[LE,RO]{\thepage}           % page number in "outer" position of footer line
% \fancyfoot[RE,LO]{\footnotesize} % other info in "inner" position of footer line

% personal data
\firstname{Abdennacer}
\familyname{HAMDI}
%\title{Curriculum Vitae}
\address{GSI Helmholtz Center}{Heckb\"uro 1.001, Planckstraße 1}{64291 Darmstadt, Germany}
\mobile{+49 6159 71 2556}
%\phone{<Phone number>} 
%\fax{<Fax number>}          
% \email{A.Hamdi@gsi.de}
\email{nacer@jlab.org}
%\extrainfo{\cvline{Skype ID:}{\small abdennacer.hamdi\normalsize}}
%\extrainfo{additional information (optional)} % optional, remove the line if not wanted
\photo[90pt]{Hamdi_Abdennacer.jpg} % The first bracket is the picture height, the second is the thickness of the frame around the picture (0pt for no frame)
%\quote{"Success is the ability to go from failure to failure without losing your enthusiasm." -- Winston Churchill} % optional, remove the line if not wanted
%\nopagenumbers{}   % uncomment to suppress automatic page numbering for CVs longer than one page
%\extrainfo{\cvline{Nationality:}{\small Algerian\normalsize}}

%----------------------------------------------------------------------------------
%            content
%----------------------------------------------------------------------------------
\begin{document}
\maketitle

%Section
%\section{Info}
%\cvline{Birth}{\small 30 may 1986 (<city>)\normalsize}
%\cvcomputer{Citizenship}{\small <Citisenship>\normalsize}{Driving License}{\small <License> \normalsize}
%\cvline{Elance}{\small \url{<Link to profile>}\normalsize}
%\cvline{LinkedIn}{\small \url{<Link to profile>}\normalsize}
%\cvline{Blog}{\small \url{<Link to profile>}\normalsize}
%\cvline{Skype}{\small abdennacer.hamdi\normalsize}

%Section
\section{Summary of Research Interests}
%\cvline{}{\Large PhD in Particle Physics}
\cvline{}{\small Currently I am a PhD researcher in physics in my last year with the GlueX and PANDA experiments. The goal of my current project is searching for a manifestation of the gluonic degrees of freedom in the hadron spectrum, which will yield fundamental insight into the non-perturbative QCD. I also had the opportunity during my masters to work with ATLAS experiment on the search for a dark matter candidate. Furthermore, I conducted a feasibilty study of a precision measurement of the top quark coupling to the Higgs boson with the future leptonic accelerators, as well as studying the performance of a new hadronic calorimeter prototype proposed by the CALICE collaboration. All these experiences have strengthened my skills in scientific research and I have gained experience working with different research teams. Team work, the capacity to adapt to multicultural environment and the ability to coordinate and monitor scientific projects gained during my education and research internships. As a next step, I would like to join a research team in experimental physics. I am confident that I can contribute to the efforts of the team both, in physics analysis and addressing the needs of the experiment in hardware and software development.}

%Section
\section{Education}
\cventry{2016 - 2020}{Ph.D in Physics}{Goethe-Universität Frankfurt am Main}{Frankfurt am Main}{Germany}{} % arguments 3 to 6 are optional
\cventry{2013 - 2015}{Masters in Particle Physics}{Blaise Pascal university - Science and Technology}{Clermont-Ferrand}{France}{} % arguments 3 to 6 are optional
\cventry{2005 - 2010}{Nuclear engineering}{Ferhat Abbas University}{Sétif}{Algeria}{} % arguments 3 to 6 are optional

%Section
%\section{Master thesis}
%\cvline{title}{\emph{Title}}
%\cvline{supervisors}{Supervisors}
%\cvline{description}{\small Short thesis abstract}


%Section
\section{Experience \& Training}

%\subsection{Academic}
\cventry{2016 - 2019}{Graduate Researcher}{GSI Helmholtz Centre}{Darmstadt}{Germany}{Thesis title: Search for the $Y(2175)$ photoproduction at GlueX experiment.
\begin{itemize}
    \item Delivering an optimal mean energy loss truncation in GlueX Central Drift Chamber detector.
    \item Making an impact as an effective researcher, by delivering first measurements in photo-production for the physics channels: $\gamma p\rightarrow\phi\pi^{+}\pi^{-}/\phi f_{0}$ and $Y(2175)\rightarrow\phi\pi^{+}\pi^{-}/\phi f_{0}$.
    \item Representing my team in the collaboration during physics working group meetings.
    \item Optimization of statistical methods to overcome the physics analysis challenges.
    \item Strengthening my computing skills.
    \item Extending my knowledge of particle physics and hadron spectroscopy.
\end{itemize}} % arguments 3 to 6 are optional

% \cventry{2016 - 2019}{HGS-HIRe Softskill Workshops}{Helmholtz Graduate School}{Frankfurt Am Main}{Germany}{
% \begin{itemize}
%     \item Making an impact as an effective researcher, by delivering first measurements in photo-production for an important physics channel ($\gamma p\rightarrow\phi\pi^{+}\pi^{-}$), and 
%     \item Leading teams in a research environment, by representing my physics analysis group in the collaboration in many work events.
% \end{itemize}} % arguments 3 to 6 are optional

\cventry{2017 - 2019}{GlueX data taking}{Thomas Jefferson Lab}{Newport News}{USA}{Participation in experimental data collection and evaluation of the detectors performance.} % arguments 3 to 6 are optional

\cventry{2015}{Master Internship in Particle Physics}{Laboratoire de Physique Corpusculaire}{Clermont-Ferrand}{France}{Report title: Study of the Higgs Yukawa coupling to the Top quark, and of the High Granularity Calorimeter for the future Electron - Positron experiments.
\begin{itemize}
    \item Learning the phenomenology of the Higgs and Top quark in the Standard Model.
    \item Optimization of measurements by statistical methods, using jet multiplication and same charge leptons to overcome the challenging background.
    \item Studying the performance of the hadronic calorimeter of the CALICE collaboration, and participating in the data collection using the SPS beam at CERN.
    \item Building a Monte Carlo simulation to study the separation power of particles produced in quark and gluon jets.
    \item Improve my computing skills.
    \item Extend knowledge of Particle Physics, detectors, especially of calorimeters.
\end{itemize}} % arguments 3 to 6 are optional

% \cventry{2015}{$\bm{1^{st}}$ BCD International School on High Energy Physics}{Scientific Study Institut Cargese}{Cargese, Corsica}{France}{The School comprises lectures on experiments and phenomenology in the field of High Energy Physics, including a presentation of my experimental results, and is organized by the universities of Bologna (Italy), Clermont-Ferrand (France) and Dortmund (Germany)}

\cventry{2014}{Master Internship in Particle Physics}{Laboratoire de Physique Corpusculaire}{Clermont-Ferrand}{France}{Report title: Search for monotop events in the ATLAS experiment.
\begin{itemize}
    \item Develop a data analysis relying on Monte-Carlo simulations, to select a single top quark with transverse missing energy final state, which is a potential candidate for dark matter in the search for New Physics.
    \item Learning the main analysis tools and understanding the proposed theoretical models.
\end{itemize}} % arguments 3 to 6 are

\cventry{2012 - 2013}{Physicist Engineer in Radiation Protection}{Algiers Nuclear Research Center}{Algiers}{Algeria}
{\begin{itemize}
    \item Working in a research team carrying out in-depth qualitative and quantitative evaluation of radioactivity in the environment.
    \item Teaching master students about basics of the radiation protection and assisting them during practical lab manipulation.
\end{itemize}} % arguments 3 to 6 are optional

\cventry{2011 - 2012}{Radiation Protection Training}{Algiers Nuclear Research Center}{Algiers}{Algeria}
{\begin{itemize}
    \item Measure and record radiation levels. 
    \item Calibrate radiation protection instruments and equipment. 
    \item Ensuring the safety of employees working in radiation areas, as well as the facility’s compliance with radiation requirements.
\end{itemize}} % arguments 3 to 6 are optional

% \cventry{2010}{Master internship in Nuclear Physics}{Draria Nuclear Research Center}{Algiers}{Algeria}{Master thesis submitted to the University of Sétif for the Master's degree in Nuclear physics.\\
% Thesis Title: “control of radioactivity in the air of the NUR reactor hall, influence of the ventilation system on the detection limit”.} % arguments 3 to 6 are optional

\section{Presentations}

\cventry{May 2019}{FAIR next generation scientist - 6th Edition Workshop}{Genova}{Italy}{}{Search for exotic states in photo-production at GlueX} % arguments 3 to 6 are optional
\cventry{March 2019}{DPG Spring Meeting}{Technischen Universität München}{München}{Germany}{Search for the Y(2175) in photo-production at GlueX} % arguments 3 to 6 are optional
\cventry{May 2018}{Panda PID Computing Workshop}{GSI}{Darmstadt}{German}{Energy loss estimation in the Central Drift Chamber} % arguments 3 to 6 are optional
\cventry{March 2018}{DPG Spring Meeting}{Ruhr-Universität Bochum}{Bochum}{Germany}{Search for the Y(2175) in Photo-Production at GlueX} % arguments 3 to 6 are optional
\cventry{May 2017}{GlueX Collaboration Meeting}{Thomas Jefferson Lab}{Newport News}{USA}{Energy loss estimation in the Central Drift Chamber} % arguments 3 to 6 are optional
\cventry{August 2017}{A5 Seminar}{Justus-Liebig-Universität Gießen}{Gießen}{Germany}{The GlueX Experiment} % arguments 3 to 6 are optional

\section{Publications}

\cventry{}{Search for exotic states in photoproduction at GlueX}{}{}{}{GlueX Collaboration et al. 2019 arXiv:1908.11786}
\cventry{}{Beam Asymmetry $\Sigma$ for the Photoproduction of $\eta$ and $\eta'$ Mesons at $E_\gamma=8.8$ GeV}{}{}{}{GlueX Collaboration et al. 2019 arXiv:1908.05563}
% \cventry{}{Baryon -- anti-Baryon Photoproduction}{}{}{}{GlueX Collaboration et al. 2019 arXiv:1909.08091}
\cventry{}{First measurement of near-threshold $J/\psi$ exclusive photoproduction off the proton}{}{}{}{GlueX Collaboration et al. 2019 arXiv:1905.10811}
\cventry{}{Precision resonance energy scans with the PANDA experiment at FAIR: Sensitivity study for width and line-shape measurements of the X(3872)}{}{}{}
{PANDA Collaboration et al. 2019 arXiv:1812.05132}
\cventry{}{Strange Hadron Spectroscopy with a Secondary $K_{L}$ Beam at GlueX}{}{}{}{GlueX Collaboration (S. Adhikari (Florida Intl. U.) et al.) 2017 arXiv:1707.05284}

%\subsection{Working}
%\cventry{start-end}{<Position Held>}{<Name of employer>}{<Place>}{<Country>}{<Description>} % arguments 3 to 6 are optional
%\cventry{start-end}{<Position Held>}{<Name of employer>}{<Place>}{<Country>}{<Description>} % arguments 3 to 6 are optional
%\subsection{Various}
%\cventry{start-end}{<Brief Description>}{<Institution>}{<Place>}{<Country>}{<Description>} % arguments 3 to 6 are optional

%Section
\section{languages}

\hspace{25mm}Fluent in \textbf{English}, \textbf{French}, \textbf{Arabic} and \textbf{Berber}, basic \textbf{German}

%\hspace{25mm}\small Self-assessment European level \href{http://europass.cedefop.europa.eu/en/resources/european-language-levels-cefr}{CEFR} (C2 maximum evaluation)\normalsize
%\vspace{5mm}

%\begin{tabular}{p{67mm} p{40mm} p{40mm} p{20mm}}
%& \textbf{Understanding} & \textbf{Speaking} & \textbf{Writing} \\
%\end{tabular}

%\begin{tabular}{p{67mm} p{20mm} p{20mm} p{20mm} p{20mm} p{20mm}}
%& Listening & Reading & Interaction & Production & \\
%\end{tabular}

%\vspace{3mm}
%lvl should be in this range A1 < A2 < B1 < B2 < C1 < C2
%\cvlanguage{<Lang 1>}{<Level>}{
%	\begin{tabular}{p{20mm} p{20mm} p{20mm} p{20mm} p{21mm}}
%		lvl & lvl & lvl & lvl & lvl
%	\end{tabular}}
%\cvlanguage{<Lang 2>}{<Level>}{
%	\begin{tabular}{p{20mm} p{20mm} p{20mm} p{20mm} p{21mm}}
%		lvl & lvl & lvl & lvl & lvl
%	\end{tabular}}

%Section
\section{Computer Skills} 
\cvcomputer{Languages}{C/C++, Python, Bash, Csh, \LaTeX, HTML}{}{}
\cvcomputer{Platforms}{Linux, Mac OS, Windows}{}{}
\cvcomputer{Tools}{\small ROOT, Matlab, LabVIEW, Office Package}{}{}

%Section
%\section{Interests and Hobbies}
%\cvline{}{\small <Description>}

%Section
%\section{Extra}
%\cvline{<Extra Content>}{\small <Description>}
%\small
%\cvlistitem{\href{...}{<Eventual link>}}
%\cvlistitem{\href{...}{<Eventual link>}}

\closesection{}                   % needed to renewcommands
\renewcommand{\listitemsymbol}{-} % change the symbol for lists

% Publications from a BibTeX file
%\nocite{*}
%\bibliographystyle{plain}
%\bibliography{publications}       % 'publications' is the name of a BibTeX file

\end{document}