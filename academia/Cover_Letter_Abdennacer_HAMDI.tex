\documentclass[11pt]{letter}
 \usepackage[utf8]{inputenc} % un package
 \usepackage[T1]{fontenc} % un second package
%\usepackage[latin1]{inputenc}
 \usepackage[english]{babel} % un troisième package
 \usepackage{textcomp}
 \usepackage{ifpdf}
\ifpdf
 \usepackage[pdftex]{graphicx}
 \else
 \usepackage[dvips]{graphicx}\fi
\pagestyle{empty}
\usepackage[scale=0.88]{geometry}
\setlength{\parindent}{1em}
\addtolength{\parskip}{0.5pt}
\renewcommand{\ttdefault}{pcr}
 \begin{document}
 \sffamily
 \hfill
 %start
 \begin{flushleft}
 {\bfseries Abdennacer Hamdi}\\[.35ex]
 \small\itshape
 GSI Helmholtzzentrum f\"ur Schwerionenforschung GmbH\\ % GSI Helmholtzzentrum f\"ur Schwerionenforschung GmbH /GSI Helmholtz Centre
 Planckstraße 1,\\
 64291 Darmstadt,\\
 Germany\\ %[.35ex]
%  +49 (0) 1781043207\\
%  a.hamdi@gsi.de
 \end{flushleft}
 %
 \begin{flushleft}
 To {\bfseries Prof. Dr. Anna Nelles}\\[.35ex]
 \small\itshape
 Physics Insitute \\
 Erwin-Rommel-Str. 1 \\
 91058 Erlangen \\
 Germany

 \end{flushleft}
 %
 \hfill
 %
 \begin{flushleft}
 Darmstadt, \today \\
 
 \end{flushleft}
 %
% \textit{Subject:} Application for postdoctoral position in experimental elementary particle physics.\\
\par Dear Prof. Dr. Anna Nelles,
%Content of the letter
~\par It is with great pleasure that I am writing to you to express my interest in applying for the postdoc position in the radio detection of neutrinos at DESY, to work together with the physicists of DESY and IceCube groups. This project appeals to me because of my strong interest in exploring the nature of elementary particles, in testing the Standard Model and cosmology predictions, with its peculiar signatures in particle accelerators or direct observations in the universe, such as searching for neutrinos in the radio neutrino observatory. Due to my experience regarding particle identification and hardware, I see also a potential to contribute to the efforts of the team addressing the needs of the experiment by developing the new software and hardware, in particular participating in the R$\&$D of the IceCube-Gen2.
% with its peculiar signatures in particle accelerators or direct observations in the universe such as DM candidates
% with its peculiar signatures in particle accelerators, such as studying the photoproduction of light hybrid meson candidates
%, in particular participating in the R$\&$D of the IceCube-Gen2.
% in High energy colliders such as Supersymmetry and DM candidates
% particularly the challenges of High energy neutrino measurements and their unique ability to probe some of the most enigmatic processes at the core of stars and galaxies from across cosmological distances.
% As well as using the knowledge I have gained through graduate research and education. 
I have been seeking just such an opportunity as this, it is clear to me that DESY is genuinely devoted to the advancement of research and education, I will have the opportunity to work and learn from the best and the privilege to work in an international environment. I believe I can contribute to that mission through my passion for physics, my education and background would be a good match for your requirements.
~\par I mention that I am currently finishing my Ph.D project at the Goethe University of Frankfurt in Germany, where I work on testing Quantum ChromoDynamics (QCD) predictions. My goal is to search for mesons that manifest the gluon degrees of freedom for the first time in photoproduction at the GlueX experiment. I have a master's degree in particle physics at the Clermont Auvergne University in France. During this program, I was involved in two research projects, one was with the ATLAS team, searching for a candidate to the Dark Mater (neutralino) predicted by New Physics theories, and a second project studying the potential of a precise measurement of the Higgs coupling to the top quark in the future electron-positron colliders. Furthermore, I studied the performance of the new hadronic calorimeter proposed by the CALICE collaboration. I have worked as an Engineer Physicist specialized in Radiation Protection at the Algiers Nuclear Research Center, where I had the opportunity to teach master students about radiation protection and guide them through laboratory practical manipulations. I graduated in 2010, from S\'etif University in Algeria and got a Master degree in Nuclear Physics.
% , completed with a Master thesis at the Draria Nuclear Research Center. 
All these challenges have strengthened my skills in laboratory work and I have gained experience working with different research teams.
~\par I am eager to talk with you about the research project you proposed this would offer me the possibility of receiving the complete picture of this postdoctoral position. I thank you for the opportunity to submit this application for your consideration. Please contact me if there is anything else I need to provide. I look forward to hearing from you in the near future.\\\\
% I am interested in exploring the experimental nature of elementary particles, especially hadron spectroscopy, . 
% grand unification theories and physics beyond the standard model, such as Supersymmetry and dark matter phenomenology. After completing this PhD programme, I plan to apply for research positions in the field of Particle Physics.
% My recommendation letters will be sent soon in a separate email
% theoretical and, especially grand unification theories and physics beyond the standard model, particularly its signatures in colliders such as Supersymmetry and dark matter candidates experiments
%Please find enclosed my CV. \\ \\
Yours sincerely,

\begin{flushleft}
 {\bfseries Abdennacer Hamdi}
 \end{flushleft}
 \vfill
 \end{document}