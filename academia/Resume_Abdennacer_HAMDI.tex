\documentclass[11pt,a4paper,roman]{moderncv}
\moderncvtheme[blue]{classic} % optional argument are 'blue' (default), 'orange', 'red', 'green', 'grey' 
%\moderncvtheme[green]{classic}                % idem
\usepackage[T1]{fontenc}
% character encoding
\usepackage[utf8x]{inputenc}                   % replace by the encoding you are using
\usepackage[english]{babel}
\usepackage{bm}
% adjust the page margins
\usepackage[scale=0.92]{geometry} %0.9
\recomputelengths                             % required when changes are made to page layout lengths
\fancyfoot{} % clear all footer fields
% \fancyfoot[LE,RO]{\thepage}           % page number in "outer" position of footer line
% \fancyfoot[RE,LO]{\footnotesize} % other info in "inner" position of footer line

% personal data
\firstname{Abdennacer}
\familyname{Hamdi}
%\title{Curriculum Vitae}
\address{GSI Helmholtz Center}{Heckb\"uro 1.001, Planckstraße 1}{64291 Darmstadt, Germany}
\mobile{+49 6159 71 2556}
%\phone{<Phone number>} 
%\fax{<Fax number>}          
\email{A.Hamdi@gsi.de}
% \email{nacer@jlab.org}
%\extrainfo{\cvline{Skype ID:}{\small abdennacer.hamdi\normalsize}}
%\extrainfo{additional information (optional)} % optional, remove the line if not wanted
% \photo[90pt]{Hamdi_Abdennacer.jpg} % The first bracket is the picture height, the second is the thickness of the frame around the picture (0pt for no frame)
% \photo[100pt]{photo/Hamdi-A-14.jpg}
\photo[90pt]{photo/nacer_pic_2.jpg}
% \photo[100pt]{photo/nacer_pic_6.jpg}
% \photo[220pt]{photo/Hamdi-A-12.jpg}
% \photo[210pt]{photo/Hamdi-B-06.jpg}
% \photo[210pt]{photo/Hamdi-C-02.jpg}
%\quote{"Success is the ability to go from failure to failure without losing your enthusiasm." -- Winston Churchill} % optional, remove the line if not wanted
%\nopagenumbers{}   % uncomment to suppress automatic page numbering for CVs longer than one page
%\extrainfo{\cvline{Nationality:}{\small Algerian\normalsize}}

%----------------------------------------------------------------------------------
%            content
%----------------------------------------------------------------------------------
\begin{document}
\maketitle

%Section
%\section{Info}
%\cvline{Birth}{\small 30 may 1986 (<city>)\normalsize}
% \cvline{Birth}{30 may 1986 (Hammam Guergour, Algeria)\normalsize}
%\cvcomputer{Citizenship}{\small <Citisenship>\normalsize}{Driving License}{\small <License> \normalsize}
%\cvline{Elance}{\small \url{<Link to profile>}\normalsize}
%\cvline{LinkedIn}{\small \url{<Link to profile>}\normalsize}
%\cvline{Blog}{\small \url{<Link to profile>}\normalsize}
%\cvline{Skype}{\small abdennacer.hamdi\normalsize}

%Section
\section{Summary of Research Interests}
%\cvline{}{\Large PhD in Particle Physics}
\cvline{}{\small PhD researcher in physics with the GlueX and PANDA experiments. The project revolve around the search for exotic mesons that will yield fundamental insight into the non-perturbative QCD. I also worked with the ATLAS experiment on the search for a DM candidate, and with the future electron-positron colliders on the potential of a precise measurement of the top quark and Higgs boson coupling. Furthermore, I have studied the performance of a new hadronic calorimeter prototype proposed by the CALICE collaboration, and participated in the CERN Beam tests. I have mentored master students during the lab manipulations of the gamma spectroscopy detectors, and offered them basic courses in radiation protection. All these challenges have strengthened my skills in scientific research, and gained experience in coordinating and monitoring scientific projects. 
% As a next step, I would like to join a research team in experimental physics. I am confident that I can contribute to the efforts of the team both, in physics analysis and addressing the needs of the experiment in hardware and software development.
}

%Section
\section{Education}
\cventry{2017 - 2020}{Ph.D in Physics}{Goethe-Universität Frankfurt am Main}{Frankfurt am Main}{Germany}{} % arguments 3 to 6 are optional
\cventry{2013 - 2015}{Masters in Particle Physics}{Clermont Auvergne University}{Clermont-Ferrand}{France}{} % arguments 3 to 6 are optional
\cventry{2005 - 2010}{Diploma in Nuclear Engineering}{Ferhat Abbas University}{Sétif}{Algeria}{} % arguments 3 to 6 are optional

%Section
%\section{Master thesis}
%\cvline{title}{\emph{Title}}
%\cvline{supervisors}{Supervisors}
%\cvline{description}{\small Short thesis abstract}


%Section
\section{Experience \& Training}

%\subsection{Academic}
\cventry{2017 - 2020}{Graduate Researcher}{GSI Helmholtz Centre}{Darmstadt}{Germany}{Thesis title: Exotic Meson Photoproduction at GlueX.
\begin{itemize}
    \item First measurements of cross sections in some particular physics channels in photoproduction.
    \item Delivering an optimal mean energy loss truncation in GlueX Central Drift Chamber detector.
    \item Representing my team in the collaboration on the physics working group and collaboration meetings.
    % \item Optimizing statistical methods to overcome the physics analysis challenges.
    % \item Extending my knowledge of particle physics and hadron spectroscopy.
\end{itemize}} % arguments 3 to 6 are optional

% \cventry{2016 - 2019}{HGS-HIRe Softskill Workshops}{Helmholtz Graduate School}{Frankfurt Am Main}{Germany}{
% \begin{itemize}
%     \item Making an impact as an effective researcher, by delivering first measurements in photoproduction for an important physics channel ($\gamma p\rightarrow\phi\pi^{+}\pi^{-}$), and 
%     \item Leading teams in a research environment, by representing my physics analysis group in the collaboration in many work events.
% \end{itemize}} % arguments 3 to 6 are optional

\cventry{2017 - 2019}{Data Collection}{Thomas Jefferson Lab}{Newport News}{USA}{Collecting data and evaluating the detectors performance.} % arguments 3 to 6 are optional

\cventry{2015}{Research Internship in Particle Physics}{Laboratoire de Physique de Clermont}{Clermont-Ferrand}{France}{Report title: Study of the Higgs Yukawa coupling to the Top quark, and of the High Granularity Calorimeter for the future Electron - Positron experiments.
\begin{itemize}
    % \item Learning the phenomenology of the Higgs and Top quark in the Standard Model.
    \item Optimizing measurements by statistical methods, using jet multiplication and same charge leptons to overcome the challenging background.
    \item Studying the performance of the hadronic calorimeter of the CALICE collaboration, and participating in the data collection using the SPS beam at CERN.
    \item Building a Monte Carlo simulation to study the separation power of particles produced in quark and gluon jets.
\end{itemize}} % arguments 3 to 6 are optional

% \cventry{2015}{$\bm{1^{st}}$ BCD International School on High Energy Physics}{Scientific Study Institut Cargese}{Cargese, Corsica}{France}{The School comprises lectures on experiments and phenomenology in the field of High Energy Physics, including a presentation of my experimental results, and is organized by the universities of Bologna (Italy), Clermont-Ferrand (France) and Dortmund (Germany)}

\cventry{2014}{Research Internship in Particle Physics}{Laboratoire de Physique de Clermont}{Clermont-Ferrand}{France}{Report title: Search for Monotop Events in The ATLAS Experiment.
\begin{itemize}
    \item Developing a data analysis relying on Monte-Carlo simulations, to select a single top quark with transverse missing energy final state, which is a potential candidate for dark matter in the search for New Physics.
    \item Learning the main analysis tools and understanding the proposed theoretical models.
\end{itemize}} % arguments 3 to 6 are

\cventry{2012 - 2013}{Physicist Engineer in Radiation Protection}{Algiers Nuclear Research Center}{Algiers}{Algeria}
{\begin{itemize}
    \item Working in a research team carrying out in-depth qualitative and quantitative evaluation of radioactivity in the environment.
    \item Mentoring master students about basics of the radiation protection and assisting them during practical lab manipulation.
\end{itemize}} % arguments 3 to 6 are optional

\cventry{2011 - 2012}{Radiation Protection Training}{Algiers Nuclear Research Center}{Algiers}{Algeria}
{\begin{itemize}
    \item Measuring and recording radiation levels. 
    \item Calibrating radiation protection instruments and equipment. 
    \item Ensuring the safety of employees working in radiation areas, as well as the facility’s compliance with radiation requirements.
\end{itemize}} % arguments 3 to 6 are optional

% \cventry{2010}{Master internship in Nuclear Physics}{Draria Nuclear Research Center}{Algiers}{Algeria}{Master thesis submitted to the University of Sétif for the Master's degree in Nuclear physics.\\
% Thesis Title: “control of radioactivity in the air of the NUR reactor hall, influence of the ventilation system on the detection limit”.} % arguments 3 to 6 are optional

\section{Selected Presentations}

\cventry{Feb. 2020}{GlueX Collaboration Meeting}{Thomas Jefferson Lab}{Newport News}{USA}{Search for the Y(2175) in photoproduction at GlueX} % arguments 3 to 6 are optional
\cventry{May 2019}{FAIR next generation scientist - 6th Edition Workshop}{Genova}{Italy}{}{Search for exotic states in photoproduction at GlueX} % arguments 3 to 6 are optional
\cventry{Mar. 2019}{German Physical Society (DPG) Spring Meeting}{Technischen Universität München}{München}{Germany}{Search for the Y(2175) in photoproduction at GlueX} % arguments 3 to 6 are optional
\cventry{May 2018}{PID Computing Workshop - PANDA experiment}{GSI}{Darmstadt}{German}{Energy loss estimation in the Central Drift Chamber} % arguments 3 to 6 are optional
\cventry{Mar. 2018}{German Physical Society (DPG) Spring Meeting}{Ruhr-Universität Bochum}{Bochum}{Germany}{Search for the Y(2175) in photoproduction at GlueX} % arguments 3 to 6 are optional
\cventry{May 2017}{GlueX Collaboration Meeting}{Thomas Jefferson Lab}{Newport News}{USA}{Energy loss estimation in the Central Drift Chamber} % arguments 3 to 6 are optional
\cventry{Aug. 2017}{Dramstdat-Gießen-Heidelberg Seminar}{Justus-Liebig-Universität Gießen}{Gießen}{Germany}{The GlueX Experiment} % arguments 3 to 6 are optional

\section{Selected Publications}

\cventry{}{Search for exotic states in photoproduction at GlueX}{}{}{}{GlueX Collaboration et al. 2019 arXiv:1908.11786}
\cventry{}{Beam Asymmetry $\Sigma$ for the Photoproduction of $\eta$ and $\eta'$ Mesons at $E_\gamma=8.8$ GeV}{}{}{}{S. Adhikari et al. (The GlueX Collaboration), Phys. Rev. C 100, 052201(R), 2019. arXiv:1908.05563}
% \cventry{}{Baryon -- anti-Baryon Photoproduction}{}{}{}{GlueX Collaboration et al. 2019 arXiv:1909.08091}
\cventry{}{First measurement of near-threshold $J/\psi$ exclusive photoproduction off the proton}{}{}{}{GlueX Collaboration et al. ,Phys.Rev.Lett. 123 (2019). arXiv:1905.10811}
\cventry{}{Precision resonance energy scans with the PANDA experiment at FAIR: Sensitivity study for width and line-shape measurements of the X(3872)}{}{}{}
{PANDA Collaboration et al. ,Eur.Phys.J. A55 (2019). arXiv:1812.05132}
\cventry{}{Strange Hadron Spectroscopy with a Secondary $K_{L}$ Beam at GlueX}{}{}{}{GlueX Collaboration (S. Adhikari (Florida Intl. U.) et al.) 2017 arXiv:1707.05284}

%\subsection{Working}
%\cventry{start-end}{<Position Held>}{<Name of employer>}{<Place>}{<Country>}{<Description>} % arguments 3 to 6 are optional
%\cventry{start-end}{<Position Held>}{<Name of employer>}{<Place>}{<Country>}{<Description>} % arguments 3 to 6 are optional
%\subsection{Various}
%\cventry{start-end}{<Brief Description>}{<Institution>}{<Place>}{<Country>}{<Description>} % arguments 3 to 6 are optional

%Section
\section{languages}

\hspace{25mm}Fluent in \textbf{English} and \textbf{French}. \textbf{German} (B1).

%\hspace{25mm}\small Self-assessment European level \href{http://europass.cedefop.europa.eu/en/resources/european-language-levels-cefr}{CEFR} (C2 maximum evaluation)\normalsize
%\vspace{5mm}

%\begin{tabular}{p{67mm} p{40mm} p{40mm} p{20mm}}
%& \textbf{Understanding} & \textbf{Speaking} & \textbf{Writing} \\
%\end{tabular}

%\begin{tabular}{p{67mm} p{20mm} p{20mm} p{20mm} p{20mm} p{20mm}}
%& Listening & Reading & Interaction & Production & \\
%\end{tabular}

%\vspace{3mm}
%lvl should be in this range A1 < A2 < B1 < B2 < C1 < C2
%\cvlanguage{<Lang 1>}{<Level>}{
%	\begin{tabular}{p{20mm} p{20mm} p{20mm} p{20mm} p{21mm}}
%		lvl & lvl & lvl & lvl & lvl
%	\end{tabular}}
%\cvlanguage{<Lang 2>}{<Level>}{
%	\begin{tabular}{p{20mm} p{20mm} p{20mm} p{20mm} p{21mm}}
%		lvl & lvl & lvl & lvl & lvl
%	\end{tabular}}

%Section
\section{Computer Skills} 
\cvcomputer{Languages}{C/C++, Python, Bash, Csh, \LaTeX, HTML}{}{}
\cvcomputer{Platforms}{Linux, Mac OS, Windows}{}{}
\cvcomputer{Tools}{\small ROOT, Matlab, LabVIEW, Office Package}{}{}

%Section
%\section{Interests and Hobbies}
%\cvline{}{\small <Description>}

%Section
%\section{Extra}
%\cvline{<Extra Content>}{\small <Description>}
%\small
%\cvlistitem{\href{...}{<Eventual link>}}
%\cvlistitem{\href{...}{<Eventual link>}}

\closesection{}                   % needed to renewcommands
\renewcommand{\listitemsymbol}{-} % change the symbol for lists

% Publications from a BibTeX file
%\nocite{*}
%\bibliographystyle{plain}
%\bibliography{publications}       % 'publications' is the name of a BibTeX file

\end{document}